\documentclass{article}

\usepackage{amsmath, amssymb, amsthm, amscd, alltt, graphicx}
\usepackage[utf8]{inputenc}
\usepackage[T1]{fontenc}
\usepackage[capitalise]{cleveref}
\usepackage[matrix,arrow,curve]{xy}
\usepackage[notref, notcite]{showkeys}

\usepackage[backend=biber, bibencoding=utf8, giveninits=true, citestyle=numeric-comp, sortlocale=en_US, url=false, doi=false, eprint=true, maxbibnames=4]{biblatex}
\addbibresource{f4paper.bib}
\renewbibmacro*{volume+number+eid}{\ifentrytype{article}{\- \iffieldundef{volume}{}{Vol.~\printfield{volume},}\iffieldundef{number}{}{ No.~\printfield{number},}}}
\renewbibmacro{in:}{\ifentrytype{article}{}{\printtext{\bibstring{in}\intitlepunct}}}
\newbibmacro{string+doi}[1]{\iffieldundef{doi}{\iffieldundef{url}{#1}{\href{\thefield{url}}{#1}}}{\href{https://dx.doi.org/\thefield{doi}}{#1}}}
\DeclareFieldFormat[article, inproceedings, inbook, book, online]{title}{\usebibmacro{string+doi}{\mkbibquote{#1}}}
\renewcommand*{\bibfont}{\footnotesize}

\newtheorem{lemma}{Lemma} \numberwithin{lemma}{section}
\newtheorem{prop}{Proposition} \numberwithin{prop}{section}
\newtheorem{theorem}{Theorem}
\newtheorem*{theorem*}{Theorem}

\theoremstyle{definition} 
\newtheorem{df}[lemma]{Definition} \Crefname{df}{Definition}{Definitions}
\newtheorem{example}[lemma]{Example} \Crefname{example}{Example}{Examples}

\theoremstyle{remark} 
\newtheorem{rem}[lemma]{Remark}
\newtheorem{conv}[lemma]{Convention} \Crefname{conv}{Convention}{Conventions}

\newenvironment{psmallmatrix}{\left(\begin{smallmatrix}}{\end{smallmatrix}\right)}

\DeclareMathOperator\mat{M}
\DeclareMathOperator\elin{E}
\DeclareMathOperator\stlin{St}
\DeclareMathOperator\St{St}
\DeclareMathOperator\Ker{Ker}
\DeclareMathOperator\GG{G}
\DeclareMathOperator\gstlin{GSt}
\DeclareMathOperator\klin{K}
\DeclareMathOperator\stmap{st}
\DeclareMathOperator\glin{GL}
\DeclareMathOperator\diag{D}
\DeclareMathOperator\Jac{J}
\DeclareMathOperator{\id}{id}
\DeclareMathOperator{\Pro}{Pro}
\DeclareMathOperator{\Ad}{Ad}

\newcommand{\lar}{\leftarrow}
\newcommand{\leqt}{\trianglelefteq}
\newcommand{\sub}[2]{{_{#1\!}{#2}}}
\newcommand{\up}[2]{{^{#1}\!{#2}}}

\newcommand{\Set}{\mathbf{Set}}
\newcommand{\Group}{\mathbf{Grp}}
\newcommand{\Rng}{\mathbf{Rng}}
\newcommand{\Fun}{\mathbf{Fun}}
\newcommand{\Mod}{\mathbf{Mod}}
\newcommand{\op}{\mathrm{op}}

\begin{document}

\section{Preliminaries}
\subsection{Generalities on pro-objects}
Let \(\mathcal C\) be an arbitrary category.
In this section we recall the construction of the pro-completion of \(\mathcal C\) (cf. \cite[Section~6.1]{SK06}).

Recall that a nonempty small category \(\mathcal I\) is called {\it filtered} if
\begin{itemize}
 \item for any two objects \(i, k \in Ob(\mathcal{I})\) there is a diagram \(i \to j \lar k\) in \(\mathcal I\);
 \item every two parallel morphisms \(i \rightrightarrows j\) are equalized by some morphism \(j \to k\) in \(\mathcal I\).
\end{itemize}
A {\it pro-object} in \(\mathcal C\) is, by definition, a functor $X^{(\infty)}\colon \mathcal{I}_X^{\op} \to \mathcal{C}$, i.\,e. a contravariant functor from a filtered category \(\mathcal I_X\), called {\it the category of indices of $X^{(\infty)}$}, to the category \(\mathcal C\). 

We denote by \(X^{(i)}\) the value of \(X^{(\infty)}\) on an index \(i\).
The values of the functor $X^{(\infty)}$ on the arrows of $\mathcal{I}$ are called the {\it structure morphisms} of $X^{(\infty)}$.

The category of pro-objects is denoted by \(\Pro(\mathcal C)\). The hom-sets in this category are given by the formula
\begin{equation} \label{eq:pro-c-hom} \Pro(\mathcal C)(X^{(\infty)}, Y^{(\infty)}) = \varprojlim_{j \in \mathcal I_Y} \varinjlim_{i \in \mathcal I_X} \mathcal C(X^{(i)}, Y^{(j)}). \end{equation}
Let us recall a more explicit description of morphisms in \(\Pro(\mathcal C)\). 
By definition, a {\it pre-morphism} \(f \colon X^{(\infty)} \to Y^{(\infty)}\) consists of the following data:
\begin{itemize}
\item a set-theoretic function \(f^* \colon Ob(\mathcal{I_Y}) \to Ob(\mathcal{I_X})\);
\item a collection of morphisms \(f^{(i)} \colon X^{(f^*(i))} \to Y^{(i)}\) in $\mathcal{C}$ parametrized by $i \in Ob(\mathcal{I_Y})$. These morphisms are required to satisfy the following additional assumption: for every morphism \(i \to j\) in \(\mathcal I_Y\) there exists a sufficiently large index \(k \in Ob(\mathcal{I_X})\) such that the composite morphisms \(X^{(k)} \to X^{(f^*(i))} \to Y^{(i)}\) and \(X^{(k)} \to X^{(f^*(j))} \to Y^{(j)} \to Y^{(i)}\) are equal. \end{itemize}
The composition of two pre-morphisms \(f \colon X^{(\infty)} \to Y^{(\infty)}\) and \(g \colon Y^{(\infty)} \to Z^{(\infty)}\) is defined as the pre-morphism \(g \circ f\), where \((g \circ f)^*(i) = f^*(g^*(i))\) and \((g \circ f)^{(i)} = g^{(i)} \circ f^{(g^*(i))}\).

Two parallel pre-morphisms \(f, g \colon X^{(\infty)} \to Y^{(\infty)}\) are called equivalent if for every \(i \in Ob(\mathcal I_Y)\) there exists a sufficiently large index \(j \in Ob(\mathcal I_X)\) such that the composite morphisms \(X^{(j)} \to X^{(f^*(i))} \to Y^{(i)}\) and \(X^{(j)} \to X^{(g^*(i))} \to Y^{(i)}\) are equal. Finally, a {\it morphism} \(X^{(\infty)} \to Y^{(\infty)}\) is an equivalence class of pre-morphisms. Note that the equivalence relation is preserved by the composition operation.

There is a fully faithful functor $\mathcal{C} \to \Pro(\mathcal{C})$ sending \(X \in Ob(\mathcal C)\) to the pro-object $X \colon \mathbf{1}^\op \to \mathcal{C}$. It is clear from~\eqref{eq:pro-c-hom}  that
\begin{align}
 \Pro(\mathcal C)(X, Y) &\cong \mathcal C(X, Y);\\
 \Pro(\mathcal C)(X^{(\infty)}, Y) &\cong \varinjlim_{i \in \mathcal I_X} \mathcal C(X^{(i)}, Y); \label{eq:pro-c-hom-2}\\
 \Pro(\mathcal C)(X, Y^{(\infty)}) &\cong \varprojlim_{i \in \mathcal I_Y} \mathcal C(X, Y^{(i)}).
\end{align}

Moreover, it is also clear from~\eqref{eq:pro-c-hom} and~\eqref{eq:pro-c-hom-2} that the following assertion holds.
\begin{lemma} \label{lem:proobj-is-a-limit}
  \(X^{(\infty)}\) is the projective limit of \(X^{(i)}\) in the category \(\Pro(\mathcal C).\)
\end{lemma}

The category of pro-sets \(\Pro(\Set)\) has all finite limits by \cite[Prop.~6.1.18]{SK06} and therefore is a cartesian monoidal category.

Let us describe an explicit construction of limits in one important special case. Let $X^{(\infty)}, Y^{(\infty)} \colon \mathcal{I}^\op \to \Set\) be a pair of pro-sets with the same index category. Recall that the pointwise product of functors $X^{(\infty)} \times Y^{(\infty)} \colon \mathcal{I}^\op \to \Set$ is given by $(X^{(\infty)} \times Y^{(\infty)})^{(i)} = X^{(i)} \times Y^{(i)}$. Clearly, $X^\infty \times Y^\infty$ is a product of $X^{(\infty)}$ and $Y^{(\infty)}$  in the category $\Fun(\mathcal{I}^\op, \Set)$. 

There is an obvious identity-on-objects functor $\Fun(\mathcal{I}^\op, \Set) \to \Pro(\Set)$ sending a natural transformation $\varphi \colon X^{(\infty)} \to Y^{(\infty)}$ to the morphism of pro-sets given by the pre-morphism $\varphi^* = \mathrm{id}_{Ob(\mathcal{I})}$, $\varphi^{(i)} = \varphi_i$. The diagonal morphism $\Delta \colon X^{(\infty)} \to X^{(\infty)} \times X^{(\infty)}$ and the canonical projection morphisms \(\pi_X\colon X^{(\infty)} \times Y^{(\infty)} \to X^{(\infty)}, \pi_Y\colon X^{(\infty)} \times Y^{(\infty)} \to Y^{(\infty)}\) can be defined in the category $\Pro(\Set)$ as the images of the corresponding natural transformations in $\Fun(\mathcal{I}^\op, \Set)$ under this functor.

We claim that the point-wise product $X^{(\infty)} \times Y^{(\infty)}$ satisfies the universal property of products in the category $\Pro(\Set)$. Indeed, this follows from~\eqref{eq:pro-c-hom} and the fact that filtered colimits commute with finite limits. This argument also shows that arbitrary finite limits of pro-sets with the same index category can be computed pointwise.

\subsection{Group and ring objects in pro-sets}
In this paper we consider only commutative rings.
To distinguish between unital and non-unital rings we reserve the word ``ring'' only for unital rings and refer to non-unital rings as {\it rngs}.
We denote the category of rings (resp. rngs) as $\textbf{Ring}$ (resp. $\textbf{Rng}$).

Let $T$ be an algebraic theory. Throughout this paper we will be mostly interested in the case when $T$ is the theory of groups or the theory of rngs.

Since the category $\Pro(\Set)$ is cartesian monoidal, we may speak of the category $\Mod(T, \Pro(\Set))$ of models of $T$ in \(\Pro(\Set)\).
In the special case when $T$ is the theory of groups (resp. rngs) this category is precisely the category of group (resp. rng) objects in $\Pro(\Set)$.

The category of pro-groups \(\Pro(\Group)\) embeds into \(\Pro(\Set)\). Every pro-group is a group object in \(\Pro(\Set)\). It is easy to see that a morphism \(f \in \Pro(\Set)(G^{(\infty)}, H^{(\infty)})\) between pro-groups comes from \(\Pro(\Group)\) if and only if it is a morphism of group objects. Thus, $\Pro(\Group)$ is a full subcategory of the category of group objects in $\Pro(\Set)$.

\begin{df} \label{df-pro-set-morphisms} 
 Let $n$ be a natural number and $F_{n, T}$ be the free algebra on $n$ generators in the theory $T$.
 If $T$ is the theory of groups, then $F_n$ is the free group $F(t_1,\ldots, t_n)$.
 Similarly, if $T$ is the theory of rngs, $F_{n, T}$ is the sub-rng of the ring $\mathbb{Z}[t_1,\ldots, t_n]$ consisting of polynomials over $\mathbb{Z}$ without free terms.
 
 Given a word $w \in F_{n, T}$ and a $T$-model $X^{(\infty)}$ in $\Pro(\Set)$, one can construct the ``interpretation'' morphism in $\Pro(\Set)$
 \[ w^{(\infty)} \colon \underbrace{X^{(\infty)} \times \ldots \times X^{(\infty)}}_{n\text{ times}} \to X^{(\infty)}\]
 as an appropriate composition of diagonal morphisms $\Delta_{X^{(\infty)}}$, projections $\pi_i$ 
 (which are part of the structure of a cartesian monoidal category on $\Pro(\Set)$) 
 and the morphisms defining the structure of a $T$-model on $X^{(\infty)}$.
 
 In order to shorten the notation, to denote the interpretation morphism $w^{(\infty)}$ we often use the original term for $w$,
  in which every symbol of a free variable $t_i$ is replaced by the symbol $t_i^{(\infty)}$.
\end{df}

\begin{example}\label{example-commutator}
Consider the case when $T$ is the theory of groups.
Let $G^{(\infty)}$ be a group object in $\Pro(\Set)$ and $w = [t_1, t_2] = t_1 t_2 t_1^{-1} t_2^{-1} \in F(t_1, t_2)$ be the word representing the generic commutator.
In this case the interpretation morphism $w^{(\infty)}$ (also denoted $[t_1^{(\infty)}, t_2^{(\infty)}]$) can be defined as the composition
 \[ \xymatrix{G^{(\infty)} \times G^{(\infty)} \ar[r]^(.35){\Delta \times \Delta} & G^{(\infty)} \times G^{(\infty)} \times G^{(\infty)} \times G^{(\infty)} \ar[d]_{\langle \pi_1, \pi_3, i\pi_2, i\pi_4 \rangle} & \\
    & G^{(\infty)} \times G^{(\infty)} \times G^{(\infty)} \times G^{(\infty)} \ar[r]^(.76){m(m\times m)} & G^{(\infty)},} \]
where the structure of a group object on $G^{(\infty)}$ is given by the triple $(m, i, e)$.
\end{example}

\subsection{Homotopes of rings and pro-rings}
Let $R$ be an arbitrary ring and let \(S\) be some fixed multiplicative subset of $R$ containing a unit. Denote by $\mathcal{S}$ the category, whose objects are the elements of \(S\) and whose morphisms \(\mathcal{S}(s, s')\) are all \(s'' \in S\) such that \(ss'' = s'\). The composition and the identity morphisms are induced by the ring structure on $R$. It is clear that $\mathcal{S}$ is a filtered category. Unless stated otherwise, all the pro-sets that we encounter in the sequel have \(\mathcal S\) as their category of indices.

We introduce the notion of a {\it homotope} of a ring inspired by a similar notion from nonassociative algebra (cf. McCrimmon?). 
\begin{df} \label{ring-homotope}
 Let $s$ be an element of $R$.  
 By definition, the {\it \(s\)-homotope} of \(R\) is the rng \(R^{(s)} = \{a^{(s)} \mid a \in R\}\) with the operations of addition and multiplication given by
 \[ a^{(s)} + b^{(s)} = (a + b)^{(s)},\ \ a^{(s)} b^{(s)} = (asb)^{(s)},\ a, b\in R.\]
 Clearly, $R^{(s)}$ has the structure of an \(R\)-algebra given by the formula \[a \cdot b^{(s)} = (ab)^{(s)},\ a, b \in R.\] For \(s, s' \in S\) there is a homomorphism of \(R\)-algebras \[R^{(ss')} \to R^{(s')}, a^{(ss')} \mapsto (as)^{(s')}.\]
 Denote by \(R^{(\infty)}\) the formal projective limit of the projective system \(R^{(s)}\), where \(s \in Ob(\mathcal S)\).
 Thus, $R^{(\infty)}$ is an object of the category $\Pro(\Rng)$. 
 It can also be considered as an rng object in \(\Pro(\Set)\).  
\end{df}

\begin{rem}\label{rem:prorings-comment}
 Denote by $sR$ the principal ideal of $R$ generated by $s$. 
 There is an rgn homomorphism $R^{(s)} \to sR$ given by $r^{(s)}\mapsto sr$.
 In the case when $R$ is an integral domain this homomorphism is easily seen to be an rng isomorphism.
 
 The reason why we use homotopes rather than principal ideals in the definition of $R^{(\infty)}$ is that
 there is always a ``division by $s$'' homomorphism of $R$-algebras $R^{(ss')} \to R^{(s')}$ given by $r^{(ss')} \mapsto r^{(s')},$
 while the similar map $ss'R \to s'R$ may not exist if $R$ does not happen to be a domain.
 The existence of this division homomorphism will be important in the sequel.
 
 Notice also that the projective limit of $R^{(s)}$ in $\Rng$ is often trivial.
 Indeed, if $R$ is a domain, then the limit of $R^{(s)}$ in $\Rng$ computes the intersection of the principal ideals $\bigcap_{s\in S} sR$, which often coincides with the zero ring. 
 This also shows that the set of {\it global elements} of $R^{(\infty)}$ (i.\,e. the hom-set in $\Pro(\Set)$ from the terminal object $1$ to $R^{(\infty)}$) is often trivial and therefore will be of little interest to us.
\end{rem}

Recall that a morphism \(f \in \mathcal C(X, Y)\) is called a split epimorphism (or a retraction) if it admits a section,
 i.\,e. there exists $g \in \mathcal{C}(Y, X)$ such that $fg = \id_{Y}$. Retractions are preserved under pullbacks.

\begin{lemma}\label{RingGeneration}
The rng multiplication morphism $m \colon R^{(\infty)} \times R^{(\infty)} \to R^{(\infty)}$ is a split epimorphism of pro-sets.
\end{lemma}
\begin{proof}
Consider the following pre-morphism of pro-sets:
\[u \colon R^{(\infty)} \to R^{(\infty)} \times R^{(\infty)}, \enskip u^*(s) = s^2, \enskip u^{(s)} \colon c^{(s^2)} \mapsto 1^{(s)} \otimes c^{(s)}.\]
Clearly, this is indeed a pre-morphism and
\[m^{(s)}\bigl(u^{(s)}\bigl(c^{(s^2)}\bigr)\bigr) = 1^{(s)} c^{(s)} = (sc)^{(s)},\]
which shows that $mu = \mathrm{id}_{R^{(\infty)}}$.
\end{proof}

\begin{conv} \label{conv:notation}
Before we proceed further let us introduce the following conventions, 
 which will help us to keep our notations short:
 \begin{itemize}
  \item any algebraic expression containing the symbol $(\infty)$ in the upper index of a variable (e.\,g. $a^{(\infty)}$, $b^{(\infty)}$ etc.) denotes a certain morphism of pro-sets, whose domain is a certain product of pro-sets (often a power of a single pro-set), such morphisms are typically obtained from~\cref{df-pro-set-morphisms};
  \item the number of factors in this product coincides with the number of different variables occuring in the expression;
  \item in order to specify the domain of a variable $a^{(\infty)}$ occuring in the expression, we use the notation $a^{(\infty)} \in X^{(\infty)}$ (this notation is not confusing since we never refer to the actual elements of pro-sets, recall that the elements of our pro-sets are usually uninteresting, cf.~\cref{rem:prorings-comment});
  \item the notation for the multiplication operation is usually suppressed, i.\,e. we write $a^{(\infty)} b^{(\infty)}$ more often than $m(a^{(\infty)}, b^{(\infty)})$ or $a^{(\infty)} \cdot b^{(\infty)}$;
  \item the syntax of tuples is used to denote the product of morphisms (e.\,g. if $f,g \colon X^{(\infty)} \to Y^{(\infty)}$ are morphisms of pro-sets, then we use the notation $(f(x_1^{(\infty)}), g(x_2^{(\infty)}))$ instead of $f\times g$);
  \item if $g$ is a morphism of pro-sets with $X^{(\infty)} \times Y^{\infty}$ as its domain, then we write $g(a^{(\infty)}, b^{(\infty)})$ instead of $g((a^{(\infty)}, b^{(\infty)}))$;
  \item to denote the operation of composition of morphisms we use the syntax of substituted expressions;
  \item notice that the trivial group $1$ is a zero object in the category $\Pro(\Group)$, therefore for any pro-groups $G^{(\infty)}, H^{(\infty)}$
   there is only one morphism $G^{(\infty)} \to H^{(\infty)}$ passing through $1$, this morphism will also be denoted by $1$.
 \end{itemize}
\end{conv} 
Now we are ready to formulate our next result.
\begin{lemma}\label{RingPresentation}
Let \(G^{(\infty)}\) be a pro-group, $R^{(\infty)}$ be a pro-rng and \(g \colon R^{(\infty)} \times R^{(\infty)} \to G^{(\infty)}\) be a morphism of pro-sets. There is a morphism \(f \colon R^{(\infty)} \to G^{(\infty)}\) of pro-groups such that
\[g\bigl(a^{(\infty)} , b^{(\infty)}\bigr) = f\bigl(a^{(\infty)} b^{(\infty)}\bigr)\]
if and only if \(g\) satisfies the following identities in $\Pro(\Set)$:
\begin{itemize}
\item \(\bigl[g\bigl(a_1^{(\infty)}, b_1^{(\infty)}\bigr), g\bigl(a_2^{(\infty)}, b_2^{(\infty)}\bigr)\bigr] = 1\);
\item \(g\bigl(\bigl(a_1^{(\infty)} + a_2^{(\infty)}\bigr), b^{(\infty)}\bigr) = g\bigl(a_1^{(\infty)}, b^{(\infty)}\bigr)\, g\bigl(a_2^{(\infty)}, b^{(\infty)}\bigr)\);
\item \(g\bigl(a^{(\infty)}, \bigl(b_1^{(\infty)} + b_2^{(\infty)}\bigr)\bigr) = g\bigl(a^{(\infty)}, b_1^{(\infty)}\bigr)\, g\bigl(a^{(\infty)}, b_2^{(\infty)}\bigr)\);
\item \(g\bigl(a^{(\infty)} b^{(\infty)}, c^{(\infty)}\bigr) = g\bigl(a^{(\infty)}, b^{(\infty)} c^{(\infty)}\bigr)\).
\end{itemize}
\end{lemma}
\begin{proof}
The necessity of the identities is clear.
By~\cref{RingGeneration} the morphism \(f\) is unique, so it suffices to show that it exists.
By~\cref{lem:proobj-is-a-limit} it suffices to consider the case when \(G\) is a group. 
Let \(g\) be a morphism satisfying the above identities.
By definition, there exists $s\in S$ such that $g$ is given by a map \(g' \colon R^{(s)} \times R^{(s)} \to G\) satisfying the first three identities and the identity 
\[g'\bigl((asb)^{(s)}, c^{(s)}\bigr) = g'\bigl(a^{(s)}, (bsc)^{(s)}\bigr).\]
Consider the map \(f' \colon R^{(s^2)} \to G\) given by
\[f'\bigl(c^{(s^2)}\bigr) = g'\bigl(1^{(s)}, c^{(s)}\bigr),\]
it is a homomorphism by the first and the third identities.
From the last two identities we conclude that for all \(a, b \in R\) one has
\begin{align*}
f'\bigl(a^{(s^2)} b^{(s^2)})
&= g' \bigl( 1^{(s)}, (s^2 ab)^{(s)} \bigr) =\\
&= g' \bigl( (sa)^{(s)}, (sb)^{(s)} \bigr) =\\
&= g' \bigl(a^{(s^2)}, b^{(s^2)}\bigr).
\end{align*}
It is clear that \(f'\) gives the required morphism \(f\) of pro-groups.
\end{proof}
\subsection{Steinberg groups and Steinberg pro-groups}
Let $\Phi$ be an irreducible root system of rank $\geq 3$.
We start with the classical definition of the Steinberg group of unital rings.
Let $R$ be a ring.
\begin{df} \label{def:Steinberg}
Recall the {\it Steinberg group $\St(\Phi, R)$} is given by generators $x_\alpha(a)$, where $\alpha \in \Phi$ and $a \in R$ and the following list of defining relations:
\begin{align}
 x_\alpha(a) \cdot x_\alpha(b)    &= x_\alpha(a+b); \tag{R1} \label{R1} \\
 [x_\alpha(a),\ x_\beta(b)] &= 1, \tag{R2} \label{R2} \\ 
 \multispan2{\hfil if $\alpha + \beta \not\in\Phi \cup \{0\};$} \nonumber \\
 [x_\alpha(a),\ x_\beta(b)] &= x_{\alpha + \beta}(N_{\alpha,\beta} \cdot ab), \tag{R3} \label{R3} \\
 \multispan2{\hfill if $\alpha+\beta\in\Phi$ but $\alpha+2\beta,\ 2\alpha+\beta\not\in\Phi;$} \nonumber \\
 [x_\alpha(a),\ x_\beta(b)] &= x_{\alpha + \beta}(N_{\alpha,\beta} \cdot ab) \cdot x_{2\alpha+\beta}(N_{\alpha,\beta} \cdot \widehat{N}_{\alpha, \alpha+\beta}\cdot a^2b), \tag{R4} \label{R4} \\ \multispan2{ \hfill if $\alpha+\beta,2\alpha+\beta\in\Phi$.} \nonumber  \end{align}
The constants $N_{\alpha,\beta}$ appearing in the above relations are called the {\it structure constants} of the Chevalley group of type $\Phi$.
Since we excluded the case $\Phi=\mathsf{G}_2$ the only possibilities for $N_{\alpha, \beta}$ are $\pm 1$ or $\pm 2$.
Notice that $|N_{\alpha,\beta}| = 2$ provided $\alpha$ and $\beta$ are short, in which case we set $\widehat{N}_{\alpha, \beta} = \frac{1}{2} N_{\alpha, \beta}$.
In the other cases $|N_{\alpha, \beta}| = 1$. \end{df}

Notice that the multiplicative identity of $R$ is actually never used in the above definition, 
 which allows one to use it in the situation when $R$ is an rng.
This also allows one to ``deform'' the above definition and, by analogy with the homotopes of rngs introduced in~\cref{ring-homotope}, come to the notion of a ``homotope'' of the Steinberg group.
\begin{df}\label{def:Steinberg-homotope}
 Let $R$ be an rng with a fixed element $s \in R$ and, as before, $\Phi$ be a root system of rank $\geq 3$.
 The {\it $s$-homotope $\St^{(s)}(\Phi, R)$} of the Steinberg group, by definition, is the group given by generators $x_\alpha^{(s)}(a)$, $a\in R$, $\alpha\in\Phi$ and the following list of relations: \begin{align}
 x^{(s)}_\alpha(a) \cdot x^{(s)}_\alpha(b)    &= x^{(s)}_\alpha(a+b); \tag{S$1_s$} \label{S1} \\
 [x^{(s)}_\alpha(a),\ x^{(s)}_\beta(b)] &= 1, \tag{S$2_s$} \label{S2} \\ 
 \multispan2{\hfil if $\alpha + \beta \not\in\Phi \cup \{0\};$} \nonumber \\
 [x^{(s)}_\alpha(a),\ x^{(s)}_\beta(b)] &= x^{(s)}_{\alpha + \beta}(N_{\alpha,\beta} \cdot sab), \tag{S$3_s$} \label{S3} \\
 \multispan2{\hfill if $\alpha+\beta\in\Phi$ but $\alpha+2\beta,\ 2\alpha+\beta\not\in\Phi;$} \nonumber \\
 [x^{(s)}_\alpha(a),\ x^{(s)}_\beta(b)] &= x^{(s)}_{\alpha + \beta}(N_{\alpha,\beta} \cdot sab) \cdot x^{(s)}_{2\alpha+\beta}(N_{\alpha,\beta} \cdot \widehat{N}_{\alpha, \alpha+\beta}\cdot s^2a^2b), \tag{S$4_s$} \label{S4} \\ \multispan2{ \hfill if $\alpha+\beta,2\alpha+\beta\in\Phi$.} \nonumber  \end{align} 
 
 Now if $R$ is a ring with a multiplicative system $S$, then for every $s, s' \in S$ there is a group homomorphism $\St^{(ss')}(\Phi, R) \to \St^{(s)}(\Phi, R)$ given by the obvious map $x_\alpha^{(ss')}(a) \mapsto x_\alpha^{(s')}(sa)$.
 These homomorphisms together form a projective system in $\Group$, whose formal projective limit will be called the {\it Steinberg pro-group} and will be denoted by $\St^{(\infty)}(\Phi, R)$.
 By definition, $\St^{(\infty)}(\Phi, R)$ is an object of $\Pro(\Group)$. It can also be considered as a group object in $\Pro(\Set)$. 
\end{df}

\begin{rem} \label{rem:pro-Steinberg-comment}
There is a group homomorphism $\St^{(s)}(\Phi, R) \to \St(\Phi, sR)$ given by $x_\alpha^{(s)}(a)\mapsto x_\alpha(sa)$, which is easily seen to be an isomorphism in the case when $R$ is an integral domain (cf.~\cref{rem:prorings-comment}). Similarly to pro-rngs, Steinberg pro-groups often do not have global elements.
\end{rem}
 
For every root $\alpha \in \Phi$ there is a ``root subgroup'' morphism $x_{\alpha} \colon R^{(\infty)} \to \St^{(\infty)}(\Phi, R)$ in $\Pro(\Group)$ given by the premorphism $x_\alpha^* = id_S$, $x_\alpha^{(s)}(a^{(s)}) = x_\alpha^{(s)}(a)$.

\begin{df}
 Let $R$ be a ring with a multiplicative system $S$.
 Consider the projective system of relative Chevalley groups (also called congruence subgroups) 
 $\GG(\Phi, R, sR) = \Ker\left(\GG_{sc}(\Phi, R) \to \GG_{sc}(\Phi, R/sR)\right)$ with the structure morphisms given by the usual inclusions of congruence subgroups. Define the {\it simply-connected Chevalley pro-group} $\GG^{(\infty)}(\Phi, R)$ as the formal projective limit of this system.
\end{df}

Recall that there is well-defined homomorphism $\mathrm{st}\colon \St(\Phi, R) \to \GG_{sc}(\Phi, R)$ sending each generator $x_\alpha(a)$ to the root unipotent $t_\alpha(a)$. 

The pro-group analogue \(\mathrm{st} \colon \St^{(\infty)}(\Phi, R) \to \GG^{(\infty)}(\Phi, R)\) of the above homomorphism is given by the pre-morphism \(\mathrm{st}^* = id_S\), \(\mathrm{st}^{(s)}\bigl(x_{\alpha}^{(s)}(a)\bigr) = t_\alpha(sa)\). 
%It follows that all maps \(x_{ij}^{(s)} \colon R_{ij} \to \stlin(R)^{(s)}\) are injective.

\begin{lemma}\label{SteinbergPresentation}
Let \(G^{(\infty)}\) be a pro-group. The morphisms \[x_{\alpha} \colon R^{(\infty)} \to \St^{(\infty)}(\Phi, R)\] generate $\St^{(\infty)}(\Phi, R)$ in the categorical sense, i.\,e. to ensure that the morphisms $f_1,f_2\colon\St^{(\infty)}(\Phi, R) \to G^{(\infty)}$ are equal it suffices to verify the equalities $f_1 x_{\alpha} = f_2 x_\alpha$ for all $\alpha\in\Phi$.

Conversely, a morphism $f \colon \St^{(\infty)}(\Phi, R) \to G^{(\infty)}$ can be obtained from a collection of pro-group morphisms \(f_{\alpha} \colon R^{(\infty)} \to G^{(\infty)}\), $\alpha \in \Phi$ satisfying the identities~\eqref{R1}--\eqref{R4} (in which $f_{\alpha}$'s are substituted in place of $x_\alpha$'s, the sign $(\infty)$ is added to the upper index of the variables $a, b$ and the resulting equalities are understood as equalities of morphisms in \(\Pro(\Set)\) according to~\cref{conv:notation}).
\end{lemma}
\begin{proof}
Notice that by~\cref{lem:proobj-is-a-limit} both of the assertions can be verified in the special case when $G^{(\infty)} = G$ is a group.

Let us verify the first assertion. Since there is only a finite number of roots in $\Phi$ one can find a sufficiently large index $s \in S$ such that the homomorphisms $f_1^{(s)}, f_2^{(s)}\colon \St^{(s)}(\Phi, R) \to G$ become equal after precomposition with $x_\alpha$ for all $\alpha \in \Phi$.
Since the root elements $x_\alpha^{(s)}(a)$ generate $\St^{(s)}(\Phi, R)$ we obtain that the morphisms $f_1$ and $f_2$ are equal.

Let us verify the second assertion. 
Notice that~\eqref{R1}--\eqref{R4} specify only a {\it finite} collection of identities in $\Pro(\Set)$ to which $f_\alpha$'s must satisfy.
Unwinding the definitions, we find a sufficiently large index $s \in S$ such that group homomorphisms 
 $f_\alpha^{(s)}\colon R^{(s)} \to G$ satisfy the same identities (with $f_\alpha$'s replaced by $f_\alpha^{(s)}$'s).
Thus, we obtain a group homomorphism $\St^{(s)}(\Phi, R) \to G$, which is then easily seen to determine a morphism $\St^{(\infty)}(\Phi, R) \to G$.
\end{proof}
Notice that the assumption that $G^{(\infty)}$ is a pro-group and not merely a group object in $\Pro(\Set)$ is essential in the proof of the above lemma.


\printbibliography
\end{document}
