\documentclass{article}

\usepackage{amsmath, amssymb, amsthm, amscd, alltt, graphicx}
\usepackage[utf8]{inputenc}
\usepackage[T1]{fontenc}
\usepackage[capitalise]{cleveref}
\usepackage[matrix,arrow,curve]{xy}
\usepackage[notref, notcite]{showkeys}

\usepackage[backend=biber, bibencoding=utf8, giveninits=true, citestyle=numeric-comp, sortlocale=en_US, url=false, doi=false, eprint=true, maxbibnames=4]{biblatex}
\addbibresource{f4paper.bib}
\renewbibmacro*{volume+number+eid}{\ifentrytype{article}{\- \iffieldundef{volume}{}{Vol.~\printfield{volume},}\iffieldundef{number}{}{ No.~\printfield{number},}}}
\renewbibmacro{in:}{\ifentrytype{article}{}{\printtext{\bibstring{in}\intitlepunct}}}
\newbibmacro{string+doi}[1]{\iffieldundef{doi}{\iffieldundef{url}{#1}{\href{\thefield{url}}{#1}}}{\href{https://dx.doi.org/\thefield{doi}}{#1}}}
\DeclareFieldFormat[article, inproceedings, inbook, book, online]{title}{\usebibmacro{string+doi}{\mkbibquote{#1}}}
\renewcommand*{\bibfont}{\footnotesize}

\numberwithin{equation}{section}

\newtheorem{lemma}{Lemma} \numberwithin{lemma}{section}
\newtheorem{prop}{Proposition} \numberwithin{prop}{section}
\newtheorem{theorem}{Theorem}
\newtheorem*{theorem*}{Theorem}

\theoremstyle{definition} 
\newtheorem{df}[lemma]{Definition} \Crefname{df}{Definition}{Definitions}
\newtheorem{example}[lemma]{Example} \Crefname{example}{Example}{Examples}

\theoremstyle{remark} 
\newtheorem{rem}[lemma]{Remark}
\newtheorem{conv}[lemma]{Convention} \Crefname{conv}{Convention}{Conventions}

\newenvironment{psmallmatrix}{\left(\begin{smallmatrix}}{\end{smallmatrix}\right)}

\DeclareMathOperator\St{St}
\DeclareMathOperator\Ker{Ker}
\DeclareMathOperator\GG{G}
\DeclareMathOperator\Torus{T}
\DeclareMathOperator{\Pro}{Pro}


\newcommand{\Set}{\mathbf{Set}}
\newcommand{\Group}{\mathbf{Grp}}
\newcommand{\Rng}{\mathbf{Rng}}
\newcommand{\Fun}{\mathbf{Fun}}
\newcommand{\Mod}{\mathbf{Mod}}
\newcommand{\op}{\mathrm{op}}

\newcommand{\up}[2]{{^{#1}\!{#2}}}

\newcommand{\rA}{\mathsf{A}}
\newcommand{\rB}{\mathsf{B}}
\newcommand{\rC}{\mathsf{C}}
\newcommand{\rD}{\mathsf{D}}
\newcommand{\rE}{\mathsf{E}}
\newcommand{\rF}{\mathsf{F}}
\newcommand{\rG}{\mathsf{G}}
\begin{document}

\section{Preliminaries}
\subsection{Generalities on pro-objects}
Let \(\mathcal C\) be an arbitrary category.
In this section we recall the construction of the pro-completion of \(\mathcal C\) (cf. \cite[Section~6.1]{SK06}).

Recall that a nonempty small category \(\mathcal I\) is called {\it filtered} if
\begin{itemize}
 \item for any two objects \(i, k \in Ob(\mathcal{I})\) there is a diagram \(i \to j \leftarrow k\) in \(\mathcal I\);
 \item every two parallel morphisms \(i \rightrightarrows j\) are equalized by some morphism \(j \to k\) in \(\mathcal I\).
\end{itemize}
A {\it pro-object} in \(\mathcal C\) is, by definition, a functor $X^{(\infty)}\colon \mathcal{I}_X^{\op} \to \mathcal{C}$, i.\,e. a contravariant functor from a filtered category \(\mathcal I_X\), called {\it the category of indices of $X^{(\infty)}$}, to the category \(\mathcal C\). 

We denote by \(X^{(i)}\) the value of \(X^{(\infty)}\) on an index \(i\).
The values of the functor $X^{(\infty)}$ on the arrows of $\mathcal{I}$ are called the {\it structure morphisms} of $X^{(\infty)}$.

The category of pro-objects is denoted by \(\Pro(\mathcal C)\). The hom-sets in this category are given by the formula
\begin{equation} \label{eq:pro-c-hom} \Pro(\mathcal C)(X^{(\infty)}, Y^{(\infty)}) = \varprojlim_{j \in \mathcal I_Y} \varinjlim_{i \in \mathcal I_X} \mathcal C(X^{(i)}, Y^{(j)}). \end{equation}
Let us recall a more explicit description of morphisms in \(\Pro(\mathcal C)\). 
By definition, a {\it pre-morphism} \(f \colon X^{(\infty)} \to Y^{(\infty)}\) consists of the following data:
\begin{itemize}
\item a set-theoretic function \(f^* \colon Ob(\mathcal{I_Y}) \to Ob(\mathcal{I_X})\);
\item a collection of morphisms \(f^{(i)} \colon X^{(f^*(i))} \to Y^{(i)}\) in $\mathcal{C}$ parametrized by $i \in Ob(\mathcal{I_Y})$. These morphisms are required to satisfy the following additional assumption: for every morphism \(i \to j\) in \(\mathcal I_Y\) there exists a sufficiently large index \(k \in Ob(\mathcal{I_X})\) such that the composite morphisms \(X^{(k)} \to X^{(f^*(i))} \to Y^{(i)}\) and \(X^{(k)} \to X^{(f^*(j))} \to Y^{(j)} \to Y^{(i)}\) are equal. \end{itemize}
The composition of two pre-morphisms \(f \colon X^{(\infty)} \to Y^{(\infty)}\) and \(g \colon Y^{(\infty)} \to Z^{(\infty)}\) is defined as the pre-morphism \(g \circ f\), where \((g \circ f)^*(i) = f^*(g^*(i))\) and \((g \circ f)^{(i)} = g^{(i)} \circ f^{(g^*(i))}\).

Two parallel pre-morphisms \(f, g \colon X^{(\infty)} \to Y^{(\infty)}\) are called equivalent if for every \(i \in Ob(\mathcal I_Y)\) there exists a sufficiently large index \(j \in Ob(\mathcal I_X)\) such that the composite morphisms \(X^{(j)} \to X^{(f^*(i))} \to Y^{(i)}\) and \(X^{(j)} \to X^{(g^*(i))} \to Y^{(i)}\) are equal. Finally, a {\it morphism} \(X^{(\infty)} \to Y^{(\infty)}\) is an equivalence class of pre-morphisms. Note that the equivalence relation is preserved by the composition operation.

There is a fully faithful functor $\mathcal{C} \to \Pro(\mathcal{C})$ sending \(X \in Ob(\mathcal C)\) to the pro-object $X \colon \mathbf{1}^\op \to \mathcal{C}$. It is clear from~\eqref{eq:pro-c-hom}  that
\begin{align}
 \Pro(\mathcal C)(X, Y) &\cong \mathcal C(X, Y);\\
 \Pro(\mathcal C)(X^{(\infty)}, Y) &\cong \varinjlim_{i \in \mathcal I_X} \mathcal C(X^{(i)}, Y); \label{eq:pro-c-hom-2}\\
 \Pro(\mathcal C)(X, Y^{(\infty)}) &\cong \varprojlim_{i \in \mathcal I_Y} \mathcal C(X, Y^{(i)}).
\end{align}

Moreover, it is also clear from~\eqref{eq:pro-c-hom} and~\eqref{eq:pro-c-hom-2} that the following assertion holds.
\begin{lemma} \label{lem:proobj-is-a-limit}
  \(X^{(\infty)}\) is the projective limit of \(X^{(i)}\) in the category \(\Pro(\mathcal C).\)
\end{lemma}

The category of pro-sets \(\Pro(\Set)\) has all finite limits by \cite[Prop.~6.1.18]{SK06} and therefore is a cartesian monoidal category.

Let us describe an explicit construction of limits in one important special case. Let $X^{(\infty)}, Y^{(\infty)} \colon \mathcal{I}^\op \to \Set\) be a pair of pro-sets with the same index category. Recall that the pointwise product of functors $X^{(\infty)} \times Y^{(\infty)} \colon \mathcal{I}^\op \to \Set$ is given by $(X^{(\infty)} \times Y^{(\infty)})^{(i)} = X^{(i)} \times Y^{(i)}$. Clearly, $X^{(\infty)} \times Y^{(\infty)}$ is a product of $X^{(\infty)}$ and $Y^{(\infty)}$  in the category $\Fun(\mathcal{I}^\op, \Set)$. 

There is an obvious identity-on-objects functor $\Fun(\mathcal{I}^\op, \Set) \to \Pro(\Set)$ sending a natural transformation $\varphi \colon X^{(\infty)} \to Y^{(\infty)}$ to the morphism of pro-sets given by the pre-morphism $\varphi^* = \mathrm{id}_{Ob(\mathcal{I})}$, $\varphi^{(i)} = \varphi_i$. The diagonal morphism $\Delta \colon X^{(\infty)} \to X^{(\infty)} \times X^{(\infty)}$ and the canonical projection morphisms \(\pi_X\colon X^{(\infty)} \times Y^{(\infty)} \to X^{(\infty)}, \pi_Y\colon X^{(\infty)} \times Y^{(\infty)} \to Y^{(\infty)}\) can be defined in the category $\Pro(\Set)$ as the images of the corresponding natural transformations in $\Fun(\mathcal{I}^\op, \Set)$ under this functor.

We claim that the point-wise product $X^{(\infty)} \times Y^{(\infty)}$ satisfies the universal property of products in the category $\Pro(\Set)$. Indeed, this follows from~\eqref{eq:pro-c-hom} and the fact that filtered colimits commute with finite limits. This argument also shows that arbitrary finite limits of pro-sets with the same index category can be computed pointwise.

\subsection{Group and ring objects in pro-sets}
In this paper we consider only commutative rings.
To distinguish between unital and non-unital rings we reserve the word ``ring'' only for unital rings and refer to non-unital rings as {\it rngs}.
We denote the category of rings (resp. rngs) as $\textbf{Ring}$ (resp. $\textbf{Rng}$).

Let $T$ be an algebraic theory. Throughout this paper we will be mostly interested in the case when $T$ is the theory of groups or the theory of rngs.

Since the category $\Pro(\Set)$ is cartesian monoidal, we may speak of the category $\Mod(T, \Pro(\Set))$ of models of $T$ in \(\Pro(\Set)\).
In the special case when $T$ is the theory of groups (resp. rngs) this category is precisely the category of group (resp. rng) objects in $\Pro(\Set)$.

The category of pro-groups \(\Pro(\Group)\) embeds into \(\Pro(\Set)\). Every pro-group is a group object in \(\Pro(\Set)\). It is easy to see that a morphism \(f \in \Pro(\Set)(G^{(\infty)}, H^{(\infty)})\) between pro-groups comes from \(\Pro(\Group)\) if and only if it is a morphism of group objects. Thus, $\Pro(\Group)$ is a full subcategory of the category of group objects in $\Pro(\Set)$.

\begin{df} \label{df-pro-set-morphisms} 
 Let $n$ be a natural number and $F_{n, T}$ be the free algebra on $n$ generators in the theory $T$.
 If $T$ is the theory of groups, then $F_n$ is the free group $F(t_1,\ldots, t_n)$.
 Similarly, if $T$ is the theory of rngs, $F_{n, T}$ is the sub-rng of the ring $\mathbb{Z}[t_1,\ldots, t_n]$ consisting of polynomials over $\mathbb{Z}$ without free terms.
 
 Given a word $w \in F_{n, T}$ and a $T$-model $X^{(\infty)}$ in $\Pro(\Set)$, one can construct the $\Pro(\Set)$-morphism
 \[ w^{(\infty)} \colon \underbrace{X^{(\infty)} \times \ldots \times X^{(\infty)}}_{n\text{ times}} \to X^{(\infty)}, \]
 which ``interprets'' the word $w$. This morphism can be obtained as an appropriate composition of diagonal morphisms $\Delta_{X^{(\infty)}}$, projections $\pi_i$ 
 (which are part of the structure of a cartesian monoidal category on $\Pro(\Set)$) 
 and the morphisms defining the structure of a $T$-model on $X^{(\infty)}$.
 
 In order to shorten the notation, to denote the interpretation morphism $w^{(\infty)}$ we often use the original term for $w$, in which every symbol of a free variable $t_i$ is replaced by the symbol $t_i^{(\infty)}$.
\end{df}

\begin{example}\label{example-commutator}
Consider the case when $T$ is the theory of groups.
Let $G^{(\infty)}$ be a group object in $\Pro(\Set)$ and $w = [t_1, t_2] = t_1 t_2 t_1^{-1} t_2^{-1} \in F(t_1, t_2)$ be the word representing the generic commutator.
In this case the interpretation morphism $w^{(\infty)}$ (also denoted $[t_1^{(\infty)}, t_2^{(\infty)}]$) can be defined as the composition
 \[ \xymatrix{G^{(\infty)} \times G^{(\infty)} \ar[r]^(.35){\Delta \times \Delta} & G^{(\infty)} \times G^{(\infty)} \times G^{(\infty)} \times G^{(\infty)} \ar[d]_{\langle \pi_1, \pi_3, i\pi_2, i\pi_4 \rangle} & \\
    & G^{(\infty)} \times G^{(\infty)} \times G^{(\infty)} \times G^{(\infty)} \ar[r]^(.76){m(m\times m)} & G^{(\infty)},} \]
where the structure of a group object on $G^{(\infty)}$ is given by the triple $(m, i, e)$.
\end{example}

\subsection{Homotopes of rings and pro-rings}
Let $R$ be an arbitrary ring and let \(S\) be some fixed multiplicative subset of $R$ containing a unit. Denote by $\mathcal{S}$ the category, whose objects are the elements of \(S\) and whose morphisms \(\mathcal{S}(s, s')\) are all \(s'' \in S\) such that \(ss'' = s'\). The composition and the identity morphisms are induced by the ring structure on $R$. It is clear that $\mathcal{S}$ is a filtered category. Unless stated otherwise, all the pro-sets that we encounter in the sequel have \(\mathcal S\) as their category of indices.

We introduce the notion of a {\it homotope} of a ring inspired by a similar notion from nonassociative algebra (cf. McCrimmon?). 
\begin{df} \label{ring-homotope}
 Let $s$ be an element of $R$.  
 By definition, the {\it \(s\)-homotope} of \(R\) is the rng \(R^{(s)} = \{a^{(s)} \mid a \in R\}\) with the operations of addition and multiplication given by
 \[ a^{(s)} + b^{(s)} = (a + b)^{(s)},\ \ a^{(s)} b^{(s)} = (asb)^{(s)},\ a, b\in R.\]
 Clearly, $R^{(s)}$ has the structure of an \(R\)-algebra given by the formula \[a \cdot b^{(s)} = (ab)^{(s)},\ a, b \in R.\] For \(s, s' \in S\) there is a homomorphism of \(R\)-algebras \[R^{(ss')} \to R^{(s')}, a^{(ss')} \mapsto (as)^{(s')}.\]
 Denote by \(R^{(\infty)}\) the formal projective limit of the projective system \(R^{(s)}\), where \(s \in Ob(\mathcal S)\).
 Thus, $R^{(\infty)}$ is an object of the category $\Pro(\Rng)$. 
 It can also be considered as an rng object in \(\Pro(\Set)\).  
\end{df}

\begin{rem}\label{rem:prorings-comment}
 Denote by $sR$ the principal ideal of $R$ generated by $s$. 
 There is an rgn homomorphism $R^{(s)} \to sR$ given by $r^{(s)}\mapsto sr$.
 In the case when $R$ is an integral domain this homomorphism is easily seen to be an rng isomorphism.
 
 The reason why we use homotopes rather than principal ideals in the definition of $R^{(\infty)}$ is that
 there is always a ``division by $s$'' homomorphism of $R$-modules $R^{(ss')} \to R^{(s')}$ given by $r^{(ss')} \mapsto r^{(s')},$
 while the similar map $ss'R \to s'R$ may not exist if $R$ does not happen to be a domain.
 The existence of this division homomorphism will be important in the sequel.
 
 Notice also that the projective limit of $R^{(s)}$ in $\Rng$ is often trivial.
 Indeed, if $R$ is a domain, then the limit of $R^{(s)}$ in $\Rng$ computes the intersection of the principal ideals $\bigcap_{s\in S} sR$, which often coincides with the zero ring. 
 This also shows that the set of {\it global elements} of $R^{(\infty)}$ (i.\,e. the hom-set in $\Pro(\Set)$ from the terminal object $1$ to $R^{(\infty)}$) is often trivial and therefore will be of little interest to us.
\end{rem}

Recall that a morphism \(f \in \mathcal C(X, Y)\) is called a split epimorphism (or a retraction) if it admits a section,
 i.\,e. there exists $g \in \mathcal{C}(Y, X)$ such that $fg = \mathrm{id}_{Y}$. Retractions are preserved under pullbacks.

\begin{lemma}\label{RingGeneration}
The rng multiplication morphism $m \colon R^{(\infty)} \times R^{(\infty)} \to R^{(\infty)}$ is a split epimorphism of pro-sets.
\end{lemma}
\begin{proof}
Consider the following pre-morphism of pro-sets:
\[u \colon R^{(\infty)} \to R^{(\infty)} \times R^{(\infty)}, \enskip u^*(s) = s^2, \enskip u^{(s)} \colon c^{(s^2)} \mapsto (1^{(s)}, c^{(s)}).\]
Clearly, this is indeed a pre-morphism and
\[m^{(s)}\bigl(u^{(s)}\bigl(c^{(s^2)}\bigr)\bigr) = 1^{(s)} c^{(s)} = (sc)^{(s)},\]
which shows that $mu = \mathrm{id}_{R^{(\infty)}}$.
\end{proof}

\begin{conv} \label{conv:notation}
Before we proceed further we need to introduce certain notational conventions 
 which will help us to keep our notations short and avoid using bulky commutative diagrams.
 \begin{itemize}
  \item Any algebraic expression containing the symbol $(\infty)$ in the upper index of a variable (e.\,g. $a^{(\infty)}$, $b^{(\infty)}$ etc.) denotes a certain morphism of pro-sets, whose domain is a certain product of pro-sets (often a power of a single pro-set), such morphisms are typically obtained from~\cref{df-pro-set-morphisms};
  \item The number of factors in this product coincides with the number of different variables occuring in the expression.
  \item In order to specify the domain of a variable $a^{(\infty)}$ occuring in the expression, we use the notation $a^{(\infty)} \in X^{(\infty)}$ (this notation is not confusing since we never refer to the actual elements of pro-sets, recall that the elements of our pro-sets are usually uninteresting, cf.~\cref{rem:prorings-comment}).
  \item The notation for the multiplication operation is usually suppressed, i.\,e. we write $a^{(\infty)} b^{(\infty)}$ more often than $m(a^{(\infty)}, b^{(\infty)})$ or $a^{(\infty)} \cdot b^{(\infty)}$.
  \item The syntax of tuples is used to denote the product of morphisms (e.\,g. if $f,g \colon X^{(\infty)} \to Y^{(\infty)}$ are morphisms of pro-sets, then we use the notation $(f(x_1^{(\infty)}), g(x_2^{(\infty)}))$ instead of $f\times g$).
  \item If $g$ is a morphism of pro-sets with $X^{(\infty)} \times Y^{\infty}$ as its domain, then we write $g(a^{(\infty)}, b^{(\infty)})$ instead of $g((a^{(\infty)}, b^{(\infty)}))$.
  \item To denote the composition of morphisms we use the syntax of substituted expressions. Thus, for example, if $f, g \colon X^{(\infty)} \to Y^{(\infty)}$ are morphisms of pro-groups and $[t_1^{(\infty)}, t_2^{(\infty)}]$ is the generic commutator from~\cref{example-commutator}, then $[f(a^{(\infty)}), g(b^{(\infty)})]$ denotes the composition \[[t_1^{(\infty)}, t_2^{(\infty)}] \circ (f\times g) \colon X^{(\infty)} \times X^{(\infty)} \to Y^{(\infty)}\]
  \item Notice that the trivial group $1$ is a zero object in the category $\Pro(\Group)$, therefore for any pro-groups $G^{(\infty)}, H^{(\infty)}$
   there is only one morphism $G^{(\infty)} \to H^{(\infty)}$ passing through $1$, this morphism will also be denoted by $1$.
 \end{itemize}
\end{conv} 
Now we are ready to formulate our next result.
\begin{lemma}\label{RingPresentation}
Let \(G^{(\infty)}\) be a pro-group, $R^{(\infty)}$ be a pro-rng and \(g \colon R^{(\infty)} \times R^{(\infty)} \to G^{(\infty)}\) be a morphism of pro-sets. There is a morphism \(f \colon R^{(\infty)} \to G^{(\infty)}\) of pro-groups such that
\[g\bigl(a^{(\infty)} , b^{(\infty)}\bigr) = f\bigl(a^{(\infty)} b^{(\infty)}\bigr)\]
if and only if \(g\) satisfies the following identities in $\Pro(\Set)$:
\begin{itemize}
\item \(\bigl[g\bigl(a_1^{(\infty)}, b_1^{(\infty)}\bigr), g\bigl(a_2^{(\infty)}, b_2^{(\infty)}\bigr)\bigr] = 1\);
\item \(g\bigl(a_1^{(\infty)} + a_2^{(\infty)}, b^{(\infty)}\bigr) = g\bigl(a_1^{(\infty)}, b^{(\infty)}\bigr)\, g\bigl(a_2^{(\infty)}, b^{(\infty)}\bigr)\);
\item \(g\bigl(a^{(\infty)}, b_1^{(\infty)} + b_2^{(\infty)}\bigr) = g\bigl(a^{(\infty)}, b_1^{(\infty)}\bigr)\, g\bigl(a^{(\infty)}, b_2^{(\infty)}\bigr)\);
\item \(g\bigl(a^{(\infty)} b^{(\infty)}, c^{(\infty)}\bigr) = g\bigl(a^{(\infty)}, b^{(\infty)} c^{(\infty)}\bigr)\).
\end{itemize}
\end{lemma}
\begin{proof}
The necessity of the identities is clear.
By~\cref{RingGeneration} the morphism \(f\) is unique, so it suffices to show that it exists.
By~\cref{lem:proobj-is-a-limit} it suffices to consider the case when \(G\) is a group. 
Let \(g\) be a morphism satisfying the above identities.
By definition, there exists $s\in S$ such that $g$ is given by a map \(g' \colon R^{(s)} \times R^{(s)} \to G\) satisfying the first three identities and the identity 
\[g'\bigl((asb)^{(s)}, c^{(s)}\bigr) = g'\bigl(a^{(s)}, (bsc)^{(s)}\bigr).\]
Consider the map \(f' \colon R^{(s^2)} \to G\) given by
\[f'\bigl(c^{(s^2)}\bigr) = g'\bigl(1^{(s)}, c^{(s)}\bigr),\]
it is a homomorphism by the first and the third identities.
From the last two identities we conclude that for all \(a, b \in R\) one has
\begin{align*}
f'\bigl(a^{(s^2)} b^{(s^2)})
&= g' \bigl( 1^{(s)}, (s^2 ab)^{(s)} \bigr) =\\
&= g' \bigl( (sa)^{(s)}, (sb)^{(s)} \bigr) =\\
&= g' \bigl(a^{(s^2)}, b^{(s^2)}\bigr).
\end{align*}
It is clear that \(f'\) gives the required morphism \(f\) of pro-groups.
\end{proof}

\subsection{Steinberg groups and Steinberg pro-groups}
Let $\Phi$ be an irreducible root system of rank $\geq 3$.
We assume that the root system is $\Phi$ contained in a Euclidean space $V = \mathbb{R}^\ell$ whose scalar product we denote by $(-, -)$.
For a pair of roots $\alpha, \beta \in \Phi$ we denote by $\langle \alpha, \beta \rangle$ the integer $\tfrac{2(\alpha, \beta)}{|\beta|^2}$.

We start by recalling the classical definition of the Steinberg group.
\begin{df} \label{def:Steinberg}
Let $R$ be a ring. Recall the {\it Steinberg group $\St(\Phi, R)$} is given by generators $x_\alpha(a)$, where $\alpha \in \Phi$ and $a \in R$ and the following list of defining relations:
\begin{align}
 x_\alpha(a) \cdot x_\alpha(b)    &= x_\alpha(a+b); \tag{R1} \label{R1} \\
 [x_\alpha(a),\ x_\beta(b)] &= 1, \tag{R2} \label{R2} \\ 
 \multispan2{\hfil if $\alpha + \beta \not\in\Phi \cup \{0\};$} \nonumber \\
 [x_\alpha(a),\ x_\beta(b)] &= x_{\alpha + \beta}(N_{\alpha,\beta} \cdot ab), \tag{R3} \label{R3} \\
 \multispan2{\hfill if $\alpha+\beta\in\Phi$ but $\alpha+2\beta,\ 2\alpha+\beta\not\in\Phi;$} \nonumber \\
 [x_\alpha(a),\ x_\beta(b)] &= x_{\alpha + \beta}(N_{\alpha,\beta} \cdot ab) \cdot x_{2\alpha+\beta}(N_{\alpha,\beta} \cdot \widehat{N}_{\alpha, \alpha+\beta}\cdot a^2b), \tag{R4} \label{R4} \\ \multispan2{ \hfill if $\alpha+\beta,2\alpha+\beta\in\Phi$.} \nonumber  \end{align}
\end{df}
The constants $N_{\alpha,\beta}$ appearing in the above relations are called the {\it structure constants} of the Chevalley group of type $\Phi$.
Different methods of computing signs of these constants have been proposed in the literature, see e.\,g.~\cite{VP}.
For the purposes of the present article, the standard properties of structure constants suffice. Let us briefly recall them.

First of all, notice that we excluded the case $\Phi=\mathsf{G}_2$ so the only possibilities for $N_{\alpha, \beta}$ appearing in the above relations are $\pm 1$ or $\pm 2$.
Notice that $|N_{\alpha,\beta}| = 2$ only when $\alpha$ and $\beta$ are short but $\alpha+\beta$ is long, in which case we set $\widehat{N}_{\alpha, \beta} = \frac{1}{2} N_{\alpha, \beta}$.
In the other cases $|N_{\alpha, \beta}| = 1$.

Recall from~\cite[\S~14]{VP}) that
\begin{equation} \label{eq:sc-ids-sl} N_{\alpha, \beta} = -N_{\beta,\alpha} = - N_{-\alpha, -\beta} = \tfrac{|\alpha+\beta|^2}{|\alpha|^2} N_{\beta, -\alpha-\beta} = \tfrac{|\alpha+\beta|^2}{|\beta|^2} N_{-\alpha-\beta, \alpha}. \end{equation}
These identities will be used in the sequel without explicit reference. 

We also need yet another identity for structure constants.
To formulate it succintly, we extend the domain of the structure constant function $N_{-,-}$ by setting $N_{\alpha, \beta} = 0$ whenever $\alpha+\beta\not\in\Phi\setminus\{0\}$.
Now if $\alpha, \beta, \gamma, \delta$ are pairwise linearly independent roots such that $\alpha+\beta+\gamma+\delta = 0$ then the following identity holds:
\begin{equation} \label{eq:cocycle} \tfrac{1}{|\alpha+\beta|^2} N_{\alpha,\beta} N_{\gamma,\delta} + \tfrac{1}{|\beta+\gamma|^2} N_{\beta,\gamma} N_{\alpha,\delta} + \tfrac{1}{|\gamma+\alpha|^2} N_{\gamma,\alpha} N_{\beta,\delta} = 0. \end{equation}

Notice that the multiplicative identity of $R$ is actually never used in the definition of the Steinberg group, 
 which allows one to use it in the situation when $R$ is an rng.
This also allows one to ``deform'' the above definition and, by analogy with the homotopes of rngs introduced in~\cref{ring-homotope}, come to the notion of a ``homotope'' of the Steinberg group.
\begin{df}\label{def:Steinberg-homotope}
 Let $R$ be an rng with a fixed element $s \in R$ and, as before, $\Phi$ be a root system of rank $\geq 3$.
 The {\it $s$-homotope $\St^{(s)}(\Phi, R)$} of the Steinberg group, by definition, is the group given by generators $x_\alpha^{(s)}(a)$, $a\in R$, $\alpha\in\Phi$ and the following list of relations: \begin{align}
 x^{(s)}_\alpha(a) \cdot x^{(s)}_\alpha(b)    &= x^{(s)}_\alpha(a+b); \tag{S$1_s$} \label{S1} \\
 [x^{(s)}_\alpha(a),\ x^{(s)}_\beta(b)] &= 1, \tag{S$2_s$} \label{S2} \\ 
 \multispan2{\hfil if $\alpha + \beta \not\in\Phi \cup \{0\};$} \nonumber \\
 [x^{(s)}_\alpha(a),\ x^{(s)}_\beta(b)] &= x^{(s)}_{\alpha + \beta}(N_{\alpha,\beta} \cdot sab), \tag{S$3_s$} \label{S3} \\
 \multispan2{\hfill if $\alpha+\beta\in\Phi$ but $\alpha+2\beta,\ 2\alpha+\beta\not\in\Phi;$} \nonumber \\
 [x^{(s)}_\alpha(a),\ x^{(s)}_\beta(b)] &= x^{(s)}_{\alpha + \beta}(N_{\alpha,\beta} \cdot sab) \cdot x^{(s)}_{2\alpha+\beta}(N_{\alpha,\beta} \cdot \widehat{N}_{\alpha, \alpha+\beta}\cdot s^2a^2b), \tag{S$4_s$} \label{S4} \\ \multispan2{ \hfill if $\alpha+\beta,2\alpha+\beta\in\Phi$.} \nonumber  \end{align} 
 
 Now if $R$ is a ring with a multiplicative system $S$, then for every $s, s' \in S$ there is a group homomorphism $\St^{(ss')}(\Phi, R) \to \St^{(s)}(\Phi, R)$ given by the obvious map $x_\alpha^{(ss')}(a) \mapsto x_\alpha^{(s')}(sa)$.
 These homomorphisms together form a projective system in $\Group$, whose formal projective limit will be called the {\it Steinberg pro-group} and will be denoted by $\St^{(\infty)}(\Phi, R)$.
 By definition, $\St^{(\infty)}(\Phi, R)$ is an object of $\Pro(\Group)$. It can also be considered as a group object in $\Pro(\Set)$. 
\end{df}

\begin{rem} \label{rem:pro-Steinberg-comment}
There is a group homomorphism $\St^{(s)}(\Phi, R) \to \St(\Phi, sR)$ given by $x_\alpha^{(s)}(a)\mapsto x_\alpha(sa)$, which is easily seen to be an isomorphism in the case when $R$ is an integral domain (cf.~\cref{rem:prorings-comment}). Similarly to pro-rngs, Steinberg pro-groups often do not have global elements.
\end{rem}
 
For every root $\alpha \in \Phi$ there is a ``root subgroup'' morphism $x_{\alpha} \colon R^{(\infty)} \to \St^{(\infty)}(\Phi, R)$ in $\Pro(\Group)$ given by the pre-morphism $x_\alpha^* = \mathrm{id}_S$, $x_\alpha^{(s)}(a^{(s)}) = x_\alpha^{(s)}(a)$.

\begin{df}
 Let $R$ be a ring with a multiplicative system $S$.
 Consider the projective system of relative Chevalley groups (also called congruence subgroups) 
 $\GG(\Phi, R^{(s)} \rtimes R, R^{(s)}) = \Ker\left(\GG_{sc}(\Phi, R^{(s)} \rtimes R) \to \GG_{sc}(\Phi, R)\right)$ with the structure morphisms given by the structure morphisms of \(R^{(s)}\). Define the {\it simply-connected Chevalley pro-group} $\GG^{(\infty)}(\Phi, R)$ as the formal projective limit of this system.
\end{df}

% If R is a domain, then this should be the projective limit of G(Phi, R, sR)...

Recall that there is well-defined homomorphism $\mathrm{st}\colon \St(\Phi, R) \to \GG_{sc}(\Phi, R)$ sending each generator $x_\alpha(a)$ to the root unipotent $t_\alpha(a)$. 

The pro-group analogue \(\mathrm{st} \colon \St^{(\infty)}(\Phi, R) \to \GG^{(\infty)}(\Phi, R)\) of the above homomorphism is given by the pre-morphism \(\mathrm{st}^* = \mathrm{id}_S\), \(\mathrm{st}^{(s)}\bigl(x_{\alpha}^{(s)}(a)\bigr) = t_\alpha(a^{(s)})\). 
%It follows that all maps \(x_{ij}^{(s)} \colon R_{ij} \to \stlin(R)^{(s)}\) are injective.

\begin{lemma}\label{SteinbergPresentation}
Let \(G^{(\infty)}\) be a pro-group. The morphisms \[x_{\alpha} \colon R^{(\infty)} \to \St^{(\infty)}(\Phi, R), \ \alpha\in \Phi\] ``generate'' $\St^{(\infty)}(\Phi, R)$. In other words, to ensure that a pair of morphisms $f_1,f_2\colon\St^{(\infty)}(\Phi, R) \to G^{(\infty)}$ are equal it suffices to verify the equalities $f_1 x_{\alpha} = f_2 x_\alpha$ for all $\alpha\in\Phi$.

To obtain a morphism $f \colon \St^{(\infty)}(\Phi, R) \to G^{(\infty)}$ it suffices to construct a collection of pro-group morphisms \(f_{\alpha} \colon R^{(\infty)} \to G^{(\infty)}\) satisfying the identities~\eqref{R1}--\eqref{R4} in which $f_{\alpha}$'s are substituted in place of $x_\alpha$'s, the sign $(\infty)$ is added to the upper index of the variables $a, b$ and the resulting identities are understood as equalities of morphisms in \(\Pro(\Set)\) according to~\cref{conv:notation}.
\end{lemma}
\begin{proof}
Notice that by~\cref{lem:proobj-is-a-limit} both of the assertions can be verified in the special case when $G^{(\infty)} = G$ is a group.

Let us verify the first assertion. Since there is only a finite number of roots in $\Phi$ one can find a sufficiently large index $s \in S$ such that the homomorphisms $f_1^{(s)}, f_2^{(s)}\colon \St^{(s)}(\Phi, R) \to G$ become equal after precomposition with $x_\alpha$ for all $\alpha \in \Phi$.
Since the root elements $x_\alpha^{(s)}(a)$ generate $\St^{(s)}(\Phi, R)$ we obtain that the morphisms $f_1$ and $f_2$ are equal.

Let us verify the second assertion. By definition, a morphism $f_\alpha \colon R^{(\infty)} \to G$ corresponds to a single group homomorphism 
 $f_\alpha^{\circ} \colon R^{(s_\alpha)} \to G$ for some $s_\alpha \in S$.
Notice that~\eqref{R1}--\eqref{R4} specify only a {\it finite} collection of identities in $\Pro(\Set)$ to which $f_\alpha$'s must satisfy.
Unwinding the definitions, we find a sufficiently large index $s \in S$ such that composite homomorphisms $\widetilde{f}_\alpha$ defined as the composites of $f_\alpha^{(\circ)}\colon R^{(s_\alpha)} \to G$ with the structure morphisms $R^{(s)} \to R^{(s_\alpha)}$ satisfy the same identities (with $f_\alpha$'s replaced by $\widetilde{f}_\alpha$'s).
Thus, we obtain a group homomorphism $\St^{(s)}(\Phi, R) \to G$, which, in turn, determines a morphism $\St^{(\infty)}(\Phi, R) \to G$.
\end{proof}
Notice that the assumption that $G^{(\infty)}$ is a pro-group and not merely a group object in $\Pro(\Set)$ is essential in the proof of the above lemma.

\section{Main result}
Throughout this section we make use of the following commutator identities:
\begin{align}
\label{eq:comm-mult-rhs}[x, yz]& =   [x, y] \cdot {}^{y}\![x, z],
\end{align}
\begin{equation} \label{eq:HW-corr} [x,z] = 1 \text{ implies } [x, [y,z]] = [[x,y],{}^yz].
\end{equation}

\subsection{Elimination of roots}
Recall that a root subset $\Psi \subseteq \Phi$ is called {\it closed} if $\alpha, \beta \in \Psi$, $\alpha+\beta\in\Phi$ imply $\alpha+\beta\in \Psi$. A closed root subset $\Psi$ will be called a {\it subsystem} if $\Psi = -\Psi$.

\begin{lemma}\label{ThreeRoots}
Let \(\Phi\) be a root system of rank \(\geq 3\) and not of type \(\rB_\ell\) or \(\rC_\ell\). Let \(\alpha, \beta, \gamma \in \Phi\) be a triple of roots having the same length such that $\alpha = \beta + \gamma$.  
Then there exist roots \(\beta_1, \beta_2 \in \Phi \setminus (\mathbb R\beta + \mathbb R\gamma)\) having the same length as $\beta$ and, moreover, such that \(\beta_1 + \beta_2 = \beta\).

The roots $\alpha, \gamma, \beta_1, \beta_2$ are contained in a root subsystem $\Phi_0 \subseteq \Phi$, of type either \(\rA_3\) or \(\rC_3\). There is an embedding \(\Phi_0 \hookrightarrow  \mathbb R^3\) such that \(\beta = \mathrm e_1 - \mathrm e_2\), \(\gamma = \mathrm e_2 - \mathrm e_3\) and 
\begin{itemize}
 \item \(\{\beta_1, \beta_2\} = \{\mathrm e_1 - \mathrm e_4, \mathrm e_4 - \mathrm e_2\}\) in the case \(\Phi_0\cong\rA_3\);
 \item \(\{\beta_1, \beta_2\} = \{\mathrm e_1 + \mathrm e_3, -\mathrm e_2 - \mathrm e_3\}\) in the case \(\Phi_0\cong\rC_3\).
\end{itemize}
\end{lemma}
\begin{proof}
Denote by $\Phi'$ the subset of $\Phi$ consisting of roots having the same length as $\beta$.
It follows from our assumptions that either $\Phi$ is simply-laced and $\Phi' = \Phi$ or $\Phi = \rF_4$, in which case $\Phi'$ is a root subset isomorphic to $\rD_4$ (notice that $\Phi'$ is closed only if $\beta$ is long). In either case the Dynkin diagram of $\Phi'$ is simply laced and has at least $3$ vertices, which implies the first assertion.

Notice that the smallest root subsystem $\Phi_0$ containing $\alpha, \gamma, \beta_1, \beta_2$ is irreducible and has rank $3$, which implies that it is either of type  $\rC_3$ (if $\Phi = \rF_4$ and $\beta$ is short) or $\rA_3$ (otherwise). \end{proof}

\begin{df}
Let $R$ be a ring, $S$ be its multiplicative system of and $\alpha$ be a root of an irreducible root system $\Phi$ of rank $\geq 3$. 

We denote by $\St(\Phi\setminus\{\pm \alpha\}, R)$ the group given by generators $x_\beta(a)$, $\beta \neq \pm \alpha$, $a\in R$ and the subset of the set of relations \eqref{R1}--\eqref{R4} consisting of those relations, expressions for which do not contain $x_{\pm\alpha}(a)$.
Similarly to~\cref{def:Steinberg-homotope} one can define the $s$-homotope $\St^{(s)}(\Phi\setminus\{\pm \alpha\}, R)$ of this group for every $s\in R$. 
As before, these $s$-homotopes form a projective system, whose formal projective limit will be denoted $\St^{(\infty)}(\Phi \setminus\{\pm \alpha\}, R)$. 
\end{df}

For $\beta \neq \pm\alpha$ we use the same notation $x_\beta$ for the ``root subgroup'' morphisms $R^{(\infty)}\to \St^{(\infty)}(\Phi \setminus\{\pm\alpha\}, R)$.

There is a morphism of pro-groups
\[ F_\alpha \colon \St^{(\infty)}(\Phi \setminus\{\pm\alpha\}, R) \to \St^{(\infty)}(\Phi, R). \]
given by the pre-morphism $F_\alpha^{*} = \mathrm{id}_S$ and the homomorphisms \[F_\alpha^{(s)} \colon \St^{(s)}(\Phi \setminus\{\pm\alpha\}, R) \to \St^{(s)}(\Phi, R),\] which are induced by the obvious embedding of generators.

Notice that, in general, the individual homomorphisms $F_\alpha^{(s)}$ need {\it not} be isomorphisms (at least if $s \neq 1$). On the other hand, as the following result shows, the morphism of pro-groups $F_\alpha$ is often an isomorphism. 
\begin{theorem}\label{SingleRootElimination}
 Let \(\Phi\) be a root system of rank \(\geq 3\) and not of type \(\rB_\ell\) or \(\rC_\ell\). Then for every root \(\alpha \in \Phi\) the morphism $F_\alpha$ is an isomorphism of pro-groups.
\end{theorem}
The proof of~\cref{SingleRootElimination} occupies the rest of this subsection.
Our immediate goal is to construct the root subgroup morphisms
\[\widetilde x_{\pm \alpha} \colon R^{(\infty)} \to \St^{(\infty)}(\Phi \setminus \{\pm \alpha\}, R) \] ``missing'' from the presentation of $\St^{(\infty)}(\Phi\setminus\{\pm\alpha\}, R)$. The construction of $\widetilde{x}_{-\alpha}$ is analogous to $\widetilde{x}_\alpha$, so it suffices to construct only the latter.

%To verify the assertion of the theorem we need to construct a pro-group morphism \[G_\alpha \colon \St^{(\infty)}(\Phi, R) \to \St^{(\infty)}(\Phi \setminus \{\pm\alpha\}, R)\] such that \(G_\alpha \circ x_\beta = x_\beta\) for all \(\beta \neq \pm \alpha\). By~\cref{SteinbergPresentation} it suffices to construct a pair of morphisms \(\widetilde x_{\pm \alpha} \colon R^{(\infty)} \to \St^{(\infty)}(\Phi \setminus \{\pm \alpha\}, R)\) which together with $\widetilde{x}_\beta$, $\beta\neq \pm\alpha$, satisfy the relations \eqref{R1}--\eqref{R4} missing from the presentation of $\St(\Phi\setminus\{\pm\alpha\}, R)$. As before, we understand Steinberg relations as equalities of morphisms in $\Pro(\Set)$ according to~\cref{conv:notation}.

\begin{df}
Let $\beta, \gamma \in \Phi$ be arbitrary roots such that $\alpha = \beta+\gamma$ and the length of $\beta$, $\gamma$ is the same as $\alpha$'s. 
Denote by $x_{\beta,\gamma}(b^{(\infty)}, c^{(\infty)})$ the morphism
 \[[x_\beta(N_{\beta,\gamma}b^{(\infty)}),\ x_\gamma(c^{(\infty)})]\colon R^{(\infty)} \times R^{(\infty)} \to \St^{(\infty)}(\Phi \setminus\{\pm\alpha\}, R).\]
See~\cref{conv:notation} for the explanation of the notation used in the above formula.

%Of course, alternatively, $x_{\beta,\gamma}$ can be defined as the class of the pre-morphism $x_{\beta,\gamma}^* = \mathrm{id}_S$, $x_{\beta,\gamma}^{(s)} \colon R^{(s)}\times R^{(s)} \to \St^{(s)}(\Phi\setminus\{\pm\alpha\}, R)$, $x_{\beta,\gamma}^{(s)}(b^{(s)}, c^{(s)}) = [x_\beta^{(s)}(b), x_\gamma^{(s)}(c)].$
\end{df}

\begin{lemma} The morphism $x_{\beta, \gamma}$ satisfies the identities of~\cref{RingPresentation} and therefore gives rise to a morphism
 $R^{(\infty)} \to \St^{(\infty)}(\Phi\setminus\{\pm\alpha\}, R),$ which we denote by $\widetilde{x}_\alpha$. The resulting morphism $\widetilde{x}_\alpha$ does not depend on the choice of $\beta$ and $\gamma$ and satisfies the relation~\eqref{R1}.
\end{lemma}
\begin{proof}
 Suppose that \(\alpha = \beta_1 + \gamma_1 = \beta_2 + \gamma_2\) are two different decompositions for $\alpha$ such that $\alpha$, $\beta_i$ and $\gamma_i$ have the same length.
 By~\cref{ThreeRoots} the roots $\beta_i$, $\gamma_i$ span a root subsystem $\Phi_0$ of type \(\rA_3\) or \(\rC_3\). 
 Replacing $\beta_2$ with $\gamma_2$ if necessary, we may assume that \(\beta_1\), \(\beta_2\) form an acute angle.
 Set $\delta = \gamma_1 - \gamma_2 = \beta_2 - \beta_1$ and
 \[\varepsilon_1 = N_{\beta_1, \gamma_1},\ \varepsilon_2 = N_{\beta_2, \gamma_2},\ \varepsilon_3 = N_{\delta,\gamma_2},\ \varepsilon_4 = N_{\beta_1, \delta}.\] 
 
 Consider the case \(\Phi_0 \cong \rA_3\), in which $\delta$ is a root forming an obtuse angle with both $\beta_1$ and $\gamma_2$. 
 Under our assumptions $\beta_1\perp \gamma_2$, $\gamma_1 \perp \beta_2$, therefore 
 substituting the quadruple of roots $\beta_1, \gamma_1, -\beta_2, -\gamma_2$ into~\eqref{eq:cocycle} we obtain that
 $\varepsilon_1 \varepsilon_2 \varepsilon_3 \varepsilon_4 = 1$.
 Notice that
 \begin{align}
 x_{\beta_1, \gamma_1}(b^{(\infty)}, c^{(\infty)} d^{(\infty)}) = [x_{\beta_1}(\varepsilon_1b^{(\infty)}),\ x_{\gamma_1}(c^{(\infty)}d^{(\infty)})] \label{eq:a3-start} \\ 
 = [x_{\beta_1}(\varepsilon_1b^{(\infty)}),\ [x_\delta(\varepsilon_3 c^{(\infty)}),\ x_{\gamma_2}(d^{(\infty)})]] & \text{ by~\eqref{R3}.} \nonumber
 \end{align}
 
Now consider the case \(\Phi_0 = \rC_3\), in which the root $\delta$ is long.
By our assumptions, $\beta_1$ (resp. $\beta_2$) forms an acute angle with $\gamma_2$ (resp. $\gamma_1$) and $\gamma_1 + \gamma_2$ is a root orthogonal to $\beta_1$.
Substituting the same quadruple of roots as above into~\eqref{eq:cocycle} we obtain that
 \[-\varepsilon_1 \varepsilon_2 +  N_{\gamma_1,-\beta_2} N_{\beta_1,-\gamma_2} + 2 \varepsilon_3 \varepsilon_4 = 0\]
from which, again, it follows that $\varepsilon_1 \varepsilon_2 \varepsilon_3 \varepsilon_4 = 1$.
In this case we have that
% we may assume that \(\alpha = \mathrm e_1 + \mathrm e_3\), \(\beta_1 = \mathrm e_1 - \mathrm e_2\), \(\gamma_1 = \mathrm e_2 + \mathrm e_3\), \(\beta_2 = \mathrm e_1 + \mathrm e_2\), \(\gamma_2 = \mathrm e_3 - \mathrm e_2\). Let \(\delta = 2 \mathrm e_2\), then
 \begin{align}
 x_{\beta_1, \gamma_1}(b^{(\infty)}, c^{(\infty)} d^{(\infty)}) = [x_{\beta_1}(\varepsilon_1b^{(\infty)}), x_{\gamma_1}(c^{(\infty)} d^{(\infty)})] & \nonumber \\
 [x_{\beta_1}(\varepsilon_1b^{(\infty)}), x_{\gamma_1}(c^{(\infty)} d^{(\infty)}) x_{\gamma_1 + \gamma_2}(\widehat{N}_{\delta,\gamma_1}c^{(\infty)} d^{(\infty)})] & \text{ by~\eqref{R2},\eqref{eq:comm-mult-rhs}} \label{eq:c3-start}\\
 = [x_{\beta_1}(\varepsilon_1b^{(\infty)}), [x_{\delta}(\varepsilon_3c^{(\infty)}), x_{\gamma_2}(d^{(\infty)})]] &\text{ by~\eqref{R4}.} \nonumber
 \end{align}
Now both~\eqref{eq:a3-start} and~\eqref{eq:c3-start} can be continued as follows:
 \begin{align*}
  \ldots = [[x_{\beta_1}(\varepsilon_1b^{(\infty)}),\ x_\delta(\varepsilon_3c^{(\infty)})],\ {}^{x_\delta(\varepsilon_3c^{(\infty)})}x_{\gamma_2}(d^{(\infty)})] & \text{ by~\eqref{R2},\eqref{eq:HW-corr}}\\
 = [x_{\beta_2}(\varepsilon_2b^{(\infty)}c^{(\infty)}),\ x_{\gamma_2}(d^{(\infty)}) x_{\gamma_1}(c^{(\infty)}d^{(\infty)}) ] & \text{ by~\eqref{R3}}\\
 = [x_{\beta_2}(\varepsilon_2b^{(\infty)}c^{(\infty)}),\ x_{\gamma_2}(d^{(\infty)})] &\text{ by~\eqref{R2},\eqref{eq:comm-mult-rhs}} \\
 = x_{\beta_2, \gamma_2}(b^{(\infty)} c^{(\infty)}, d^{(\infty)}). \end{align*}
 In both cases we obtain that \(x_{\beta_1, \gamma_1}(b^{(\infty)} c^{(\infty)}, d^{(\infty)}) = x_{\beta_2, \gamma_2}(b^{(\infty)}, c^{(\infty)} d^{(\infty)})\),
 from which and~\cref{RingPresentation} it follows that there exists a unique morphism \[\widetilde x_\alpha \colon R^{(\infty)} \to \St^{(\infty)}(\Phi \setminus\{\pm \alpha\}, R)\] such that \(x_{\beta_1, \gamma_1}(N_{\beta_1, \gamma_1}b^{(\infty)}, c^{(\infty)}) = \widetilde x_\alpha(b^{(\infty)} c^{(\infty)}).\) It is clear that \(\widetilde x_\alpha\) satisfies~\eqref{R1}.
\end{proof}

The next step of the proof of~\cref{SingleRootElimination} is to verify the Steinberg relations, which are ``missing'' from the presentation of $\St^{(\infty)}(\Phi\setminus\{\pm\alpha\}, R)$.
This is accomplished in a series of lemmas below.
Notice that none of these Steinberg relations simultaneously involves both \(\alpha\) and \(-\alpha\).
Thus, by symmetry it suffices to verify only Steinberg relations which involve \(\alpha\).

%Although we have shown that $\widetilde{x}_{\alpha}$ 

\begin{lemma}
 The root subgroup morphism $\widetilde{x}_\alpha$ satisfies Steinberg relations~\eqref{R3}--\eqref{R4} in which it occurs on the right-hand side.
\end{lemma}
\begin{proof}
 It remains to prove (\ref{S3}) and (\ref{S4}) with \(\alpha\) from the right. There are three cases (and all of them may arise only for \(F_4\)): \(\alpha = \beta + \gamma\) with \(\alpha\) long and \(\beta\), \(\gamma\) short, \(\alpha = \beta + \gamma\) with \(\alpha\), \(\beta\) short and \(\gamma\) long, \(\alpha = 2\beta + \gamma\) for \(\beta\) short and \(\alpha\), \(\gamma\) long.

 \begin{enumerate}
 \item If \(\alpha = \beta + \gamma\) with \(\alpha\), \(\beta\) short and \(\gamma\) long, consider a decomposition \(\beta = \beta_1 + \beta_2\) for short roots \(\beta_i\). The smallest root subsystem containing \(\beta_i\) and \(\gamma\) is of type \(C_3\), so we may assume that \(\alpha = \mathrm e_1 + \mathrm e_3\), \(\beta = \mathrm e_1 - \mathrm e_3\), \(\beta_1 = \mathrm e_1 - \mathrm e_2\), \(\beta_2 = \mathrm e_2 - \mathrm e_3\), \(\gamma = 2\mathrm e_3\). Then
 \begin{align*}
  \up{x_\gamma(c^{(\infty)})}
   {x_\beta(N_{\beta_1, \beta_2} b_1^{(\infty)} b_2^{(\infty)})}
  &= [\up{x_\gamma(c^{(\infty)})}
   {x_{\beta_1}(b_1^{(\infty)})},
  \up{x_\gamma(c^{(\infty)})}
   {x_{\beta_2}(b_2^{(\infty)})}] =\\
  %
  &= [x_{\beta_1}(b_1^{(\infty)}),
  x_{\beta_2}(b_2^{(\infty)})
  x_{\mathrm e_2 + \mathrm e_3}(\ldots)
  x_{2\mathrm e_2}(\ldots)] =\\
  %
  &= x_\beta(N_{\beta_1, \beta_2} b_1^{(\infty)} b_2^{(\infty)})
  \widetilde x_\alpha(\ldots)
  x_{\mathrm e_1 + \mathrm e_2}(\ldots)
  x_{\mathrm e_1 + \mathrm e_2}(\ldots)
  x_{2\mathrm e_1}(\ldots).
 \end{align*}
 \item Suppose that \(\alpha = 2\beta + \gamma\) for \(\beta\) short and \(\alpha\), \(\gamma\) long. Then consider a decomposition \(\gamma = \gamma_1 + \gamma_2\) for long roots \(\gamma_i\). The smallest root subsystem containing \(\beta\) and \(\gamma_i\) is of type \(B_3\), so we may assume that \(\alpha = \mathrm e_1 + \mathrm e_3\), \(\beta = \mathrm e_3\), \(\gamma = \mathrm e_1 - \mathrm e_3\), \(\gamma_1 = \mathrm e_1 - \mathrm e_2\), \(\gamma_2 = \mathrm e_2 - \mathrm e_3\). Then
 \begin{align*}
  \up{x_\beta(b^{(\infty)})}
   {x_\gamma(N_{\gamma_1, \gamma_2} c_1^{(\infty)} c_2^{(\infty)})}
  &= [\up{x_\beta(b^{(\infty)})}
   {x_{\gamma_1}(c_1^{(\infty)})},
  \up{x_\beta(b^{(\infty)})}
   {x_{\gamma_2}(c_2^{(\infty)})}] =\\
  %
  &= [x_{\gamma_1}(c_1^{(\infty)}),
  x_{\gamma_2}(c_2^{(\infty)})
  x_{\mathrm e_2}(N_{\beta, \gamma_2} b^{(\infty)} c_2^{(\infty)})
  x_{\mathrm e_2 + \mathrm e_3}(N_{\beta, \gamma_2} \widehat N_{\beta, \beta + \gamma_2} (b^{(\infty)})^2 c_2^{(\infty)})] =\\
  %
  &= x_\gamma(N_{\gamma_1, \gamma_2} c_1^{(\infty)})
  x_{\mathrm e_1}(\ldots)
  \widetilde x_\alpha(\ldots).
 \end{align*}
 \item If \(\beta\) and \(\gamma\) are short, then the smallest root subsystem containing \(\alpha\), \(\beta\), \(\gamma\) is of type \(B_2\). Without loss of generality let \(\alpha = \mathrm e_1 + \mathrm e_2\), \(\beta = \mathrm e_1\), and \(\gamma = \mathrm e_2\). Then using the previous case we get
 \begin{align*}
  &\up{x_\beta(d^{(\infty)})}
   {\bigl(x_\gamma(N_{\beta, \gamma - \beta} b^{(\infty)} c^{(\infty)})
   \widetilde x_\alpha(N_{\beta, \gamma - \beta} \widehat N_{\beta, \gamma}
   (b^{(\infty)})^2 c^{(\infty)})\bigr)} =\\
  %
  &= [x_\beta(b^{(\infty)}),
  \up{x_\beta(d^{(\infty)})}
   {x_{\gamma - \beta}(c^{(\infty)})}] =\\
  %
  &= \up{x_\beta(b^{(\infty)} + d^{(\infty)})}
   {x_{\gamma - \beta}(c^{(\infty)})}\,
  \up{x_\beta(d^{(\infty)})}
   {x_{\gamma - \beta}(-c^{(\infty)})} =\\
  %
  &= x_\gamma(N_{\beta, \gamma - \beta} b^{(\infty)} c^{(\infty)})
  \widetilde x_\alpha(N_{\beta, \gamma - \beta} \widehat N_{\beta, \gamma}
   (b^{(\infty)})^2 c^{(\infty)}
   + N_{\beta, \gamma} N_{\beta, \gamma - \beta}
   b^{(\infty)} c^{(\infty)} d^{(\infty)}).
 \end{align*}
 \end{enumerate}
\end{proof}
 
 If \(\delta \in \Phi \setminus (\mathbb R \beta + \mathbb R \gamma)\), then
 \begin{align*}
  [x_{\beta, \gamma}(b^{(\infty)}, c^{(\infty)}),
  x_\delta(d^{(\infty)})]
  &= [x_\beta(b^{(\infty)}) x_\gamma(c^{(\infty)})
   x_\beta(-b^{(\infty)}) x_\gamma(-c^{(\infty)}),
  x_\delta(d^{(\infty)})] =\\
  %
  &= \prod_{\substack{i\beta + j\gamma + k\delta \in \Phi\\ k > 0}}
  x_{i\beta + j\gamma + k\delta}
   (A_{\beta, \gamma, \delta, i, j, k}
   (b^{(\infty)})^i (c^{(\infty)})^j (d^{(\infty)})^k)
 \end{align*}
 for some integers \(A_{\beta, \gamma, \delta, i, j, k}\),
 hence we have almost all Steinberg relations with \(\alpha\) and \(\delta\) in the left.

By lemma \ref{ThreeRoots} there are roots \(\beta_1, \beta_2 \in \Phi \setminus (\mathbb R \beta + \mathbb R \gamma)\) such that \(|\beta_1| = |\beta_2| = |\beta|\) and \(\beta = \beta_1 + \beta_2\). Moreover, we know all possible arrangements of these roots in a root subsystem of type \(A_3\) or \(C_3\). Then
 \begin{align*}
  \up{x_{\beta, \gamma}(b^{(\infty)}, c^{(\infty)})}
  {x_\beta(N_{\beta_1, \beta_2}
   b_1^{(\infty)} b_2^{(\infty)})}
  &= [\up{x_{\beta, \gamma}(b^{(\infty)}, c^{(\infty)})}
   {x_{\beta_1}(b_1^{(\infty)})},
  \up{x_{\beta, \gamma}(b^{(\infty)}, c^{(\infty)})}
   {x_{\beta_2}(b_2^{(\infty)})}] =\\
  %
  &= [x_{\beta_1}(b_1^{(\infty)})
   x_{\alpha + \beta_1}(N_{\alpha, \beta_1} N_{\beta, \gamma}
   b_1^{(\infty)} b^{(\infty)} c^{(\infty)}),\\
  %
  &\qquad x_{\beta_2}(b_2^{(\infty)})
   x_{\alpha + \beta_2}(N_{\alpha, \beta_2} N_{\beta, \gamma}
   b_2^{(\infty)} b^{(\infty)} c^{(\infty)})] =\\
  %
  &= x_\beta(N_{\beta_1, \beta_2} b_1^{(\infty)} b_2^{(\infty)}).
 \end{align*}

 Similarly,
 \begin{align*}
  \up{x_{\beta, \gamma}(b^{(\infty)}, c^{(\infty)})}
   {x_{-\beta}(N_{-\beta_1, -\beta_2} b_1^{(\infty)} b_2^{(\infty)})}
  &= [\up{x_{\beta, \gamma}(b^{(\infty)}, c^{(\infty)})}
   {x_{-\beta_1}(b_1^{(\infty)})},
  \up{x_{\beta, \gamma}(b^{(\infty)}, c^{(\infty)})}
   {x_{-\beta_2}(b_2^{(\infty)})}] =\\
  %
  &= [x_{-\beta_1}(b_1^{(\infty)})
   x_{\alpha - \beta_1}(N_{\alpha, -\beta_1} N_{\beta, \gamma}
   b_1^{(\infty)} b^{(\infty)} c^{(\infty)}),\\
  %
  &\qquad x_{-\beta_2}(b_2^{(\infty)})
   x_{\alpha - \beta_2}(N_{\alpha, -\beta_2} N_{\beta, \gamma}
   b_2^{(\infty)} b^{(\infty)} c^{(\infty)})] =\\
  %
  &= x_{-\beta}(N_{-\beta_1, -\beta_2}
   b_1^{(\infty)} b_2^{(\infty)})
   x_\gamma(\ldots).
 \end{align*}
 Hence we have all cases of (\ref{S3}) with \(\alpha\) in the left. It follows that the morphism \(x_{\beta, \gamma}\) of pro-sets satisfy the first three conditions of lemma \ref{RingPresentation}. Also \(x_{\gamma, \beta}(c^{(\infty)}, b^{(\infty)}) = x_{\beta, \gamma}(-b^{(\infty)}, c^{(\infty)})\).

 
 Now clearly \(G_\alpha\) is the inverse of \(F_\alpha\).


 \begin{theorem}\label{DoubleRootElimination}
  Let \(R\) be a ring, \(S \subseteq R\) be a multiplicative subset, \(\Phi\) be a root system of rank \(\geq 3\) different from \(\rB_l\) and \(\rC_l\), \(\alpha, \beta \in \Phi\) be linearly independent roots. Then \(\St^{(\infty)}(\Phi / \{\alpha, \beta\}, R) \to \St^{(\infty)}(\Phi, R)\) is an epimorphism of pro-groups.
 \end{theorem}
 \begin{proof}
  If \(\Phi_0\) is not of type \(A_2\), then every root from \(\Phi_0\) may be decomposed into a sum of two roots with the same length, these roots do not lie in \(\Phi_0\) and we are done by lemma \ref{SteinbergPresentation}. If \(\Phi_0\) is of type \(A_2\), apply lemma \ref{ThreeRoots}.
 \end{proof}


\subsection{Localization}

Let \(R\) be a ring and \(S \subseteq R\) be a multiplicative subset. Then \(R^{(\infty)}\) is an \(S^{-1} R\)-algebra in the following sense: for every element \(\frac rs \in S^{-1} R\) there is a well-defined pro-group endomorphism \(m_{\frac rs} \colon R^{(\infty)} \to R^{(\infty)}\) given by \(m_{\frac rs}^*(s') = ss'\) and \(m_{\frac rs}^{(s')}(a^{(ss')}) = (ra)^{(s')}\) for some choice of \(r\) and \(s\). Moreover, \(m\) is actually a homomorphism of rings \(S^{-1} R \to \Pro(\Group)(R^{(\infty)})\) and \(m_{\frac rs}(a^{(\infty)} b^{(\infty)}) = m_{\frac rs}(a^{(\infty)}) b^{(\infty)} = a^{(\infty)} m_{\frac rs}(b^{(\infty)})\). We usually write \(\frac rs a^{(\infty)}\) instead of \(m_{\frac rs}(a^{(\infty)})\). Similarly, \(\GG_{\mathrm{sc}}(\Phi, S^{-1} R)\) acts on the Chevalley pro-group \(\GG^{(\infty)}(\Phi, R)\) by automorphisms.

\begin{df}\label{root-action}
 Let \(R\) be a ring, \(S \subseteq R\) be a multiplicative subset, and \(\Phi\) be a root system of rank \(\geq 3\) different from \(\rB_l\) and \(\rC_l\). For every root \(\beta \in \Phi\) and every \(g\) in the standard torus \(\Torus(\Phi, S^{-1} R)\) of \(\GG_{\mathrm{sc}}(\Phi, S^{-1} R)\) let
 \[\up g{x_\beta(b^{(\infty)})} = x_\beta(\beta(g) b^{(\infty)}).\]
 For every linearly independent \(\alpha, \beta \in \Phi\) and \(u \in S^{-1} R\) let
 \begin{align*}
 \up{x_\alpha(u)}{x_\beta(b^{(\infty)})}
 &= x_\beta(b^{(\infty)}), \\
 \multispan2{\hfil if $\alpha + \beta \notin \Phi$;} \\
 %
 \up{x_\alpha(u)}{x_\beta(b^{(\infty)})}
 &= x_{\alpha + \beta}\bigl(N_{\alpha, \beta} u b^{(\infty)}\bigr)
  x_\beta(b^{(\infty)}), \\
 \multispan2{\hfill if $\alpha + \beta \in \Phi$ but $\alpha + 2\beta, 2\alpha + \beta \notin \Phi$;} \\
 %
 \up{x_\alpha(u)}{x_\beta(b^{(\infty)})}
 &= x_{\alpha + \beta}\bigl(N_{\alpha, \beta} u b^{(\infty)}\bigr)
  x_{2\alpha + \beta}\bigl(N_{\alpha, \beta} \widehat N_{\alpha, \alpha + \beta} u^2 b^{(\infty)}\bigr)
  x_\beta(b^{(\infty)}), \\
 \multispan2{\hfill if $\alpha + \beta, 2\alpha + \beta \in \Phi$;} \\
 %
 \up{x_\alpha(u)}{x_\beta(b^{(\infty)})}
 &= x_{\alpha + \beta}\bigl(N_{\alpha, \beta} u b^{(\infty)}\bigr)
  x_{\alpha + 2\beta}\bigl(N_{\alpha, \beta} \widehat N_{\beta, \alpha + \beta} u (b^{(\infty)})^2\bigr)
  x_\beta(b^{(\infty)}), \\
 \multispan2{\hfill if $\alpha + \beta, \alpha + 2\beta \in \Phi$.}
 \end{align*} 
\end{df}

\begin{lemma}\label{SteinbergLocalAction}
 Let \(R\) be a ring with a multiplicative subset \(S \subseteq R\), \(\Phi\) be a root system of rank \(\geq 3\) different from \(\rB_l\) and \(\rC_l\). Then the group \(\St(\Phi, S^{-1} R) \rtimes \Torus(\Phi, S^{-1} R)\) acts on the pro-group \(\St^{(\infty)}(\Phi, R)\) (by pro-group automorphisms) as in the definition \ref{root-action}, the morphism \(\mathrm{st} \colon \St^{(\infty)}(\Phi, R) \to \GG^{(\infty)}(\Phi, R)\) is equivariant with respect to this action. This action also satisfies
 \[\up{x_\alpha(u)}{x_\alpha(a^{(\infty)})} = x_\alpha(a^{(\infty)}).\]
\end{lemma}
\begin{proof}
 In is clear from the definitions that \(\Torus(\Phi, S^{-1} R)\) acts on \(\St^{(\infty)}(\Phi, R)\). For every root \(\alpha \in \Phi\) the root subgroup \(x_\alpha(S^{-1} R) \leq \St(\Phi, S^{-1} R)\) acts on \(\St^{(\infty)}(\Phi \setminus \{\alpha, -\alpha\}, R)\), hence on \(\St^{(\infty)}(\Phi, R)\) by theorem \ref{SingleRootElimination}. Moreover, the actions of \(g x_\alpha(u) g^{-1}\) and \(x_\alpha(\alpha(g) u)\) coincide for all \(g \in \Torus(\Phi, S^{-1} R)\), \(u \in R\). We still have to check that the actions of \(x_\alpha(u)\) and \(x_\beta(v)\) satisfy the Steinberg relation on \(x_\alpha(u)\) and \(x_\beta(v)\) if \(\alpha \neq -\beta\). But this is obvious if we consider the action on \(\St^{(\infty)}(\Phi / \{\alpha, \beta\}, R)\), so the claim follows by theorem \ref{DoubleRootElimination}.
 % !!!!!!!!!!!!!!! Here Phi / {alpha, beta} may be replaced by a better notation. !!!!!!!!!!!!!!!

 Equivariance of the morphism \(\mathrm{st}\) is clear. If \(\alpha = \beta + \gamma\) for roots \(\beta, \gamma \in \Phi\) with \(|\alpha| = |\beta| = |\gamma|\), then
 \begin{align*}
  \up{x_\alpha(u)}{x_\alpha(N_{\beta, \gamma} b^{(\infty)} c^{(\infty)})} &= [\up{x_\alpha(u)}{x_\beta(b^{(\infty)})}, \up{x_\alpha(u)}{x_\gamma(c^{(\infty)})}] =\\
  &= [x_\beta(b^{(\infty)}), x_\gamma(c^{(\infty)})] = x_\alpha(N_{\beta, \gamma} b^{(\infty)} c^{(\infty)}),
 \end{align*}
 hence the last claim follows.
\end{proof}

\begin{theorem}\label{ChevalleyLocalAction}
 Let \(R\) be a ring, \(\mathfrak m \trianglelefteq R\) be a maximal ideal, \(\Phi\) be a root system of rank \(\geq 3\) different from \(\rB_l\) and \(\rC_l\). Consider a multiplicative subset \(S = R \setminus \mathfrak m\). Then the action from lemma \ref{SteinbergLocalAction} factors through an action of the group \(\GG_{\mathrm{sc}}(\Phi, R_{\mathfrak m})\) on \(\St^{(\infty)}(\Phi, R)\). The morphism \(\mathrm{st} \colon \St^{(\infty)}(\Phi, R) \to \GG^{(\infty)}(\Phi, R)\) is \(\GG_{\mathrm{sc}}(\Phi, R_{\mathfrak m})\)-equivariant.
\end{theorem}
\begin{proof}
 For \(a \in R_{\mathfrak m}^*\) and \(\alpha \in \Phi\) let \(w_\alpha(a) = x_\alpha(a) x_{-\alpha}(-a^{-1}) x_\alpha(a)\) and \(h_\alpha(a) = w_\alpha(a) w_\alpha(1)^{-1}\). It is well-known that \(\GG_{\mathrm{sc}}(\Phi, R_{\mathfrak m})\) is the factor-group of \(\St(\Phi, R_{\mathfrak m}) \rtimes \Torus(\Phi, R_{\mathfrak m})\) by the relations
 \[h_\alpha(a) = \alpha^\vee(a)\]
 for all \(\alpha \in \Phi\) and \(a \in R_{\mathfrak m}^*\) (recall that \(R_{\mathfrak m}\) is a local ring). We only have to check that \(h_\alpha(a)\) and \(\alpha^\vee(a)\) acts in the same way on \(\St^{(\infty)}(\Phi, R)\). Indeed, if \(\beta \in \Phi\) is linearly independent with \(\alpha\), then
 \[\up{h_\alpha(a)}{x_\beta(b^{(\infty)})} = \prod_{\substack{i\alpha + j\beta \in \Phi\\ j > 0}} x_{i\alpha + j\beta}(A_{i, j, \alpha, \beta} a^i (b^{(\infty)})^j)\]
 for some universal \(A_{i, j, \alpha, \beta} \in \mathbb Z\). But then in the Steinberg group \(\St(\Phi, \mathbb Z[a, a^{-1}, b])\) we have the same equation
 \[\up{h_\alpha(a)}{x_\beta(b)} = \prod_{\substack{i\alpha + j\beta \in \Phi\\ j > 0}} x_{i\alpha + j\beta}(A_{i, j, \alpha, \beta} a^i b^j).\]
 Here the right hand side lies in the unipotent radical of this Steinberg group, this unipotent radical embeds into the Chevalley group \(\GG_{\mathrm{sc}}(\Phi, \mathbb Z[a, a^{-1}, b])\). The image of the left hand side is the same as the image of \(\up{\alpha^\vee(a)}{x_\beta(b)}\). Hence \(A_{i, j, \alpha, \beta} = 0\) with the only exception \(A_{(\beta, \alpha), 1, \alpha, \beta} = 1\), and we are done.
\end{proof}


\subsection{Steinberg crossed module}

Here we prove that the Steinberg group \(\St(\Phi, R)\) is a crossed module over \(\GG(\Phi, R)\). In the proofs we always take \(S = R \setminus \mathfrak m\) for a given maximal ideal \(\mathfrak m \trianglelefteq R\).

\begin{lemma}\label{CentralityK2}
 Let \(R\) be a commutative ring, \(\Phi\) be a root system of rank \(\geq 3\) different from \(\rB_l\) and \(\rC_l\). Then \(\mathrm K_2(\Phi, R) = \Ker(\mathrm{st} \colon \St(\Phi, R) \to \GG(\Phi, R))\) is a central subgroup of \(\St(\Phi, R)\).
\end{lemma}
\begin{proof}
 Let \(g \in \mathrm K_2(\Phi, R)\) and \(\alpha \in \Phi\) be a root. Consider the ideal
 \[\mathfrak a = \{a \in R \mid \up g{x_\alpha(ra)} = x_\alpha(ra) \text{ for all } r \in R\}.\]
 We have to show that \(\mathfrak a = R\). Take a maximal ideal \(\mathfrak m \trianglelefteq R\), then by theorem \ref{ChevalleyLocalAction}, \(g\) trivially acts on the pro-group \(\St^{(\infty)}(\Phi, R)\). This means that \(\mathfrak a \not \leq \mathfrak m\).
\end{proof}

The next result is actually well-known. % !!!!!!!!! Add references !!!!!!!!!!!

\begin{lemma}\label{Normality}
 Let \(R\) be a commutative ring, \(\Phi\) be a root system of rank \(\geq 3\) different from \(\rB_l\) and \(\rC_l\). Then \(\mathrm E(\Phi, R) = \mathrm{Im}(\mathrm{st} \colon \St(\Phi, R) \to \GG(\Phi, R))\) is a normal subgroup of \(\GG(\Phi, R)\).
\end{lemma}
\begin{proof}
 Let \(g \in \GG(\Phi, R)\) and \(\alpha \in \Phi\) be a root. Consider the ideal
 \[\mathfrak a = \{a \in R \mid \up g{x_\alpha(Ra)} \in \mathrm E(\Phi, R)\}.\]
 Again, we have to show that \(\mathfrak a = R\). Take a maximal ideal \(\mathfrak m \trianglelefteq R\), then by theorem \ref{ChevalleyLocalAction}, \(g\) acts on the pro-group \(\St^{(\infty)}(\Phi, R)\) such that \(\mathrm{st} \colon \St^{(\infty)}(\Phi, R) \to \GG^{(\infty)}(\Phi, R)\) is equivariant. As above, this means that \(\mathfrak a \not \leq \mathfrak m\).
\end{proof}

For any ring \(R\) with a multiplicative subset \(S \subseteq R\) let \(\pi_R \colon R^{(\infty)} \to R\) and \(\pi_{\St} \colon \St^{(\infty)}(\Phi, R) \to R\) be the canonical morphisms.

\begin{theorem}\label{SteinbergCrossedModule}
 Let \(R\) be a commutative ring, \(\Phi\) be a root system of rank \(\geq 3\) different from \(\rB_l\) and \(\rC_l\). Then \(\mathrm{st} \colon \St(\Phi, R) \to \GG(\Phi, R)\) is a crossed module in a unique way.
\end{theorem}
\begin{proof}
 We already know that the Steinberg group \(\St(\Phi, R)\) is a central (see lemma \ref{CentralityK2}) perfect extension of the elementary group \(\mathrm E(\Phi, R)\). For any \(g \in \GG(\Phi, R)\) it remains to construct an endomorphism \(\up g{(-)} \colon \St(\Phi, R) \to \St(\Phi, R)\) such that \(\mathrm{st}\) is equivariant. Such an endomorphism is necessarily unique and multuplicative on \(g\). Fix an element \(g \in \GG(\Phi, R)\).

 Take a root \(\alpha \in \Phi\). Since \(\mathrm E(\Phi, R) \trianglelefteq \GG(\Phi, R)\) by lemma \ref{Normality}, the set
 \[Y_\alpha(a) = \mathrm{st}^{-1}\bigl(\up g{t_\alpha(a)}\bigr)\]
 is non-empty for all \(a \in R\). Moreover, it is a coset of \(\mathrm K_2(\Phi, R)\) and \(Y_\alpha(a + a') = Y_\alpha(a) Y_\alpha(a')\).

 First of all, we prove that \([Y_\alpha(a), Y_\beta(b)] = 1\) if \(\alpha + \beta \notin \Phi \cup \{0\}\). For any such \(\alpha, \beta\) and for any \(a \in R\) fix an element \(h \in Y_\alpha(a)\) and consider the ideal
 \[\mathfrak b = \{b \in R \mid [h, Y_\beta(bR)] = 1\}.\]
 If \(\mathfrak m \trianglelefteq R\) is a maximal ideal, then
 \[[h, \pi_{\St}(\up g {x_\beta(b^{(\infty)})})] = \pi_{\St}(\up{g t_\alpha(a)} {x_\beta(-b^{(\infty)})}) \pi_{\St}(\up g {x_\beta(-b^{(\infty)})}) = 1.\]
 In other words, \(\mathfrak b \not \leq \mathfrak m\), hence \(\mathfrak b = R\) and we are done.

 Fix a root \(\alpha \in \Phi\). There are roots \(\beta, \gamma \in \Phi\) of the same length such that \(\alpha = \beta + \gamma\). Let \(y_\alpha(a)\) be the only element of \([Y_\beta(N_{\beta, \gamma}^{-1}), Y_\gamma(a)]\). Clearly, \(y_\alpha \colon R \to \St(\Phi, R)\) is a homomorphism. For any maximal ideal \(\mathfrak m \trianglelefteq R\) we have
 \begin{align*}
  y_\alpha(\pi_R(a^{(\infty)}))
  &= [Y_\beta(N_{\beta, \gamma}^{-1}), \pi_{\St}(\up g{x_\gamma(a^{(\infty)})})] =\\
  %
  &= \pi_{\St}\bigl(\up{g t_\beta(N_{\beta, \gamma}^{-1})}{x_\gamma(a^{(\infty)})}\, \up g{x_\gamma(-a^{(\infty)})}\bigr) = \pi_{\St}(\up g{x_\alpha(a^{(\infty)})}).
 \end{align*}

 It remains to check that \(y_\alpha(a)\) satisfy the Steinberg relations \ref{R3} and \ref{R4}. Take roots \(\alpha, \beta \in \Phi\) such that \(\alpha + \beta \in \Phi\), but \(2\alpha + \beta\) and \(\alpha + 2\beta\) are not roots. Fix an element \(a \in R\) and consider the ideal
 \[\mathfrak b_1 = \{b \in R \mid [y_\alpha(a), y_\beta(br)] = y_{\alpha + \beta}(N_{\alpha, \beta} abr) \text{ for all } r \in R\}.\]
 For any maximal ideal \(\mathfrak m \trianglelefteq R\) we have
 \begin{align*}
 [y_\alpha(a), y_\beta(\pi_R(b^{(\infty)}))] &= [y_\alpha(a), \pi_{\St}(\up g{x_\beta(b^{(\infty)})})] =\\
 %
 &= \pi_{\St}\bigl(\up{g t_\alpha(a)}{x_\beta(b^{(\infty)})}\, \up g{x_\beta(-b^{(\infty)})}\bigr) =\\
 %
 &= \pi_{\St}(\up g{x_{\alpha + \beta}(N_{\alpha, \beta} a b^{(\infty)})}) = y_{\alpha + \beta}(N_{\alpha, \beta} a \pi_R(b^{(\infty)})).
 \end{align*}
 It follows that \(\mathfrak b_1 \not \leq \mathfrak m\), i.e. \(\mathfrak b_1 = R\) and we have \ref{R3}.

 Similarly, take roots \(\alpha, \beta \in \Phi\) such that \(\alpha + \beta, 2\alpha + \beta \in \Phi\). Fix an element \(a \in R\) and consider the ideal
 \begin{align*}
 \mathfrak b_2 = \{b \in R \mid [y_\alpha(a), y_\beta(br)] &= y_{\alpha + \beta}(N_{\alpha, \beta} abr)\\
 &\quad y_{2\alpha + \beta}(N_{\alpha, \beta} \widehat N_{\alpha, \alpha + \beta} a^2 br) \text{ for all } r \in R\}.
 \end{align*}
 For any maximal ideal \(\mathfrak m \trianglelefteq R\) we have
 \begin{align*}
 [y_\alpha(a), y_\beta(\pi_R(b^{(\infty)}))] &= [y_\alpha(a), \pi_{\St}(\up g{x_\beta(b^{(\infty)})})] =\\
 %
 &= \pi_{\St}\bigl(\up{g t_\alpha(a)}{x_\beta(b^{(\infty)})}\, \up g{x_\beta(-b^{(\infty)})}\bigr) =\\
 %
 &= \pi_{\St}\bigl(\up g{x_{\alpha + \beta}(N_{\alpha, \beta} a b^{(\infty)})}\, \up g{x_{2\alpha + \beta}(N_{\alpha, \beta} \widehat N_{\alpha, \alpha + \beta} a^2 b^{(\infty)})}\bigr) =\\
 %
 &= y_{\alpha + \beta}(N_{\alpha, \beta} a \pi_R(b^{(\infty)})) y_{2\alpha + \beta}(N_{\alpha, \beta} \widehat N_{\alpha, \alpha + \beta} a^2 \pi_R(b^{(\infty)})).
 \end{align*}
 It follows that \(\mathfrak b_2 \not \leq \mathfrak m\), i.e. \(\mathfrak b_2 = R\).
\end{proof}


\printbibliography
\end{document}
