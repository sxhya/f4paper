\documentclass{article}

\usepackage{amsmath, amssymb, amsthm, amscd, alltt, graphicx}
\usepackage[utf8]{inputenc}
\usepackage[T1]{fontenc}
\usepackage[capitalise]{cleveref}
\usepackage[matrix,arrow,curve]{xy}

\usepackage[backend=biber, bibencoding=utf8, giveninits=true, citestyle=numeric-comp, sortlocale=en_US, url=false, doi=false, eprint=true, maxbibnames=4]{biblatex}
\addbibresource{f4paper.bib}
\renewbibmacro*{volume+number+eid}{\ifentrytype{article}{\- \iffieldundef{volume}{}{Vol.~\printfield{volume},}\iffieldundef{number}{}{ No.~\printfield{number},}}}
\renewbibmacro{in:}{\ifentrytype{article}{}{\printtext{\bibstring{in}\intitlepunct}}}
\newbibmacro{string+doi}[1]{\iffieldundef{doi}{\iffieldundef{url}{#1}{\href{\thefield{url}}{#1}}}{\href{https://dx.doi.org/\thefield{doi}}{#1}}}
\DeclareFieldFormat[article, inproceedings, inbook, book, online]{title}{\usebibmacro{string+doi}{\mkbibquote{#1}}}
\renewcommand*{\bibfont}{\footnotesize}

\newtheorem{lemma}{Lemma} \numberwithin{lemma}{section}
\newtheorem{prop}{Proposition} \numberwithin{prop}{section}
\newtheorem{theorem}{Theorem}
\newtheorem*{theorem*}{Theorem}

\theoremstyle{definition} 
\newtheorem{df}[lemma]{Definition} \Crefname{df}{Definition}{Definitions}
\newtheorem{example}[lemma]{Example} \Crefname{example}{Example}{Examples}

\theoremstyle{remark} 
\newtheorem{rem}[lemma]{Remark}

\newenvironment{psmallmatrix}{\left(\begin{smallmatrix}}{\end{smallmatrix}\right)}

\DeclareMathOperator\mat{M}
\DeclareMathOperator\elin{E}
\DeclareMathOperator\stlin{St}
\DeclareMathOperator\gstlin{GSt}
\DeclareMathOperator\klin{K}
\DeclareMathOperator\stmap{st}
\DeclareMathOperator\glin{GL}
\DeclareMathOperator\diag{D}
\DeclareMathOperator\Jac{J}
\DeclareMathOperator{\id}{id}
\DeclareMathOperator{\Pro}{Pro}
\DeclareMathOperator{\Ad}{Ad}

\newcommand{\lar}{\leftarrow}
\newcommand{\leqt}{\trianglelefteq}
\newcommand{\sub}[2]{{_{#1\!}{#2}}}
\newcommand{\up}[2]{{^{#1}\!{#2}}}

\newcommand{\Set}{\mathbf{Set}}
\newcommand{\Group}{\mathbf{Grp}}
\newcommand{\Rng}{\mathbf{Rng}}
\newcommand{\Fun}{\mathbf{Fun}}
\newcommand{\Mod}{\mathbf{Mod}}
\newcommand{\op}{\mathrm{op}}

\begin{document}

\section{Preliminaries}
\subsection{Generalities on pro-objects}
Let \(\mathcal C\) be an arbitrary category.
In this section we recall the construction of the pro-completion of \(\mathcal C\) (cf. \cite[Section~6.1]{SK06}).

Recall that a nonempty small category \(\mathcal I\) is called {\it filtered} if
\begin{itemize}
 \item for any two objects \(i, k \in Ob(\mathcal{I})\) there is a diagram \(i \to j \lar k\) in \(\mathcal I\);
 \item every two parallel morphisms \(i \rightrightarrows j\) are equalized by some morphism \(j \to k\) in \(\mathcal I\).
\end{itemize}
A {\it pro-object} in \(\mathcal C\) is, by definition, a functor $X^{(\infty)}\colon \mathcal{I}_X^{\op} \to \mathcal{C}$, i.\,e. a contravariant functor from a filtered category \(\mathcal I_X\), called {\it the category of indices of $X^{(\infty)}$}, to the category \(\mathcal C\). %The objects of \(\mathcal I_X\) are called {\it the indices of \(X^{(\infty)}\)}.

We denote by \(X^{(i)}\) the value of \(X^{(\infty)}\) on an index \(i\) and suppress the notation for the value of \(X^{(\infty)}\) on a morphism \(i \to j\) if it is clear from the context.
%(for example, if \(j\) is a sufficiently large index). 
%Thus, a pro-object \(X^{(\infty)}\) is the formal projective limit of \(X^{(i)}\). 
%We use the notation with upper indices since our pro-objects consist of homotopes of various algebraic objects.

The category of pro-objects is denoted by \(\Pro(\mathcal C)\). The hom-sets in this category are given by the formula
\begin{equation} \label{eq:pro-c-hom} \Pro(\mathcal C)(X^{(\infty)}, Y^{(\infty)}) = \varprojlim_{j \in \mathcal I_Y} \varinjlim_{i \in \mathcal I_X} \mathcal C(X^{(i)}, Y^{(j)}). \end{equation}
Let us recall a more explicit description of morphisms in \(\Pro(\mathcal C)\). 
By definition, a {\it pre-morphism} \(f \colon X^{(\infty)} \to Y^{(\infty)}\) consists of the following data:
\begin{itemize}
\item a set-theoretic function \(f^* \colon Ob(\mathcal{I_Y}) \to Ob(\mathcal{I_X})\);
\item a collection of morphisms \(f^{(i)} \colon X^{(f^*(i))} \to Y^{(i)}\) in $\mathcal{C}$ parametrized by $i \in Ob(\mathcal{I_Y})$. These morphisms are required to satisfy the following additional assumption: for every morphism \(i \to j\) in \(\mathcal I_Y\) there exists a sufficiently large index \(k \in Ob(\mathcal{I_X})\) such that the composite morphisms \(X^{(k)} \to X^{(f^*(i))} \to Y^{(i)}\) and \(X^{(k)} \to X^{(f^*(j))} \to Y^{(j)} \to Y^{(i)}\) are equal. \end{itemize}
The composition of two pre-morphisms \(f \colon X^{(\infty)} \to Y^{(\infty)}\) and \(g \colon Y^{(\infty)} \to Z^{(\infty)}\) is defined as the pre-morphism \(g \circ f\), where \((g \circ f)^*(i) = f^*(g^*(i))\) and \((g \circ f)^{(i)} = g^{(i)} \circ f^{(g^*(i))}\).

Two parallel pre-morphisms \(f, g \colon X^{(\infty)} \to Y^{(\infty)}\) are called equivalent if for every \(i \in Ob(\mathcal I_Y)\) there exists a sufficiently large index \(j \in Ob(\mathcal I_X)\) such that the composite morphisms \(X^{(j)} \to X^{(f^*(i))} \to Y^{(i)}\) and \(X^{(j)} \to X^{(g^*(i))} \to Y^{(i)}\) are equal. Finally, a {\it morphism} \(X^{(\infty)} \to Y^{(\infty)}\) is an equivalence class of pre-morphisms. Note that the equivalence relation is preserved by the composition operation.

There is a fully faithful functor $\mathcal{C} \to \Pro(\mathcal{C})$ sending \(X \in Ob(\mathcal C)\) to the pro-object $X \colon \mathbf{1}^\op \to \mathcal{C}$. It is clear from~\eqref{eq:pro-c-hom}  that
\begin{align}
 \Pro(\mathcal C)(X, Y) &\cong \mathcal C(X, Y);\\
 \Pro(\mathcal C)(X^{(\infty)}, Y) &\cong \varinjlim_{i \in \mathcal I_X} \mathcal C(X^{(i)}, Y); \label{eq:pro-c-hom-2}\\
 \Pro(\mathcal C)(X, Y^{(\infty)}) &\cong \varprojlim_{i \in \mathcal I_Y} \mathcal C(X, Y^{(i)}).
\end{align}
It is also clear from~\eqref{eq:pro-c-hom} and~\eqref{eq:pro-c-hom-2} that \(X^{(\infty)}\) is the projective limit of \(X^{(i)}\) in the category \(\Pro(\mathcal C)\).

The category of pro-sets \(\Pro(\Set)\) has all finite limits by \cite[Prop.~6.1.18]{SK06} and therefore is a cartesian monoidal category.

Let us describe an explicit construction of limits in one important special case. Let $X^{(\infty)}, Y^{(\infty)} \colon \mathcal{I}^\op \to \Set\) be a pair of pro-sets with the same index category. Recall that the pointwise product of functors $X^{(\infty)} \times Y^{(\infty)} \colon \mathcal{I}^\op \to \Set$ is given by $(X^{(\infty)} \times Y^{(\infty)})^{(i)} = X^{(i)} \times Y^{(i)}$. Clearly, $X^\infty \times Y^\infty$ is a product of $X^{(\infty)}$ and $Y^{(\infty)}$  in the category $\Fun(\mathcal{I}^\op, \Set)$. 

There is an obvious identity-on-objects functor $\Fun(\mathcal{I}^\op, \Set) \to \Pro(\Set)$ sending a natural transformation $\varphi \colon X^{(\infty)} \to Y^{(\infty)}$ to the morphism of pro-sets given by the pre-morphism $\varphi^* = \mathrm{id}_{Ob(\mathcal{I})}$, $\varphi^{(i)} = \varphi_i$. The diagonal morphism $\Delta \colon X^{(\infty)} \to X^{(\infty)} \times X^{(\infty)}$ and the canonical projection morphisms \(\pi_X\colon X^{(\infty)} \times Y^{(\infty)} \to X^{(\infty)}, \pi_Y\colon X^{(\infty)} \times Y^{(\infty)} \to Y^{(\infty)}\) can be defined in the category $\Pro(\Set)$ as the images of the corresponding natural transformations in $\Fun(\mathcal{I}^\op, \Set)$ under this functor.

We claim that the point-wise product $X^{(\infty)} \times Y^{(\infty)}$ satisfies the universal property of products in the category $\Pro(\Set)$. Indeed, this follows from~\eqref{eq:pro-c-hom} and the fact that filtered colimits commute with finite limits. This argument also shows that arbitrary finite limits of pro-sets with the same index category can be computed pointwise.

\subsection{Group and ring objects in pro-sets}
In this paper we consider only commutative rings.
To distinguish between unital and non-unital rings we reserve the word ``ring'' only for unital rings and refer to non-unital rings as {\it rngs}.
We denote the category of rings (resp. rngs) as $\textbf{Ring}$ (resp. $\textbf{Rng}$).

Let $T$ be an algebraic theory. Throughtout this paper we will be mostly interested in the case when $T$ is the theory of groups or the theory of rngs.

Since the category $\Pro(\Set)$ is cartesian monoidal, we may speak of the category $\Mod(T, \Pro(\Set))$ of models of $T$ in \(\Pro(\Set)\).
In the special case when $T$ is the theory of groups (resp. rngs) this category is precisely the category of group (resp. rng) objects in $\Pro(\Set)$.

The category of pro-groups \(\Pro(\Group)\) embeds into \(\Pro(\Set)\). Every pro-group is a group object in \(\Pro(\Set)\). It is easy to see that a morphism \(f \in \Pro(\Set)(G^{(\infty)}, H^{(\infty)})\) between pro-groups comes from \(\Pro(\Group)\) if and only if it is a morphism of group objects. Thus, $\Pro(\Group)$ is a full subcategory of the category of group objects in $\Pro(\Set)$.

\begin{df} \label{df-pro-set-morphisms} 
 Let $n$ be a natural number and $F_{n, T}$ be the free algebra on $n$ generators in the theory $T$.
 If $T$ is the theory of groups, then $F_n$ is the free group $F(t_1,\ldots, t_n)$.
 Similarly, if $T$ is the theory of rngs, $F_{n, T}$ is the sub-rng of the ring $\mathbb{Z}[t_1,\ldots, t_n]$ consisting of polynomials over $\mathbb{Z}$ without free terms.
 
 Given a word $w \in F_{n, T}$ and a $T$-model $X^{(\infty)}$, one can construct the ``evaluation'' morphism in $\Pro(\Set)$
 \[ w^{(\infty)} \colon \underbrace{X^{(\infty)} \times \ldots \times X^{(\infty)}}_{n\text{ times}} \to X^{(\infty)}\]
 as an appropriate composition of diagonal morphisms $\Delta_{X^{(\infty)}}$, projections $\pi_i$ 
 (which are part of the structure of a cartesian monoidal category on $\Pro(\Set)$) 
 and the morphisms defining the structure of a $T$-model on $X^{(\infty)}$.
 
 In order to shorten the notation, to denote the evaluation morphism $w^{(\infty)}$ we often use the original term for $w$,
  in which every symbol of a free variable $t_i$ is replaced by the symbol $t_i^{(\infty)}$.
\end{df}

\begin{example}\label{example-commutator}
Consider the case when $T$ is the theory of groups.
Let $G^{(\infty)}$ be a group object in $\Pro(\Set)$ and $w = [t_1, t_2] = t_1 t_2 t_1^{-1} t_2^{-1} \in F(t_1, t_2)$ be the word representing the generic commutator.
In this case the evaluation morphism $w^{(\infty)}$ (also denoted $[t_1^{(\infty)}, t_2^{(\infty)}]$) can be defined as the composition
 \[ \xymatrix{G^{(\infty)} \times G^{(\infty)} \ar[r]^(.35){\Delta \times \Delta} & G^{(\infty)} \times G^{(\infty)} \times G^{(\infty)} \times G^{(\infty)} \ar[d]_{\pi_1 \times \pi_3 \times (i\pi_2) \times (i\pi_4)} & \\
    & G^{(\infty)} \times G^{(\infty)} \times G^{(\infty)} \times G^{(\infty)} \ar[r]^(.76){m(m\times m)} & G^{(\infty)},} \]
where the structure of a group object on $G^{(\infty)}$ is given by the triple $(m, i, e)$.
\end{example}

\subsection{Homotopes of rings and pro-rings}
Let $R$ be an arbitrary ring and let \(S\) be some fixed multiplicative subset of $R$ containing a unit. Denote by $\mathcal{S}$ the category, whose objects are the elements of \(S\) and whose morphisms \(\mathcal{S}(s, s')\) are all \(s'' \in S\) such that \(ss'' = s'\). The composition and the identity morphisms are induced by the ring structure on $R$. It is clear that $\mathcal{S}$ is a filtered category. Unless stated otherwise, all the pro-sets that we encounter in the sequel have \(\mathcal S\) as their category of indices.

We introduce the notion of a {\it homotope} of a ring inspired by a similar notion from nonassociative algebra (cf. McCrimmon?). 
\begin{df}
 Let $s$ be an element of $R$.  
 By definition, the {\it \(s\)-homotope} of \(R\) is the rng \(R^{(s)} = \{a^{(s)} \mid a \in R\}\) with the operations of addition and multiplication given by
 \[ a^{(s)} + b^{(s)} = (a + b)^{(s)},\ \ a^{(s)} b^{(s)} = (asb)^{(s)},\ a, b\in R.\]
 Clearly, $R^{(s)}$ has the structure of an \(R\)-algebra given by the formula \[a \cdot b^{(s)} = (ab)^{(s)},\ a, b \in R.\] For \(s, s' \in S\) there is a homomorphism of \(R\)-algebras \[R^{(ss')} \to R^{(s')}, a^{(ss')} \mapsto (as)^{(s')}.\]
 Denote by \(R^{(\infty)}\) the formal projective limit of the projective system \(R^{(s)}\), where \(s \in Ob(\mathcal S)\).
 Thus, $R^{(\infty)}$ is an object of the category $\Pro(\Rng)$. 
 It can also be considered as an rng object in \(\Pro(\Set)\).  
\end{df}

\begin{rem}\label{rem:prorings-comment}
 Denote by $sR$ the principal ideal of $R$ generated by $s$. 
 There is always a homomorphism $R^{(s)} \to sR$ given by $r^{(s)}\mapsto sr$.
 In the case when $R$ is an integral domain this homomorphism is easily seen to be an rng isomorphism.
 
 The reason why we use homotopes rather than principal ideals in the definition of $R^{(\infty)}$ is that
 there is always a ``division by $s$'' homomorphism of $R$-algebras $R^{(ss')} \to R^{(s')}$ given by $r^{(ss')} \mapsto r^{(s')},$
 while the similar map $ss'R \to s'R$ may not exist if $R$ does not happen to be a domain.
 The existence of this division homomorphism will be important in the sequel.
 
 Notice also that the projective limit of $R^{(s)}$ in $\Rng$ is often trivial.
 Indeed, if $R$ is a domain, then the limit of $R^{(s)}$ in $\Rng$ computes the intersection of the principal ideals $\bigcap_{s\in S} sR$, which often coincides with the zero ring. 
 This also shows that the set of {\it global elements} of $R^{(\infty)}$ (i.\,e. the hom-set in $\Pro(\Set)$ from the terminal object $1$ to $R^{(\infty)}$) is often trivial and therefore will be of little interest to us.
\end{rem}

Recall that a morphism \(f \in \mathcal C(X, Y)\) is called a split epimorphism (or a retraction) if it admits a section,
 i.\,e. there exists $g \in \mathcal{C}(Y, X)$ such that $fg = \id_{Y}$. Retractions are preserved under pullbacks.

\begin{lemma}\label{RingGeneration}
The rng multiplication morphism $m \colon R^{(\infty)} \times R^{(\infty)} \to R^{(\infty)}$ is a split epimorphism of pro-sets.
\end{lemma}
\begin{proof}
Consider the following pre-morphism of pro-sets:
\[u \colon R^{(\infty)} \to R^{(\infty)} \times R^{(\infty)}, \enskip u^*(s) = s^2, \enskip u^{(s)} \colon c^{(s^2)} \mapsto 1^{(s)} \otimes c^{(s)}.\]
Clearly, this is indeed a pre-morphism and
\[m^{(s)}\bigl(u^{(s)}\bigl(c^{(s^2)}\bigr)\bigr) = 1^{(s)} c^{(s)} = (sc)^{(s)},\]
which shows that $mu = \mathrm{id}_{R^{(\infty)}}$.
\end{proof}

Before we proceed further let us introduce the following conventions, 
 which will help us to keep our notations short:
 \begin{itemize}
  \item any algebraic expression containing the symbol $(\infty)$ in the upper index of a variable (e.\,g. $a^{(\infty)}$, $b^{(\infty)}$ etc.) denotes a certain morphism of pro-sets, whose domain is a certain product of pro-sets (often a power of a single pro-set), such morphisms are typically obtained from~\cref{df-pro-set-morphisms};
  \item the number of factors in this product coincides with the number of different variables occuring in the expression;
  \item in order to specify the domain of a variable $a^{(\infty)}$ occuring in the expression, we use the notation $a^{(\infty)} \in X^{(\infty)}$ (this notation is not confusing since we never refer to the actual elements of pro-sets, recall that the elements of our pro-sets are usually uninteresting, cf.~\cref{rem:prorings-comment});
  \item the notation for the multiplication operation is always suppressed, i.\,e. we write $a^{(\infty)} b^{(\infty)}$ instead of $m(a^{(\infty)}, b^{(\infty)})$ or $a^{(\infty)} \cdot b^{(\infty)}$;
  \item the syntax of tuples is used to denote the product of morphisms (e.\,g. if $f,g \colon X^{(\infty)} \to Y^{(\infty)}$ are morphisms of pro-sets, then we use the notation $(f(x_1^{(\infty)}), g(x_2^{(\infty)}))$ instead of $f\times g$);
  \item if $g$ is a morphism of pro-sets with $X^{(\infty)} \times Y^{\infty}$ as its domain, then we always write $g(a^{(\infty)}, b^{(\infty)})$ instead of $g((a^{(\infty)}, b^{(\infty)}))$;
  \item to denote the operation of composition of morphisms we use the syntax of substituted expressions.
 \end{itemize}
Now we are ready to formulate our next result.
\begin{lemma}\label{RingPresentation}
Let \(G^{(\infty)}\) be a pro-group, \(g \colon R^{(\infty)} \times R^{(\infty)} \to G^{(\infty)}\) be a morphism of pro-sets. There is a morphism \(f \colon R^{(\infty)} \to G^{(\infty)}\) of pro-groups such that
\[g\bigl(a^{(\infty)} , b^{(\infty)}\bigr) = f\bigl(a^{(\infty)} b^{(\infty)}\bigr)\]
if and only if \(g\) satisfies the following identities in $\Pro(\Set)$:
\begin{itemize}
\item \(\bigl[g\bigl(a_1^{(\infty)}, b_1^{(\infty)}\bigr), g\bigl(a_2^{(\infty)}, b_2^{(\infty)}\bigr)\bigr] = 1\);
\item \(g\bigl(\bigl(a_1^{(\infty)} + a_2^{(\infty)}\bigr), b^{(\infty)}\bigr) = g\bigl(a_1^{(\infty)}, b^{(\infty)}\bigr)\, g\bigl(a_2^{(\infty)}, b^{(\infty)}\bigr)\);
\item \(g\bigl(a^{(\infty)}, \bigl(b_1^{(\infty)} + b_2^{(\infty)}\bigr)\bigr) = g\bigl(a^{(\infty)}, b_1^{(\infty)}\bigr)\, g\bigl(a^{(\infty)}, b_2^{(\infty)}\bigr)\);
\item \(g\bigl(a^{(\infty)} b^{(\infty)}, c^{(\infty)}\bigr) = g\bigl(a^{(\infty)}, b^{(\infty)} c^{(\infty)}\bigr)\).
\end{itemize}
\end{lemma}
\begin{proof}
The necessity of the identities is clear.
By~\cref{RingGeneration} the morphism \(f\) is unique, so it suffices to show that it exists.
It suffices to consider the case when \(G\) is a group. 
Let \(g\) be a morphism satisfying the above identities.
By definition, there exists $s\in S$ such that $g$ is given by a map \(g' \colon R^{(s)} \times R^{(s)} \to G\) satisfying the first three identities and the identity 
\[g'\bigl((asb)^{(s)}, c^{(s)}\bigr) = g'\bigl(a^{(s)}, (bsc)^{(s)}\bigr).\]
Consider the map \(f' \colon R^{(s^2)} \to G\) given by
\[f'\bigl(c^{(s^2)}\bigr) = g'\bigl(1^{(s)}, c^{(s)}\bigr),\]
it is a homomorphism by the first and the third identities.
From the last two identities we conclude that for all \(a, b \in R\) one has
\begin{align*}
f'\bigl(a^{(s^2)} b^{(s^2)})
&= g' \bigl( 1^{(s)}_p, (s^2 ab)^{(s)} \bigr) =\\
&= g' \bigl( (sa)^{(s)}, (sb)^{(s)} \bigr) =\\
&= g' \bigl(a^{(s^2)}, b^{(s^2)}\bigr).
\end{align*}
It is clear that \(f'\) gives the required morphism \(f\) of pro-groups.
\end{proof}



\section{Homotopes of Steinberg groups}
In this section we construct a certain pro-group \(\stlin(R)^{(\infty)}\) and prove its basic properties. The ``homotopes'' of \(\stlin(R)\) are the groups \(\stlin(R)^{(s)}\) parameterized by \(s \in S\). They are generated by symbols \(x_{ij}^{(s)}(a)\) for \(i \neq j\), \(a \in R_{ij}\) with the modified Steinberg relations
\begin{enumerate}
\item \(x_{ij}^{(s)}(a)\, x_{ij}^{(s)}(b) = x_{ij}^{(s)}(a + b)\);
\item \([x_{ij}^{(s)}(a), x_{kl}^{(s)}(b)] = 1\) for \(j \neq k\) and \(i \neq l\);
\item \([x_{ij}^{(s)}(a), x_{jk}^{(s)}(b)] = x_{ik}^{(s)}(asb)\) for \(i \neq k\).
\end{enumerate}
The structure homomorphisms are given by \(x^{(ss')}(a) \mapsto x^{(s)}(s'a)\), and a pro-group \(\stlin(R)^{(\infty)}\) is the formal projective limit of \(\stlin(R)^{(s)}\) for \(s \in \mathcal S\). There are pre-morphisms \(x_{ij} \colon R_{ij}^{(\infty)} \to \stlin(R)^{(\infty)}\) of pro-groups with \(x_{ij}^*(s) = s\) and \(x_{ij}^{(s)}\bigl(a^{(s)}\bigr) = x_{ij}^{(s)}(a)\). There is also a pre-morphism \(\mathrm{st} \colon \stlin(R)^{(\infty)} \to \glin(R)^{(\infty)}\) of pro-groups, where \(\glin(R)^{(s)} = \bigl(R^{(s)}\bigr)^\circ\) (so \(\glin(R)^{(1)}\) is only isomorphic to \(\glin(R)\)), \(\stmap^*(s) = s\), \(\stmap^{(s)}\bigl(x_{ij}^{(s)}(a)\bigr) = a^{(s)}\). It follows that all maps \(x_{ij}^{(s)} \colon R_{ij} \to \stlin(R)^{(s)}\) are injective.

\begin{lemma}\label{SteinbergPresentation}
Let \(G^{(\infty)}\) be a pro-group. Then every morphism \(f \colon \stlin(R)^{(\infty)} \to G^{(\infty)}\) of pro-groups is uniquely determined by its compositions with all \(x_{ij} \colon R_{ij}^{(\infty)} \to \stlin(R)^{(\infty)}\). Morphisms \(g_{ij} \colon R_{ij}^{(\infty)} \to G^{(\infty)}\) of pro-groups for \(i \neq j\) are obtained in this way if and only if they satisfy (St1)--(St3) in \(\Pro(\Set)\).
\end{lemma}
\begin{proof}
This follows directly from the definitions. Here we need that \(G^{(\infty)}\) is actually a pro-group, not just a group object in \(\Pro(\Set)\), and there is only a finite number of ``generators'' \(x_{ij}\).
\end{proof}

\printbibliography
\end{document}
