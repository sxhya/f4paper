\documentclass{article}

\usepackage[utf8]{inputenc}
\usepackage[T1]{fontenc}
\usepackage{graphicx}
\usepackage{amsmath}
\usepackage{amssymb}
\usepackage{amsthm}
\usepackage{amscd}
\usepackage{alltt}
\usepackage{xcolor}

\usepackage[backend=biber, bibencoding=utf8, giveninits=true, citestyle=numeric-comp, sortlocale=en_US, url=false, doi=false, eprint=true, maxbibnames=4]{biblatex}
\addbibresource{f4paper.bib}
\renewbibmacro*{volume+number+eid}{\ifentrytype{article}{\- \iffieldundef{volume}{}{Vol.~\printfield{volume},}\iffieldundef{number}{}{ No.~\printfield{number},}}}
\renewbibmacro{in:}{\ifentrytype{article}{}{\printtext{\bibstring{in}\intitlepunct}}}
\newbibmacro{string+doi}[1]{\iffieldundef{doi}{\iffieldundef{url}{#1}{\href{\thefield{url}}{#1}}}{\href{https://dx.doi.org/\thefield{doi}}{#1}}}
\DeclareFieldFormat[article, inproceedings, inbook, book, online]{title}{\usebibmacro{string+doi}{\mkbibquote{#1}}}
\renewcommand*{\bibfont}{\footnotesize}

\newtheorem{lemma}{Lemma}
\newtheorem{prop}{Proposition}
\newtheorem{theorem}{Theorem}
\newtheorem*{theorem*}{Theorem}

\newcommand{\rar}{\rightarrow}
\newcommand{\lar}{\leftarrow}

\DeclareMathOperator\mat{M}
\DeclareMathOperator\elin{E}
\DeclareMathOperator\stlin{St}
\DeclareMathOperator\gstlin{GSt}
\DeclareMathOperator\klin{K}
\DeclareMathOperator\stmap{st}
\DeclareMathOperator\glin{GL}
\DeclareMathOperator\diag{D}
\DeclareMathOperator\Jac{J}
\newcommand{\leqt}{\trianglelefteq}
\DeclareMathOperator{\Ad}{Ad}

\newcommand{\sub}[2]{{_{#1\!}{#2}}}
\newcommand{\up}[2]{{^{#1}\!{#2}}}

\newcommand{\Set}{\mathbf{Set}}
\newcommand{\Group}{\mathbf{Grp}}
\newcommand{\Fun}{\mathbf{Fun}}
\newcommand{\op}{\mathrm{op}}
\DeclareMathOperator{\Pro}{Pro}

\begin{document}



\section{Pro-objects}
Let \(\mathcal C\) be an arbitrary category.
In this section we recall the construction of the pro-completion of \(\mathcal C\) (cf. \cite[Section~6.1]{SK06}).

Recall that a nonempty small category \(\mathcal I\) is called {\it filtered} if
\begin{itemize}
 \item for any two objects \(i, k \in Ob(\mathcal{I})\) there is a diagram \(i \rar j \lar k\) in \(\mathcal I\);
 \item every two parallel morphisms \(i \rightrightarrows j\) are equalized by some morphism \(j \rar k\) in \(\mathcal I\).
\end{itemize}
A {\it pro-object} in \(\mathcal C\) is, by definition, a functor $X^{(\infty)}\colon \mathcal{I}_X^{\op} \to \mathcal{C}$, i.\,e. a contravariant functor from a filtered category \(\mathcal I_X\), called {\it the category of indices of $X^{(\infty)}$}, to the category \(\mathcal C\). %The objects of \(\mathcal I_X\) are called {\it the indices of \(X^{(\infty)}\)}.

We denote by \(X^{(i)}\) the value of \(X^{(\infty)}\) on an index \(i\) and suppress the notation for the value of \(X^{(\infty)}\) on a morphism \(i \rar j\) if it is clear from the context.
%(for example, if \(j\) is a sufficiently large index). 
%Thus, a pro-object \(X^{(\infty)}\) is the formal projective limit of \(X^{(i)}\). 
%We use the notation with upper indices since our pro-objects consist of homotopes of various algebraic objects.

The category of pro-objects is denoted by \(\Pro(\mathcal C)\). The hom-sets in this category are given by the formula
\begin{equation} \label{eq:pro-c-hom} \Pro(\mathcal C)(X^{(\infty)}, Y^{(\infty)}) = \varprojlim_{j \in \mathcal I_Y} \varinjlim_{i \in \mathcal I_X} \mathcal C(X^{(i)}, Y^{(j)}). \end{equation}
Let us recall a more explicit description of morphisms in \(\Pro(\mathcal C)\). 
By definition, a {\it pre-morphism} \(f \colon X^{(\infty)} \rar Y^{(\infty)}\) consists of the following data:
\begin{itemize}
\item a set-theoretic function \(f^* \colon Ob(\mathcal{I_Y}) \to Ob(\mathcal{I_X})\);
\item a collection of morphisms \(f^{(i)} \colon X^{(f^*(i))} \rar Y^{(i)}\) in $\mathcal{C}$ parametrized by $i \in Ob(\mathcal{I_Y})$. These morphisms are required to satisfy the following additional assumption: for every morphism \(i \rar j\) in \(\mathcal I_Y\) there exists a sufficiently large index \(k \in Ob(\mathcal{I_X})\) such that the composite morphisms \(X^{(k)} \rar X^{(f^*(i))} \rar Y^{(i)}\) and \(X^{(k)} \rar X^{(f^*(j))} \rar Y^{(j)} \rar Y^{(i)}\) are equal. \end{itemize}
The composition of two pre-morphisms \(f \colon X^{(\infty)} \rar Y^{(\infty)}\) and \(g \colon Y^{(\infty)} \rar Z^{(\infty)}\) is defined as the pre-morphism \(g \circ f\), where \((g \circ f)^*(i) = f^*(g^*(i))\) and \((g \circ f)^{(i)} = g^{(i)} \circ f^{(g^*(i))}\).

Two parallel pre-morphisms \(f, g \colon X^{(\infty)} \rar Y^{(\infty)}\) are called equivalent if for every \(i \in Ob(\mathcal I_Y)\) there exists a sufficiently large index \(j \in Ob(\mathcal I_X)\) such that the composite morphisms \(X^{(j)} \rar X^{(f^*(i))} \rar Y^{(i)}\) and \(X^{(j)} \rar X^{(g^*(i))} \rar Y^{(i)}\) are equal. Finally, a {\it morphism} \(X^{(\infty)} \rar Y^{(\infty)}\) is an equivalence class of pre-morphisms. Note that the equivalence relation is preserved by the composition operation.

There is a fully faithful functor $\mathcal{C} \to \Pro(\mathcal{C})$ sending \(X \in Ob(\mathcal C)\) to the pro-object $X \colon \mathbf{1}^\op \to \mathcal{C}$. It is clear from~\eqref{eq:pro-c-hom}  that
\begin{align}
 \Pro(\mathcal C)(X, Y) &\cong \mathcal C(X, Y);\\
 \Pro(\mathcal C)(X^{(\infty)}, Y) &\cong \varinjlim_{i \in \mathcal I_X} \mathcal C(X^{(i)}, Y); \label{eq:pro-c-hom-2}\\
 \Pro(\mathcal C)(X, Y^{(\infty)}) &\cong \varprojlim_{i \in \mathcal I_Y} \mathcal C(X, Y^{(i)}).
\end{align}
It is also clear from~\eqref{eq:pro-c-hom} and~\eqref{eq:pro-c-hom-2} that \(X^{(\infty)}\) is the projective limit of \(X^{(i)}\) in the category \(\Pro(\mathcal C)\).

The category of pro-sets \(\Pro(\Set)\) has all finite limits by \cite[Prop.~6.1.18]{SK06}. Let us describe an explicit construction of limits in one important special case. Let $X^{(\infty)}, Y^{(\infty)} \colon \mathcal{I}^\op \to \Set\) be a pair of pro-sets with the same index category. Recall that the pointwise product of functors $X^{(\infty)} \times Y^{(\infty)} \colon \mathcal{I}^\op \to \Set$ is given by $(X^{(\infty)} \times Y^{(\infty)})^{(i)} = X^{(i)} \times Y^{(i)}$. Clearly, $X^\infty \times Y^\infty$ is a product of $X^{(\infty)}$ and $Y^{(\infty)}$  in the category $\Fun(\mathcal{I}^\op, \Set)$. 

There is an obvious identity-on-objects functor $\Fun(\mathcal{I}^\op, \Set) \to \Pro(\Set)$ sending a natural transformation $\varphi \colon X^{(\infty)} \to Y^{(\infty)}$ to the morphism of pro-sets given by the pre-morphism $\varphi^* = \mathrm{id}_{Ob(\mathcal{I})}$, $\varphi^{(i)} = \varphi_i$. The diagonal morphism $\Delta \colon X^{(\infty)} \to X^{(\infty)} \times X^{(\infty)}$ and the canonical projection morphisms \(\pi_X\colon X^{(\infty)} \times Y^{(\infty)} \to X^{(\infty)}, \pi_Y\colon X^{(\infty)} \times Y^{(\infty)} \to Y^{(\infty)}\) can be defined in the category $\Pro(\Set)$ as the images of the corresponding natural transformations in $\Fun(\mathcal{I}^\op, \Set)$ under this functor.

We claim that the point-wise product $X^{(\infty)} \times Y^{(\infty)}$ satisfies the universal property of products in the category $\Pro(\Set)$. Indeed, this follows from~\eqref{eq:pro-c-hom} and the fact that filtered colimits commute with finite limits. This argument also shows that arbitrary finite limits of pro-sets with the same index category can be computed pointwise.

Since the category $\Pro(\Set)$ admits finite products, we may speak of algebraic objects in \(\Pro(\Set)\) (e.\,g. group and ring objects).
Any algebraic formula (say, a commutator or a polynomial with integer coefficients) defines a morphism in \(\Pro(\Set)\) from a product of copies of an algebraic object to this algebraic object. 
If \(a^{(\infty)}\) is a variable in such a formula, then the notation \(a^{(\infty)} \in X^{(\infty)}\) means that \(X^{(\infty)}\) is the domain of \(a^{(\infty)}\).

The category of pro-groups \(\Pro(\Group)\) embeds into \(\Pro(\Set)\). Every pro-group is a group object in \(\Pro(\Set)\). It is easy to see that a morphism \(f \in \Pro(\Set)(G^{(\infty)}, H^{(\infty)})\) between pro-groups comes from \(\Pro(\Group)\) if and only if it is a morphism of group objects.

Now let us return to the Steinberg groups. From now on fix a multiplicative subset \(S \subseteq K\). We construct a filtered category \(\mathcal S\). Its objects are the elements of \(S\), its morphisms \(s \rar s'\) are all \(s'' \in S\) such that \(ss'' = s'\), composition and the identity morphisms are obvious. By default, our pro-sets have \(\mathcal S\) as the category of indices.

Recall the definition of homotopes from non-associative algebra. The \(s\)-homotope of \(K\) is the non-unital ring \(K^{(s)} = \{a^{(s)} \mid a \in K\}\) with the addition \(a^{(s)} + b^{(s)} = (a + b)^{(s)}\) and the multiplication \(a^{(s)} b^{(s)} = (asb)^{(s)}\) for \(a, b \in K\). It is even a \(K\)-algebra with \(ab^{(s)} = (ab)^{(s)}\) for \(a, b \in K\). If \(s, s' \in S\), then there is a homomorphism of \(K\)-algebras \(K^{(ss')} \rar K^{(s')}, a^{(ss')} \mapsto (as)^{(s')}\).

Let \(K^{(\infty)}\) be the formal projective limit of \(K^{(s)}\) for \(s \in \mathcal S\), it is a non-unital ring object in \(\Pro(\Set)\). It is convenient to sometimes write the elements of \(K^{(s)} \times K^{(s)}\) as \(a^{(s)} \otimes b^{(s)}\) and the variables with the domain \(K^{(\infty)} \times K^{(\infty)}\) as \(a^{(\infty)} \otimes b^{(\infty)}\), so the multiplication morphism is \(K^{(\infty)} \times K^{(\infty)} \rar K^{(\infty)}, a^{(\infty)} \otimes b^{(\infty)} \mapsto a^{(\infty)} b^{(\infty)}\) (this is actually a notation of the morphism, not the definition).

Recall that a morphism \(f \in \mathcal C(X, Y)\) is a split epimorphism if it admits a section. Split epimorphisms are preserved under pullbacks.

\begin{lemma}\label{RingGeneration}
The multiplication morphism
\[m \colon K^{(\infty)} \times K^{(\infty)} \rar K^{(\infty)}, \enskip a^{(\infty)} \otimes b^{(\infty)} \mapsto a^{(\infty)} b^{(\infty)}\]
is a split epimorphism of pro-sets.
\end{lemma}
\begin{proof}
Consider a pre-morphism of pro-sets
\[u \colon K^{(\infty)} \rar K^{(\infty)} \times K^{(\infty)}, \enskip u^*(s) = s^2, \enskip u^{(s)} \colon c^{(s^2)} \mapsto 1^{(s)} \otimes c^{(s)}.\]
Clearly, this is indeed a pre-morphism and
\[
m^{(s)}\bigl(u^{(s)}\bigl(c^{(s^2)}\bigr)\bigr) = 1^{(s)} c^{(s)} = (sc)^{(s)}.\qedhere
\]
\end{proof}

\begin{lemma}\label{RingPresentation}
Let \(G^{(\infty)}\) be a pro-group, \(g \colon K^{(\infty)} \times K^{(\infty)} \rar G^{(\infty)}\) be a morphism of pro-sets. There is a morphism \(f \colon K^{(\infty)} \rar G^{(\infty)}\) of pro-groups such that
\[g\bigl(a^{(\infty)} \otimes b^{(\infty)}\bigr) = f\bigl(a^{(\infty)} b^{(\infty)}\bigr)\]
if and only if \(g\) satisfies the identities
\begin{itemize}
\item \(\bigl[g\bigl(a_1^{(\infty)} \otimes b_1^{(\infty)}\bigr), g\bigl(a_2^{(\infty)} \otimes b_2^{(\infty)}\bigr)\bigr] = 1\);
\item \(g\bigl(\bigl(a_1^{(\infty)} + a_2^{(\infty)}\bigr) \otimes b^{(\infty)}\bigr) = g\bigl(a_1^{(\infty)} \otimes b^{(\infty)}\bigr)\, g\bigl(a_2^{(\infty)} \otimes b^{(\infty)}\bigr)\);
\item \(g\bigl(a^{(\infty)} \otimes \bigl(b_1^{(\infty)} + b_2^{(\infty)}\bigr)\bigr) = g\bigl(a^{(\infty)} \otimes b_1^{(\infty)}\bigr)\, g\bigl(a^{(\infty)} \otimes b_2^{(\infty)}\bigr)\);
\item \(g\bigl(a^{(\infty)} b^{(\infty)} \otimes c^{(\infty)}\bigr) = g\bigl(a^{(\infty)} \otimes b^{(\infty)} c^{(\infty)}\bigr)\).
\end{itemize}
\end{lemma}
\begin{proof}
The necessity of the identities is clear. Since such \(f\) is unique by lemma \ref{RingGeneration}, it suffices to consider a group \(G\) instead of a pro-group. Take a morphism \(g\) satisfying the identities. It is given by a map \(g' \colon K^{(s)} \times K^{(s)} \rar G\) for sufficiently large \(s \in S\) satisfying the first three identities and the identity 
\[g'\bigl((asb)^{(s)} \otimes c^{(s)}\bigr) = g'\bigl(a^{(s)} \otimes (bsc)^{(s)}\bigr).\]
Consider the map \(f' \colon K^{(s^2)} \rar G\) given by
\[f'\bigl(c^{(s^2)}\bigr) = g'\bigl(1^{(s)} \otimes c^{(s)}\bigr),\]
it is a homomorphism by the first and the third identites. For all \(a, b \in K\) we have
\begin{align*}
f'\bigl(a^{(s^2)} b^{(s^2)})
&= g' \bigl( 1^{(s)}_p \otimes (s^2 ab)^{(s)} \bigr) =\\
&\quad g' \bigl( (sa)^{(s)} \otimes (sb)^{(s)} \bigr) =\\
&\quad g' \bigl(a^{(s^2)} \otimes b^{(s^2)}\bigr)
\end{align*}
by the second and the last identities. It is clear that \(f'\) gives the required morphism \(f\) of pro-groups.
\end{proof}

\printbibliography
\end{document}
