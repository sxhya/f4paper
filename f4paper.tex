\documentclass{article}

\usepackage[utf8]{inputenc}
\usepackage[T1]{fontenc}
\usepackage{graphicx}
\usepackage{amsmath}
\usepackage{amssymb}
\usepackage{amsthm}
\usepackage{amscd}
\usepackage{alltt}

\newtheorem{lemma}{Lemma}
\newtheorem{prop}{Proposition}
\newtheorem{theorem}{Theorem}
\newtheorem*{theorem*}{Theorem}

\newcommand{\rar}{\rightarrow}
\newcommand{\lar}{\leftarrow}

\DeclareMathOperator\mat{M}
\DeclareMathOperator\elin{E}
\DeclareMathOperator\stlin{St}
\DeclareMathOperator\gstlin{GSt}
\DeclareMathOperator\klin{K}
\DeclareMathOperator\stmap{st}
\DeclareMathOperator\glin{GL}
\DeclareMathOperator\diag{D}
\DeclareMathOperator\Jac{J}
\newcommand{\leqt}{\trianglelefteq}
\DeclareMathOperator{\Ad}{Ad}

\newcommand{\sub}[2]{{_{#1\!}{#2}}}
\newcommand{\up}[2]{{^{#1}\!{#2}}}

\newcommand{\Set}{\mathbf{Set}}
\newcommand{\Group}{\mathbf{Grp}}
\DeclareMathOperator{\Pro}{Pro}

\begin{document}



\section{Pro-objects}

Recall a construction of the pro-completion of a given category \(\mathcal C\) from \cite{}, section 6.1. A small category \(\mathcal I\) is called filtered if it is non-empty, for any two objects \(i, j\) there is a diagram \(i \rar j \lar k\) in \(\mathcal I\), and every pair of parallel morphisms \(i \rightrightarrows j\) is equalized by some morphism \(j \rar k\) in \(\mathcal I\). A pro-object in \(\mathcal C\) is a contravariant functor \(X^{(\infty)}\) from a filtered category \(\mathcal I_X\) to \(\mathcal C\). Objects of \(\mathcal I_X\) are called indices of \(X^{(\infty)}\). We write \(X^{(i)}\) for the values of \(X^{(\infty)}\) on indices \(i\) and omit the values of \(X^{(\infty)}\) on morphisms \(i \rar j\) in our formulas if the morphism \(i \rar j\) is clear from the context (for example, if \(j\) is a sufficiently large index). So a pro-object \(X^{(\infty)}\) is the formal projective limit of \(X^{(i)}\). We use the notation with upper indices since our pro-objects consist of homotopes of various algebraic objects.

The category of pro-objects is denoted by \(\Pro(\mathcal C)\). By definition,
\[
\Pro(\mathcal C)(X^{(\infty)}, Y^{(\infty)}) = \varprojlim_{j \in \mathcal I_Y} \varinjlim_{i \in \mathcal I_X} \mathcal C(X^{(i)}, Y^{(j)}).
\]
There is a more explicit description of morphisms in \(\Pro(\mathcal C)\). We say that a pre-morphism \(f \colon X^{(\infty)} \rar Y^{(\infty)}\) consists of a function \(f^*\) from the set of indices of \(Y\) to the set of indices of \(X\), and of morphisms \(f^{(i)} \colon X^{(f^*(i))} \rar Y^{(i)}\) for all \(i \in \mathcal I_Y\) such that for any morphism \(i \rar j\) in \(\mathcal I_Y\) there exists a sufficiently large index \(k \in \mathcal I_X\) making the composition \(X^{(k)} \rar X^{(f^*(i))} \rar Y^{(i)}\) equal to \(X^{(k)} \rar X^{(f^*(j))} \rar Y^{(j)} \rar Y^{(i)}\) (here \(f^*\) is not a functor between index categories). A composition of pre-morphisms \(f \colon X^{(\infty)} \rar Y^{(\infty)}\) and \(g \colon Y^{(\infty)} \rar Z^{(\infty)}\) is the pre-morphism \(g \circ f\), where \((g \circ f)^*(i) = f^*(g^*(i))\) and \((g \circ f)^{(i)} = g^{(i)} \circ f^{(g^*(i))}\). Two parallel pre-morphisms \(f, g \colon X^{(\infty)} \rar Y^{(\infty)}\) are called equivalent if for every \(i \in \mathcal I_Y\) there exists a sufficiently large index \(j \in \mathcal I_X\) making the composition \(X^{(j)} \rar X^{(f^*(i))} \rar Y^{(i)}\) equal to \(X^{(j)} \rar X^{(g^*(i))} \rar Y^{(i)}\). Finally, a morphism \(X^{(\infty)} \rar Y^{(\infty)}\) is an equivalence class of pre-morphisms. Note that equivalence is preserved under composition.

The category \(\mathcal C\) embeds into \(\Pro(\mathcal C)\) (i.e. there is a fully faithful functor between them), because we may consider every object from \(\mathcal C\) as a pro-object with a single index and a single morphism between indices. So \(\Pro(\mathcal C)(X, Y) \cong \mathcal C(X, Y)\), \(\Pro(\mathcal C)(X^{(\infty)}, Y) \cong \varinjlim_{i \in \mathcal I_X} \mathcal C(X^{(i)}, Y)\), and \(\Pro(\mathcal C)(X, Y^{(\infty)}) \cong \varprojlim_{i \in \mathcal I_Y} \mathcal C(X, Y^{(i)})\). Also \(X^{(\infty)}\) is the projective limit of \(X^{(i)}\) in the category \(\Pro(\mathcal C)\).

The category of pro-sets \(\Pro(\Set)\) has all finite limits by \cite{}, proposition 6.1.18. If \(X^{(\infty)}\) and \(Y^{(\infty)}\) are pro-sets with the same index category \(\mathcal I\), then we may construct their product \(Z^{(\infty)}\) as follows. The index category of \(Z^{(\infty)}\) is \(\mathcal I\), \(Z^{(i)} = X^{(i)} \times Y^{(i)}\), the projection \(Z^{(\infty)} \rar X^{(\infty)}\) is given by the pre-morphism \(\pi_X^*(i) = i\), \(\pi_X^{(i)}(x^{(i)}, y^{(i)}) = x^{(i)}\) for \(x^{(i)} \in X^{(i)}\) and \(y^{(i)} \in Y^{(i)}\), and similarly for the projection \(Z^{(\infty)} \rar Y^{(\infty)}\). Also the diagonal morphism \(X^{(\infty)} \rar X^{(\infty)} \times X^{(\infty)}\) is given by the pre-morphism \(\Delta^*(i) = i\), \(\Delta^{(i)}(x^{(i)}) = (x^{(i)}, x^{(i)})\) for \(x^{(i)} \in X^{(i)}\).

Hence we may consider algebraic objects in \(\Pro(\Set)\) such as rings and groups. Any algebraic formula (say, the commutator or a polynomial with integer coefficients) defines a morphism in \(\Pro(\Set)\) from a product of algebraic objects to an algebraic object. If \(a^{(\infty)}\) is a variable in such a formula, then \(a^{(\infty)} \in X^{(\infty)}\) means that \(X^{(\infty)}\) is the domain of \(a^{(\infty)}\).

The category of pro-groups \(\Pro(\Group)\) embeds into \(\Pro(\Set)\). Every pro-group is a group object in \(\Pro(\Set)\). It is easy to see that a morphism \(f \in \Pro(\Set)(G^{(\infty)}, H^{(\infty)})\) between pro-groups comes from \(\Pro(\Group)\) if and only if it is a morphism of group objects.

Now let us return to the Steinberg groups. From now on fix a multiplicative subset \(S \subseteq K\). We construct a filtered category \(\mathcal S\). Its objects are the elements of \(S\), its morphisms \(s \rar s'\) are all \(s'' \in S\) such that \(ss'' = s'\), composition and the identity morphisms are obvious. By default, our pro-sets have \(\mathcal S\) as the category of indices.

Recall the definition of homotopes from non-associative algebra. The \(s\)-homotope of \(K\) is the non-unital ring \(K^{(s)} = \{a^{(s)} \mid a \in K\}\) with the addition \(a^{(s)} + b^{(s)} = (a + b)^{(s)}\) and the multiplication \(a^{(s)} b^{(s)} = (asb)^{(s)}\) for \(a, b \in K\). It is even a \(K\)-algebra with \(ab^{(s)} = (ab)^{(s)}\) for \(a, b \in K\). If \(s, s' \in S\), then there is a homomorphism of \(K\)-algebras \(K^{(ss')} \rar K^{(s')}, a^{(ss')} \mapsto (as)^{(s')}\).

Let \(K^{(\infty)}\) be the formal projective limit of \(K^{(s)}\) for \(s \in \mathcal S\), it is a non-unital ring object in \(\Pro(\Set)\). It is convenient to sometimes write the elements of \(K^{(s)} \times K^{(s)}\) as \(a^{(s)} \otimes b^{(s)}\) and the variables with the domain \(K^{(\infty)} \times K^{(\infty)}\) as \(a^{(\infty)} \otimes b^{(\infty)}\), so the multiplication morphism is \(K^{(\infty)} \times K^{(\infty)} \rar K^{(\infty)}, a^{(\infty)} \otimes b^{(\infty)} \mapsto a^{(\infty)} b^{(\infty)}\) (this is actually a notation of the morphism, not the definition).

Recall that a morphism \(f \in \mathcal C(X, Y)\) is a split epimorphism if it admits a section. Split epimorphisms are preserved under pullbacks.

\begin{lemma}\label{RingGeneration}
The multiplication morphism
\[m \colon K^{(\infty)} \times K^{(\infty)} \rar K^{(\infty)}, \enskip a^{(\infty)} \otimes b^{(\infty)} \mapsto a^{(\infty)} b^{(\infty)}\]
is a split epimorphism of pro-sets.
\end{lemma}
\begin{proof}
Consider a pre-morphism of pro-sets
\[u \colon K^{(\infty)} \rar K^{(\infty)} \times K^{(\infty)}, \enskip u^*(s) = s^2, \enskip u^{(s)} \colon c^{(s^2)} \mapsto 1^{(s)} \otimes c^{(s)}.\]
Clearly, this is indeed a pre-morphism and
\[
m^{(s)}\bigl(u^{(s)}\bigl(c^{(s^2)}\bigr)\bigr) = 1^{(s)} c^{(s)} = (sc)^{(s)}.\qedhere
\]
\end{proof}

\begin{lemma}\label{RingPresentation}
Let \(G^{(\infty)}\) be a pro-group, \(g \colon K^{(\infty)} \times K^{(\infty)} \rar G^{(\infty)}\) be a morphism of pro-sets. There is a morphism \(f \colon K^{(\infty)} \rar G^{(\infty)}\) of pro-groups such that
\[g\bigl(a^{(\infty)} \otimes b^{(\infty)}\bigr) = f\bigl(a^{(\infty)} b^{(\infty)}\bigr)\]
if and only if \(g\) satisfies the identities
\begin{itemize}
\item \(\bigl[g\bigl(a_1^{(\infty)} \otimes b_1^{(\infty)}\bigr), g\bigl(a_2^{(\infty)} \otimes b_2^{(\infty)}\bigr)\bigr] = 1\);
\item \(g\bigl(\bigl(a_1^{(\infty)} + a_2^{(\infty)}\bigr) \otimes b^{(\infty)}\bigr) = g\bigl(a_1^{(\infty)} \otimes b^{(\infty)}\bigr)\, g\bigl(a_2^{(\infty)} \otimes b^{(\infty)}\bigr)\);
\item \(g\bigl(a^{(\infty)} \otimes \bigl(b_1^{(\infty)} + b_2^{(\infty)}\bigr)\bigr) = g\bigl(a^{(\infty)} \otimes b_1^{(\infty)}\bigr)\, g\bigl(a^{(\infty)} \otimes b_2^{(\infty)}\bigr)\);
\item \(g\bigl(a^{(\infty)} b^{(\infty)} \otimes c^{(\infty)}\bigr) = g\bigl(a^{(\infty)} \otimes b^{(\infty)} c^{(\infty)}\bigr)\).
\end{itemize}
\end{lemma}
\begin{proof}
The necessity of the identities is clear. Since such \(f\) is unique by lemma \ref{RingGeneration}, it suffices to consider a group \(G\) instead of a pro-group. Take a morphism \(g\) satisfying the identities. It is given by a map \(g' \colon K^{(s)} \times K^{(s)} \rar G\) for sufficiently large \(s \in S\) satisfying the first three identities and the identity 
\[g'\bigl((asb)^{(s)} \otimes c^{(s)}\bigr) = g'\bigl(a^{(s)} \otimes (bsc)^{(s)}\bigr).\]
Consider the map \(f' \colon K^{(s^2)} \rar G\) given by
\[f'\bigl(c^{(s^2)}\bigr) = g'\bigl(1^{(s)} \otimes c^{(s)}\bigr),\]
it is a homomorphism by the first and the third identites. For all \(a, b \in K\) we have
\begin{align*}
f'\bigl(a^{(s^2)} b^{(s^2)})
&= g' \bigl( 1^{(s)}_p \otimes (s^2 ab)^{(s)} \bigr) =\\
&\quad g' \bigl( (sa)^{(s)} \otimes (sb)^{(s)} \bigr) =\\
&\quad g' \bigl(a^{(s^2)} \otimes b^{(s^2)}\bigr)
\end{align*}
by the second and the last identities. It is clear that \(f'\) gives the required morphism \(f\) of pro-groups.
\end{proof}



\end{document}
