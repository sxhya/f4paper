\documentclass[oneside, 11pt]{amsart} \pdfoutput=1
\usepackage{amsmath, mathtools, amssymb, amsthm, amscd, alltt, graphicx, comment}
\usepackage[utf8]{inputenc}
\usepackage[T1]{fontenc}
\usepackage[breaklinks=true,unicode]{hyperref}
\usepackage[capitalise]{cleveref}
\usepackage{tikz,tikz-cd}
\usepackage{scalerel}
%\usepackage[notref, notcite]{showkeys}
\usepackage[a4paper, left=25mm, right=15mm, top=25mm, bottom=35mm]{geometry}
\usepackage{enumitem}

\usepackage[backend=bibtex, bibencoding=utf8, giveninits=true, citestyle=numeric-comp, sortlocale=en_US, url=false, doi=false, eprint=true, maxbibnames=4]{biblatex}
\addbibresource{nisnevich2.bib}

\renewbibmacro*{volume+number+eid}{\ifentrytype{article}{\- \iffieldundef{volume}{}{Vol.~\printfield{volume},}\iffieldundef{number}{}{ No.~\printfield{number},}}}
\renewbibmacro{in:}{\ifentrytype{article}{}{\printtext{\bibstring{in}\intitlepunct}}}
\newbibmacro{string+doi}[1]{\iffieldundef{doi}{\iffieldundef{url}{#1}{\href{\thefield{url}}{#1}}}{\href{https://dx.doi.org/\thefield{doi}}{#1}}}
\DeclareFieldFormat[article, inproceedings, inbook, book, online]{title}{\usebibmacro{string+doi}{\mkbibquote{#1}}}
\renewcommand*{\bibfont}{\footnotesize}

\begin{document}
\renewcommand{\Im}{\mathop{\mathrm{Im}}\nolimits}
\newcommand{\Ker}{\mathop{\mathrm{Ker}}\nolimits}
\newcommand{\K}{{\mathrm{K}}}
\newcommand{\St}{\mathop{\mathrm{St}}\nolimits}
\newcommand{\E}{\mathrm{E}}
\newcommand{\Gsc}{\mathrm{G}_\mathrm{sc}}
\newcommand{\eval}{\mathrm{ev}}
\newcommand{\Max}{\mathrm{Max}}
\numberwithin{equation}{section}
\newtheorem{lemma}{Lemma} \numberwithin{lemma}{section}
\newtheorem*{lemma*}{Lemma}
\newtheorem{prop}[lemma]{Proposition} 
\newtheorem{theorem}[lemma]{Theorem}
\newtheorem{corollary}[lemma]{Corollary} 
\newtheorem*{theorem*}{Theorem} 
\newtheorem*{corollary*}{Corollary} 

\crefname{prop}{prop}{propss}

\theoremstyle{definition} 
\newtheorem{df}[lemma]{Definition} 
\newtheorem{rem}[lemma]{Remark}

\newcommand{\Set}{\mathbf{Set}}
\newcommand{\Group}{\mathbf{Grp}}
\newcommand{\Rng}{\mathbf{Rng}}
\newcommand{\Fun}{\mathbf{Fun}}
\newcommand{\Mod}{\mathbf{Mod}}
\newcommand{\op}{\mathrm{op}}
\newcommand{\ZZ}{\mathbb{Z}}
\newcommand{\otimeshat}{\mathbin{\widehat{\otimes}}}
\newcommand{\up}[2]{{^{#1}\!{#2}}}
\newcommand{\rA}{\mathsf{A}}
\newcommand{\rB}{\mathsf{B}}
\newcommand{\rC}{\mathsf{C}}
\newcommand{\rD}{\mathsf{D}}
\newcommand{\rE}{\mathsf{E}}
\newcommand{\rF}{\mathsf{F}}
\newcommand{\rG}{\mathsf{G}}

\newcommand{\catname}[1]{{\normalfont\textbf{#1}}} %Category name


\section{General formalism}
\begin{comment}
The aim of this section is to formulate and prove Lindel--Popescu Theorem, a certain formal statement, which gives a sufficient condition for a general group-valued functor to be homotopy invariant. Later, in~\cref{sec:main} this result is applied to the functor $K_2$. Some of the machinery developed in this section also will be useful in the study of the $K_2$-analogue of Gersten's conjecture, see~\cref{sec:K2-GC}.

Let us briefly recall the historical context, which might help to explain our choice of the name for~\cref{lpb}. Recall that H.~Lindel's paper~\cite{Li81} has settled the geometric case of the Bass--Quillen conjecture for all regular $k$-algebras $R$ essentially of finite type over $k$. Later T.~Vorst has used Lindel's result in the proof of the homotopy invariance for the unstable $\K_1$-functor, see~\cite{Vo81}. Recall that Vorst's result asserts that $\K_{1,r}(R[t]) \cong \K_{1,r}(R)$ for $r\geq 3$ and a regular ring $R$ essentially finite type over a field. Next, D.~Popescu proved Bass--Quillen conjecture in equicharacteristic case~\cite{Po90}.  Finally, in her recent work~\cite{Sta14} A.~Stavrova has combined Vorst's theorem with a theorem of Popescu and obtained a more general homotopy invariance $\K_1^G(A[t]) \cong \K_1^G(A)$ for an arbitrary regular algebra $A$ over a perfect field $F$ and a sufficiently isotropic reductive group $G$ defined over $F$, see~\cite[Theorem~1.3]{Sta14}.

\subsection{A general Lindel--Popescu-type theorem}
\end{comment}
For a commutative ring $k$ we denote by $\catname{Alg}_k$ the category of commutative unital $k$-algebras (not neccessarily of finite type). In the statement of~\cref{lpb}, which is the main result of this section, $k$ will be assumed to be a field. However, some of the intermediate steps in the proof of~\cref{lpb} can be obtained for $k=\ZZ$, so for the sake of greater generality we make no blanket assumption that $k$ is a field.  

For a commutative ring $R$ and an element $a\in R$ we denote by $\lambda_a$ the homomorphism of principal localization $R \to R_a$. Similarly, we denote by $\lambda_P$ the homomorphism $R \to R_P = (R\setminus P)^{-1}R$ of localization in a prime ideal $P \trianglelefteq R$. 

Let $A$ be an $R$-algebra, $f\in R[t]$ and $a \in A$. We denote by $ev_t(a)$ (or just $ev(a)$ when $t$ is clear from the context) the unique $R$-algebra homomorphism $R[t] \to A$ mapping $t$ to $a$. For an element $g \in K(R[t])$ we often shorten the notation for the element $K(ev(a))(g)$ to just $g(a)$.

\begin{df}\label{df:NK}
Let $K$ be a functor $\catname{Alg}_k \to \catname{Grp}$.
Denote by $NK(R)$ the kernel of the homomorphism $K(R[t]) \to K(R)$ induced by the homomorphism of evaluation at $t=0$. It is clear that $K(R[t]) \cong NK(R) \rtimes K(R)$ and $NK$ is also a functor $\catname{Alg}_k \to \catname{Grp}$. If $K(R[t])$ is abelian then $K(R[t]) \cong NK(R) \oplus K(R)$ and our definition agrees with~\cite[Def.~III.3.3]{Kbook}.
\end{df}

Notice that $R_a$ is the zero ring if $a$ is nilpotent. It will be convenient for us to extend the domain of the definition for $K$ and $NK$ so that it includes the zero ring. We do this by setting $K(0) = NK(0) = 1$.

For a commutative ring $R$ and $a \in R$ consider the following commutative square:
  \begin{equation} \label{M-sq} \begin{tikzcd} R \ltimes t R_a[t] \ar{r}{l} \ar{d}[swap]{e} & R_a[t] \ar{d}{ev(0)} \\ R \ar{r}{\lambda_a} & R_a. \end{tikzcd}\end{equation}
  The ring $R \ltimes tR_a[t]$ can be defined either via the semidirect product construction (see e.\,g. \cite[Definition~3.2]{S15}) or as the pullback of~\eqref{M-sq}.
  It is also clear that~\eqref{M-sq} is a Milnor square, see~\cite[Example~I.2.6]{Kbook}.

Now suppose that $k$ is a field. Following Swan we say that a $k$-algebra $R$ is {\it geometrically regular} (over $k$), if for any finite field extension $E$ of $k$ the ring $R\otimes_kE$ is regular (cf.~\cite[p.~137]{Sw98}). In particular, a geometrically regular algebra is noetherian. If $k$ is perfect then a $k$-algebra $R$ is geometrically regular over $k$ if and only if it is regular.

The following result gives a sufficient condition for a functor $K$ to be $\mathbb{A}^1$-invariant, cf. e.\,g. with~\cite[Proposition~2.2]{AHW20}.
\begin{theorem} \label{lpb}
 Let $k$ be a field.
 Suppose that a functor $K \colon \catname{Alg}_k \to \catname{Grp}$ satisfies the following axioms:
 \begin{enumerate}[label=\textnormal{(A\arabic*)}]
  \item \label{CFC} {\it $K$ is finitary}, i.\,e. commutes with filtered colimits.
  \item \label{DP} %{\it $K$ maps each square of the form~\eqref{M-sq} to a pullback square.}
  %Since $e$ is split by the obvious embedding, this requirement is equivalent to the following one: 
  For a $k$-algebra $R$ and $a \in R$ consider the diagram obtained from~\eqref{M-sq} by applying $K$. Then the homomorphism $\Ker(K(e)) \to NK(R_a)$ between the kernels of vertical arrows is injective.
  \item \label{LPP} {\it $K$ satisfies weak affine Nisnevich excision for domains.} By this we mean the following. Let $\iota \colon B \hookrightarrow A$ be an {\'e}tale inclusion of domains contained in $\catname{Alg}_k$. Let $h$ be an element of $B$ not invertible in $A$ such that $\iota$ induces an isomorphism $B / hB \cong A / hA$. Consider the following commutative square (sometimes called an elementary Nisnevich square): 
  \[\begin{tikzcd} B \ar{r}{\iota} \ar{d} & A \ar{d}\\ B_h \ar{r} & A_h. \end{tikzcd}\]
  Then the natural homomorphism \[\Ker(K(B) \to K(B_h)) \to \Ker(K(A) \to K(A_h))\] induced by $\iota$ is surjective.
  \item \label{PGP} {\it $K$ satisfies $\mathbb{P}^1$-glueing property for local domains.}  By definition, this means that for every local domain $R \in \catname{Alg}_k$ the following diagram whose arrows are induced by natural embeddings is a pullback square: \begin{equation}\label{eq:P1-square} \begin{tikzcd} K(R) \ar[r] \ar[d] & K(R[t]) \arrow{d} \\ K(R[t^{-1}]) \ar{r} & K(R[t, t^{-1}]). \end{tikzcd} \end{equation}  
  It is easy to see that if~\eqref{eq:P1-square} is pullback, then all its arrows are injective.
  \item \label{HIF} {\it $K$ is homotopy invariant for fields}, i.\,e. for every field $F \in \catname{Alg}_k$ the canonical embedding induces an isomorphism $K(F) \cong K(F[t])$.
 \end{enumerate}
 Then for every geometrically regular $R\in \catname{Alg}_k$ the natural embedding $R \hookrightarrow R[t]$ induces an isomorphism $K(R)\cong K(R[t]).$
\end{theorem}
\begin{rem}
 In the literature a functor satisfying the axiom~\ref{PGP} is sometimes called {\it acyclic} (cf.~\cite[Def.~III.4.1.1]{Kbook}).
 The choice of the name for~\eqref{LPP} is inspired by axiom (P3) from~\cite[Proposition~3.3.4]{AHW18}.
 Notice also that our axioms can be slightly relaxed, so that the assertion of the Theorem becomes slightly stronger, see Remark~\ref{rem:relax} below.
\end{rem}

The proof is based on a series of lemmas and is deferred until the end of this section. We start by deducing the following useful corollary of the axiom~\ref{DP}. 

Let $k$ be an arbitrary commutative ring and $K\colon \catname{Alg}_k \to \catname{Grp}$ be a functor.
We say that $K$ satisfies the {\it Quillen--Suslin local-global principle} if for every $R\in \catname{Alg}_k$ the map 
\begin{equation} \label{QS-def} \begin{tikzcd} NK(R) \ar{r}{\prod \lambda_M} & \prod\limits_{M \in \Max(R)} NK(R_M) \end{tikzcd} \end{equation} is injective.
\begin{lemma}\label{LGP}
Suppose that $K$ is a finitary functor satisfying~\ref{DP}. Then $K$ satisfies the Quillen--Suslin local-global principle. In particular, $NK$ is a separated presheaf in the Zariski topology.
\end{lemma}

\begin{proof}
 Fix a $k$-algebra $R$ and $a \in R$. 
 Consider the following diagram of $k$-algebras. Its objects $A_i$ are copies of $R[t]$ indexed by natural numbers $i$. The only arrows of this diagram are the homomorphisms $ev (a^{j-i}t)\colon A_i \to A_j$ defined for $1 \leq i \leq j$. It is clear that this diagram is filtered and its colimit is $R \ltimes tR_a[t]$ (cf. e.\,g.~\cite[Lemma~15]{S15}). 
 
 The first step of the proof of the lemma is to show that for every $g \in \Ker(NK(R) \to NK(R_a))$ there exists some natural $n$ such that $g(a^nt)$ is the trivial element of $K(R[t])$. Let $g$ be such an element. From~\ref{DP} we obtain that the image of $g$ in $K(R \ltimes tR_a[t])$ is trivial. The required assertion now follows from~\ref{CFC} and the previous paragraph.
 
 The next step of the proof is to verify that for any coprime elements $a, b \in R$ the homomorphism $\langle K(\lambda_a), K(\lambda_b) \rangle \colon NK(R) \to NK(R_a) \times NK(R_b)$ is injective. Fix $g \in \Ker(\langle \lambda_a, \lambda_b \rangle)$. We argue as in the proof of~\cite[Lemma~2.5]{Tu83}. Set $S := R[t, t_1]$. Consider the element $h(t, t_1, t_2) = g(t_1 t) \cdot g((t_1 + t_2)t)^{-1}\in K(S[t_2]).$ Clearly, $h$ lies in the kernel of the homomorphism $K(ev_{t_2}(0))$. Since evaluation commutes with localization, we obtain that the element $K(\lambda_a)(h) \in K(S_a[t_2])$ is trivial, therefore by the previous paragraph there exists $n$ such that $h(t, t_1, a^nt_2)$ is trivial in $K(S[t_2])$. Similarly, we find $m$ such that $h(t, t_1, b^m t_2)$ is trivial. Since $a^n$ and $b^m$ are still coprime, we can find $x, y \in R$ such that $xa^n + yb^m = 1$. The required assertion now follows from the following calculation:
 $$1 = h(t, 1, -yb^m) \cdot h(t, xa^n, -xa^n) = g(t)\cdot g(xa^n\cdot t)^{-1} \cdot g(xa^n\cdot t) \cdot g(0)^{-1} = g(t).$$
 
 Now we can finish the proof of the lemma. We argue as in the proof of~\cite[Theorem~2]{S15}. Let $g$ be an element of the kernel of~\eqref{QS-def}. Denote by $Q(g)$ the set consisting of all elements $c \in R$ for which $K(\lambda_c)(g)$ is trivial. Let us check that this set is, in fact, an ideal. Fix $a, b \in Q(g)$ and let $c$ be an element of the ideal $\langle a, b \rangle$. Notice that $\overline{a} = \lambda_c(a)$, $\overline{b} = \lambda_c(b)$ are coprime elements of $R_c$. From the identities $\lambda_{\overline{a}}\lambda_c = \lambda_{\lambda_a(c)}\lambda_a$ and $\lambda_{\overline{b}}\lambda_c = \lambda_{\lambda_b(c)}\lambda_b$ we obtain that $g' = K(\lambda_c)(g)$ lies in the kernel of $\langle K(\lambda_{\overline{a}}), K(\lambda_{\overline{b}}) \rangle.$ By the previous paragraph, we obtain that $g' = 1$ and hence that $c \in Q(g)$. We have shown that $Q(g)$ is an ideal of $R$. If $Q(g)$ is proper then it is contained in a maximal ideal $M \trianglelefteq R$, in which case from $K(\lambda_M)(g) = 1$ and~\ref{CFC} we obtain that $K(\lambda_s)(g) = 1$ for some $s \in R \setminus M$. Thus, we obtain a contradiction, so $Q(g) = R$ and $g = 1$.
\end{proof}

First of all, notice that the local-global principle allows one to obtain the following global version of~\ref{PGP} (cf. e.\,g. with the proof of~\cite[Theorem~1]{LS20}). 
\begin{lemma}[$\mathbb{P}^1$-glueing property for domains] \label{ght}
Suppose that $K$ satisfies the Quillen--Suslin local-global principle and the $\mathbb{P}^1$-glueing property for local domains. Then for an arbitrary domain $R \in \catname{Alg}_k$ the square~\eqref{eq:P1-square} is a pullback square. In particular, all its arrows are injective. \end{lemma}
\begin{proof}
Let $R \in \catname{Alg}_k$ be an arbitrary domain. Suppose that the images of $g \in K(R[t])$ and $h \in K(R[t^{-1}])$ coincide in $K(R[t, t^{-1}])$. Then so do the images of $g' = g \cdot g(0)^{-1}$ and $h' = h \cdot g(0)^{-1}$. For a maximal ideal $M\trianglelefteq R$ set $g'_M := K(\lambda_M)(g') \in K(R_M[t])$ and $h'_M := K(\lambda_M)(h') \in K(R_M[t^{-1}]))$. Clearly, their images in $K(R_M[t, t^{-1}])$ coincide. By our assumptions, the square~\eqref{eq:P1-square} is pullback for $R_M$, therefore $g'_M = 1$ for all $M$. Since $g' \in NK(R)$, by the local-global principle we obtain that $g' = 1$, hence $g$ is the image of some element of $K(R)$. By symmetry, $h$ also is the image of some element of $K(R)$. Since $K(R)\to K(R[t, t^{-1}])$ is injective, we conclude that $g$ and $h$ are the images of the same element of $K(R)$, which completes the proof.  \end{proof}

Let us also note the following useful particular special case of weak affine Nisnevich excision.
\begin{lemma}[weak affine Zariski excision for domains]
	\label{zgl} Suppose that a functor $K\colon \catname{Alg}_k \to \catname{Grp}$ satisfies~\ref{LPP}. Let $R \in \catname{Alg}_k$ be a domain and $a, b$ be a pair of coprime elements of $R$. Consider the diagram
$$\begin{tikzcd}
	R \ar{r}{\lambda_a} \ar{d}[swap]{\lambda_b} & R_a \ar{d}{\overline{\lambda_b}}\\
	R_b \ar{r}{\overline{\lambda_a}} & R_{ab}.
\end{tikzcd}$$
	Then the natural map $\Ker(K(\lambda_b)) \to \Ker(K(\overline{\lambda_b}))$ induced by $\lambda_a$ is surjective.
\end{lemma}
%\begin{proof}
%	Set $B=R$, $A=R_a$ и $h=b$. It is clear that $B \to A$ is an {\'e}tale homomorphism. To see that the present situation is a special case of~\ref{LPP} it is enough to  check that the map $j\colon B/hB \to A/hA$ is an isomorphism. Let us first verify the surjectivity of $j$, or what is the same, the inclusion $A \subseteq Ah+B$. Fix an element $r/a^s\in A=R_a$, we need to show that it lies in $Ah+B$. We may assume that $s\geq 1$, so that $a^s$ and $b^s$ are still coprime. Choose $x, y \in R$ such that $xa^s+yb^s=1$. Thus, $r/a^s$ can be decomposed into the sum of $rx\in B$ and $ry(b/a)^s\in bR_a=hA$.
%
%	Now let us verify the injectivity of $j$, i.\,e. the inclusion $Ah\cap B \subseteq Bh$. Suppose that $rb/a^s=c\in B=R$. We may assume $s\geq 1$ otherwise there is nothing to prove. From $rb=a^sc$ and $xa^s+yb^s=1$ we obtain the required inclusion $c=cxa^s+cyb^s=xrb+cyb^s=(xr+cyb^{s-1})b \in Rb = Bh$. 
%\end{proof}

The following injectivity result is inspired by~\cite[Corollary~5.2]{Tu83}.
\begin{lemma} \label{lmp}
Suppose that $K$ satisfies weak affine Zariski excision and $\mathbb{P}^1$-glueing properties for domains. Then for any domain $R\in \catname{Alg}_k$ and any monic polynomial $f\in R[t]$ the localization homomorphism $\lambda_f\colon R[t]\rightarrow R[t]_f$ induces an injection $K(R[t])\hookrightarrow K(R[t]_f).$ \end{lemma}
\begin{proof}
	Fix a presentation $f=\sum_{i=0}^n a_it^i$, in which $a_n=1$. Set $$g=1+a_{n-1}t^{-1}+\ldots+a_0t^{-n}\in R[t^{-1}].$$ It is clear that $R[t, t^{-1}]_f \cong R[t, t^{-1}]_g$. Consider the following commutative diagram:
$$\begin{tikzcd}
	R[t] \ar{r}{\lambda_t} \ar{d}{\lambda_f} & R[t, t^{-1}] \ar{d}{\overline{\lambda}} & R[t^{-1}] \ar{l}[']{\lambda_{t^{-1}}} \ar{d}{\lambda_g}\\
	R[t]_f \ar{r} & R[t, t^{-1}]_f & R[t^{-1}]_g. \ar{l}
\end{tikzcd}$$
Let $x$ be an element of $\Ker(K(\lambda_f))$. Then $K(\lambda_t)(x)$ lies in $\Ker(K(\overline{\lambda}))$ so by~\cref{zgl} applied to the right square we find $y \in \Ker(K(\lambda_{g}))$ such that $K(\lambda_t)(x) = K(\lambda_{t^{-1}})(y)$. Now by~\cref{ght} $x$ and $y$ are images of some $z \in K(R)$. Since $K(R) \to K(R[t, t^{-1}])$ is injective, we conclude that $x=1$, which completes the proof. \end{proof}

For the rest of this section we assume that $k$ is a field.
Recall that a $k$-algebra $R$ is called {\it essentially smooth} if it is geometrically regular and essentially of finite type over $k$. 
Equivalently, $R$ is essentially smooth over $k$ if it is essentially of finite type over $k$ and $R\otimes_k\overline k$ is regular (see e.\,g.~\cite[p.~137]{Sw98}). 
If, moreover, $R$ is of finite type over $k$, it is called {\it smooth over $k$}.

\begin{theorem}[Panin]
\label{paninthm}
Let $k$ be a field, $R$ be a domain smooth over $k$. Let $M_1,\ldots, M_n$ be a finite set of maximal ideals of $R$. Denote by $A=R_{M_1,\ldots,M_n}$ the correponding semi-localization of $R$. Let $f$ be an element of the intersection  $\cap_{i=1}^nM_i \subseteq R$. Then there exists a monic polynomial $h(t)\in A[t]$, a domain $S$ essentially smooth over $k$ and homomorphisms $\tau$, $p$, $p'$ and $\delta$ such that $A[t]/hA[t]\cong S/\tau(h)S$ and the following diagram commutes:
\begin{equation}\label{eq:panin-diag}
 \begin{tikzcd}[column sep=4em]
   & A & \\ A[t] \ar{d}{\lambda_h} \ar{ru}{ev(0)} \ar{r}{\tau} & S \ar{u}{\delta} \ar{d}{\lambda_{\tau(h)}}  & R \ar{ul}[swap]{\lambda_{M_1,\ldots,M_n}} \ar{d}{\lambda_f} \ar{l}[swap]{p} \\
   A[t]_h \ar{r}[swap]{\tau\otimes_{A[t]}A[t]_h}              & S_{\tau(h)}      & R_f. \ar{l}{p'}\end{tikzcd}
\end{equation}
\end{theorem}
\begin{proof}
 This is a ring-theoretic restatement of the assertions (i)--(iii) of~\cite[Theorem~2.5]{Pa19}.
\end{proof}

\begin{corollary}
\label{esssmooth}
Let $k$ be a field.
Suppose that $K\colon\catname{Alg}_k\rightarrow\catname{Grp}$ is finitary and satisfies weak affine Nisnevich excision and $\mathbb{P}^1$-glueing properties. Let $R, M_1, \ldots, M_n, A$ be as in the statement of the above theorem. Denote by $E$ the fraction field of $A$ {\rm(}which coincides with the fraction field of $R${\rm)}.
Then for any $m\geq 0$ the natural homomorphism $K(A[x_1,\ldots, x_m])\to K(E[x_1,\ldots,x_m])$ is injective.
\end{corollary}
\begin{proof}
We argue as in the proof of~\cite[Theorem~3.2]{Sta20}. Set $B=k[x_1,\ldots x_m]$. Let $g$ be an element of the kernel of $K(B \otimes_k A)\rightarrow K(B \otimes_k E)$. First of all, notice that both $A$ and $E$ are filtered colimits of principal localizations of $R$. Since $K$ is finitary, there exists $f' \in \cap_{i=1}^n(R \setminus M_i)$ such that $g$ is the image of some $g_1 \in K(R_{f'}[x_1,\ldots x_m])$ under $K(\lambda_{M_1',\ldots M_n'})$, where $M_i' = R_{f'} \cdot M_i$. Set $R' := R_{f'}$.
Clearly, there exists $f \in R'$ such that $K(\lambda_{f})(g_1) = 1$ in $K(R'_{f})$. 
Without loss of generality, we may assume that $f \in \cap_{i=1}^n M_i'$. 

We apply Theorem~\ref{paninthm} to the ring $R'$, maximal ideals $M_i'$ and the polynomial $f$ as above.
Tensoring~\eqref{eq:panin-diag} with $B$ and applying functor $K$ we obtain the following commutative diagram (we use the convention that tensoring with $B$ does not change the notation for the arrows of the diagram):
\begin{equation*} 
 \begin{tikzcd}[column sep=4em]
   & K(B \otimes_k A) & \\ 
   K(B \otimes_k A[t]) \ar{d}{K(\lambda_h)} \ar{ru}{K(ev_t(0))} \ar{r}{K(\tau)} & K(B \otimes_k S) \ar{u}{K(\delta)} \ar{d}{K(\lambda_{\tau(h)})}  & K(B \otimes_k R') \ar{ul}[swap]{K(\lambda)} \ar{d}{K(\lambda_f)} \ar{l}[swap]{K(p)} \\
   K(B \otimes_k A[t]_h) \ar{r}[swap]{}              & K(B \otimes_k S_{\tau(h)})      & K(B \otimes_k R'_f). \ar{l}{}\end{tikzcd}
\end{equation*}
Notice that $K(p)(g_1)$ lies in the kernel of $K(\lambda_{\tau(h)})$, therefore by~\ref{LPP} there exists $g_2\in K(B \otimes_k A[t])$ such that $K(\lambda_h)(g_2)=1$ and $K(\tau)(g_2)=K(p)(g_1)$. Since $h \in A[t]$ is monic, by~\cref{lmp} the homomorphism $K(\lambda_h)$ is injective, therefore $g_2=1$. It remains to see that
$$g=K(\lambda)(g_1)=K(\delta)(K(p)(g_1))=K(\delta)(K(\tau)(g_2))=g_2(0)=1.\qedhere$$
\end{proof}

\begin{rem}
For the proof of~\cref{lpb} we only need the special cases $m=0,1$ of the above result.
\end{rem}

\begin{theorem}[Popescu]
\label{popescu} Let $k$ be a field, and $R$ a ring geometrically regular over $k$. Then $R$ is a filtered colimit of smooth $k$-algebras. \end{theorem}
\begin{proof} See~\cite{Po85}, ~\cite[Theorem~1.1]{Sw98}.
\end{proof}

Now we are ready to finish the proof of \cref{lpb}. %Our argument follows~\cite{Sta14, Vo81}.
\begin{proof}[Proof of~\cref{lpb}]
Our goal is to prove the triviality of $NK(R)$.
By~\cref{popescu} $R$ is a filtered colimit of smooth $k$-algebras.
Since filtered colimits commute with finite limits and the functor $K$ is finitary, the functor $NK$ is also finitary.
Thus, it suffices to verify the triviality of $NK(R)$ for a smooth $k$-algebra $R$.
Further, by~\cref{LGP} we are left to prove that $NK(R_M)$ is trivial for every maximal ideal $M$ in a such an algebra.
Since every smooth scheme over $k$ is a disjoint union of smooth irreducible $k$-schemes, we may assume, without loss of generality, that $R$ is a domain.
Denote by $E$ the fraction field of $R_M$. Consider the diagram
\[\begin{tikzcd}
K\bigl(R_M[t]\bigr) \ar[hookrightarrow]{r} \ar{d}{ev_t(0)} & K(E[t]) \ar{d}{\cong}\\
K\bigl(R_M\bigr) \ar[hookrightarrow]{r} & K(E),
\end{tikzcd}\]
in which the right vertical arrow is an isomorphism by~\ref{HIF} and the horizontal arrows are injective by Corollary~\ref{esssmooth}.


Thus, the left vertical arrow is also injective and $NK(R_M) = 1$, as required.
\end{proof}
\begin{rem}\label{rem:relax}
It is clear from the above proof that it possible to weaken the requirements on $K$ in the statement of~\cref{lpb}. For example, we could require that axioms \ref{DP}--\ref{PGP} hold true only for the algebras essentially smooth over $k$. Further, the assertion of the axiom~\ref{CFC} can be restricted to algebras geometrically regular over $k$.
\end{rem}

We can also prove the following result analogous to~\cite[Theorem~1.2]{Sta20}
\begin{theorem}
Let $k$ be a field.
Suppose that $K\colon\catname{Alg}_k\rightarrow\catname{Grp}$
is a finitary functor satisfying weak affine Nisnevich excision and $\mathbb{P}^1$-glueing properties for domains. 
Let $A$ be a semi-local domain geometrically regular over $k$. 
Denote by $E$ the fraction field of $A$.
Then the natural homomorphism $K(A) \to K(E)$ is injective.
\end{theorem}
\begin{proof}
By Theorem~\ref{popescu} $A$ is a filtered colimit of smooth $k$-algebras. Since $A$ is a domain we may assume additionally that these smooth $k$-algebras are domains.

Since $K$ is finitary, we may assume that $g \in \Ker(K(A) \to K(E))$ comes from some element $g_1\in\mathrm K(R)$, where $R$ is a smooth $k$-domain. We may assume, additionally, that $g_1$ lies in the kernel of the natural homomorphism $K(R) \to K(\mathrm{Frac}(R))$.

Denote by $P_i$ the preimages of the maximal ideals of $A$ and choose some maximal ideals $M_i$ of $R$ such that $M_i$ contains $P_i$. Then by \cref{esssmooth}, $g_1$ vanishes in the semi-localization of $R$ in the set of maximal ideals $M_i$, therefore it vanishes in the semi-localization of $R$ in the set of prime ideals $P_i$.
But $R\rightarrow A$ factors through the latter semi-localization, so we conclude that $g=1$. 
\end{proof}

\section{Preliminaries}

\subsection{Steinberg groups and pro-groups}

Now for $a, b \in R$ set $y_\alpha(a, b) = [x_\alpha(a), x_{-\alpha}(b)]$.
For an ideal $I \trianglelefteq R$ we denote by $\overline{\St}(\Phi, R, I)$ the normal closure in $\St(\Phi, R)$ of the subgroup generated by $x_\alpha(a)$, $a\in I$.

\begin{lemma} \label{lem:c-identities} For an arbitrary irreducible root system $\Phi$ of rank $\geq 2$, arbitrary ideals $A, B \trianglelefteq R$ and all $a \in A$, $b \in B$, $c \in R$ the following congruences hold:
\begin{itemize}
 \item $y_\alpha(a, cb) \equiv y_\alpha(ac, b)\ (\mathrm{mod}\ \overline{\St}(\Phi, R, AB))$ in the case when either $\Phi \neq \rC_{\ell}$ or $\alpha$ is short.
 \item $y_\alpha(a, c^2b) \equiv y_\alpha(ac^2, b)\ (\mathrm{mod}\ \overline{\St}(\Phi, R, AB))$, $y_\alpha(a, cb)^2 \equiv y_\alpha(ac, b)^2\ (\mathrm{mod}\ \overline{\St}(\Phi, R, AB))$ in the case $\Phi = \rC_{\ell}$ and $\alpha$ is long.
\end{itemize} \end{lemma}
\begin{proof}
 Observe that the proof of~\cite[Theorem~5]{VZ20} is based solely on computations with Chevalley commutator formula, which all can be reproduced verbatim in the context of Steinberg groups.
\end{proof}

Notice that the Steinberg group functor {\it does not} commute with general finite limits. However, it satisfies the following weaker property, which we are going to use in the sequel.
\begin{lemma} \label{lem:fprod} For an arbitrary irreducible root system $\Phi$ of rank $\geq 2$ the Steinberg group functor $\St(\Phi, -)$ commutes with finite direct products. \end{lemma}
\begin{proof} 
It suffices to verify the assertion for binary direct products.
Observe that canonical projections $R_1 \times R_2 \to R_i$, $i=1,2$ split in the category~\catname{Rngs}, therefore the groups $G_i = \St(\Phi, R_i)$ embed as subgroups into $G = \St(\Phi, R_1 \times R_2)$. It is also clear that $G_i$ together generate $G$. Thus, to verify the isomorphism $G \cong G_1 \times G_2$ it suffices to show the triviality of the commutator subgroup $[G_1, G_2] \leq G$.

Set $A = R_1\times 0$, $B = 0 \times R_2$. It is clear that $A, B \trianglelefteq R_1 \times R_2$. 
It follows directly from Chevalley commutator formula that the commutators $[x_{\alpha}(a),\ x_\beta(b)]$ are trivial for all $\beta \neq -\alpha$. On the other hand, in order to verify the triviality of $y_\alpha(a, b) = [x_{\alpha}(a),\ x_{-\alpha}(b)]$ for $a\in A$, $b\in B$ we can apply the congruences of~\cref{lem:c-identities} (since $AB=0$ these congruences turn into equalities).
Indeed, setting $c = c^2 = (1, 0)$ we obtain that $y_\alpha(a, b) = y_\alpha(ac, b) = y_\alpha(a, bc) = y_\alpha(a, 0) = 1$. \end{proof}

\subsection{Overview of Steinberg pro-groups}
The aim of this section is to recall the technique of Steinberg pro-groups, which has been used in~\cite{LSV20} to prove the centrality of $\K_2$.
We briefly present the key constructions from~\cite{LSV20} and also obtain some new results essential for the purposes of this article. 

Throughout this section $R$ is an integral domain and $\Phi$ is a reduced irreducible root system of rank at least $3$.
We denote by $\mathbb{N}$ the set of nonnegative integers.
%All root systems in this text are reduced and irreducible. By \(\mathbb N\) we mean the set of non-negative integer numbers.

We start by recalling the definition of the category $\catname{Pro}(\catname{Grp})$ of pro-groups, which is the formal pro-completion of the category of groups (cf.~\cite[\S~2.1]{LSV20}).
By definition, its objects are contravariant functors $X^{\bullet}\colon\mathcal{C}^{\mathrm{op}} \to \catname{Grp}$ from an arbitrary filtered category $\mathcal{C}$ to the category of groups. For $X^{(\bullet)} \in 
\catname{Pro}(\catname{Grp})$ and $c \in \mathcal{C}$ we denote by $X^{(c)}$ the value of $X^{(\bullet)}$ on $c$. 
Let $X^{(\bullet)}\colon\mathcal{C}\to\catname{Grp}$ and $Y^{(\bullet)}\colon\mathcal{D}\to\catname{Grp}$ be a pair of pro-groups.

By definition, a {\it pre-morphism} $\eta\colon X^{(\bullet)} \to Y^{(\bullet)}$ is a pair $(\eta^*, \{\eta_d\}_{d\in\mathrm{Ob}(\mathcal{D})})$ consisting of a function $\eta^*\colon \mathrm{Ob}(\mathcal{D})\to\mathrm{Ob}(\mathcal{C})$ and a collection of group-homomorphisms $\eta_d\colon X^{(\eta^*(d))}\to Y^{(d)}$ such that for every arrow $d' \to d$ there exists $c$ such that the composite homomorphisms $X^{(c)} \to X^{(\eta^*(d'))} \xrightarrow{\eta_{d'}} Y^{(d')}$ and $X^{(c)} \to X^{(\eta^*(d))} \xrightarrow{\eta_d} Y^{(d)} \to Y^{(d')}$ are equal. 
Premorphisms $\eta, \epsilon \colon X^{(\bullet)} \to Y^{(\bullet)}$ are declared equivalent if for every $d \in \mathrm{Ob}(\mathcal{D})$ there exists $c \in \mathrm{Ob}(\mathcal{C})$ such that the composite homomorphisms $X^{(c)} \to X^{(\eta^*(d))} \xrightarrow{\eta_d} Y^{(d)}$ and $X^{(c)} \to X^{(\epsilon^*(d))} \xrightarrow{\epsilon_d} Y^{(d)}$ are equal. By definition, the set of morphisms $\catname{Pro}(\catname{Grp})(X^{(\bullet)}, Y^{(\bullet)})$ consists of equivalence classes $[\eta]$ of pre-morphisms $\eta \colon X^{(\bullet)} \to Y^{(\bullet)}$ with respect to the equivalence relation defined above.
There is an obvious fully faithful embedding $\catname{Grp}\to\catname{Pro}(\catname{Grp})$ sending $G$ to the obvious functor $G\colon \{ * \} \to \catname{Grp}$.

Let $S$ be a multiplicative subset of $R$, containing a unit. 
We denote by $\lambda_S$ the canonical localization homomorphism $R \to R_S$. We also use the same letter for the induced homomorphism $\St(\Phi, R) \to \St(\Phi, R_S)$.
Now let us recall the definition of the {\it Steinberg pro-group} (cf.~\cite[\S~2.4]{LSV20}). 
It is clear that $S$ can be interpreted as a filtered category via $\mathrm{Ob}(S) = S$ and $S(s, t) = \{ u \in S \mid su = t \}$.
For every $s$ the principal ideal $sR$ can be considered as a ring without unit (with the multiplication given by $sr \cdot sr' = s(srr')$).
By definition, the Steinberg pro-group $\St^{\bullet}(\Phi, R, S)$ (denoted $\St^{(\infty)}(\Phi, R)$ in~\cite{LSV20}) is the functor $S \to \catname{Grp}$, whose
value on $(s \to t)$ is given by the homomorphism $\St(\Phi, tR) \to \St(\Phi, sR)$ induced by the obvious embedding $tR \subseteq sR$.
There is an obvious pro-group morphism $\pi \colon \St^{\bullet}(\Phi, R, S) \to \St(\Phi, R)$ given by $\pi^* \colon \{* \} \to S$, $\pi^*(*) = 1$, $\pi_{*} = \mathrm{id}_{\St(\Phi, R)}$. Postcomposing this morphism with the homomorphism $\lambda_S$ we obtain a morphism $\St^\bullet(\Phi, R, S) \to \St(\Phi, R_S)$, which we denote by $\pi_S$.

%The importance of the Steinberg pro-group is illustrated by the following result, see~\cite[Proposition~4.2]{LSV20}.
\begin{prop}
 For a simply laced $\Phi$ and a multiplicative system $S\subseteq R$ there exists a well-defined group homomorphism $conj\colon \St(\Phi, R_S) \to \mathrm{Aut}(\St^\bullet(\Phi, R, S))$ compatible with the obvious conjugation action of $\St(\Phi, R_S)$ on itself, i.\,e. such that for every $g \in \St(\Phi, R_S)$ the following diagram of pro-groups commutes:
 \[ \begin{tikzcd} \St^\bullet(\Phi, R, S) \ar{r}{conj(g)} \ar{d}{\pi_S} & \St^\bullet(\Phi, R, S) \ar{d}{\pi_S} \\ \St(\Phi, R_S) \ar{r}{(-)^g} & \St(\Phi, R_S). \end{tikzcd} \]
\end{prop}
In down-to-earth terms the latter condition states that for every $g\in \St(\Phi, R_S)$ there exists a representative $c \in conj(g)$ which amounts to the pair $c^*\colon \{*\} \to S$ and $c_{*}\colon \St(\Phi, c^*(pt)R) \to \St(\Phi, R)$ such that $\lambda_S(c_1(x)) = g\lambda_S(i(x))g^{-1}$ for all $x\in \St(\Phi, c^*(1)R)$, where $i$ denotes the canonical homomorphism $\St(\Phi, c^*(1)R) \to \St(\Phi, R)$.

Consider a diagram of Steinberg groups $\St(\Phi, s^n R)$ for $n \in \mathbb N$, i.e. the groups with the generators $x_{\alpha}(a)$ for $\alpha \in \Phi$, $a \in s^n R$ and the ordinary Steinbrg relations. The homomorphisms in the chain
$$ \ldots \to \St(\Phi, s^3 R) \to \St(\Phi, s^2 R) \to \St(\Phi, s R) \to \St(\Phi, R) $$
are defined in the natural way. Such a tower is called a {\it Steinberg pro-group}, it is denoted by $\St^\infty(\Phi, s^\bullet R)$. We usually do not write explicitly the structure homomorphisms of $\St^\infty(\Phi, s^\bullet R)$ in diagrams and formulas. In particular, we denote the elements of $\St(\Phi, s^n R)$ and their images in $\St(\Phi, s^{n - 1} R)$ by the same letters.

We cannot conjugate elements of $\mathrm{St}(\Phi, s^nR)$ by elements of $\mathrm{St}(\Phi, R_s)$. Fortunately, there is such a conjugation action on the whole Steinberg pro-group.

\begin{df}
An {\it endomorphism representative} $\eta$ of the Steinberg pro-group $\St^\infty(\Phi,\,s^\bullet R)$ is the following data: 
\begin{itemize}
\item
a map $\eta^* \colon \mathbb N \to \mathbb N$,
\item
for each $i \in \mathbb N$ a group homomorphism $\eta_i \colon \St(\Phi, s^{\eta^*(i)} R) \to \St(\Phi, s^i R)$,
\end{itemize}
such that for every $i \leq j$ there is $k \geq \eta^*(i), \eta^*(j)$ making the diagram
$$\begin{tikzcd}[row sep=tiny]
& \St(\Phi, s^{\eta^*(j)} R)\ar{r}{\eta_j} & \St(\Phi, s^j R) \ar{dd}\\
\St(\Phi, s^k R) \ar{ur} \ar{dr} & &\\
&\St(\Phi, s^{\eta^*(i)} R) \ar{r}{\eta_i} & \St(\Phi, s^i R)
\end{tikzcd}$$
commutative.

Endomorphism representatives $\eta$ and $\theta$ are called {\it equivalent} if for every $i \in \mathbb N$ there is $k \geq \eta^*(i), \theta^*(i)$ making the diagram
$$\begin{tikzcd}
\St(\Phi, s^k R) \ar{r} \ar{d} & \St(\Phi, s^{\eta^*(i)} R) \ar{d}{\eta_i}\\
\St(\Phi, s^{\theta^*(i)} R) \ar{r}{\theta_i} & \St(\Phi, s^i R)
\end{tikzcd}$$
commutatuve.

For example, if we consider any map $\theta^*$ such that $\theta^*(i) \geq \eta^*(i)$ for all $i$ and define $\theta_i$ as the composition of $\eta_i$ and the structure homomorphism of $\St^\infty(\Phi, s^\bullet R)$, then such a representative $\theta$ is equivalent to $\eta$.

An {\it endomorphism} $g$ of the pro-group $\St^\infty(\Phi,\,s^\bullet R)$ is the equivalence class of endomorphism representatives. Clearly, endomorphisms of $\St^\infty(\Phi,\,s^\bullet R)$ form a monoid $\mathrm{End}\bigl(\St^\infty(\Phi,\,s^\bullet R)\bigr)$ under composition. We are interested in its invertible elements, i.e. the automorphisms of the pro-group \(\St^\infty(\Phi, s^\bullet R)\).

Suppose that there is a fixed group homomorphism $$\Xi \colon \St(\Phi, R_s) \to \mathrm{Aut}\bigl(\St^\infty(\Phi, s^\bullet R)\bigr).$$ For any $g \in \St(\Phi, R_s)$ we say that a representative $\eta$ of the automorphism $\Xi(g)$ is {\it strict} if for every $i$ and for every $x \in \St(\Phi, s^{\eta^*(i)} R)$ the identity
$$
\lambda_s\bigl(\eta_i(x)\bigr) = g \lambda_s(x) g^{-1} \in \St(\Phi, R_s)
$$
holds, where $\lambda_s$ is the principal localization at $s$. Clearly, a composition of strict representatives is strict.
\end{df}


\begin{theorem}[Voronetsky]
\label{vor}
There is a group homomorphism 
$$
\Xi \colon \St(\Phi, R_s) \to \mathrm{Aut}\bigl(\St^\infty(\Phi, s^\bullet R)\bigr)
$$
such that every automorphism $\Xi(g)$ for $g \in \St(\Phi, R_s)$ has a strict representative.
\end{theorem}

Recall a possible construction of such strict representatives. Clearly, it suffices to construct a strict representative for \(\Xi(g)\), where $g = x_\alpha(\frac a {s^k})$ is a generator of $\St(\Phi, R_s)$. Take $\eta^*(i) = 2p(i + k)$ for sufficiently large constant \(p\) and define
$$
\eta_i \colon \St(\Phi, s^{2p(i + k)} R) \to \St(\Phi, s^i R)
$$
by the following rule. For $\beta \in \Phi$ such that \(\alpha + \beta \notin \Phi \cup \{0\}\) let 
$$\eta_i \bigl(x_\beta(s^{2p(i + k)} b)\bigr) = x_\beta(s^{2p(i + k)}b).$$
If $\alpha + \beta$ is a root, let
$$
\eta_i\bigl(x_\beta(s^{2p(i + k)} b)\bigr) = x_{\alpha + \beta}(N_{\alpha \beta} s^{2p(i + k) - k} ab)\, x_\beta(s^{2p(i + k)} b).
$$
Finally, if $-\alpha = \beta + \gamma$ is any decomposition into a sum of roots, then let
$$
\eta_i\bigl(x_{-\alpha}(s^{2p(i + k)} b)\bigr) = \bigl[x_{-\gamma}(N_{\alpha\beta} s^{p(i + k) - k} ab)\,
x_\beta(s^{p(i + k)} b),
x_{-\beta}(N_{\alpha\gamma} N_{\beta\gamma} s^{p(i + k) - k} a)\,
x_\gamma(N_{\beta \gamma} s^{p(i + k)})\bigr].
$$

It is proved in [LSV] that the homomorphisms $\eta_i$ are well-deined and they give an action of $\St(\Phi, R_s)$ on $\St^\infty(\Phi, s^\bullet R)$.

Such consruction shows that the choice of strict representatives is functorial is some natural sense. 
\begin{corollary}
\label{vorcor}
Let $f \colon B \to A$ be a homomorphism of integral domains, $h \in B$, $s = f(h) \neq 0$, and $u \in \St(\Phi, B_h)$. Then there is a strict representative $\eta$ of the automorphism $\Xi(u)$ of the pro-group $\St^\infty(\Phi, h^\bullet B)$ and a strict representative $\theta$ of the automorphism $\Xi\bigl(f(u)\bigr)$ of the pro-group $\St^\infty(\Phi, s^\bullet A)$ such that $\eta^*(i) = \theta^*(i)$ and
$$f\bigl(\eta_i(x)\bigr) = \theta_i\bigl(f(x)\bigr)$$
for all $x \in \St(\Phi, h^{\eta^*(0)}B)$ and for all \(i\).
\end{corollary}

\section{A patching theorem for \texorpdfstring{$\K_2(\Phi, R)$}{K2(Ф,R)}}
Using Theorem~\ref{vor}, we will prove Theorem~\ref{glueing} about patching for $\St(\Phi,-)$. 
Similar result for projective modules has been originally proved by Lindel-Lutkebohrnert and independently by Mohan Kumar as an ingredient in the proof of Bass--Quillen conjecture for the power series ring over a field. 
Consequently, Lindel has used similar technique in his proof of Bass--Quillen problem for essentially smooth domains over a perfect field.
As in the previous section $\Phi$ is a simply-laced root system of rank at least $3$.

\begin{theorem}\label{glueing}
Let $B$ be a subring of an integral domain $A$, $h\in B \setminus \{0\}$ be such that $A / hA = B / hB$. Then the following square is pullback:
$$\begin{tikzcd}
\St(\Phi, B) \ar{r}{\iota} \ar{d}{\lambda_h^B} & \St(\Phi, A) \ar{d}{\lambda_h^A}\\
\St(\Phi, B_h) \ar{r}{\overline{\iota}} & \St(\Phi, A_h).
\end{tikzcd}$$
\end{theorem}
\begin{proof}
%It is well known that subgroup factorizations of $\mathrm G(\Phi,\,R)$ are an important ingredient in the study of the $\K_1$-functor. It turns out that in the study of $\K_2$-functors the $\St(\Phi,\,R)$-torsors play similar role, see~\cite{LS20}. TODO:

First of all, recall that the condition $A/hA = B/hB$ is equivalent to the fact that for every $n \geq 0$ one has $A = Ah^n + B$ and $Ah^n \cap B = Bh^n$. FOO

We let $\St(\Phi, B)$ act on $\St(\Phi, B_h) \times \St(\Phi, A)$ on the left via the formula 
\[w \star (u, v) = (u \lambda_h^B(w)^{-1}, \iota(w) v),\text{ where }w \in \St(\Phi, B),\ u \in \St(\Phi, B_h),\ v \in \St(\Phi, A).\]
Denote by $V$ the orbit set for this action. Our goal is to show that there is a bijection between $V$ and $\St(\Phi, A_h)$ defined by the formula $(u, v) \mapsto \overline{\iota}(u) \cdot \lambda_h^A(v)$.

We start by constructing an action of $\St(\Phi, A_h)$ on $V$.
Take $u \in \St(\Phi, B_h)$, $v \in \St(\Phi, A)$, $\alpha \in \Phi$, and $\frac c {h^s} \in A_h$. Choose a strict representative $\eta$ of the automorphism $\Xi(u)^{-1}$ of the pro-group $\St^\infty(\Phi, h^\bullet B)$ and a strict representative $\theta$ of the automorphism $\Xi\bigl(\iota(u)\bigr)^{-1}$ of the pro-group $\St^\infty(\Phi, h^\bullet A)$ using Corollart~\ref{vorcor} in such a way that $\eta^*(0) = \theta^*(0)$ and $$\iota\bigl(\eta_0(x)\bigr) = \theta_0\bigl(\iota(x)\bigr)$$ for all $x\in\St(\Phi, h^{\eta^*(0)}B)$.

Choose $n \geq \theta^*(0) + s$ and a decomposition $c = ah^n + b$ for $a \in A$, $b \in B$. Define
$$\textstyle
x_\alpha(\frac c {h^s}) (u, v) = \bigl(x_\alpha(\frac b {h^s})\, u, \theta_0(x_\alpha(ah^{n - s}))\, v\bigr).
$$
In lemma \ref{well-def} below we show that this construction is independent on all choices. Then the actions of all Steinberg generators $x_\alpha(\frac c {h^s})$ on $V$ are well-defined and it remains to check that they give an action of the Steinberg group $\St(\Phi, A_h)$.

At first we show that 
$$\textstyle
x_\alpha(\frac c {h^s})\, \bigl(x_\alpha(\frac{c'}{h^s})\, (u, v)\bigr) = x_\alpha(\frac{c+c'}{h^s})\, (u, v).
$$
Choose decompositions $c = ah^n + b$, $c' = a'h^n + b'$ as in the definition of the action, where \(n\) is sufficiently large. Since $x_\alpha(\frac{b'}{h^s})$ trivially acts on $x_\alpha(ah^{n-s})$ for sufficiently large $n$, there are strict representatives $\theta$ of $\Xi(\iota(u))^{-1}$ and $\theta'$ of $\Xi(x_\alpha(\frac{b'}{h^s})\, \iota(u))^{-1}$ such that
$$
\theta'_0\bigl(x_\alpha(ah^{n-s})\bigr) = \theta_0\bigl(x_\alpha(ah^{n-s})\bigr)
$$
and similarly for the corresponding $\eta$ and $\eta'$. Then
$$\textstyle
x_\alpha(\frac c {h^s})\, \bigl(x_\alpha(\frac{b'}{h^s})\, u, \theta_0(x_\alpha(a' h^{n-s}))\, v\bigr) = \bigl(x_\alpha(\frac{b + b'}{h^s})\, u, \theta'_0(x_\alpha(a h^{n - s}))\, \theta_0(x_\alpha(a' h^{n - s}))\, v\bigr),
$$
this proves the identity. The proof that $x_\alpha(\frac c {h^s})\, x_\beta(\frac{c'}{h^s})$ and $x_\beta(\frac{c'}{h^s})\, x_\alpha(\frac c {h^s})$ identically act for $\alpha + \beta \notin \Phi \cup \{0\}$ is the same.

In the case $\alpha + \beta \in \Phi$ we have to show that
$$\textstyle
f = x_\alpha(\frac c {h^s})\, x_\beta(\frac{c'}{h^s})
\text{ and }
g = x_{\alpha + \beta}(N_{\alpha \beta} \frac{cc'}{h^{2s}})\, x_\beta(\frac{c'}{h^s})\, x_\alpha(\frac c {h^s})
$$ 
act identically on $V$. Choose decompositions $c = ah^n + b$, $c' = a'h^n + b'$ and $cc' = (aa' h^n + ab' + a'b) h^n + bb'$ for sufficiently large \(n\).
Then
$$\textstyle
f\,(u, v) = \bigl(x_\alpha(\frac b {h^s})\, x_\beta(\frac{b'}{h^s})\, u,
\theta'_0(x_\alpha(ah^{n - s}))\, \theta_0(x_\beta(a'h^{n - s}))\, v\bigr),
$$
where $\theta$ is a strict representative of $\Xi(\iota(u))^{-1}$ and $\theta'$ is a strict representative of $\Xi(x_\beta(\frac{b'}{h^s})\, \iota(u))^{-1}$. It is possible to choose $\theta'$ in such a way that
$$
\theta'_0(x_\alpha(ah^{n - s})) = \theta_0\bigl(x_\alpha(ah^{n - s})\, x_{\alpha + \beta}(N_{\alpha\beta} ab'h^{n - 2s})\bigr).
$$
Next, 
\begin{multline*}\textstyle
g\, (u, v) = \bigl(x_{\alpha+\beta}(N_{\alpha \beta} \frac{bb'}{h^{2s}})\, x_\beta(\frac{b'}{h^s})\, x_\alpha(\frac b {h^s})\, u,\\
\theta'''_0\bigl(x_{\alpha + \beta}\bigl(N_{\alpha \beta} (aa' h^n + ab' + a'b) h^{n - 2s}\bigr)\bigr)\, \theta''_0(x_\beta(a'h^{n-s}))\, \theta_0(x_\alpha(ah^{n - s}))\, v\bigr),
\end{multline*}
where $\theta''$ is a strict representative of $\Xi\bigl(x_\alpha(\frac b {h^s})\, \iota(u)\bigr)^{-1}$ and $\theta'''$ is a strict representative of $\Xi\bigl(x_\beta(\frac{b'}{h^s})\, x_\alpha(\frac b{h^s})\, \iota(u)\bigr)^{-1}$. We may choose $\theta''$ and $\theta'''$ in such a way that 
$$
\theta''_0(x_\beta(a' h^{n - s})) = \theta_0\bigl(x_{\alpha + \beta}(-N_{\alpha \beta} a'bh^{n - 2s})\, x_\beta(a'h^{n - s})\bigr),
$$
and
$$
\theta'''_0\bigl(x_{\alpha + \beta}\bigl(N_{\alpha \beta} (aa'h^n + ab' + a'b) h^{n - 2s}\bigr)\bigr) = \theta_0\bigl(x_{\alpha + \beta}\bigl(N_{\alpha \beta} (aa'h^n + ab' + a'b) h^{n - 2s}\bigr)\bigr).
$$
It remains to check that
\begin{multline*}
x_\alpha(ah^{n - s})\, x_{\alpha + \beta}(N_{\alpha \beta} ab' h^{n - 2s})\, x_\beta(a' h^{n - s}) =\\
= x_{\alpha + \beta}\bigl(N_{\alpha \beta} (aa' h^n + ab' + a'b) h^{n - 2s}\bigr)\, x_{\alpha + \beta}(-N_{\alpha\beta} a'bh^{n - 2s})\, x_\beta(a'h^{n - s})\, x_\alpha(ah^{n - s}).
\end{multline*}
This is exactly a Steinberg relation.

It follows that the action of $\St(\Phi, A_h)$ on $V$ is well-defined. By construction, the canonical map \(V \to \St(\Phi, A_h)\) preserves the action, where \(\St(\Phi, A_h)\) acts on itself by left multiplication. Since \((1, 1) \in V\) maps to \(1 \in \St(\Phi, A_h)\), it remains to check transitivity of the action on \(V\). But the construction implies that \((u, v) = \iota(u)\, \bigl(\lambda_h(v)\, (1, 1)\bigr)\). Hence \(V \to \St(\Phi, A_h)\) is a bijection.

Now let \(u \in \St(\Phi, B_h)\) and \(v \in \St(\Phi, A)\) be such that \(\iota(u) = \lambda_h(v)\). The class of \((u, v^{-1})\) in \(V\) maps to \(1 \in \St(\Phi, A_h)\). Hence \((u, v^{-1}) \sim (1, 1)\).
\end{proof}

\begin{lemma}\label{well-def}
The action of \(x_\alpha(\frac c{h^s})\) on the element \((u, v) \in V\) is independent on the choices of a decomposition of \(c\), coherent strict representatives of \(\Sigma(u^{-1})\) and \(\Sigma(\iota(u^{-1}))\), and a representative \((u, v)\) of its equivalence class in \(V\).
\end{lemma}
\begin{proof}
Let us show that the action is independent on the choice of a decomposition of $c$. Indeed, let $k = \theta^*(0) + s$ and $c = a' h^k + b'$. Since $a' h^k - a h^n = b - b' \in Ah^k \cap B = Bh^k$, there is $d \in B$ such that $b = b' + dh^k$. Then
$$\textstyle
x_\alpha(\frac b {h^s})\, u = \bigl(x_\alpha(\frac{b'}{h^s})\, u\bigr) \bigl(u^{-1}\, x_\alpha(dh^{k - s})\, u\bigr).
$$
Since $\eta$ is a strict representative of $\Xi(u)^{-1}$, it follows that $$u^{-1}\, x_\alpha(dh^{k-s})\, u = \lambda_h \bigl(\eta_0(x_\alpha(dh^{k - s}))\bigr)$$
and
$$\textstyle
\bigl(x_\alpha(\frac b {h^s})\, u, \theta_0(x_\alpha(ah^{n - s}))\, v\bigr) \sim \bigl(x_\alpha(\frac{b'}{h^s})\, u, \iota\bigl(\eta_0(x_\alpha(dh^{k-s}))\bigr)\, \theta_0(x_\alpha(ah^{n - s}))\, v\bigr).
$$

A coherent choice of $\eta$ and $\theta$ guarantees that
$$
\iota\bigl(\eta_0(x_\alpha(dh^{k-s}))\bigr) = \theta_0(x_\alpha(dh^{k - s})),
$$
and, since $\theta_0$ is a homomorphism,
$$
\theta_0(x_\alpha(dh^{k - s}))\, \theta_0(x_\alpha(ah^{n - s})) = \theta_0(x_\alpha(dh^{k-s} + ah^{n - s})) = \theta_0(x_\alpha(a'h^{k - s})).
$$

Now we check that the action is independent on the choice of strict representatives. Let the strict representatives $\eta'$ and $\theta'$ be equivalent to $\eta$ and $\theta$ respectively. By definition of the equivalence, there is $n \geq \max(\theta^*(0) + s, {\theta'}^*(0) + s)$ such that 
$$
\theta_0(x_\alpha(ah^{n - s})) = \theta_0'(x_\alpha(ah^{n - s}))
$$
for all $a \in A$. Since the action is independent on the choice of a decomposition of $c$, it follows that it is also independent on the choices of $\eta$ and $\theta$.

Finally, we prove that the action is independent on the choice of $(u, v)$ in its equivalence class. Let $u = u' \lambda_h(w)$ for some $w \in \St(\Phi,\,B)$. Choose coherent strict representatives $\eta$, $\theta$, $\eta'$, $\theta'$ of $\Xi(u)^{-1}$, $\Xi(\iota(u))^{-1}$, $\Xi(u')^{-1}$, $\Xi(\iota(u'))^{-1}$, by Corollary~\ref{vorcor}. Clearly, for some strict representative $\zeta$ of $\Xi(\iota(w))^{-1}$ we have $\zeta_0(x) = \iota(w)^{-1}\, x\, \iota(w)$ for all $x \in \St(\Phi, h^{\zeta^*(0)} A)$.
Then
$$\textstyle
\bigl(x_\alpha(\frac b {h^s})\, u'\, \lambda_h(w), \theta_0(x_\alpha(ah^{n - s}))\, v\bigr) \sim \bigl(x_\alpha(\frac b {h^s})\, u', \iota(w)\, \theta_0(x_\alpha(ah^{n - s}))\, v\bigr),
$$
so
$$
\iota(w)\, \theta_0\bigl(x_\alpha(ah^{n-s})\bigr) = \theta'_0\bigl(x_\alpha(ah^{n-s})\bigr)\, \iota(w)
$$
for all \(a\) if \(n\) is large enough.
\end{proof}

\section{Main results}\label{sec:main}
\subsection{Analogue of Serre's problem for \texorpdfstring{$\K_2$}{K2}}

In this section $\Phi$ denotes an arbitrary irreducible root system.

\begin{lemma}
\label{k2cdc} 
The functors $\Gsc(\Phi,\,-)$, $\St(\Phi,\,-)$ and $\K_2(\Phi,\,-)$ commute with filtered colimits.
\end{lemma}
\begin{proof}
The assertion for $\mathrm G_{\mathrm{sc}}(\Phi,\,-)$ follows from the fact that it is represented in the category $\catname{Ring}$ by a finitely presented Hopf $\ZZ$-algebra (see e.\,g. \cite[Lemma~10.127.3]{stacks-project}). The assertion for $\St(\Phi, -)$ is obvious from its definition. 

The assertion for the functor $\K_2(\Phi, -)$ now follows from the assertions for $\St(\Phi, -)$ and $\Gsc(\Phi, R)$ using the fact that filtered colimits commute with finite limits in $\catname{Grp}$ (in particular, they commute with kernels).
\end{proof}

The following theorem has been proved by U.~Rehmann in \cite{Re75}, see Korollar of Satz 1.

\begin{theorem}
For an arbitrary field $F$ one has $\K_2(\Phi,\,F[t])\cong\K_2(\Phi,\,F)$.
\end{theorem}

\begin{theorem} \label{theorem:LP-for-K2}
Let $R$ be a regular ring containing a field $F$ and let $\Phi$ be either the root system $\rA_l$  for $l\geq4$, or $\rD_l$ for $l\geq7$. In the latter case assume additionally that the characteristic of $F$ is not $2$. Then one has
\[\K_2(\Phi,\,R[t])\cong\K_2(\Phi,\,R).\]
\end{theorem}
\begin{proof}
This is a direct consequence of~\ref{lpb}. One should only use that for $k=\mathbb Z$ or $k=\mathbb Z[\frac12]$ and a $k$-algebra $F$ which is a field, the prime subfield of $F$ is also a $k$-algebra and perfect, therefore $R$ is geometrically regular over it.
\end{proof}

\begin{corollary}
Let $F$ be a field $\mathrm{char}\,F\neq2$, $l\geq7$, then
$$
\mathrm H_2\big(\mathrm{Spin}_{2l}(F[t_1,\ldots,t_n]),\,\mathbb Z\big)=\cfrac{F^\times\otimes F^\times}{a\otimes(1-a)}.
$$
\end{corollary}
\subsection{An analogue of Gersten's conjecture for $\K_2$.} \label{sec:K2-GC}

\subsection{\texorpdfstring{$\K_2(\Phi, R)$}{K2(R)} as $\mathbb{A}^1$-fundamental group}
For a ring $R$ we denote by $R[\Delta^\bullet]$ the standard simplicial ring, see e.\,g.~\cite{Jar83}. The application of a functor $G\colon\catname{Ring}\rightarrow\catname{Grp}$ to $R[\Delta^\bullet]$ yields a simplicial group $G(R[\Delta^\bullet])$, which in the context of $\mathbb{A}^1$-homotopy theory is usually called the {\it simplicial resolution} of $G$ and is denoted by $\mathrm{Sing}^{\mathbb{A}^1}_\bullet(G)(R)$.

The aim of this section is to show that Stein's group $\K_2(\Phi, R)$ can be interpreted as the fundamental group of the simplicial group $\Gsc(\Phi, R[\Delta^\bullet])$ under the assumption that $\Phi$ and $R$ satisfy the requirements of~\cref{theorem:LP-for-K2}. This is achieved in~\cref{theorem:pi1-GRDelta} below. This result parallels the computation of the fundamental group of the simplicial group $G(k[\Delta^\bullet])$ modeled on an arbitrary isotropic reductive group $G$, see~\cite[Proposition~3.2]{VW16}. As a corollary of~\cref{theorem:pi1-GRDelta} and a general representability result of A.~Asok, M.~Hoyois and M.~Wendt we obtain an interpretation of $\K_2(\Phi, R)$ in terms of $\mathbb{A}^1$-homotopy theory, see~\cref{cor:motivic-pi1}.

Recall from~\cite[\S~17]{May67} that the homotopy group $\pi_n(G)$ of a simplicial group $(G_\bullet, d_i, s_i)$ can be computed as $n$-th homology group of the normalized Moore complex
\[
1 \leftarrow N_0(G) \xleftarrow{\partial_1} N_1(G) \xleftarrow{\partial_2} N_2(G) \xleftarrow{\partial_3} \ldots\,,
\]
\iffalse\[\begin{tikzcd} 1 & N_0(G) \ar[l] & N_1(G) \ar{l}[swap]{\partial_1} & N_2(G) \ar{l}[swap]{\partial_2} & \ar{l}[swap]{\partial_3} \ldots, \end{tikzcd} \] \fi
in which $N_n(G) = \cap_{i=1}^n\Ker(d_i) \trianglelefteq G_n$ and the differential $\partial_k$ is obtained from $d_0\colon G_k \to G_{k-1}$ by restriction. In other words, $\pi_n(G) \cong \mathrm{H}_n(N_\bullet G) = \Ker(\partial_n) / \Im(\partial_{n+1})$.

\begin{prop}\label{prop:pi1-StDelta} Let $\Phi$ be an arbitrary irreducible root system. Then for any ring $R$ the simplicial groups $\E(\Phi,\,R[\Delta^\bullet])$ and $\St(\Phi,\,R[\Delta^\bullet])$ are connected. If, moreover, $\Phi$ has rank at least $2$ then the simplicial group $\St(\Phi, R[\Delta^\bullet])$ is simply-connected. \end{prop}
\begin{proof}
In our computations we identify the ring $R[\Delta^n] = R[t_0,\ldots t_n]/\langle \sum_{i=0}^n t_i -1 \rangle$ with $R[t_1, \ldots, t_n]$ via $t_0 = 1 - \sum_{i=1}^n t_i$. We also use concrete formulas from~\cite{Jar83} for the face maps of $R[\Delta^\bullet]$.

The map $d_1\colon\St(\Phi,\,R[\Delta^1])\rightarrow\St(\Phi,\,R[\Delta^0])$ is given by $t_1\mapsto0$, therefore \[N_1\St(\Phi,\,R[\Delta^\bullet])=\Ker(d_1)=\St(\Phi,\,R[t_1],\,\langle t_1\rangle),\]
and $\partial_1\colon N_1\St(\Phi,\,R[\Delta^\bullet])\rightarrow N_0\St(\Phi,\,R[\Delta^\bullet])=\St(\Phi,\,R)$ is induced by $d_0$, which sends $t_1$ to $t_0=1$. 
It is clear that $x_\alpha(r)$ of $\St(\Phi, R)$ is the image of $x_\alpha(rt_1)$ under $\partial_1$.
This shows that $\St(\Phi, R[\Delta^\bullet])$ is connected. The argument for $\mathrm{E}(\Phi, R[\Delta^\bullet])$ is identical.

Now let us verify the second assertion. Notice that the kernel of $\partial_1$ coincides with the intersection $\overline{\St}(\Phi,\,R[t_1],\,\langle t_1\rangle)\cap\overline{\St}(\Phi,\,R[t_1],\,\langle t_1-1 \rangle )$, or, what is the same, with the kernel of the homomorphism
\[\St(\Phi,\,R[t_1])\rightarrow\St(\Phi,\,R)\times\St(\Phi,\,R)\]
sending $g(t_1)$ to $\big(g(0),\,g(1)\big)$. By~\cref{lem:fprod} this homomorphism can be identified with the homomorphism $\St(\Phi, R[t_1]) \to \St(\Phi, R\times R)$ induced by the ring homomorphism of evaluation of $t_1$ at $(0, 1)$.

Since the ideals $\langle t_1 \rangle$ and $\langle t_1-1 \rangle$ are coprime, we can identify $R\times R$ with $R[t_1]/t_1(t_1-1)$ by the Chinese remainder theorem. Thus, we obtain that $\Ker(\partial_1)$ coincides with $\overline{\St}(\Phi,\,R[t_1],\,\langle t_1(t_1-1) \rangle)$ and, in particular, is generated by $x_\alpha\big(t_1(t_1-1)f(t_1)\big)^{g(t_1)}$, where $\alpha \in \Phi$, $f\in R[t_1]$, $g \in \St(\Phi, R[t_1])$.

Notice that the face maps $d_1, d_2\colon\mathrm{St}(\Phi,\,R[\Delta^2])\rightarrow\mathrm{St}(\Phi,\,R[\Delta^1])$ are given by evaluations ($t_1\mapsto0$, $t_2\mapsto t_1$) and ($t_1\mapsto t_1$, $t_2\mapsto0$), respectively. Thus, we obtain that \[N_2\mathrm{St}(\Phi,\,R[\Delta^\bullet])=\Ker(d_1)\cap\Ker(d_2)=\overline{\St}(\Phi,\,R[t_1,\,t_2],\,\langle t_1\rangle )\cap\overline{\St}(\Phi,\,R[t_1,\,t_2],\,\langle t_2\rangle).\]
The differential $\partial_2$ is induced by the face map $d_0 \colon \St(\Phi, R[\Delta^2]) \to \St(\Phi, R[\Delta^1])$, which, in turn, is given by the evaluation ($t_1 \mapsto 1-t_1$, $t_2 \mapsto t_1$). 

It remains to see that the elements $x_{\alpha}\big(t_1t_2\,f(t_2)\big)^{g(t_2)}$ belong to $N_2\St(\Phi,\,R[\Delta^\bullet])$ and are mapped by $\partial_2$ onto the generating set of $\Ker(\partial_1)$ mentioned above. Thus, the normalized Moore complex for $\St(\Phi, R[\Delta^\bullet])$ is exact at $N_1$-term, which completes the proof of the proposition. \end{proof}

Let $f\colon G_\bullet\twoheadrightarrow Q_\bullet$ be a degreewise surjective morphism of simplicial groups (i.\,e. $f_n\colon G_n\to Q_n$ are surjective for all $n$). Recall from~\cite[Theorem~1.3]{Ina75} that in this situation the degreewise kernel $K_\bullet$ (i.\,e. the simplicial group given by $K_n = \Ker(f_n)$ with face and degeneracy maps induced from those of $G_\bullet$) yields a long exact sequence of groups
\begin{equation} \label{eq:simplicial-les} \begin{tikzcd} \ldots \ar{r} & \pi_{1}(G_\bullet) \ar{r} & \pi_1(Q_\bullet) \ar{r} & \pi_0(K_\bullet) \ar{r} & \pi_0(G_\bullet). \end{tikzcd} \end{equation}

\begin{theorem} \label{theorem:pi1-GRDelta}
 For $\Phi$ and $R$ as in~\cref{theorem:LP-for-K2} one has $\pi_1(\Gsc(\Phi, R[\Delta^\bullet])) = \K_2(\Phi, R)$.
\end{theorem}
\begin{proof}
First of all, notice that by the homotopy invariance for $\K_1$ (see e.\,g.~\cite[Theorem~1.3]{Sta14}) the simplicial group $\K_1(\Phi, R[\Delta^\bullet])$ is discrete. Applying the exact sequence~\eqref{eq:simplicial-les} to the canonical morphism $\Gsc(\Phi\,,R[\Delta^\bullet]) \to \K_1(\Phi\,,R[\Delta^\bullet])$, we obtain that $\pi_1(\Gsc(\Phi, R[\Delta^\bullet])) \cong \pi_1(\E(\Phi, R[\Delta^\bullet]))$.
 
Now consider the simplicial map $\mathrm{st}_\bullet \colon \St(\Phi, R[\Delta^\bullet]) \to \E(\Phi, R[\Delta^\bullet])$ given by canonical projections $\mathrm{st}$ in each degree. The application of~\eqref{eq:simplicial-les} yields an exact sequence of groups
\[
\pi_1\bigl(\St(\Phi, R[\Delta^\bullet])\bigr) \to \pi_1\bigl(\E(\Phi, R[\Delta^\bullet])\bigr) \to \pi_0\bigl(\K_2(\Phi, R[\Delta^\bullet])\bigr) \to \pi_0\bigl(\St(\Phi, R[\Delta^\bullet])\bigr).
\]
\iffalse \[
\begin{tikzcd} \pi_1(\St(\Phi, R[\Delta^\bullet])) \ar[r] & \pi_1(\E(\Phi, R[\Delta^\bullet])) \ar[r] & \pi_0(\K_2(\Phi, R[\Delta^\bullet])) \ar[r] & \pi_0(\St(\Phi, R[\Delta^\bullet])). \end{tikzcd}
\] \fi
 Proposition~\ref{prop:pi1-StDelta} implies that the first and the last groups in this exact sequence are trivial, so the two central groups are isomorphic. On the other hand, \cref{theorem:LP-for-K2} implies that $\K_2(\Phi, R[\Delta^\bullet])$ is discrete, so $\pi_0(\K_2(\Phi, R[\Delta^\bullet])) = \K_2(\Phi, R)$, as required.
\end{proof}

Now combining the above theorem with~\cite[Theorem~4.3.1]{AHW18} we obtain the following.
\begin{corollary} \label{cor:motivic-pi1} Let $k$ be an infinite field, $A$ be an arbitrary essentially smooth $k$-algebra. Then for $\Phi$ as above one has $\pi_1^{\mathbb{A}^1}(\Gsc(\Phi, -))(A) = [S^1 \wedge (\mathrm{Spec}(A))_+, \Gsc(\Phi, -)]_{ \mathbb{A}^1,*} = \K_2(\Phi, A).$
\end{corollary}

\printbibliography
\end{document}
