\documentclass[oneside, 11pt]{amsart} \pdfoutput=1
\usepackage{amsmath, mathtools, amssymb, amsthm, amscd, alltt, graphicx, comment}
\usepackage[utf8]{inputenc}
\usepackage[T1]{fontenc}
\usepackage[breaklinks=true,unicode]{hyperref}
\usepackage[capitalise]{cleveref}
\usepackage{tikz,tikz-cd}
\usepackage{scalerel}
\usepackage[a4paper, left=25mm, right=15mm, top=25mm, bottom=35mm]{geometry}
\usepackage{enumitem}
\usepackage[backend=bibtex, bibencoding=utf8, giveninits=true, citestyle=numeric-comp, sortlocale=en_US, url=false, doi=false, eprint=true, maxbibnames=4]{biblatex}

%\usepackage[notref, notcite]{showkeys}

\addbibresource{nisnevich2.bib}
\renewbibmacro*{volume+number+eid}{\ifentrytype{article}{\- \iffieldundef{volume}{}{Vol.~\printfield{volume},}\iffieldundef{number}{}{ No.~\printfield{number},}}}
\renewbibmacro{in:}{\ifentrytype{article}{}{\printtext{\bibstring{in}\intitlepunct}}}
\newbibmacro{string+doi}[1]{\iffieldundef{doi}{\iffieldundef{url}{#1}{\href{\thefield{url}}{#1}}}{\href{https://dx.doi.org/\thefield{doi}}{#1}}}
\DeclareFieldFormat[article, inproceedings, inbook, book, online]{title}{\usebibmacro{string+doi}{\mkbibquote{#1}}}
\renewcommand*{\bibfont}{\footnotesize}

\title{On the $\mathbb{A}^1$-invariance of $\K_2$ modeled on linear and even orthogonal groups.}
\keywords {Steinberg group, Chevalley group, unstable $\K_2$-functor, $\mathbb{A}^1$-fundamental group. {\em Mathematical Subject Classification (2010):} 19C09, 20G35, 20H05}

\author[1] {Andrei Lavrenov} \email{avlavrenov at gmail.com} 
\author[2] {Sergey Sinchuk} \email{sinchukss at gmail.com}
\author[3] {Egor Voronetsky} \email{voronetckiiegor at yandex.ru}
\address{St. Petersburg State University, 14th Line V.O., 29B, Saint-Petersburg, 199178 Russia}

\date {\today}

\renewcommand{\Im}{\mathop{\mathrm{Im}}\nolimits}
\newcommand{\Ker}{\mathop{\mathrm{Ker}}\nolimits}
\newcommand{\K}{{\mathrm{K}}}
\newcommand{\St}{\mathop{\mathrm{St}}\nolimits}
\newcommand{\E}{\mathrm{E}}
\newcommand{\Gsc}{\mathrm{G}_\mathrm{sc}}
\newcommand{\eval}{\mathrm{ev}}
\newcommand{\Max}{\mathrm{Max}}
\numberwithin{equation}{section}
\newtheorem{lemma}{Lemma} \numberwithin{lemma}{section}
\newtheorem*{lemma*}{Lemma}
\newtheorem{prop}[lemma]{Proposition} 
\newtheorem{theorem}[lemma]{Theorem}
\newtheorem{corollary}[lemma]{Corollary} 
\newtheorem*{theorem*}{Theorem} 
\newtheorem*{corollary*}{Corollary} 

\theoremstyle{definition} 
\newtheorem{df}[lemma]{Definition} 
\newtheorem{rem}[lemma]{Remark}

\newcommand{\Set}{\mathbf{Set}}
\newcommand{\Group}{\mathbf{Grp}}
\newcommand{\Rng}{\mathbf{Rng}}
\newcommand{\Fun}{\mathbf{Fun}}
\newcommand{\Mod}{\mathbf{Mod}}
\newcommand{\op}{\mathrm{op}}
\newcommand{\ZZ}{\mathbb{Z}}
\newcommand{\otimeshat}{\mathbin{\widehat{\otimes}}}
\newcommand{\up}[2]{{^{#1}\!{#2}}}
\newcommand{\rA}{\mathsf{A}}
\newcommand{\rB}{\mathsf{B}}
\newcommand{\rC}{\mathsf{C}}
\newcommand{\rD}{\mathsf{D}}
\newcommand{\rE}{\mathsf{E}}
\newcommand{\rF}{\mathsf{F}}
\newcommand{\rG}{\mathsf{G}}

\newcommand{\catname}[1]{{\normalfont\textbf{#1}}} %Category name
\begin{document}
\begin{abstract}
 Let $k$ be an arbitrary field. In this paper we show that in the linear case ($\Phi=\rA_\ell$, $\ell \geq 4$) and even orthogonal case ($\Phi = \rD_\ell$, $\ell\geq 7$, $\mathrm{char}(k)\neq 2$) the unstable functor $\mathrm{K}_2(\Phi, -)$ possesses the $\mathbb{A}^1$-invariance property in the geometric case, i.\,e. $\K_2(\Phi, R[t]) = \K_2(\Phi, R)$ for a regular ring $R$ containing $k$. As a consequence, the unstable $\K_2$ groups can be represented in the unstable $\mathbb{A}^1$-homotopy category $\mathcal{H}^{\mathbb{A}^1}_{k}$ as fundamental groups of the simply-connected Chevalley--Demazure group schemes $\mathrm{G}(\Phi,-)$. Our invariance result can be considered as the $\K_2$-analogue of the geometric case of Bass--Quillen conjecture. We also show for a semilocal regular $k$-algebra $A$ that $\K_2(\Phi, A)$ embeds as a subgroup into $\K^\mathrm{M}_2(\mathrm{Frac}(A))$.
\end{abstract}

\maketitle

\section{Introduction}
Many approaches to higher algebraic $\K$-theory and Hermitian $\K$-theory are known, so one might be interested in finding comparison results for them.
For example, recall from~\cite[Theorem~IV.11.8]{Kbook} that stable Quillen's groups $\K_n(R)$ (defined e.\,g. via the $+$-construction) and stable Karoubi--Villamayor groups $\mathrm{KV}_n(R)$ coincide for $n\geq 1$ if $R$ happens to be a regular ring.
These theories use infinite-dimensional algebraic groups such as $\mathrm{GL}_\infty(R)$ in their definition. 
The aim of this work is to obtain an {\it unstable} analogue of such result for the functor $\K_2$.

Recall that one can define and study the analogues of Quillen's and Karoubi--Villamayour's $\K$-groups for finite-dimensional algebraic groups, see e.\,g.~\cite{St71, Re75, St78, Su82, Tu83, Abe83, Jar83, Pa89, Sta14, VW16, Sta20}. The interest for the unstable $\K_2$-groups, in particular, comes from the fact that they appear in Steinberg's presentation of the groups of points of algebraic groups by means of generators and relations. Let us recall this presentation in greater detail. Let $R$ be a commutative ring with a unit and $\Phi$ be an irreducible root system of rank $\geq 2$. The Steinberg group $\St(\Phi, R)$ is the group defined by means of generators $x_\alpha(\xi)$ (where $\alpha\in\Phi$, $\xi\in R$) and Chevalley commutator formulas, see~\cite[Ch.~6]{St67} (see also~\cref{sec:main-results}).
If one additionally chooses a lattice $\Lambda$ located between the root lattice $Q(\Phi)$ and the weight lattice $P(\Phi)$ of $\Phi$ (i.\,e. $Q(\Phi) \leq \Lambda \leq P(\Phi)$), one can construct the Chevalley--Demazure group scheme $G = \mathrm{G}_\Lambda(\Phi, -)$, see~\cite[\S~3]{St67}. The {\it unstable functors $\K_1$ and $\K_2$ modeled on Chevalley groups} can be defined via the following exact sequence (cf.~\cite{St78}):
\begin{equation} \label{eq:main-exactSeq} \begin{tikzcd} \K_2^G(R) \ar[hookrightarrow]{r} & \St(\Phi, R) \ar{r}{\pi} & G(R) \ar[twoheadrightarrow]{r} & \K_1^G(R). \end{tikzcd} \end{equation}
Notice that the groups $\K_i(\Phi, R)$ from~\cite{St78} are precisely the groups $\K_i^G(R)$ in the case when $G$ is the simply-connected form of the Chevalley group, i.\,e. $\Lambda = P(\Phi)$ (for shortness we also write $\Gsc(\Phi,-)$ instead of $\mathrm{G}_{P(\Phi)}(\Phi, R)$). 
Thus, the group $\K_2^G(R)$ can interpreted as the group of ``nontrivial'' relations between the images of elements $x_\alpha(\xi)$ in $G(R)$, i.\,e. the relations depending on the arithmetic of $R$. Our first main result is the following
\begin{theorem}[The $\K_2$-analogue of Lindel--Popescu theorem] \label{theorem:LP-for-K2}
Let $k$ be an arbitrary field and $R$ be a regular ring containing $k$. Let $\Phi$ be either $\rA_\ell$ for $\ell\geq4$, or $\rD_\ell$ for $\ell\geq 7$. In the latter case assume additionally that the characteristic of $k$ is not $2$. Then for any lattice $\Lambda$ as above and $G = \mathrm{G}_\Lambda(\Phi, -)$ one has $\K_2^G(R[t])\cong\K_2^G(R).$
\end{theorem} 
Recall that the $\mathbb{A}^1$-invariance for the stable $\K$-functors $\K_i(R)$, $\mathrm{KO}_i(R)$, $i \geq 0$ is well-known, see e.\,g.~\cite[Theorem~V.6.3]{Kbook}, \cite[Corollary~1.12]{Ho05}. The classical Bass--Quillen conjecture asks if the $\mathbb{A}^1$-invariance for the functor $\K_0$ holds {\it unstably}, i.\,e. whether the set $\mathrm{VB}_n(R)$ of isomorphism classes of vector bundles of constant rank $n$ over $\mathrm{Spec}(R)$ has the property $\mathrm{VB}_n(R) = \mathrm{VB}_n(R[t])$. 
The classical Lindel--Popescu theorem for projective modules provides an affirmative answer to this question in the geometric case, i.\,e. when $R$ contains a field $k$, see~\cite{Li81},~\cite{Po85}. An exposition of D.~Popescu's result can also be found in~\cite{Sw98}. Much stronger results have appeared since Lindel--Popescu theorem, see e.\,g.~\cite{AHW17}.
Thus, \cref{theorem:LP-for-K2} can be considered as the solution to the $\K_2$-analogue of the Bass--Quillen conjecture in the geometric case for the split linear and orthogonal groups.
Recall also that the $\mathbb{A}^1$-invariance for the unstable $\K_1$-functor modeled on Chevalley groups of rank $\geq 2$ has been established in~\cite{Abe83} and has been subsequently generalized to reductive groups of isotropic rank $\geq 2$ in the recent works of A.~Stavrova, see~\cite[Theorem~1.3]{Sta14}, \cite[Theorem~1.1]{Sta20}.

When $R=k$ is a field the assertion of~\cref{theorem:LP-for-K2} is known from~\cite{Re75} for all $\Phi$ (see Korollar of~Satz~1). Also if $\Phi = \rA_\ell$ and $\ell \geq \mathrm{max}(4, \mathrm{dim}(R) + 2)$ the assertion of~\cref{theorem:LP-for-K2} follows from the main result of~\cite{Tu83}.

The assertion of~\cref{theorem:LP-for-K2} is likely to remain true for all $\Phi$ of rank $\geq 3$ without any assumptions on the characteristic of $k$. It is false for $\Phi$ of rank $2$, however, see~\cite{We12}. The bottleneck in our proof which leads to the restrictive assumptions $\ell\geq 7$ and $\mathrm{char}(k)\neq 2$ in the case $\Phi=\rD_\ell$ is the orthogonal Horrocks theorem \cite[Theorem~1]{LS20}. In the future papers we plan to address this issue and obtain an improved Horrocks theorem for all simply-laced $\Phi$ containing $\rA_4$ without the invertibility of $2$ assumption.

As a corollary of~\cref{theorem:LP-for-K2}, we show that the unstable groups $\K_2^G(R)$ coincide with Karoubi--Villamayor groups $\mathrm{KV}_2^G(R)$ (see~\cref{sec:K2-as-pi-1} for the definition). It follows from~\cite{AHW18} that the latter groups can be interpreted as fundamental groups of $G$ in the unstable $\mathbb{A}^1$-homotopy category $\mathcal{H}^{\mathbb{A}^1}_{k}$ of Morel and Voevodsky~\cite{MV99}. Consequently, we obtain the following
\begin{corollary} \label{cor:motivic-pi1} Let $R$ and $G$ be as in~\cref{theorem:LP-for-K2} and assume addtionally that $R$ is Noetherian of finite Krull dimension.
Then one has \[\pi_1^{\mathbb{A}^1}(G)(R) := \mathrm{Hom}_{\mathcal{H}^{\mathbb{A}^1}_{k,*}}(S^1 \wedge \mathrm{Spec}(R)_+, G) = \mathrm{KV}_2^{G}(R) = \K_2^G(R).\]
\end{corollary}
The above result is analogous to the computation of $\pi_0$ of an isotropic reductive group due to A.~Stavrova, see~\cite[Theorem~5.5]{Sta20}. The analogue of~\cref{cor:motivic-pi1} in the broader context of reductive groups of isotropic rank $\geq 2$ (albeit only when $R = k$ is an infinite field) has also been obtained in~\cite{VW16}.
Similar representability results for the stable $\K$-groups are well-known, see e.\,g.~\cite{MV99, Ho05}.

Recall also from the~\cite[Theorem~5.3]{St71} that under the assumptions on the rank $\Phi$ stated in~\cref{theorem:LP-for-K2} the first two homology groups of $\St(\Phi, R)$ vanish. Consequently from the main result of~\cite{LSV20} and \S~IV.1 of~\cite{Kbook} it is easy to conclude that the group $\K_2(\Phi, R)$ coincides with the unstable $\K_2$-group defined by means of Quillen's $+$-construction:
\begin{equation} \label{eq:H2-K2}
  \K_2^\mathrm{Q}(\Phi, R) := \pi_2(\Gsc(\Phi, R))^+_{\E_\mathrm{sc}(\Phi, R)} = \mathrm{H}_2(\mathrm{E}_\mathrm{sc}(\Phi, R), \ZZ) = \K_2(\Phi, R),
\end{equation}
where $\E_\mathrm{sc}(\Phi, R) = \mathrm{Im}(\pi)$ denotes the {\it elementary subgroup of $\Gsc(\Phi, R)$}, see~\cref{sec:main-results}. Thus, in particular, the definitions of the unstable $\K_2$-functors a la Quillen and a la Karoubi--Villamayor agree for a regular ring $R$ containing a field $k$ and $\Phi$ satisfying the assumptions of~\cref{theorem:LP-for-K2}.

One of the ingredients in the proof of~\cref{theorem:LP-for-K2} is~\cref{lpb}, which gives a sufficient condition for a general group-valued functor to be $\mathbb{A}^1$-invariant on regular $k$-algebras. In turn, the proof of~\cref{lpb} is based on a recent result of I.~Panin, see~\cite[Theorem~2.5]{Pa19}. 

The sufficient condition of~\cref{lpb} is given as a short list of axioms. The hardest of these to verify is the so-called {\it $\mathbb{P}^1$-glueing property} (also called Horrocks theorem, cf.~\cite{LS20}). Yet another axiom appearing in the statement of~\cref{lpb} is the so-called {\it weak affine Nisnevich excision property} for Steinberg groups. The verification of this property for simply-laced Steinberg groups is another new result of the paper, see~\cref{glueing}. Unlike M.~Tulenbaev, who used the technique of van der Kallen's ``another presentation'' in the proof of the weaker Zariski excision property~\cite[Proposition~1.4]{Tu83}, in our proof we use the new technique of pro-groups introduced by the third-named author in~\cite{Vor1} (see also~\cite{LSV20}). %V20?

In addition to~\cref{theorem:LP-for-K2} the proof of~\cref{cor:motivic-pi1} relies on the computation of the fundamental group of the simplicial group $\St(\Phi, R[\Delta^\bullet])$, see \cref{prop:pi1-StDelta}. This computation can be considered as a more precise version of the calculation appearing in the proof of~\cite[Proposition~3.2]{VW16} in the setting of Chevalley groups.

The method of the proof of~\cref{lpb} allows us to obtain the following assertion for the functor $\K_2(\Phi, R)$ as a byproduct, cf. e.\,g. with~\cite[Theorem~1.2]{Sta20}.
\begin{theorem} \label{theorem:Gersten} Let $\Phi$ and $k$ be as in~\cref{theorem:LP-for-K2} and let $A$ be an arbitrary semi-local regular domain containing $k$. Then the canonical map $\K_2(\Phi, A) \to \K_2(\Phi, \mathrm{Frac}(A))$ is injective. \end{theorem}

Combining the above result with the classical Matsumoto theorem~\cite[Theorem~5.10]{Ma69} one can show that under the assumptions of~\cref{theorem:LP-for-K2} $\K_2(\Phi, A)$ embeds into the Milnor $\K_2$-group $\K_2^\mathrm{M}(\mathrm{Frac}(A))$.

As a consequence~\cref{theorem:LP-for-K2} we obtain the following results.
\begin{corollary} \label{cor:various-facts}
For $k$ with $\mathrm{char}(k)\neq 2$, $\ell \geq 7$ and $n\geq 0$ the group $\mathrm{Spin}_{2\ell}(k[t_1,\ldots, t_n])$ admits presentation as in~\cite[\S~6]{St67} or \cite[\S~5]{Ma69}, i.\,e. the presentation by means of generators $x_\alpha(\xi)$, $\xi \in k[t_1,\ldots, t_n]$ relations~\eqref{R1}--\eqref{R3} and relations $h_\alpha(u)h_\alpha(v) = h_\alpha(uv)$, where $u, v\in k^\times$. Additionally,
\[\mathrm H_2(\mathrm{Spin}_{2\ell}(k[t_1,\ldots,t_n]),\,\mathbb Z\big) = \K^\mathrm{M}_2(k).\]
\end{corollary}
\begin{corollary} \label{cor:H_2-O}
 For an arbitrary field $k$ of characteristic $\neq 2$, a regular $k$-algebra $R$, $\ell \geq 7$ one has
 \begin{align}
  \mathrm H_2 (\mathrm{SO}_{2\ell}(R[t]), \ZZ) =&\ \mathrm H_2 (\mathrm{SO}_{2\ell}(R), \ZZ); \label{eq:H_2-SO} \\
  \mathrm H_2 (\mathrm{O}_{2\ell}(R[t]), \ZZ) =&\ \mathrm H_2 (\mathrm{O}_{2\ell}(R), \ZZ). \label{eq:H_2-O}
 \end{align}
\end{corollary}

\subsection{Acknowledgements} The authors would like to thank A. Stavrova for suggesting several ideas used in this paper. We would also like to thank P.~Gvozdevsky, S.~Ivanov, V.~Sosnilo and N.~Vavilov for their useful comments and interest in this work.

The work on \S~2 was supported by the Ministry of Science and Higher Education of the Russian Federation, agreement No. 075-15-2019-1619. The work on \S~3 was supported by the Russian Science Foundation grant No. 19-71-30002. The work on \S~4 was supported by the Foundation for the Advancement of Theoretical Physics and Mathematics “BASIS“. The first-named author is a winner of the Young Russian Mathematics contest and would like to thank its sponsors and the jury.

\section{General formalism}
For a commutative ring $k$ we denote by $\catname{Alg}_k$ the category of commutative unital $k$-algebras (not neccessarily of finite type). In the statement of~\cref{lpb}, which is the main result of this section, $k$ will be assumed to be a field. However, some of the intermediate steps in the proof of~\cref{lpb} can be obtained for $k=\ZZ$, so for the sake of greater generality we make no blanket assumption that $k$ is a field.  

For a commutative ring $R$ and an element $a\in R$ we denote by $\lambda_a$ the homomorphism of principal localization $R \to R_a$. Similarly, we denote by $\lambda_P$ the homomorphism $R \to R_P = (R\setminus P)^{-1}R$ of localization in a prime ideal $P \trianglelefteq R$. Notice that if $a$ is nilpotent then $R_a$ is the zero ring (which we also consider as an object of $\catname{Alg}_k$). 

Let $A$ be an $R$-algebra, $f\in R[t]$ and $a \in A$. We denote by $ev_t(a)$ (or just $ev(a)$ when $t$ is clear from the context) the unique $R$-algebra homomorphism $R[t] \to A$ mapping $t$ to $a$. For a functor $K \colon \catname{Alg}_k \to \catname{Grp}$ and an element $g \in K(R[t])$ we often shorten the notation for the element $K(ev(a))(g)$ to just $g(a)$.

\begin{df}\label{df:NK}
Let $K$ be a functor $\catname{Alg}_k \to \catname{Grp}$.
Denote by $NK(R)$ the kernel of the homomorphism $K(R[t]) \to K(R)$ induced by the homomorphism of evaluation at $t=0$. It is clear that $K(R[t]) \cong NK(R) \rtimes K(R)$ and $NK$ is also a functor $\catname{Alg}_k \to \catname{Grp}$. If $K(R[t])$ is abelian then $K(R[t]) \cong NK(R) \oplus K(R)$ and our definition agrees with~\cite[Def.~III.3.3]{Kbook}.
\end{df}

For a commutative ring $R$ and $a \in R$ consider the following commutative square:
  \begin{equation} \label{M-sq} \begin{tikzcd} R \ltimes t R_a[t] \ar{r}{l} \ar{d}[swap]{e} & R_a[t] \ar{d}{ev(0)} \\ R \ar{r}{\lambda_a} & R_a. \end{tikzcd}\end{equation}
  The ring $R \ltimes tR_a[t]$ can be defined either via the semidirect product construction (see e.\,g. \cite[Definition~3.2]{S15}) or as the pullback of~\eqref{M-sq}.
  It is also clear that~\eqref{M-sq} is a Milnor square, see~\cite[Example~I.2.6]{Kbook}.

Now suppose that $k$ is a field. Recall that a $k$-algebra $R$ is called {\it geometrically regular} over $k$, if for any finite field extension $E/k$ the ring $R\otimes_kE$ is regular (cf.~\cite[p.~137]{Sw98}). In particular, a geometrically regular algebra is noetherian. If $k$ is perfect then a $k$-algebra $R$ is geometrically regular over $k$ if and only if it is regular.

The following result gives a sufficient condition for a functor $K$ to be $\mathbb{A}^1$-invariant, cf. e.\,g. with~\cite[Proposition~2.2]{AHW20}.
\begin{theorem} \label{lpb}
 Let $k$ be a field.
 Suppose that a functor $K \colon \catname{Alg}_k \to \catname{Grp}$ satisfies the following axioms:
 \begin{enumerate}[label=\textnormal{(A\arabic*)}]
  \item \label{CFC} {\it $K$ is finitary}, i.\,e. commutes with filtered colimits.
  \item \label{DP} For a $k$-algebra $R$ and $a \in R$ consider the diagram obtained from~\eqref{M-sq} by applying $K$. Then the homomorphism $\Ker(K(e)) \to NK(R_a)$ between the kernels of vertical arrows is injective.
  \item \label{LPP} {\it $K$ satisfies weak affine Nisnevich excision for domains.} By this we mean the following. Let $\iota \colon B \hookrightarrow A$ be an injective homomorphism of domains contained in $\catname{Alg}_k$ such that $\mathrm{Spec}(A)\to\mathrm{Spec}(B)$ is an {\'e}tale morphism of affine schemes. Let $h$ be an element of $B$ not invertible in $A$ such that $\iota$ induces an isomorphism $B / hB \cong A / \iota(h)A$. Consider the following commutative square: \[\begin{tikzcd} B \ar{r}{\iota} \ar{d}{\lambda_h} & A \ar{d}{\lambda_{\iota(h)}}\\ B_h \ar{r}{\overline{\iota}} & A_h \end{tikzcd}\] Then the natural homomorphism \[\Ker(K(B) \to K(B_h)) \to \Ker(K(A) \to K(A_h))\] induced by $\iota$ is surjective.
  \item \label{PGP} {\it $K$ satisfies $\mathbb{P}^1$-glueing property for local domains.} By definition, this means that for every local domain $R \in \catname{Alg}_k$ the following diagram whose arrows are induced by natural embeddings is a pullback square: \begin{equation}\label{eq:P1-square} \begin{tikzcd} K(R) \ar[r] \ar[d] & K(R[t]) \arrow{d} \\ K(R[t^{-1}]) \ar{r} & K(R[t, t^{-1}]). \end{tikzcd} \end{equation}  
  It is easy to see that if~\eqref{eq:P1-square} is pullback, then all its arrows are injective.
  \item \label{HIF} {\it $K$ is homotopy invariant for fields}, i.\,e. $NK$ vanishes on every field $F \in \catname{Alg}_k$.
 \end{enumerate}
 Then $NK$ vanishes on every geometrically regular $R\in \catname{Alg}_k$. In other words, for every such $R$ the natural embedding $R \hookrightarrow R[t]$ induces an isomorphism $K(R)\cong K(R[t]).$
\end{theorem}
\begin{rem}
 In the literature a functor satisfying the axiom~\ref{PGP} is sometimes called {\it acyclic} (cf.~\cite[Def.~III.4.1.1]{Kbook}).
The choice of the name for~\eqref{LPP} is inspired by axiom (P3) from~\cite[Proposition~3.3.4]{AHW18}. It is easy to see that the application of $\mathrm{Spec}$ to the commutative square from~\ref{LPP} yields a distinguished Nisnevich square in the sense of~\cite[Definition~3.1.3]{MV99}. Notice also that our axioms can be slightly relaxed, so that the assertion of the Theorem becomes slightly stronger, see Remark~\ref{rem:relax} below.
\end{rem}

The proof is based on a series of lemmas and is deferred until the end of this section. We start by deducing the following useful corollary of the axiom~\ref{DP}. 

Let $k$ be an arbitrary commutative ring and $K\colon \catname{Alg}_k \to \catname{Grp}$ be a functor.
We say that $K$ satisfies the {\it Quillen--Suslin local-global principle} if for every $R\in \catname{Alg}_k$ the following map is injective:
\begin{equation} \label{QS-def} \begin{tikzcd} NK(R) \ar{r}{\prod \lambda_M} & \prod\limits_{M \in \Max(R)} NK(R_M). \end{tikzcd} \end{equation}
\begin{lemma}\label{LGP}
Suppose that $K$ is a finitary functor satisfying~\ref{DP}. Then $K$ satisfies the Quillen--Suslin local-global principle. In particular, $NK$ is a separated presheaf in the Zariski topology.
\end{lemma}

\begin{proof}
 Fix a $k$-algebra $R$ and $a \in R$. 
 Consider the following diagram of $k$-algebras. Its objects $A_i$ are copies of $R[t]$ indexed by natural numbers $i$. The only arrows of this diagram are the homomorphisms $ev (a^{j-i}t)\colon A_i \to A_j$ defined for $1 \leq i \leq j$. It is clear that this diagram is filtered and its colimit is $R \ltimes tR_a[t]$ (cf. e.\,g.~\cite[Lemma~15]{S15}). 
 
 The first step of the proof of the lemma is to show that for every $g \in \Ker(NK(R) \to NK(R_a))$ there exists some natural $n$ such that $g(a^nt)$ is the trivial element of $K(R[t])$. Let $g$ be such an element. From~\ref{DP} we obtain that the image of $g$ in $K(R \ltimes tR_a[t])$ is trivial. The required assertion now follows from~\ref{CFC} and the previous paragraph.
 
 The next step of the proof is to verify that for any coprime elements $a, b \in R$ the homomorphism $\langle K(\lambda_a), K(\lambda_b) \rangle \colon NK(R) \to NK(R_a) \times NK(R_b)$ is injective. Fix $g \in \Ker(\langle \lambda_a, \lambda_b \rangle)$. We argue as in the proof of~\cite[Lemma~2.5]{Tu83}. Set $S := R[t, t_1]$. Consider the element $h(t, t_1, t_2) = g(t_1 t) \cdot g((t_1 + t_2)t)^{-1}\in K(S[t_2]).$ Clearly, $h$ lies in the kernel of the homomorphism $K(ev_{t_2}(0))$. Since evaluation commutes with localization, we obtain that the element $K(\lambda_a)(h) \in K(S_a[t_2])$ is trivial, therefore by the previous paragraph there exists $n$ such that $h(t, t_1, a^nt_2)$ is trivial in $K(S[t_2])$. Similarly, we find $m$ such that $h(t, t_1, b^m t_2)$ is trivial. Since $a^n$ and $b^m$ are still coprime, we can find $x, y \in R$ such that $xa^n + yb^m = 1$. The required assertion now follows from the following calculation:
 $$1 = h(t, 1, -yb^m) \cdot h(t, xa^n, -xa^n) = g(t)\cdot g(xa^n\cdot t)^{-1} \cdot g(xa^n\cdot t) \cdot g(0)^{-1} = g(t).$$
 
 Now we can finish the proof of the lemma. We argue as in the proof of~\cite[Theorem~2]{S15}. Let $g$ be an element of the kernel of~\eqref{QS-def}. Denote by $Q(g)$ the set consisting of all elements $c \in R$ for which $K(\lambda_c)(g)$ is trivial. Let us check that this set is, in fact, an ideal. Fix $a, b \in Q(g)$ and let $c$ be an element of the ideal $\langle a, b \rangle$. Notice that $\overline{a} = \lambda_c(a)$, $\overline{b} = \lambda_c(b)$ are coprime elements of $R_c$. From the identities $\lambda_{\overline{a}}\lambda_c = \lambda_{\lambda_a(c)}\lambda_a$ and $\lambda_{\overline{b}}\lambda_c = \lambda_{\lambda_b(c)}\lambda_b$ we obtain that $g' = K(\lambda_c)(g)$ lies in the kernel of $\langle K(\lambda_{\overline{a}}), K(\lambda_{\overline{b}}) \rangle.$ By the previous paragraph, we obtain that $g' = 1$ and hence that $c \in Q(g)$. We have shown that $Q(g)$ is an ideal of $R$. If $Q(g)$ is proper then it is contained in a maximal ideal $M \trianglelefteq R$, in which case from $K(\lambda_M)(g) = 1$ and~\ref{CFC} we obtain that $K(\lambda_s)(g) = 1$ for some $s \in R \setminus M$. Thus, we obtain a contradiction, so $Q(g) = R$ and $g = 1$.
\end{proof}

First of all, notice that the local-global principle allows one to obtain the following global version of~\ref{PGP} (cf. e.\,g. with the proof of~\cite[Theorem~1]{LS20}). 
\begin{lemma}[$\mathbb{P}^1$-glueing property for domains] \label{ght}
Suppose that $K$ satisfies the Quillen--Suslin local-global principle and the $\mathbb{P}^1$-glueing property for local domains. Then for an arbitrary domain $R \in \catname{Alg}_k$ the square~\eqref{eq:P1-square} is a pullback square. In particular, all its arrows are injective. \end{lemma}
\begin{proof}
Let $R \in \catname{Alg}_k$ be an arbitrary domain. Suppose that the images of $g \in K(R[t])$ and $h \in K(R[t^{-1}])$ coincide in $K(R[t, t^{-1}])$. Then so do the images of $g' = g \cdot g(0)^{-1}$ and $h' = h \cdot g(0)^{-1}$. For a maximal ideal $M\trianglelefteq R$ set $g'_M := K(\lambda_M)(g') \in K(R_M[t])$ and $h'_M := K(\lambda_M)(h') \in K(R_M[t^{-1}]))$. Clearly, their images in $K(R_M[t, t^{-1}])$ coincide. By our assumptions, the square~\eqref{eq:P1-square} is pullback for $R_M$, therefore $g'_M = 1$ for all $M$. Since $g' \in NK(R)$, by the local-global principle we obtain that $g' = 1$, hence $g$ is the image of some element of $K(R)$. By symmetry, $h$ also is the image of some element of $K(R)$. Since $K(R)\to K(R[t, t^{-1}])$ is injective, we conclude that $g$ and $h$ are the images of the same element of $K(R)$, which completes the proof.  \end{proof}

Let us also note the following useful particular special case of weak affine Nisnevich excision.
\begin{lemma}[weak affine Zariski excision for domains]
	\label{zgl} Suppose that a functor $K\colon \catname{Alg}_k \to \catname{Grp}$ satisfies~\ref{LPP}. Let $R \in \catname{Alg}_k$ be a domain and $a, b$ be a pair of coprime elements of $R$. Consider the diagram
$$\begin{tikzcd}
	R \ar{r}{\lambda_a} \ar{d}[swap]{\lambda_b} & R_a \ar{d}{\overline{\lambda_b}}\\
	R_b \ar{r}{\overline{\lambda_a}} & R_{ab}.
\end{tikzcd}$$
	Then the natural map $\Ker(K(\lambda_b)) \to \Ker(K(\overline{\lambda_b}))$ induced by $\lambda_a$ is surjective.
\end{lemma}
\begin{proof}
	Set $B=R$, $A=R_a$ and $h=b$. It is clear that $\mathrm{Spec}(A)\to\mathrm{Spec}(B)$ is an {\'e}tale morphism. To see that the present situation is a special case of~\ref{LPP} it is enough to  check that the map $j\colon B/hB \to A/hA$ is an isomorphism. Let us first verify the surjectivity of $j$, or what is the same, the inclusion $A \subseteq Ah+B$. Fix an element $r/a^s\in A=R_a$, we need to show that it lies in $Ah+B$. We may assume that $s\geq 1$, so that $a^s$ and $b^s$ are still coprime. Choose $x, y \in R$ such that $xa^s+yb^s=1$. Thus, $r/a^s$ can be decomposed into the sum of $rx\in B$ and $ry(b/a)^s\in bR_a=hA$.

	Now let us verify the injectivity of $j$, i.\,e. the inclusion $Ah\cap B \subseteq Bh$. Suppose that $rb/a^s=c\in B=R$. We may assume $s\geq 1$ otherwise there is nothing to prove. From $rb=a^sc$ and $xa^s+yb^s=1$ we obtain the required inclusion $c=cxa^s+cyb^s=xrb+cyb^s=(xr+cyb^{s-1})b \in Rb = Bh$. 
\end{proof}

The following injectivity result is inspired by~\cite[Corollary~5.2]{Tu83}.
\begin{lemma} \label{lmp}
Suppose that $K$ satisfies weak affine Zariski excision and $\mathbb{P}^1$-glueing properties for domains. Then for any domain $R\in \catname{Alg}_k$ and any monic polynomial $f\in R[t]$ the localization homomorphism $\lambda_f\colon R[t]\rightarrow R[t]_f$ induces an injection $K(R[t])\hookrightarrow K(R[t]_f).$ \end{lemma}
\begin{proof}
	Fix a presentation $f=\sum_{i=0}^n a_it^i$, in which $a_n=1$. Set $$g=1+a_{n-1}t^{-1}+\ldots+a_0t^{-n}\in R[t^{-1}].$$ It is clear that $R[t, t^{-1}]_f \cong R[t, t^{-1}]_g$. Consider the following commutative diagram:
$$\begin{tikzcd}
	R[t] \ar{r}{\lambda_t} \ar{d}{\lambda_f} & R[t, t^{-1}] \ar{d}{\overline{\lambda}} & R[t^{-1}] \ar{l}[']{\lambda_{t^{-1}}} \ar{d}{\lambda_g}\\
	R[t]_f \ar{r} & R[t, t^{-1}]_f & R[t^{-1}]_g. \ar{l}
\end{tikzcd}$$
Let $x$ be an element of $\Ker(K(\lambda_f))$. Then $K(\lambda_t)(x)$ lies in $\Ker(K(\overline{\lambda}))$ so by~\cref{zgl} applied to the right square we find $y \in \Ker(K(\lambda_{g}))$ such that $K(\lambda_t)(x) = K(\lambda_{t^{-1}})(y)$. Now by~\cref{ght} $x$ and $y$ are images of some $z \in K(R)$. 
Since the composite homomorphism $K(R) \to K(R[t^{-1}]_g) \xrightarrow{K(ev_{t^{-1}}(0))} K(R)$ coincides with the identity map, we conclude that $z=1$, consequently $x = y = 1$ and the proof is complete. \end{proof}

For the rest of this section we assume that $k$ is a field.
Recall that a $k$-algebra $R$ is called {\it essentially smooth} if it is geometrically regular and essentially of finite type over $k$. 
Equivalently, $R$ is essentially smooth over $k$ if it is essentially of finite type over $k$ and $R\otimes_k\overline k$ is regular (see e.\,g.~\cite[p.~137]{Sw98}). 
If, moreover, $R$ is of finite type over $k$, it is called {\it smooth over $k$}.

\begin{theorem}[Panin]\label{thm:Panin}
\label{paninthm}
	Let $k$ be a field, $R$ be a domain smooth over $k$. Let $M_1,\ldots, M_n$ be a finite set of maximal ideals of $R$. Denote by $A=R_{M_1,\ldots,M_n}$ the corresponding semi-localization of $R$. Let $f$ be an element of the intersection  $\cap_{i=1}^nM_i \subseteq R$. Then there exist a monic polynomial $h(t)\in A[t]$, a domain $S$ essentially smooth over $k$ and homomorphisms $\tau$, $p$, $p'$ and $\delta$ such that $\tau^*\colon\mathrm{Spec}(S)\to\mathrm{Spec}(A[t])$ is an {\'e}tale morphism and the following diagram commutes:
\begin{equation}\label{eq:panin-diag}
 \begin{tikzcd}[column sep=4em]
   & A & \\ A[t] \ar{d}{\lambda_h} \ar{ru}{ev(0)} \ar{r}{\tau} & S \ar{u}{\delta} \ar{d}{\lambda_{\tau(h)}}  & R \ar{ul}[swap]{\lambda_{M_1,\ldots,M_n}} \ar{d}{\lambda_f} \ar{l}[swap]{p} \\
   A[t]_h \ar{r}[swap]{\tau\otimes_{A[t]}A[t]_h}              & S_{\tau(h)}      & R_f. \ar{l}{p'}\end{tikzcd}
\end{equation}
Additionally, the homomorphism $\tau$ is an injection and one has $A[t]/hA[t]\cong S/\tau(h)S.$ 
\end{theorem}
\begin{proof}
 All of the assertions except the last two are a direct ring-theoretic restatement of (i)--(iii) of~\cite[Theorem~2.5]{Pa19}.
	Since $\tau^*$ is open and $S\neq 0$, $\tau(S)$ is dominant. Denote by $I$ the kernel of $\tau$. The closed subscheme defined by $I$ must contain the image of $\tau^*$ therefore $I$ is radical. Since $A[t]$ is a domain we conclude that $I=0$.		

	Set $X = \mathrm{Spec}(A[t]/h)$, $Y = \mathrm{Spec}(S/h)$. We need to show that the morphism $\tau' \colon Y \to X$ obtained from $\tau^*$ by restriction is an isomorphism. Set $Y_0 := Y \times_{X} X_\mathrm{red}$. From the fact that the left square is an elementary distinguished Nisnevich square we obtain that the composite morphism $Y_\mathrm{red} \not\hookrightarrow Y_0 \xrightarrow{\tau''} X_\mathrm{red}$ is an isomorphism, consequently since $\tau''$ is {\'e}tale, the closed embedding $(Y_0)_\mathrm{red} = Y_\mathrm{red} \not\hookrightarrow Y_0$ is also {\'e}tale, hence an open embedding, hence an isomorphism (see e.\,g. Corollary~3.6, Proposition~3.10 of~\cite{Mi80}). Thus, we conclude that $\tau''$ is an isomorphism and hence by the topological invariance of {\'e}tale morphisms~\cite[Theorem~3.23]{Mi80}, so is $\tau'$.
\end{proof}

\begin{corollary}
\label{esssmooth}
Let $k$ be a field.
Suppose that $K\colon\catname{Alg}_k\rightarrow\catname{Grp}$ is finitary and satisfies weak affine Nisnevich excision and $\mathbb{P}^1$-glueing properties. Let $R, M_1, \ldots, M_n, A$ be as in the statement of the above theorem. Denote by $E$ the fraction field of $A$ {\rm(}which coincides with the fraction field of $R${\rm)}.
Then for any $m\geq 0$ the natural homomorphism $K(A[x_1,\ldots, x_m])\to K(E[x_1,\ldots,x_m])$ is injective.
\end{corollary}
\begin{proof}
We argue as in the proof of~\cite[Theorem~3.2]{Sta20}. Set $B=k[x_1,\ldots x_m]$. Let $g$ be an element of the kernel of $K(B \otimes_k A)\rightarrow K(B \otimes_k E)$. First of all, notice that both $A$ and $E$ are filtered colimits of principal localizations of $R$. Since $K$ is finitary, there exists $f' \in \cap_{i=1}^n(R \setminus M_i)$ such that $g$ is the image of some $g_1 \in K(R_{f'}[x_1,\ldots x_m])$ under $K(\lambda_{M_1',\ldots M_n'})$, where $M_i' = R_{f'} \cdot M_i$. Set $R' := R_{f'}$.
Clearly, there exists $f \in R'$ such that $K(\lambda_{f})(g_1) = 1$ in $K(R'_{f})$. 
Without loss of generality, we may assume that $f \in \cap_{i=1}^n M_i'$. 

We apply Theorem~\ref{paninthm} to the ring $R'$, maximal ideals $M_i'$ and the polynomial $f$ as above.
Tensoring~\eqref{eq:panin-diag} with $B$ and applying functor $K$ we obtain the following commutative diagram (we use the convention that tensoring with $B$ does not change the notation for the arrows of the diagram):
\begin{equation*} 
 \begin{tikzcd}[column sep=4em]
   & K(B \otimes_k A) & \\ 
   K(B \otimes_k A[t]) \ar{d}{K(\lambda_h)} \ar{ru}{K(ev_t(0))} \ar{r}{K(\tau)} & K(B \otimes_k S) \ar{u}{K(\delta)} \ar{d}{K(\lambda_{\tau(h)})}  & K(B \otimes_k R') \ar{ul}[swap]{K(\lambda)} \ar{d}{K(\lambda_f)} \ar{l}[swap]{K(p)} \\
   K(B \otimes_k A[t]_h) \ar{r}[swap]{}              & K(B \otimes_k S_{\tau(h)})      & K(B \otimes_k R'_f). \ar{l}{}\end{tikzcd}
\end{equation*}
Notice that $K(p)(g_1)$ lies in the kernel of $K(\lambda_{\tau(h)})$, therefore by~\ref{LPP} there exists $g_2\in K(B \otimes_k A[t])$ such that $K(\lambda_h)(g_2)=1$ and $K(\tau)(g_2)=K(p)(g_1)$. Since $h \in A[t]$ is monic, by~\cref{lmp} the homomorphism $K(\lambda_h)$ is injective, therefore $g_2=1$. It remains to see that
$$g=K(\lambda)(g_1)=K(\delta)(K(p)(g_1))=K(\delta)(K(\tau)(g_2))=g_2(0)=1.\qedhere$$
\end{proof}

\begin{rem}
For the proof of~\cref{lpb} we only need the special cases $m=0,1$ of the above result.
\end{rem}

\begin{theorem}[Popescu]
\label{popescu} Let $k$ be a field, and $R$ a ring geometrically regular over $k$. Then $R$ is a filtered colimit of smooth $k$-algebras. \end{theorem}
\begin{proof} See~\cite{Po85}, ~\cite[Theorem~1.1]{Sw98}.
\end{proof}

Now we are ready to finish the proof of \cref{lpb}. %Our argument follows~\cite{Sta14, Vo81}.
\begin{proof}[Proof of~\cref{lpb}]
Our goal is to prove the triviality of $NK(R)$.
By~\cref{popescu} $R$ is a filtered colimit of smooth $k$-algebras.
Since filtered colimits commute with finite limits and the functor $K$ is finitary, the functor $NK$ is also finitary.
Thus, it suffices to verify the triviality of $NK(R)$ for a smooth $k$-algebra $R$.
Further, by~\cref{LGP} we are left to prove that $NK(R_M)$ is trivial for every maximal ideal $M$ in a such an algebra.
Since every smooth scheme over $k$ is a disjoint union of smooth irreducible $k$-schemes, we may assume, without loss of generality, that $R$ is a domain.
Denote by $E$ the fraction field of $R_M$. Consider the diagram
\[\begin{tikzcd}
K\bigl(R_M[t]\bigr) \ar[hookrightarrow]{r} \ar{d}{ev_t(0)} & K(E[t]) \ar{d}{\cong}\\
K\bigl(R_M\bigr) \ar[hookrightarrow]{r} & K(E),
\end{tikzcd}\]
in which the right vertical arrow is an isomorphism by~\ref{HIF} and the horizontal arrows are injective by Corollary~\ref{esssmooth}.

Thus, the left vertical arrow is also injective and $NK(R_M) = 1$, as required.
\end{proof}
\begin{rem}\label{rem:relax}
It is clear from the above proof that it possible to weaken the requirements on $K$ in the statement of~\cref{lpb}. For example, we could require that axioms \ref{DP}--\ref{PGP} hold true only for the algebras essentially smooth over $k$. Further, the assertion of the axiom~\ref{CFC} can be restricted to algebras geometrically regular over $k$.
\end{rem}

We can also prove the following result (cf.~\cite[Th{\'e}or{\`e}me~1.1]{CTO92}, cf. also with~\cite[Proposition~2.2]{AHW20}).
\begin{theorem}\label{thm:gb}
Let $k$ be a field.
Suppose that $K\colon\catname{Alg}_k\rightarrow\catname{Grp}$
is a finitary functor satisfying weak affine Nisnevich excision and $\mathbb{P}^1$-glueing properties for domains. 
Let $A$ be a semi-local domain geometrically regular over $k$. 
Denote by $E$ the fraction field of $A$.
Then the natural homomorphism $K(A) \to K(E)$ is injective.
\end{theorem}
\begin{proof}
By Theorem~\ref{popescu} $A$ is a filtered colimit of smooth $k$-algebras. Since $A$ is a domain we may assume additionally that these smooth $k$-algebras are domains.

Since $K$ is finitary, we may assume that $g \in \Ker(K(A) \to K(E))$ comes from some element $g_1\in\mathrm K(R)$, where $R$ is a smooth $k$-domain. We may assume, additionally, that $g_1$ lies in the kernel of the natural homomorphism $K(R) \to K(\mathrm{Frac}(R))$.

Denote by $P_i$ the preimages of the maximal ideals of $A$ and choose some maximal ideals $M_i$ of $R$ such that $M_i$ contains $P_i$. Then by \cref{esssmooth}, $g_1$ vanishes in the semi-localization of $R$ in the set of maximal ideals $M_i$, therefore it vanishes in the semi-localization of $R$ in the set of prime ideals $P_i$. Since $R\rightarrow A$ factors through the latter semi-localization, we conclude that $g=1$. 
\end{proof}

\section{Proof of the main results} \label{sec:main-results}
\subsection{Proof of Theorems~\ref{theorem:LP-for-K2} and~\ref{theorem:Gersten}}
Throughout this section $\Phi$ is an irreducible root system of rank $\geq 2$ and $R$ is a commutative ring with a unit. Recall that in the case when $\Phi$ is simply-laced the Steinberg group $\St(\Phi, R)$ can be defined by means of the generators $x_\alpha(\xi)$, $\xi \in R$ and the following set of defining relations:
\begin{align}
x_{\alpha}(a)\cdot x_{\alpha}(b)&=x_{\alpha}(a+b),\text{ for } \alpha\in \Phi;\tag{R1} \label{R1}\\
[x_{\alpha}(a),\,x_{\beta}(b)]  &=1,\text{ for }\alpha+\beta\not\in\Phi\cup 0; \tag{R2} \label{R2} \\
[x_{\alpha}(a),\,x_{\beta}(b)]  &=x_{\alpha+\beta}(N_{\alpha\beta} \cdot ab)\text{ in the case }\alpha+\beta\in\Phi. \tag{R3} \label{R3} \end{align}
The integers $N_{\alpha,\beta} = \pm 1$ appearing in the last identity are called the {\it structure constants} of the Chevalley group $\mathrm{G}_\mathrm{sc}(\Phi, R)$. They coincide with the structure constants of the corresponding simple Lie algebra. More information about Steinberg groups, structure constants and further references can be found e.\,g. in~\cite[\S~3]{St71}, \cite[\S~2.4]{LSV20}. Notice that the above definition of the Steinberg group makes sense even if $R$ does not have a unit unit, this will be important later in~\cref{sec:patching}.

We denote by $\E_\Lambda(\Phi, R)$ the {\it elementary subgroup} of $G = \mathrm{G}_\Lambda(\Phi, R)$. It is defined as the abstract subgroup generated by elementary root unipotents $t_\alpha(\xi)$, $\alpha \in \Phi$, $\xi \in R$, see e.\,g.~\cite{St71, St78, VZ20, Abe83,}. It is clear that $\E_\Lambda(\Phi, R)$ is the image of the homomorphism $\pi$ from~\eqref{eq:main-exactSeq}. We write $\E_\mathrm{sc}(\Phi, R)$ instead of $\E_{P(\Phi)}(\Phi, R)$.

For $a, b \in R$ set $y_\alpha(a, b) = [x_\alpha(a), x_{-\alpha}(b)]$.
For an ideal $I \trianglelefteq R$ we denote by $\overline{\St}(\Phi, R, I)$ the normal closure in $\St(\Phi, R)$ of the subgroup generated by $x_\alpha(a)$, $a\in I$. Recall from~\cite[Lemma~5]{S15} that
\begin{equation} \label{eq:relative-st-ker}
 \overline{\St}(\Phi, R, I) = \Ker(\St(\Phi, R) \to \St(\Phi, R/I)).
\end{equation}

\begin{lemma} \label{lem:c-identities} For an arbitrary irreducible root system $\Phi$ of rank $\geq 2$, arbitrary ideals $A, B \trianglelefteq R$ and all $a \in A$, $b \in B$, $c \in R$ the following congruences hold:
\begin{itemize}
 \item $y_\alpha(a, cb) \equiv y_\alpha(ac, b)\ (\mathrm{mod}\ \overline{\St}(\Phi, R, AB))$ in the case when either $\Phi \neq \rC_{\ell}$ or $\alpha$ is short.
 \item $y_\alpha(a, c^2b) \equiv y_\alpha(ac^2, b)\ (\mathrm{mod}\ \overline{\St}(\Phi, R, AB))$, $y_\alpha(a, cb)^2 \equiv y_\alpha(ac, b)^2\ (\mathrm{mod}\ \overline{\St}(\Phi, R, AB))$ in the case $\Phi = \rC_{\ell}$ and $\alpha$ is long.
\end{itemize} \end{lemma}
\begin{proof}
 Observe that the proof of~\cite[Theorem~5]{VZ20} is based solely on computations with Chevalley commutator formula, which all can be reproduced verbatim for Steinberg groups.
\end{proof}

Notice that the Steinberg group functor {\it does not} commute with general finite limits. However, it satisfies the following weaker property, which we are going to use in the sequel.
\begin{lemma} \label{lem:fprod} For an arbitrary irreducible root system $\Phi$ of rank $\geq 2$ the Steinberg group functor $\St(\Phi, -)$ commutes with finite direct products. \end{lemma}
\begin{proof} 
It suffices to verify the assertion for binary direct products.
Observe that canonical projections $R_1 \times R_2 \to R_i$, $i=1,2$ split in the category of rings without unit, therefore the groups $G_i = \St(\Phi, R_i)$ embed as subgroups into $G = \St(\Phi, R_1 \times R_2)$. It is also clear that $G_i$ together generate $G$. Thus, to verify the isomorphism $G \cong G_1 \times G_2$ it suffices to show the triviality of the commutator subgroup $[G_1, G_2] \leq G$.

Set $A = R_1\times 0$, $B = 0 \times R_2$. It is clear that $A, B \trianglelefteq R_1 \times R_2$. 
It follows directly from~\eqref{R2}--\eqref{R3} that the commutators $[x_{\alpha}(a),\ x_\beta(b)]$ are trivial for all $\beta \neq -\alpha$, $a \in A$, $b\in B$. On the other hand, to verify the triviality of $y_\alpha(a, b)$ we can apply the congruences of~\cref{lem:c-identities} (since $AB=0$ these congruences turn into equalities).
Indeed, setting $c = c^2 = (1, 0)$ we obtain that $y_\alpha(a, b) = y_\alpha(ac, b) = y_\alpha(a, bc) = y_\alpha(a, 0) = 1$. \end{proof}

\begin{lemma} \label{k2cdc} 
For an arbitrary root system $\Phi$ the functors $\Gsc(\Phi,\,-)$, $\St(\Phi,\,-)$ and $\K_2(\Phi,\,-)$ commute with filtered colimits.
\end{lemma}
\begin{proof}
The assertion for $\mathrm G_{\mathrm{sc}}(\Phi,\,-)$ follows from the fact that it is represented in the category $\catname{Ring}$ by a finitely presented Hopf $\ZZ$-algebra (see e.\,g. \cite[Lemma~10.127.3]{stacks-project}). The assertion for $\St(\Phi, -)$ is obvious from its definition. The assertion for the functor $\K_2(\Phi, -)$ now follows from the assertions for $\St(\Phi, -)$ and $\Gsc(\Phi, R)$ using the fact that filtered colimits commute with finite limits in $\catname{Grp}$ (in particular, they commute with kernels).
\end{proof}

\begin{proof}[Proof of Theorems~\ref{theorem:LP-for-K2} and~\ref{theorem:Gersten}]
Let us verify that the axioms \ref{CFC}--\ref{HIF} are satisfied for the functor $\K_2(\Phi, -)$. Axiom~\ref{CFC} is satisfied by~\cref{k2cdc}. Axiom~\ref{DP} follows from the fact that for a simply-laced $\Phi$ of rank $\geq 3$ Steinberg groups $\St(\Phi, -)$ satisfy Tulenbaev lifting property in the sense of~\cite[Definition~2.1]{LS17} (more precisely, \ref{DP} follows from the injectivity of the left arrow in the diagram~(2.2) ibid.).
The axiom~\ref{LPP} is checked in~\cref{sec:patching} below.
In the case $\Phi = \rA_\ell$ the axiom~\ref{PGP} is satisfied by~\cite[Theorem~5.1]{Tu83}. In the orthogonal case it is satisfied by~\cite[Theorem~1]{LS20}. Finally, axiom~\ref{HIF} is satisfied by the Korollar of \cite[Satz~1]{Re75}.

Notice that every regular ring containing a field $k$ is geometrically regular over a perfect subfield of $k$. Now the assertion of~\cref{theorem:LP-for-K2} in the simply-laced case ($\Lambda = P(\Phi)$) follows from~\cref{lpb}, while the assertion of~\cref{theorem:Gersten} follows from~\cref{thm:gb}.

It remains to verify that the assertion of~\cref{theorem:LP-for-K2} holds for the Chevalley group $G = G_\Lambda(\Phi, -)$ of non-simply-connected type. Notice that there is an exact sequence of elementary subgroups
\begin{equation} \label{eq:E-seq} \begin{tikzcd} 1 \ar[r] & M(R) \ar[r] & \E_\mathrm{sc}(\Phi, R) \ar[r] & \E_\Lambda(\Phi, R), \end{tikzcd} \end{equation}
where $M$ is either $\mu_n$ for some $n$ or $\mu_2\times \mu_2$ (cf.~\cite[\S~3]{St67}).
From the snake lemma we obtain the rows of the diagram
\begin{equation}\label{eq:snake-A1} \begin{tikzcd} 
      1  \ar[r] & \K_2(\Phi, R)   \ar{d}{\cong} \ar[r]  & \K_2^G(R) \ar{d}{i_G} \ar[r] & M(R) \ar[r] \ar{d}{i_M} & 1 \\
      1  \ar[r] & \K_2(\Phi, R[t])\ar[r] & \K_2^G(R[t]) \ar[r] & M(R[t]) \ar[r] & 1.
    \end{tikzcd} \end{equation}
We already know that the left arrow is an isomorphism. The homomorphism $i_M$ is also an isomorphism since $R$ is reduced. Thus, $i_G$ is an isomorphism by 5-lemma.
\end{proof}

\begin{proof}[Proof of~\cref{cor:various-facts}]
  Set $R = k[t_1,\ldots t_n]$. Recall from~\cite{Abe83} that $\K_1(\rD_\ell, R) = \K_1(\rD_\ell, k) = 1$, so $\mathrm{Spin}_{2\ell}(R) = \E_\mathrm{sc}(\rD_\ell, R)$. The presentation a la Steinberg for $\mathrm{Spin}_{2\ell}(R)$ now follows from~\cref{theorem:LP-for-K2} and Matsumoto's theorem~\cite[Theorem~5.10]{Ma69}. The formula for the second homology group follows from~\eqref{eq:H2-K2}. \end{proof}
 
 \begin{proof}[Proof of~\cref{cor:H_2-O}]
  We will give the proof for~\eqref{eq:H_2-O}, the proof for~\eqref{eq:H_2-SO} being similar.
  From the homological Lyndon--Hochschild--Serre spectral sequence associated with the short exact sequence $\mathrm{EO}_{2\ell}(R) \hookrightarrow \mathrm{O}_{2\ell}(R) \twoheadrightarrow \mathrm{KO}_{1, 2\ell}(R)$ we obtain an exact sequence
  \[ \begin{tikzcd} \mathrm H_3(\mathrm{KO}_{1,2\ell}(R)) \ar[r] & \mathrm H_2(\mathrm{EO}_{2\ell}(R))_{\mathrm{KO}_{1,2\ell}(R)} \ar[r] & \mathrm H_2(\mathrm{O}_{2\ell}(R)) \ar[r] &  \mathrm H_2(\mathrm{KO}_{1, 2\ell}(R)) \ar[r] & 1, \end{tikzcd} \]
  where the lower index in the second term denotes the coinvariants of the action of $\mathrm{KO}_{1,2\ell}(R)$.
  The functor $\mathrm{KO}_{1, 2\ell}(-)$ is $\mathbb{A}^1$-invariant (e.\,g. by~\cite[Theorem~1.1]{Sta20}).
  The functor $\mathrm H_2(\mathrm{EO}_{2\ell}(-))$ clearly coincides with $\mathrm{KO}_{2, 2\ell} := \K_2^{\mathrm O_{2\ell}}$ and is also $\mathbb{A}^1$-invariant by~\cref{theorem:LP-for-K2}. Thus, by 5-lemma we conclude that the central term is also $\mathbb{A}^1$-invariant.
 \end{proof}
 

\subsection{\texorpdfstring{$\K_2(\Phi, R)$}{K2(R)} as $\mathbb{A}^1$-fundamental group} \label{sec:K2-as-pi-1}
Recall from~\cite{Jar83} that for an arbitrary commutative (unital) ring $R$ one can define the standard simplicial ring $R[\Delta^\bullet]$ as follows:
\begin{equation}
 R[\Delta^n] = R[t_0,\ldots t_n]/\langle \sum_{i=0}^n t_i -1 \rangle,\ d_i(t_j) = \left\{ \begin{array}{ll}t_j & j < i, \\0 & j = i, \\ t_{j-1} & j > i; \end{array}\right. s_i(t_j) = \left\{ \begin{array}{ll} t_j & j < i, \\ t_j + t_{j+1} & j = i, \\ t_{j+1} & j > i. \end{array} \right.
\end{equation}
	The application of a functor $G\colon\catname{Ring}\rightarrow\catname{Grp}$ to $R[\Delta^\bullet]$ yields a simplicial group $G(R[\Delta^\bullet])$, which in the context of $\mathbb{A}^1$-homotopy theory is usually called the {\it simplicial resolution} of $G$ and is denoted by $\mathrm{Sing}^{\mathbb{A}^1}_\bullet(G)(R)$. Recall also, that the $n$-th {\it unstable Karoubi--Villamayor K-functor attached to $G$} (denoted $\mathrm{KV}^G_n(R)$) is, by definition, the $n-1$-th simplicial homotopy group of $\Gsc(\Phi, R[\Delta^\bullet])$ (cf.~\cite[\S~4.3]{AHW18}, \cite[\S~3]{Jar83}).

Recall from~\cite[\S~17]{May67} that the homotopy group $\pi_n(G)$ of a simplicial group $(G_\bullet, d_i, s_i)$ can be computed as $n$-th homology group of the normalized Moore complex
\[ 1 \leftarrow N_0(G) \xleftarrow{\partial_1} N_1(G) \xleftarrow{\partial_2} N_2(G) \xleftarrow{\partial_3} \ldots\,, \]
in which $N_n(G) = \cap_{i=1}^n\Ker(d_i) \trianglelefteq G_n$ and the differential $\partial_k$ is obtained from $d_0\colon G_k \to G_{k-1}$ by restriction. In other words, $\pi_n(G) \cong \mathrm{H}_n(N_\bullet G) = \Ker(\partial_n) / \Im(\partial_{n+1})$.

\begin{prop}\label{prop:pi1-StDelta} Let $\Phi$ be an arbitrary irreducible root system. Then for any commutative ring $R$ the simplicial groups $\E_\Lambda(\Phi,\,R[\Delta^\bullet])$ and $\St(\Phi,\,R[\Delta^\bullet])$ are connected. If, moreover, $\Phi$ has rank at least $2$ then the simplicial group $\St(\Phi, R[\Delta^\bullet])$ is simply-connected. \end{prop}
\begin{proof}
In our computations we identify the ring $R[\Delta^n]$ with $R[t_1, \ldots, t_n]$ via $t_0 = 1 - \sum_{i=1}^n t_i$ and use~\eqref{eq:relative-st-ker} repeatedly.
The map $d_1\colon\St(\Phi,\,R[\Delta^1])\rightarrow\St(\Phi,\,R[\Delta^0])$ is given by $t_1\mapsto0$, therefore \[N_1\St(\Phi,\,R[\Delta^\bullet])=\Ker(d_1)=\overline{\St}(\Phi,\,R[t_1],\,\langle t_1\rangle),\]
and $\partial_1\colon N_1\St(\Phi,\,R[\Delta^\bullet])\rightarrow N_0\St(\Phi,\,R[\Delta^\bullet])=\St(\Phi,\,R)$ is induced by $d_0$, which sends $t_1$ to $t_0=1$. 
It is clear that $x_\alpha(r)$ of $\St(\Phi, R)$ is the image of $x_\alpha(rt_1)$ under $\partial_1$.
This shows that $\St(\Phi, R[\Delta^\bullet])$ is connected. The argument for $\E_\Lambda(\Phi, R[\Delta^\bullet])$ is identical.

Now let us verify the second assertion. Notice that the kernel of $\partial_1$ coincides with the intersection $\overline{\St}(\Phi,\,R[t_1],\,\langle t_1\rangle)\cap\overline{\St}(\Phi,\,R[t_1],\,\langle t_1-1 \rangle )$, or, what is the same, with the kernel of the homomorphism
$\St(\Phi,\,R[t_1])\rightarrow\St(\Phi,\,R)\times\St(\Phi,\,R)$
sending $g(t_1)$ to $\big(g(0),\,g(1)\big)$. By~\cref{lem:fprod} this homomorphism can be identified with the homomorphism $\St(\Phi, R[t_1]) \to \St(\Phi, R\times R)$ induced by the ring homomorphism of evaluation of $t_1$ at $(0, 1)$.

Since the ideals $\langle t_1 \rangle$ and $\langle t_1-1 \rangle$ are coprime, we can identify $R\times R$ with $R[t_1]/t_1(t_1-1)$ by the Chinese remainder theorem. Thus, we obtain that $\Ker(\partial_1)$ coincides with $\overline{\St}(\Phi,\,R[t_1],\,\langle t_1(t_1-1) \rangle)$ and, in particular, is generated by $x_\alpha\big(t_1(t_1-1)f(t_1)\big)^{g(t_1)}$, where $\alpha \in \Phi$, $f\in R[t_1]$, $g \in \St(\Phi, R[t_1])$.

Notice that the face maps $d_1, d_2\colon\mathrm{St}(\Phi,\,R[\Delta^2])\rightarrow\mathrm{St}(\Phi,\,R[\Delta^1])$ are given by evaluations ($t_1\mapsto0$, $t_2\mapsto t_1$) and ($t_1\mapsto t_1$, $t_2\mapsto0$), respectively. Thus, we obtain that \[N_2\mathrm{St}(\Phi,\,R[\Delta^\bullet])=\Ker(d_1)\cap\Ker(d_2)=\overline{\St}(\Phi,\,R[t_1,\,t_2],\,\langle t_1\rangle )\cap\overline{\St}(\Phi,\,R[t_1,\,t_2],\,\langle t_2\rangle).\]
The differential $\partial_2$ is induced by the face map $d_0 \colon \St(\Phi, R[\Delta^2]) \to \St(\Phi, R[\Delta^1])$, which, in turn, is given by the evaluation ($t_1 \mapsto 1-t_1$, $t_2 \mapsto t_1$). 

It remains to see that the elements $x_{\alpha}\big(t_1t_2\,f(t_2)\big)^{g(t_2)}$ belong to $N_2\St(\Phi,\,R[\Delta^\bullet])$ and are mapped by $\partial_2$ onto the generating set of $\Ker(\partial_1)$ mentioned above. Thus, the normalized Moore complex for $\St(\Phi, R[\Delta^\bullet])$ is exact at $N_1$-term, which completes the proof of the proposition. \end{proof}

Let $f\colon G_\bullet\twoheadrightarrow Q_\bullet$ be a degreewise surjective morphism of simplicial groups (i.\,e. $f_n\colon G_n\to Q_n$ are surjective for all $n$). Recall from~\cite[Theorem~1.3]{Ina75} that in this situation the degreewise kernel $K_\bullet$ (i.\,e. the simplicial group given by $K_n = \Ker(f_n)$ with face and degeneracy maps induced from those of $G_\bullet$) yields a long exact sequence of groups
\begin{equation} \label{eq:simplicial-les} \begin{tikzcd} \ldots \ar{r} & \pi_{1}(G_\bullet) \ar{r} & \pi_1(Q_\bullet) \ar{r} & \pi_0(K_\bullet) \ar{r} & \pi_0(G_\bullet). \end{tikzcd} \end{equation}

\begin{theorem} \label{theorem:pi1-GRDelta}
 For $G$ and $R$ as in~\cref{theorem:LP-for-K2} one has $\pi_1(G(R[\Delta^\bullet])) = \K_2^G(R)$.
\end{theorem}
\begin{proof}
First of all, notice that by the homotopy invariance for $\K_1$ (see e.\,g.~\cite[Theorem~1.1]{Sta20}) the simplicial group $\K_1^G(R[\Delta^\bullet])$ is discrete. Applying the exact sequence~\eqref{eq:simplicial-les} to the canonical morphism $G(R[\Delta^\bullet]) \twoheadrightarrow \K_1^G(R[\Delta^\bullet])$, we obtain that $\pi_1(G( R[\Delta^\bullet])) \cong \pi_1(\E_\Lambda(\Phi, R[\Delta^\bullet]))$.

Now consider the simplicial map $\mathrm{st}_\bullet \colon \St(\Phi, R[\Delta^\bullet]) \to \E_\Lambda(\Phi, R[\Delta^\bullet])$ given by canonical projections $\mathrm{st}$ in each degree. The application of~\eqref{eq:simplicial-les} yields an exact sequence of groups
\[
\pi_1\bigl(\St(\Phi, R[\Delta^\bullet])\bigr) \to \pi_1\bigl(\E_\Lambda(\Phi, R[\Delta^\bullet])\bigr) \to \pi_0\bigl(\K_2^G( R[\Delta^\bullet])\bigr) \to \pi_0\bigl(\St(\Phi, R[\Delta^\bullet])\bigr).
\]
 Proposition~\ref{prop:pi1-StDelta} implies that the first and the last groups in this exact sequence are trivial, so the two central groups are isomorphic. On the other hand, \cref{theorem:LP-for-K2} implies that $\K_2^G(R[\Delta^\bullet])$ is discrete, so $\pi_0(\K_2^G(R[\Delta^\bullet])) = \K_2^G(R)$, as claimed.
\end{proof}

\begin{proof}[Proof of~\cref{cor:motivic-pi1}] 
 By~\cite[Corollary~5.4]{Sta20} for every regular $k$-algebra $R$ one has \[\mathrm H^1_\mathrm{Nis}(R, \Gsc(\Phi, -)) = \mathrm H^1_\mathrm{Nis}(R[t], \Gsc(\Phi, -)),\] consequently by~\cite[Theorem~2.4.2]{AHW18} the $k$-scheme $\Gsc(\Phi, -)$ defines an $\mathbb{A}^1$-naive presheaf on $\mathrm{Sm}_k^{aff}$ in the sense of~\cite[Definition~2.1.1]{AHW18}.
 In particular, one has $\pi_1^{\mathbb{A}^1}(\Gsc(\Phi, -))(R) = \pi_1(\Gsc(\Phi, R[\Delta^\bullet]))$, the latter group being isomorphic to $\K_2(\Phi, R)$ by the above theorem.
\end{proof}

\section{Nisnevich glueing for \texorpdfstring{$\K_2(\Phi, R)$}{K2(Ф,R)}} \label{sec:patching}
The main result of this section is the following glueing theorem for Steinberg groups.
\begin{theorem}\label{glueing}
Let $\iota\colon B \hookrightarrow A$ be an embedding of integral domains, $h\in B \setminus \{0\}$ be such that $B / hB = A / \iota(h)A$. Let $\Phi$ be an arbitrary simply-laced root system of rank at least $3$.
Then the group $\St(\Phi, B)$ surjects onto the pullback of the homomorphisms 
$\lambda_{\iota(h)}\colon \St(\Phi, A) \to \St(\Phi, A_{\iota(h)})$, $\overline{\iota}\colon\St(\Phi, B_h) \to \St(\Phi, A_{\iota(h)})$. In particular the functors $\St(\Phi, -)$, $\K_2(\Phi, -)$ satisfy weak affine Nisnevich excision for domains and the following sequence of pointed sets is exact in the central term
\begin{equation} \begin{tikzcd} \K_2(\Phi, B) \ar{rrr}{g \mapsto (\lambda_h(g), \iota(g))} & &  & \K_2(\Phi, B_h) \times \K_2(\Phi, A) \ar{rrrr}{(g_1,g_2) \mapsto \overline{\iota}(g_1) \lambda_{\iota(h)}(g_2)^{-1}} & & & & \K_2(\Phi, A_{\iota(h)}) \end{tikzcd} \end{equation}
\end{theorem}
The above result generalizes~\cite[Proposition~1.4]{Tu83}, which is formulated for the Zariski topology and the special case $\Phi=\rA_{\geq 4}$.
The analogues of the above theorem for the functor $\K_1$ have also been obtained by E.~Abe and A.~Stavrova, cf.~\cite[Lemma~3.7]{Abe83}, \cite[Lemma~3.4]{Sta14}.
The analogous result for projective modules is the ``descent lemma'' \cite[Lemma~4.7]{Bh99} due to W.~Lutkebohmert, H.~Lindel and M.~Kumar. 

\subsection{Overview of Steinberg pro-groups}
We refer the reader to \cite[Section~6.1]{SK06} for an introduction to the formalism of ind-objects and pro-objects. We also refer to~\cite[\S~2]{LSV20} for a more detailed exposition of the material presented in this subsection.

We start by recalling the definition of the {\it pro-completion} $\catname{Pro}(\mathcal{C})$ of a category $\mathcal{C}$ (cf.~\cite[\S~2.1]{LSV20}).
By definition, the objects of $\catname{Pro}(\mathcal{C})$ are functors $X^{(\infty)}\colon\mathcal{I}^{\mathrm{op}} \to \mathcal{C}$, where $\mathcal{I}$ is a small nonempty filtered category. For $X^{(\infty)} \in 
\catname{Pro}(\mathcal{C})$ and $i \in \mathcal{I}$ we denote by $X^{(i)}$ the value of $X^{(\infty)}$ on $i$. 
We call the images of $X^{(\infty)}$ on the arrows of $\mathcal{I}$ the {\it structure morphisms} of $X^{(\infty)}$.

Let $X^{(\infty)}\colon\mathcal{I}\to\mathcal{C}$ and $Y^{(\infty)}\colon\mathcal{J}\to\mathcal{C}$ be a pair of pro-objects.
By definition, a {\it pre-morphism} $\eta\colon X^{(\infty)} \to Y^{(\infty)}$ is a pair $(\eta^*, \{\eta_j\}_{j\in\mathrm{Ob}(\mathcal{J})})$ consisting of a map $\eta^*\colon \mathrm{Ob}(\mathcal{J})\to\mathrm{Ob}(\mathcal{I})$ and a collection of $\mathcal{C}$-morphisms $\eta_j\colon X^{(\eta^*(j))}\to Y^{(j)}$ such that for every arrow $j' \to j$ there exists $i$ such that the composite morphisms $X^{(i)} \to X^{(\eta^*(j'))} \xrightarrow{\eta_{j'}} Y^{(j')}$ and $X^{(i)} \to X^{(\eta^*(j))} \xrightarrow{\eta_j} Y^{(j)} \to Y^{(j')}$ are equal. We call the individual $\mathcal{C}$-morphism $\eta_j$ {\it the degree $j$ component} of $\eta$.
Pre-morphisms $\eta, \epsilon \colon X^{(\infty)} \to Y^{(\infty)}$ are declared equivalent if for every $j \in \mathrm{Ob}(\mathcal{J})$ there exists $i \in \mathrm{Ob}(\mathcal{I})$ such that the composite morphisms $X^{(i)} \to X^{(\eta^*(j))} \xrightarrow{\eta_j} Y^{(j)}$ and $X^{(i)} \to X^{(\epsilon^*(j))} \xrightarrow{\epsilon_j} Y^{(j)}$ are equal. By definition, the set of morphisms $\catname{Pro}(\mathcal{C})(X^{(\infty)}, Y^{(\infty)})$ consists of equivalence classes $[\eta]$ of pre-morphisms $\eta \colon X^{(\infty)} \to Y^{(\infty)}$ with respect to the equivalence relation defined above. The composition of a pair of morphisms \([\eta] \colon X^{(\infty)} \to Y^{(\infty)}\) and \([\zeta] \colon Y^{(\infty)} \to Z^{(\infty)}\) is defined as the class $[\zeta \circ \eta]$, where the pre-morphism \(\zeta \circ \eta\) is given by \((\zeta \circ \eta)^*(i) = \eta^*(\zeta^*(i))\) and \((\zeta\circ \eta)_{i} = \zeta_{i} \circ \eta_{\zeta^*(i)}\). There is a fully faithful embedding $\mathcal{C}\to\catname{Pro}(\mathcal{C})$ sending $X \in \mathrm{Ob}(\mathcal{C})$ to the obvious functor $X\colon \{ * \} \to \mathcal{C}$. 

Let $R$ be an integral domain and $S$ a multiplicative subset of $R$ containing $1$. 
We denote by $\lambda_S$ the canonical localization homomorphism $R \to R_S$.
We also reuse the same notation for the induced homomorphism $\St(\Phi, R) \to \St(\Phi, R_S)$ of Steinberg groups.

Notice that $S$ can be interpreted as a filtered category with $\mathrm{Ob}(S) = S$ and $S(s, t) = \{ u \in S \mid su = t \}$.
For shortness, we call a commutative ring without unit an {\it rng} and denote the category of rngs by $\catname{Rng}$.
For $s\in S$ the principal ideal $sR$ can be considered as an rng. We define the pro-rng $R^{(\infty)}$ as the functor $S^\mathrm{op} \to \catname{Rng}$ whose value on an arrow $(s \to t)\in S$ is defined as the obvious embedding of rngs $tR \hookrightarrow sR$. Obviously, $R^{(\infty)}$ depends on $S$, however for shortness we suppress $S$ from the notation.

Recall that the {\it Steinberg pro-group} $\St^{(\infty)}(\Phi, R)$ is defined as the functor $S^\mathrm{op} \to \catname{Grp}$ obtained from the pro-rng $R^{(\infty)}$ by postcomposition with $\St(\Phi, -)$ (cf.~\cite[\S~2.4]{LSV20}). We denote by $\pi_S$ the pro-group morphism $\St^{(\infty)}(\Phi, R) \to \St(\Phi, R_S)$ defined by the pre-morphism $\pi^* \colon \{* \} \to S$, $\pi^*(*) = 1$, $\pi_{*} = \lambda_S \colon \St(\Phi, R) \to \St(\Phi, R_S)$. 

For $\alpha \in \Phi$ we define the pro-group morphism $x_\alpha \colon R^{(\infty)} \to \St^{(\infty)}(\Phi, R)$ by means of the following data: $(x_\alpha)^* = \mathrm{id}_S$, $(x_\alpha)_s \colon sR \to \St(\Phi, sR)$, $sr \mapsto x_\alpha(sr)$. These morphisms, which we call {\it root subgroup morphisms}, allow one to characterize the Steinberg pro-group $\St^{(\infty)}(\Phi, R)$ by means of a universal property (cf.~\cite[Lemma~2.16]{LSV20}). For instance, $\St^{(\infty)}(\Phi, R)$ is ``generated'' by $x_\alpha$ in the sense that a pair of pro-group morphisms $f, g \colon \St^{(\infty)}(\Phi, R) \to G^{(\infty)}$ coincide if and only if $f \circ x_\beta = g \circ x_\beta$ for all $\beta \in \Phi$.
It is enough to verify the latter equalities for $\beta \in \Phi\setminus\{\alpha\}$ where $\alpha \in \Phi$ is any fixed root (in fact, an even stronger result holds, see~\cite[Lemma~3.2]{LSV20}). % TODO: \cite[Lemma~11]{V20}?
Moreover, given a collection of pro-group morphisms $f_\alpha \colon R^{(\infty)} \to G^{(\infty)}$ satisfying the ``pro-analogues'' of Chevalley commutator identities (see~\cite[Remark~2.15]{LSV20}), there exists a unique morphism $f \colon \St^{(\infty)}(\Phi, R) \to G^{(\infty)}$ such that $f_\alpha = f \circ x_\alpha$ for all $\alpha\in\Phi$. 
\begin{prop}\label{prop:conj-action}
 For an irreducible root system $\Phi$ of rank $\geq 3$ and a multiplicative system $S\subseteq R$ there exists a group homomorphism $conj\colon \St(\Phi, R_S) \to \mathrm{Aut}(\St^{(\infty)}(\Phi, R))$ compatible with the obvious conjugation action of $\St(\Phi, R_S)$ on itself, i.\,e. such that for every $g \in \St(\Phi, R_S)$ the following diagram of pro-groups commutes:
 \begin{equation} \label{eq:conj-pis} \begin{tikzcd} \St^{(\infty)}(\Phi, R) \ar{r}{conj(g)} \ar{d}{\pi_S} & \St^{(\infty)}(\Phi, R) \ar{d}{\pi_S} \\ \St(\Phi, R_S) \ar{r}{{}^g\!(-)} & \St(\Phi, R_S). \end{tikzcd} \end{equation}
\end{prop}
\begin{proof}
The existence of the homomorphism $conj$ has been demonstrated in~\cite[Proposition~4.2]{LSV20}.
 %TODO: (for $\Phi\neq\rB_\ell, \rC_\ell$) and the beginning of Section~10 of~\cite{V20} (for $\Phi=\rB_\ell, \rC_\ell$). 
Thus, we only need to verify the commutativity of the diagram~\eqref{eq:conj-pis}. 
Since $conj$ is a homomorphism, it is enough to consider the case $g = x_\alpha(u)$, $\alpha \in \Phi$, $u \in R_S$. In view of the above discussion, to verify the commutativity of~\eqref{eq:conj-pis} it is enough to verify the equalities $\pi_S \circ conj(g) \circ x_\beta= {}^g\!(-) \circ \pi_S \circ x_\beta$ for $\beta\neq -\alpha$.
 It remains to see that the fulfillment of these equalities is a direct consequence of the definition of $conj(g) \circ x_\beta$ (the formulas for these morphisms are given by formulas (4.5)--(4.6) of~\cite{LSV20}). \end{proof} %TODO: cf. also~\cite[\S~10]{V20}
\begin{rem} \label{rem:conj-action}
Unwinding the definitions, we see that the commutativity of~\eqref{eq:conj-pis} is equivalent to the following condition: for every $g\in \St(\Phi, R_S)$ there exists an element $s_g\in S$ and a homomorphism $c_g \colon \St(\Phi, s_g R) \to \St(\Phi, R)$ such that $\lambda_S(c_g(x)) = g \cdot \lambda_S(i(x))\cdot g^{-1}$ for all $x\in \St(\Phi, s_g R)$. Here $i$ denotes the homomorphism $\St(\Phi, s_g R) \to \St(\Phi, R)$ induced by the obvious embedding $s_gR \hookrightarrow R$. The element $s_g$ and the homomorphism $c_g$ can be determined via the following procedure: choose a representative $\eta$ for the morphism $conj(g)$ and set $s_g := \eta^*(1)$ and $c_g := \eta_1$ (the degree $1$ component of $\eta$).
\end{rem}

\begin{rem} \label{rem:conj-root-action}
 Notice also that the root subgroup morphisms $x_\alpha \colon R^{(\infty)} \to \St^{(\infty)}(\Phi, R)$ satisfy the identity $conj(x_\alpha(u)) \circ x_\alpha = x_\alpha$ for any $u \in R_S$.
\end{rem}

We also need to study basic functoriality properties of the above construction.
Let $R, R'$ be a pair of domains, $S \subseteq R$, $S' \subseteq R'$ be a pair of multiplicative subsets and let $f \colon R \to R'$ be a homomorphism such that $f(S)\subseteq S'$.
It is clear that $f$ induces a ring homomorphism $\overline{f}\colon R_S \to R'_{S'}$, we also reuse the notation $\overline{f}$ for the induced homomorphism of Steinberg groups.
Now suppose that there is a map $f^* \colon S' \to S$ such that $t$ divides $f(f^*(t))$ for all $t \in S'$ (without loss of generality, we may assume that $f^*(1)=1$ ). The latter condition means, in particular, that $R'_{f(S)} \cong R'_{S'}$. We claim that $f^*$ and the homomorphisms $\{f^*(t)R \to tR'\}_{t\in S'}$ obtained from $f$ by restricting its domain and codomain specify a pre-morphism $R^{(\infty)} \to R'^{(\infty)}$. We denote by $f^{(\infty)}$ the corresponding morphism of pro-rngs and also use the same notation for the induced morphism of Steinberg pro-groups.

\begin{prop} \label{prop:functoriality}
 The action $conj$ constructed in Proposition~\ref{prop:conj-action} has the additional property that the following diagram commutes for every $g \in \St(\Phi, R_S)$:
 \begin{equation} \label{eq:conj-finfty} \begin{tikzcd} \St^{(\infty)}(\Phi, R) \ar{r}{conj(g)} \ar{d}{f^{(\infty)}} & \St^{(\infty)}(\Phi, R) \ar{d}{f^{(\infty)}} \\ \St^{(\infty)}(\Phi, R') \ar{r}{conj(\overline{f}(g))} & \St^{(\infty)}(\Phi, R'). \end{tikzcd} \end{equation}
\end{prop}
\begin{proof}
As before, we may suppose that \(g = x_\alpha(a/s)\) for some \(\alpha \in \Phi\), \(a \in R\), \(s \in S\). By \cite[Lemma~3.2]{LSV20} %TODO: and \cite[Lemma~11]{V20}
it suffices to check that the long paths in the diagram coincide after precomposition with \(x_\beta \colon R^{(\infty)} \to \St^{(\infty)}(\Phi, R)\) for all \(\beta \neq -\alpha\). This follows from the definition before~\cite[Proposition~4.2]{LSV20} %TODO: \cite[\S 10]{V20}
and the commutativity of
\[\begin{tikzcd}
R^{(\infty)} \ar{r}{x_\beta} \ar{d}{f^{(\infty)}} &
\St^{(\infty)}(\Phi, R) \ar{d}{f^{(\infty)}} \\
{R'}^{(\infty)} \ar{r}{x_\beta} & \St^{(\infty)}(\Phi, R'). \end{tikzcd}\]
\end{proof}

\begin{rem} \label{rem:functoriality}
 Unwinding the definitions, we obtain from the commutativity of~\eqref{eq:conj-finfty} that for every $g \in \St(\Phi, R_S)$ there exists $s_0 \in S$ such that the following diagram commutes:
 \[ \begin{tikzcd} \St(\Phi, s_0R) \ar{r} \ar{d} & \St(\Phi, s_g R) \ar{r}{c_{g}} & \St(\Phi, R) \ar{dd}{f} \\
                   \St(\Phi, f^*(s_{\overline{f}(g)})R) \ar{d}{f}   & & \\
                   \St(\Phi, s_{\overline{f}(g)} R')   \ar{rr}{c_{\overline{f}(g)}} & & \St(\Phi, R'). \end{tikzcd} \]
 In the above diagram $s_g$, $s_{\overline{f}(g)}$, $c_g$, $c_{\overline{f}(g)}$ have the same meaning as in Remark~\ref{rem:conj-action}.
\end{rem}

Now let us specialize the above results to a particular situation that we will encounter in~\cref{sec:proof-glueing}.
Let $\iota \colon B \to A$ be an injective homomorphism of integral domains and $h \neq 0$ be an element of $B$.
We denote by $\mathbb N$ the set of nonnegative integers. Denote by $h^{\mathbb{N}}$ the multiplicative subset $\{ h^n \mid n \in \mathbb{N} \} \subseteq B$. The pro-rng $B^{(\infty)}$ associated to $h^\mathbb{N}$ can be depicted of as the following diagram of rngs (by definition, $B^{(k)}$ is the principal ideal $h^kB$ and all the structure morphisms of $B^{(\infty)}$ are obvious inclusions):
\[ \ldots \hookrightarrow h^{n+1} B \hookrightarrow h^n B \hookrightarrow \ldots \hookrightarrow hB \hookrightarrow B. \]
 
Cosnider the multiplicative subset $\iota(h)^\mathbb{N} \subseteq A$ and the corresponding homomorphism of localizations $\overline{\iota} \colon B_h \to A_{\iota(h)}$. It is clear that if we put $\iota^*(\iota(h)^n) = h^n$, the subsets $h^\mathbb{N} \subseteq B$, $\iota(h)^\mathbb{N} \subseteq A$, and the homomorphism $\iota$ satisfy the requirements mentioned before Proposition~\ref{prop:functoriality}. Thus, there exists a morphism $\iota^{(\infty)}$ of pro-rngs $B^{(\infty)} \to A^{(\infty)}$ and the corresponding morphism of Steinberg pro-groups $\iota^{(\infty)}\colon\St^{(\infty)}(\Phi, B) \to \St^{(\infty)}(\Phi, A)$ satisfying the conclusion of Proposition~\ref{prop:functoriality}.


%Let $f \colon B \to A$ be a homomorphism of integral domains and let $h$ be an element of $B$ such that $f(h)\neq 0$.
%We denote by \(\mathbb N\) the set of nonnegative integers.
%Consider the multiplicative subsets $h^{\mathbb{N}} = \{ h^n \mid n \in \mathbb{N} \} \subset B$ and $f(h)^\mathbb{N} \subseteq A$ and the corresponding homorphism of localizations $\overline{f} \colon B_h \to A_{f(h)}$.
%It is clear that $h^\mathbb{N} \subseteq B$ and $f(h)^\mathbb{N} \subseteq A$ and $f$ satisfy the requirements mentioned before Proposition~\ref{prop:functoriality} (e.\,g. $f^*(f(h)^n) = h^n$), so there exists a morphism $f^{(\infty)}$ of pro-rngs $B^{(\infty)} \to A^{(\infty)}$ and the corresponding morphism of Steinberg pro-groups $f^{(\infty)}\colon\St^{(\infty)}(\Phi, B) \to \St^{(\infty)}(\Phi, A)$ satisfying the conclusion of Proposition~\ref{prop:functoriality}.

\begin{corollary}
\label{vorcor}
For $\iota \colon B \to A$ and $h \in B$ as above and every $g \in \St(\Phi, B_h)$ there exists $n = n(g) \in \mathbb{N}$ 
and group homomorphisms $c_g \colon \St(\Phi, h^nB) \to \St(\Phi, B)$, $c_{\overline{\iota}(g)} \colon \St(\Phi, \iota(h)^nA) \to \St(\Phi, A)$
such that the following equalities are fulfilled:
\begin{align}
 \label{eq:coherence} \iota \circ c_g &= c_{\overline{\iota}(g)} \circ \iota, &\\
 \label{eq:strictB} \lambda_h^0 (c_g (x)) &= g \cdot \lambda_h^n(x) \cdot g^{-1}&\text{ for }x \in \St(\Phi, h^nB),\\
 \label{eq:strictA} \lambda_h^0 (c_{\overline{\iota}(g)}(y)) &= g \cdot \lambda_h^n(y) \cdot \overline{\iota}(g)^{-1}&\text{ for }y \in \St(\Phi, \iota(h)^nA).
\end{align}
Here $\lambda_h^k$ (resp. $\lambda^k_{\iota(h)}$) denotes the homomorphism of Steinberg grops induced by the following composite rng homomorphism $h^kB \hookrightarrow B \to B_h$ (resp. $\iota(h)^kA \hookrightarrow A \to A_{\iota(h)}$), where $k \in \mathbb{N}$.
\end{corollary}
\begin{proof}
 The first equality is a special case of the commutativity of the diagram from Remark~\ref{rem:functoriality}.
 The second and the third equalities follow from the equality discussed in Remark~\ref{rem:conj-action}.
\end{proof}
\begin{rem}\label{rem:indendepence}
 Notice that a priori the homomorphism $c_g$ depends on the choice of a representative for $conj(g)$. On the other hand, from the definition of equivalence of pre-morphisms it follows that if $c'_g$ is the degree $1$ component for a different representative of $conj(g)$ then for sufficiently large $m$ one has $c'_{\overline{\iota}(g)}(x_{\alpha}(ah^m)) = c_{\overline{\iota}(g)}(x_{\alpha}(ah^m))$.
\end{rem}

\subsection{Proof of Theorem~\ref{glueing}} \label{sec:proof-glueing}
%It is well known that subgroup factorizations of $\mathrm G(\Phi,\,R)$ are an important ingredient in the study of the $\K_1$-functor. It turns out that in the study of $\K_2$-functors the $\St(\Phi,\,R)$-torsors play similar role, see~\cite{LS20}. TODO:

We let $\St(\Phi, B)$ act on $\St(\Phi, B_h) \times \St(\Phi, A)$ on the left via the formula 
\[w \star (u, v) = (u \lambda_h(w)^{-1}, \iota(w) v),\text{ where }w \in \St(\Phi, B),\ u \in \St(\Phi, B_h),\ v \in \St(\Phi, A).\]
Denote by $V$ the orbit set for this action. We use the notation $[u, v]$ for the orbit corresponding to a pair $(u, v) \in \St(\Phi, B_h) \times \St(\Phi, A)$.

Our goal is to show that the formula $[u, v] \mapsto \overline{\iota}(u) \cdot \lambda_{\iota(h)}(v)$ defines a bijection between $V$ and $\St(\Phi, A_{\iota(h)})$. The first step in this direction is to construct an action of $\St(\Phi, A_{\iota(h)})$ on $V$. 

Fix some $u \in \St(\Phi, B_h)$, $v \in \St(\Phi, A)$, $\alpha \in \Phi$, and $c/{h^s} \in A_{\iota(h)}$.
Applying \cref{vorcor} to the element $u$ we obtain a natural number $n=n(u)$ and homomorphisms \begin{equation} \label{eq:c-homs} c_{u^{-1}}\colon \St(\Phi, h^nB) \to \St(\Phi, B),\ c_{\overline{\iota}(u^{-1})}\colon \St(\Phi, \iota(h)^nA) \to \St(\Phi, A)\end{equation}
satisfying the identities listed in~\cref{vorcor}.

Observe that the assumption $A/hA = B/hB$ is equivalent to the condition that for every $k \geq 0$ one has $A = Ah^k + B$ and $Ah^k \cap B = Bh^k$. Now choose any $k \geq n + s$ and decompose $c \in A$ as
 \begin{equation} \label{eq:decomp} c = ah^k + b\text{ for some }a \in A,\ b \in B. \end{equation}
We define the operator $T_\alpha(c/h^s) \colon V \to V$ via the formula:
\begin{equation}\label{eq:generator-action}
\textstyle
T_\alpha(c/{h^s}) \cdot [u, v] = \bigl[x_\alpha(b/{h^s})\cdot u,\ c_{\overline{\iota}(u^{-1})}(x_\alpha(ah^{k - s})) \cdot v\bigr].
\end{equation}
\begin{lemma}\label{well-def}
The right hand side of~\eqref{eq:generator-action} is independent of either the choice of the decomposition~\eqref{eq:decomp}, the choice of representatives for the morphisms $conj(u)$, $conj(\overline{\iota}(u))$ or the choice of a representative for the orbit $[u, v]$.
\end{lemma}
\begin{proof}
First, let us show the independence of the choice of a decomposition for $c$. 
Choose some $l \geq n + s$ and $c = a' h^{l} + b'$. Without loss of generality, we may assume $l \leq k$ so that $a' h^{l} - a h^k = b - b' \in Ah^{l} \cap B = Bh^{l}$ and there exists $d \in B$ such that $b = b' + dh^{l}$. Thus, from~\eqref{eq:strictB} we obtain that
\begin{equation} \label{eq:wd1} \textstyle x_\alpha(b/{h^s})\, u = x_\alpha({b'}/{h^s})\, u \cdot u^{-1}\, x_\alpha(dh^{l-s})\, u =  x_\alpha({b'}/{h^s})\, u \cdot \lambda_h \bigl(c_{u^{-1}}(x_\alpha(dh^{l-s}))\bigr). \end{equation}
Since $a'h^l = ah^k + dh^l$ and $A$ is a domain we obtain that $a'h^{l-s} = dh^{l-s} + ah^{k-s}$ therefore
\begin{align*}
[x_\alpha(b'/{h^s})\, u,\ c_{\overline{\iota}(u^{-1})}(x_\alpha(a'h^{l - s}))\, v\bigr] = \bigl[x_\alpha(b'/{h^s})\, u,\ \iota\bigl(c_{u^{-1}}(x_\alpha(dh^{l-s}))\bigr)\, c_{\overline{\iota}(u^{-1})}(x_\alpha(ah^{k - s}))\, v\bigr] & \text{ by~\eqref{eq:coherence}} \\ = \bigl[x_\alpha(b/{h^s})\, u,\ c_{\overline{\iota}(u^{-1})}(x_\alpha(ah^{k - s}))\, v\bigr] &\text{ by~\eqref{eq:wd1},}
\end{align*}
 which coincides with the right-hand side of~\eqref{eq:generator-action}.

The independence of the right-hand side of~\eqref{eq:generator-action} of the choice of representatives for $conj(g)$ and $conj(\overline{\iota}(g))$ follows from Remark~\ref{rem:indendepence} and the fact that the number $k$ in the decomposition for $c$ could have been chosen arbitrarily large.

Now suppose that $(u', v')$ is another representative for the orbit $[u, v]$, i.\,e. $u' = u \cdot \lambda_h(w)$, $v' = \iota(w^{-1})\cdot v$ for some $w \in \St(\Phi,\,B)$. It is clear that we can choose a representative $\eta$ for $conj(\overline{\iota}(\lambda_h(w^{-1})))$ in such a way that $\eta^*(1)=1$ and its first-degree component $\eta_1 \colon \St(\Phi, A) \to \St(\Phi, A)$ coincides with $h \mapsto \iota(w^{-1}) \cdot h \cdot \iota(w)$, moreover, from the definition of the composition of pre-morphisms we conclude that $c_{\overline{\iota}({u'}^{-1})} =  \eta_1 \circ c_{\overline{\iota}(u^{-1})} $. The independence of~\eqref{eq:generator-action} of the choice of $(u, v)$ now follows from the following direct calculation:
\begin{multline*}
 [x_\alpha(b/h^s) \cdot u',\ c_{\overline{\iota}({u'}^{-1})}(x_\alpha(ah^{k-s}))\cdot v'] = [x_\alpha(b/h^s) \cdot u \cdot \lambda_h(w),\ c_{\overline{\iota}({u'}^{-1})}(x_\alpha(ah^{k-s}))\cdot \iota(w^{-1}) v] = \\
 = [x_\alpha(b/h^s) \cdot u \cdot \lambda_h(w),\ \iota(w) \cdot c_{\overline{\iota}(u^{-1})}(x_\alpha(ah^{k-s}))\cdot v] 
 = [x_\alpha(b/h^s) \cdot u,\ c_{\overline{\iota}(u^{-1})}(x_\alpha(ah^{k-s}))\cdot v]. \qedhere
\end{multline*}
\end{proof}

The next step of the proof is to verify that the above action specifies an action of $\St(\Phi, A_{\iota(h)})$ on $V$.
This is accomplished in the series of lemmas below:

\begin{lemma}\label{lem:R1} The operators $T_\alpha$ satisfy relations~\eqref{R1}. \end{lemma}
\begin{proof}
We need to show that for $c, c' \in A$ one has $\textstyle
T_\alpha(c'/{h^s}) \cdot T_\alpha(c/h^s) \cdot [u, v] = T_\alpha\bigl((c+c')/h^s\bigr) \cdot [u, v].$
Choose a decomposition $c = ah^k + b$ as in~\eqref{eq:decomp} for some $k \geq n(u) + s$.
We claim that there exists a sufficiently large number $m$ such that $c_{\overline{\iota}(u^{-1}) \cdot x_\alpha(-b/h^s)}(x_\alpha(a'h^{m-s})) = c_{\overline{\iota}(u^{-1})}(x_\alpha(a'h^{m-s}))$ for all $a'\in A$. Indeed, this follows from the definition of the composition of pre-morphisms and Remark~\ref{rem:conj-root-action}. 
Now choose a decomposition $c' = a' h^m + b'$.
The assertion now follows from the following calculation:
\begin{multline*}
 T_\alpha(c'/{h^s}) \cdot T_\alpha(c/h^s) \cdot [u,\ v] = T_\alpha(c'/h^s) \cdot [x_\alpha(b/h^s)\cdot u,\ c_{\overline{\iota}(u^{-1})}(x_\alpha(ah^{k-s})) \cdot v] = \\
 = [x_\alpha((b + b')/h^s)\cdot u,\ c_{\overline{\iota}(u^{-1}) \cdot x_\alpha(-b/h^s)}(x_\alpha(a'h^{m-s})) \cdot c_{\overline{\iota}(u^{-1})}(x_\alpha(ah^{k-s})) \cdot v] = \\ = [x_\alpha((b + b')/h^s)\cdot u,\  c_{\overline{\iota}(u^{-1})}(x_\alpha(a'h^{m-s} + ah^{k-s})) \cdot v] = T_\alpha\bigl((c+c')/h^s\bigr) \cdot [u, v]. \qedhere
\end{multline*}
\end{proof}

\begin{lemma} \label{lem:R3-check} The operators $T_\alpha$ satisfy relations~\eqref{R3}. \end{lemma}
\begin{proof}
 Fix a natural number $s$ and a pair of roots $\alpha, \beta \in \Phi$ such that $\alpha+\beta\in\Phi$. Set $\epsilon = N_{\alpha, \beta}$. From the definition of composition of pro-group morphisms and the definition of $conj$ it follows that for any $u \in \St(\Phi, B_h)$ there exists sufficiently large natural number $n$ such that for all $a \in A$, $b, b'\in B$ the following equalities are simultaneously fulfilled:
 \begin{align}
 c_{ \overline{\iota}((x_\beta(b/h^s) \cdot u)^{-1})}(x_\alpha(ah^{n-s})) &= c_{\overline{\iota}(u^{-1})}\left(x_\alpha(ah^{n-s}) \cdot x_{\alpha+\beta}(\epsilon ab h^{n-2s})\right), \label{eq:R3-check-1}\\
 c_{ \overline{\iota}((x_\alpha(b/h^s) \cdot u)^{-1})}(x_\beta(ah^{n - s})) &= c_{\overline{\iota}(u^{-1})}\bigl(x_{\alpha + \beta}(-\epsilon abh^{n - 2s})\cdot x_\beta(ah^{n - s})\bigr), \label{eq:R3-check-2} \\
 c_{ \overline{\iota}((x_\beta(b'/h^s) \cdot x_\alpha(b/h^s) \cdot u)^{-1})}\bigl(x_{\alpha + \beta}\bigl(a h^{n - 2s}\bigr)\bigr) &= c_{\overline{\iota}(u^{-1})}\bigl(x_{\alpha + \beta}\bigl(a h^{n - 2s}\bigr)\bigr). \label{eq:R3-check-3}
 \end{align} 
Now fix $c, c\in A$. Our goal is to verify the equality 
\begin{equation} \label{eq:R3-to-check} T_\alpha(c/h^s) \cdot T_\beta(c'/h^s) \cdot [u, v] = T_{\alpha+\beta}(\epsilon cc' / h^{2s}) \cdot T_\beta(c'/h^s) \cdot T_\alpha(c/h^s) \cdot [u, v].\end{equation}
As in~\eqref{eq:decomp} we can choose decompositions $c = b + ah^n,\ c' = b' + a'h^n$ for some $a, a' \in A$ and $b, b' \in B$. It is also clear that $cc' = bb' + h^{n}a''$, where $a'' = aa'h^n + ab' + a'b$. Now we can compute the left-hand side of~\eqref{eq:R3-to-check} using~\eqref{eq:generator-action} and~\eqref{eq:R3-check-1} as follows:
\begin{multline*}
 T_\alpha(c/h^s) \cdot T_\beta(c'/h^s) \cdot [u, v] = T_\alpha(c/h^s) \cdot [x_\beta(b'/h^s) \cdot u,\ c_{\overline{\iota}(u^{-1})}(x_\beta(a'h^{n-s})) \cdot v] = \\ = [x_\alpha(b/h^s) \cdot x_\beta(b'/h^s) \cdot u, c_{\overline{\iota}(u^{-1})}(x_\alpha(ah^{n-s}) \cdot x_{\alpha+\beta}(\epsilon ab'h^{n-2s}) \cdot x_\beta(a'h^{n-s}))]. \end{multline*}
Similarly, we can compute the right-hand side of~\eqref{eq:R3-to-check} using~\eqref{eq:generator-action} and~\eqref{eq:R3-check-2}--\eqref{eq:R3-check-3}:
\begin{multline*}
 T_{\alpha+\beta}(\epsilon cc'/h^{2s}) \cdot T_\beta(c'/h^s) \cdot T_\alpha(c/h^s) \cdot [u, v] = \\ = [x_{\alpha+\beta}(\epsilon bb'/h^{2s}) \cdot x_\beta(b'/h^s) \cdot x_\alpha(b/h^s) \cdot u, c_{\overline{\iota}(u^{-1})}(x_{\alpha+\beta}(\epsilon h^{n-2s}(a''- a'b)) \cdot x_\beta(a'h^{n-s}) \cdot x_\alpha(ah^{n-s}))]. \end{multline*}
Finally, it follows from~\eqref{R3} that the right-hand sides of the last two formulae coincide.
\end{proof}

\begin{comment}
\begin{lemma} The operators $T_\alpha$ satisfy relations~\eqref{R4}. \end{lemma}
The lase case is $\alpha + \beta, \alpha + 2\beta \in \Phi$ and $2\alpha + \beta \notin \Phi$. We show that
$$\textstyle
f = x_\alpha(\frac c {h^s})\, x_\beta(\frac{c'}{h^s})
\text{ and }
g = x_{\alpha + 2\beta}(N_{\alpha \beta 1 2} \frac{c {c'}^2}{h^{3s}})\, x_{\alpha + \beta}(N_{\alpha \beta} \frac{cc'}{h^{2s}})\, x_\beta(\frac{c'}{h^s})\, x_\alpha(\frac c {h^s})
$$ 
act identically on $V$. Choose decompositions $c = ah^n + b$, $c' = a'h^n + b'$, $cc' = (aa' h^n + ab' + a'b) h^n + bb'$, and $c{c'}^2 = (a{a'}^2 h^{2n} + 2aa'b' h^n + {a'}^2 b h^n + 2a'bb' + a{b'}^2) h^n + b{b'}^2$ for sufficiently large $n$.
Then
$$\textstyle
f\,(u, v) = \bigl(x_\alpha(\frac b {h^s})\, x_\beta(\frac{b'}{h^s})\, u,
\theta'_0(x_\alpha(ah^{n - s}))\, \theta_0(x_\beta(a'h^{n - s}))\, v\bigr),
$$
where $\theta$ is a strict representative of $\Xi(\iota(u))^{-1}$ and $\theta'$ is a strict representative of $\Xi(x_\beta(\frac{b'}{h^s})\, \iota(u))^{-1}$. It is possible to choose $\theta'$ in such a way that
$$
\theta'_0(x_\alpha(ah^{n - s})) = \theta_0\bigl(x_\alpha(ah^{n - s})\, x_{\alpha + \beta}(N_{\alpha\beta} ab'h^{n - 2s})\, x_{\alpha + 2\beta}(-N_{\alpha\beta 12} a{b'}^2 h^{n - 3s})\bigr).
$$
Next, 
\begin{multline*}\textstyle
g\, (u, v) = \bigl(x_{\alpha + 2\beta}(N_{\alpha \beta 12} \frac{b{b'}^2}{h^{3s}})\, x_{\alpha+\beta}(N_{\alpha \beta} \frac{bb'}{h^{2s}})\, x_\beta(\frac{b'}{h^s})\, x_\alpha(\frac b {h^s})\, u,\\
\theta^{\mathrm{IV}}_0\bigl(x_{\alpha + 2\beta}(N_{\alpha \beta 12} (a{a'}^2 h^{2n} + 2aa'b' h^n + {a'}^2 b h^n + 2a'bb' + a{b'}^2) h^{n - 3s})\bigr)\\
\theta'''_0\bigl(x_{\alpha + \beta}\bigl(N_{\alpha \beta} (aa' h^n + ab' + a'b) h^{n - 2s}\bigr)\bigr)\, \theta''_0(x_\beta(a'h^{n-s}))\, \theta_0(x_\alpha(ah^{n - s}))\, v\bigr),
\end{multline*}
where $\theta''$ is a strict representative of $\Xi\bigl(x_\alpha(\frac b {h^s})\, \iota(u)\bigr)^{-1}$, $\theta'''$ is a strict representative of $\Xi\bigl(x_\beta(\frac{b'}{h^s})\, x_\alpha(\frac b{h^s})\, \iota(u)\bigr)^{-1}$, and $\theta^{\mathrm{IV}}$ is a strict representative of $\Xi\bigl(x_{\alpha + \beta}(N_{\alpha \beta}\frac{bb'}{h^{2s}})\, x_\beta(\frac{b'}{h^s})\, x_\alpha(\frac b{h^s})\, \iota(u)\bigr)^{-1}$. We may choose $\theta''$, $\theta'''$, $\theta^{\mathrm{IV}}$ in such a way that
$$
\theta''_0(x_\beta(a' h^{n - s})) = \theta_0\bigl(x_{\alpha + 2\beta}(-N_{\alpha \beta 1 2} {a'}^2 bh^{2n - 3s})\, x_{\alpha + \beta}(-N_{\alpha \beta} a'bh^{n - 2s})\, x_\beta(a'h^{n - s})\bigr),
$$
\begin{multline*}
\theta'''_0\bigl(x_{\alpha + \beta}\bigl(N_{\alpha \beta} (aa'h^n + ab' + a'b) h^{n - 2s}\bigr)\bigr)\\
= \theta_0\bigl(x_{\alpha + 2\beta}(-N_{\beta, \alpha + \beta} N_{\alpha \beta} (aa'h^n + ab' + a'b) b' h^{n - 3s})\, x_{\alpha + \beta}\bigl(N_{\alpha \beta} (aa'h^n + ab' + a'b) h^{n - 2s}\bigr)\bigr),
\end{multline*}
and
\begin{multline*}
\theta^{\mathrm{IV}}_0\bigl(x_{\alpha + 2\beta}(N_{\alpha \beta 12} (a{a'}^2 h^{2n} + 2aa'b' h^n + {a'}^2 bh^n + 2a'bb' + a{b'}^2) h^{n - 3s})\bigr)\\
= \theta_0\bigl(x_{\alpha + 2\beta}(N_{\alpha \beta 12} (a{a'}^2 h^{2n} + 2aa'b' + {a'}^2 bh^n + 2a'bb' + a{b'}^2) h^{n - 3s})\bigr).
\end{multline*}
It remains to check that
\begin{multline*}
x_\alpha(ah^{n - s})\, x_{\alpha + \beta}(N_{\alpha\beta} ab'h^{n - 2s})\, x_{\alpha + 2\beta}(-N_{\alpha\beta 12} a{b'}^2 h^{n - 3s})\, x_\beta(a' h^{n - s})\\
= x_{\alpha + 2\beta}(N_{\alpha \beta 12} (a{a'}^2 h^{2n} + 2aa'b' h^n + 2a'bb' + a{b'}^2) h^{n - 3s} - N_{\beta, \alpha + \beta} N_{\alpha \beta} (aa'h^n + ab' + a'b) b' h^{n - 3s})\\
x_{\alpha + \beta}\bigl(N_{\alpha \beta} (aa'h^n + ab' + a'b) h^{n - 2s}\bigr)\, x_{\alpha + \beta}(-N_{\alpha \beta} a'bh^{n - 2s})\, x_\beta(a'h^{n - s})\, x_\alpha(ah^{n - s}).
\end{multline*}
This follows from the Steinberg relations and the identity \(N_{\beta, \alpha + \beta} N_{\alpha \beta} = 2 N_{\alpha \beta 1 2}\).
\end{comment} 
The proof of the fact that $T_\alpha$ satisfy relations~\eqref{R2} is similar to the proof of the above lemmas but is easier, so we leave it as an exercise to the reader.

It follows that the action of $\St(\Phi, A_{\iota(h)})$ on $V$ is well-defined. By construction, the canonical map \(V \to \St(\Phi, A_{\iota(h)})\) preserves the action, where \(\St(\Phi, A_{\iota(h)})\) acts on itself by left multiplication. Since \((1, 1) \in V\) maps to \(1 \in \St(\Phi, A_{\iota(h)})\), it remains to check the transitivity of the action on \(V\). But the construction implies that \((u, v) = \iota(u)\, \bigl(\lambda_h(v)\, (1, 1)\bigr)\). Thus, \(V \to \St(\Phi, A_{\iota(h)})\) is a bijection.
Now if \(u \in \St(\Phi, B_h)\) and \(v \in \St(\Phi, A)\) are such that \(\iota(u) = \lambda_h(v)\), the class of \((u, v^{-1})\) in \(V\) maps to \(1 \in \St(\Phi, A_{\iota(h)})\). Thus, \([u, v^{-1}] = [1, 1]\) and $u = \lambda_h(w)$, $v = \iota(w)$ for some $w \in \St(\Phi, B)$, which completes the proof of the assertion of Theorem~\ref{glueing} for Steinberg groups. Since $\iota\colon \Gsc(\Phi, B)\to \Gsc(\Phi, A)$ is injective, we also obtain the assertion for $\K_2(\Phi, -)$.

\printbibliography
\end{document}
