\documentclass[oneside, 11pt]{amsart} \pdfoutput=1
\usepackage{amsmath, mathtools, amssymb, amsthm, amscd, alltt, graphicx, comment}
\usepackage[utf8]{inputenc}
\usepackage[english,russian]{babel}
\usepackage[T1]{fontenc}
\usepackage[breaklinks=true,unicode]{hyperref}
\usepackage[capitalise]{cleveref}
%\usepackage[matrix,arrow,curve,tips]{xy}
\usepackage{tikz,tikz-cd}
\usepackage{scalerel}
\usepackage[notref, notcite]{showkeys}
\usepackage[a4paper, left=25mm, right=15mm, top=25mm, bottom=35mm]{geometry}


\usepackage[backend=biber, bibencoding=utf8, giveninits=true, citestyle=numeric-comp, sortlocale=en_US, url=false, doi=false, eprint=true, maxbibnames=4]{biblatex}
\addbibresource{nisnevich2.bib}

\renewbibmacro*{volume+number+eid}{\ifentrytype{article}{\- \iffieldundef{volume}{}{Vol.~\printfield{volume},}\iffieldundef{number}{}{ No.~\printfield{number},}}}
\renewbibmacro{in:}{\ifentrytype{article}{}{\printtext{\bibstring{in}\intitlepunct}}}
\newbibmacro{string+doi}[1]{\iffieldundef{doi}{\iffieldundef{url}{#1}{\href{\thefield{url}}{#1}}}{\href{https://dx.doi.org/\thefield{doi}}{#1}}}
\DeclareFieldFormat[article, inproceedings, inbook, book, online]{title}{\usebibmacro{string+doi}{\mkbibquote{#1}}}
\renewcommand*{\bibfont}{\footnotesize}

\begin{document}
\renewcommand{\Im}{\mathop{\mathrm{Im}}\nolimits}
%\newcommand{\Dom}{\mathop{\mathrm{Dom}}\nolimits}
%\newcommand{\Card}{\mathop{\mathrm{Card}}\nolimits}
\newcommand{\Ker}{\mathop{\mathrm{Ker}}\nolimits}
%\newcommand{\Coker}{\mathop{\mathrm{Coker}}\nolimits}
%\newcommand{\Cent}{\mathop{\mathrm{Cent}}\nolimits}
\newcommand{\K}{{\mathrm{K}}}
\newcommand{\St}{\mathop{\mathrm{St}}\nolimits}
\newcommand{\E}{\mathrm{E}}
\newcommand{\Gsc}{\mathrm{G}_\mathrm{sc}}
%\newcommand{\Sp}{\mathop{\mathrm{Sp}}\nolimits}
%\newcommand{\Ep}{\mathop{\mathrm{Ep}}\nolimits}
%\newcommand{\GL}{\mathop{\mathrm{GL}}\nolimits}
%\newcommand{\SL}{\mathop{\mathrm{SL}}\nolimits}
%\newcommand{\Kt}{\mathop{\mathrm{K_2}}\nolimits}
%\newcommand{\Ko}{\mathop{\mathrm{K_1}}\nolimits}
%\newcommand{\SKo}{\mathop{\mathrm{SK_1}}\nolimits}
%\newcommand{\Ho}{\mathop{\mathrm{H_1}}\nolimits}
%\newcommand{\Hf}{\mathop{\mathrm{H_1}}\nolimits}
%\newcommand{\Ht}{\mathop{\mathrm{H_2}}\nolimits}
%\newcommand{\Hs}{\mathop{\mathrm{H_2}}\nolimits}
%\newcommand{\epi}{\twoheadrightarrow}
%\newcommand{\sgn}{\mathrm{sgn}}
%\newcommand{\eps}[1]{\varepsilon_{#1}}
%\newcommand{\lan}{\langle}
%\newcommand{\ran}{\rangle}
%\newcommand{\inv}{^{-1}}
%\newcommand{\ur}[1]{\!\,^{(#1)}U_1}
%\newcommand{\ps}[1]{\!\,^{(#1)}\!P_1}
%\newcommand{\ls}[1]{\!\,^{(#1)}\!L_1}

%\newcommand{\M}{{\mathrm{M}}\,\!}
%\newcommand{\fp}{\mathfrak L}
%\newcommand{\ind}{\mathop{\mathrm{ind}}\nolimits}
%\newcommand{\EU}{\mathop{\mathrm{EU}}\nolimits}
%\newcommand{\KU}{\mathop{\mathrm{K_2U}}\nolimits}
%\newcommand{\GQ}{\mathop{\mathrm{GQ}}\nolimits}
%\newcommand{\GH}{\mathop{\mathrm{GH}}\nolimits}
%\newcommand{\StU}{\mathop{\mathrm{StU}}\nolimits}
%\newcommand{\sta}[1]{\StU(#1,\,R,\,\mathfrak{L})}
%\newcommand{\st}{\sta{2n}}
%\newcommand{\sign}[1]{\mathrm{sign}(#1)}

\numberwithin{equation}{section}
\newtheorem{lemma}{Lemma} \numberwithin{lemma}{section}
\newtheorem*{lemma*}{Lemma}
\newtheorem{prop}[lemma]{Proposition} 
\newtheorem{theorem}[lemma]{Theorem}
\newtheorem{corollary}[lemma]{Corollary} 
\newtheorem*{theorem*}{Theorem} 
\newtheorem*{corollary*}{Corollary} 

\crefname{prop}{prop}{propss}

\theoremstyle{definition} 
\newtheorem{df}[lemma]{Definition} 

%\newtheorem{example}[lemma]{Example} \Crefname{example}{Example}{Examples}

%\theoremstyle{remark} 
%\newtheorem{rem}[lemma]{Remark}
%\newtheorem{conv}[lemma]{Convention} \Crefname{conv}{Convention}{Conventions}

%\newenvironment{psmallmatrix}{\left(\begin{smallmatrix}}{\end{smallmatrix}\right)}

%\DeclareMathOperator\St{St}
%\DeclareMathOperator\Ker{Ker}
%\DeclareMathOperator\GG{G}
%\DeclareMathOperator\Torus{T}
%\DeclareMathOperator{\Pro}{Pro}


\newcommand{\Set}{\mathbf{Set}}
\newcommand{\Group}{\mathbf{Grp}}
\newcommand{\Rng}{\mathbf{Rng}}
\newcommand{\Fun}{\mathbf{Fun}}
\newcommand{\Mod}{\mathbf{Mod}}
\newcommand{\op}{\mathrm{op}}
\newcommand{\ZZ}{\mathbb{Z}}

\newcommand{\otimeshat}{\mathbin{\widehat{\otimes}}}

\newcommand{\up}[2]{{^{#1}\!{#2}}}

\newcommand{\rA}{\mathsf{A}}
\newcommand{\rB}{\mathsf{B}}
\newcommand{\rC}{\mathsf{C}}
\newcommand{\rD}{\mathsf{D}}
\newcommand{\rE}{\mathsf{E}}
\newcommand{\rF}{\mathsf{F}}
\newcommand{\rG}{\mathsf{G}}

\newcommand{\catname}[1]{{\normalfont\textbf{#1}}} %Category name

\section{Introduction}
\subsection{Steinberg groups and pro-groups}

Now for $a, b \in R$ set $y_\alpha(a, b) = [x_\alpha(a), x_{-\alpha}(b)]$.
For an ideal $I \trianglelefteq R$ we denote by $\overline{\St}(\Phi, R, I)$ the normal closure in $\St(\Phi, R)$ of the subgroup generated by $x_\alpha(a)$, $a\in I$.

\begin{lemma} \label{lem:c-identities} For an arbitrary irreducible root system $\Phi$ of rank $\geq 2$, arbitrary ideals $A, B \trianglelefteq R$ and all $a \in A$, $b \in B$, $c \in R$ the following congruences hold:
\begin{itemize}
 \item $y_\alpha(a, cb) \equiv y_\alpha(ac, b)\ (\mathrm{mod}\ \overline{\St}(\Phi, R, AB))$ in the case when either $\Phi \neq \rC_{\ell}$ or $\alpha$ is short.
 \item $y_\alpha(a, c^2b) \equiv y_\alpha(ac^2, b)\ (\mathrm{mod}\ \overline{\St}(\Phi, R, AB))$, $y_\alpha(a, cb)^2 \equiv y_\alpha(ac, b)^2\ (\mathrm{mod}\ \overline{\St}(\Phi, R, AB))$ in the case $\Phi = \rC_{\ell}$ and $\alpha$ is long.
\end{itemize} \end{lemma}
\begin{proof}
 Observe that the proof of~\cite[Theorem~5]{VZ20} is based solely on computations with Chevalley commutator formula, which all can be reproduced verbatim in the context of Steinberg groups.
\end{proof}

Notice that the Steinberg group functor {\it does not} commute with general finite limits. However, it satisfies the following weaker property, which we are going to use in the sequel.
\begin{lemma} \label{lem:fprod} For an arbitrary irreducible root system $\Phi$ of rank $\geq 2$ the Steinberg group functor $\St(\Phi, -)$ commutes with finite direct products. \end{lemma}
\begin{proof} 
It suffices to verify the assertion for binary direct products.
Observe that canonical projections $R_1 \times R_2 \to R_i$, $i=1,2$ split in the category~\catname{Rngs}, therefore the groups $G_i = \St(\Phi, R_i)$ embed as subgroups into $G = \St(\Phi, R_1 \times R_2)$. It is also clear that $G_i$ together generate $G$. Thus, to verify the isomorphism $G \cong G_1 \times G_2$ it suffices to show the triviality of the commutator subgroup $[G_1, G_2] \leq G$.

Set $A = R_1\times 0$, $B = 0 \times R_2$. It is clear that $A, B \trianglelefteq R_1 \times R_2$. 
It follows directly from Chevalley commutator formula that the commutators $[x_{\alpha}(a),\ x_\beta(b)]$ are trivial for all $\beta \neq -\alpha$. On the other hand, in order to verify the triviality of $y_\alpha(a, b) = [x_{\alpha}(a),\ x_{-\alpha}(b)]$ for $a\in A$, $b\in B$ we can apply the congruences of~\cref{lem:c-identities} (since $AB=0$ these congruences turn into equalities).
Indeed, setting $c = c^2 = (1, 0)$ we obtain that $y_\alpha(a, b) = y_\alpha(ac, b) = y_\alpha(a, bc) = y_\alpha(a, 0) = 1$. \end{proof}

\subsection{Overview of the pro-group technique}
Let $R$ be an integral domain, $s \in R \setminus \{0\}$, $\Phi$ be a simply-laced root system of rank at least $3$. All root systems in this text are reduced and irreducible. By \(\mathbb N\) we mean the set of non-negative integer numbers.

Recall a definition of Steinberg pro-groups. Consider a chain of {\it unrelativized} Steinberg groups $\St(\Phi, s^n R)$ for $n \in \mathbb N$, i.e. the groups with the generators $x_{\alpha}(a)$ for $\alpha \in \Phi$, $a \in s^n R$ and the ordinary Steinbrg relations. The homomorphisms in the chain
$$
\ldots \to \St(\Phi, s^3 R) \to \St(\Phi, s^2 R) \to \St(\Phi, s R) \to \St(\Phi, R)
$$
are defined in the natural way. Such a tower is called a {\it Steinberg pro-group}, it is denoted by $\St^\infty(\Phi, s^\bullet R)$. We usually do not write explicitly the structure homomorphisms of $\St^\infty(\Phi, s^\bullet R)$ in diagrams and formulas. In particular, we denote the elements of $\St(\Phi, s^n R)$ and their images in $\St(\Phi, s^{n - 1} R)$ by the same letters.

We cannot conjugate elements of $\mathrm{St}(\Phi, s^nR)$ by elements of $\mathrm{St}(\Phi, R_s)$. Fortunately, there is such a conjugation action on the whole Steinberg pro-group.

\begin{df}
An {\it endomorphism representative} $\eta$ of the Steinberg pro-group $\St^\infty(\Phi,\,s^\bullet R)$ is the following data: 
\begin{itemize}
\item
a map $\eta^* \colon \mathbb N \to \mathbb N$,
\item
for each $i \in \mathbb N$ a group homomorphism $\eta_i \colon \St(\Phi, s^{\eta^*(i)} R) \to \St(\Phi, s^i R)$,
\end{itemize}
such that for every $i \leq j$ there is $k \geq \eta^*(i), \eta^*(j)$ making the diagram
$$\begin{tikzcd}[row sep=tiny]
& \St(\Phi, s^{\eta^*(j)} R)\ar{r}{\eta_j} & \St(\Phi, s^j R) \ar{dd}\\
\St(\Phi, s^k R) \ar{ur} \ar{dr} & &\\
&\St(\Phi, s^{\eta^*(i)} R) \ar{r}{\eta_i} & \St(\Phi, s^i R)
\end{tikzcd}$$
commutative.

Endomorphism representatives $\eta$ and $\theta$ are called {\it equivalent} if for every $i \in \mathbb N$ there is $k \geq \eta^*(i), \theta^*(i)$ making the diagram
$$\begin{tikzcd}
\St(\Phi, s^k R) \ar{r} \ar{d} & \St(\Phi, s^{\eta^*(i)} R) \ar{d}{\eta_i}\\
\St(\Phi, s^{\theta^*(i)} R) \ar{r}{\theta_i} & \St(\Phi, s^i R)
\end{tikzcd}$$
commutatuve.

For example, if we consider any map $\theta^*$ such that $\theta^*(i) \geq \eta^*(i)$ for all $i$ and define $\theta_i$ as the composition of $\eta_i$ and the structure homomorphism of $\St^\infty(\Phi, s^\bullet R)$, then such a representative $\theta$ is equivalent to $\eta$.

An {\it endomorphism} $g$ of the pro-group $\St^\infty(\Phi,\,s^\bullet R)$ is the equivalence class of endomorphism representatives. Clearly, endomorphisms of $\St^\infty(\Phi,\,s^\bullet R)$ form a monoid $\mathrm{End}\bigl(\St^\infty(\Phi,\,s^\bullet R)\bigr)$ under composition. We are interested in its invertible elements, i.e. the automorphisms of the pro-group \(\St^\infty(\Phi, s^\bullet R)\).

Suppose that there is a fixed group homomorphism $$\Xi \colon \St(\Phi, R_s) \to \mathrm{Aut}\bigl(\St^\infty(\Phi, s^\bullet R)\bigr).$$ For any $g \in \St(\Phi, R_s)$ we say that a representative $\eta$ of the automorphism $\Xi(g)$ is {\it strict} if for every $i$ and for every $x \in \St(\Phi, s^{\eta^*(i)} R)$ the identity
$$
\lambda_s\bigl(\eta_i(x)\bigr) = g \lambda_s(x) g^{-1} \in \St(\Phi, R_s)
$$
holds, where $\lambda_s$ is the principal localization at $s$. Clearly, a composition of strict representatives is strict.
\end{df}


\begin{theorem}[Voronetsky]
\label{vor}
There is a group homomorphism 
$$
\Xi \colon \St(\Phi, R_s) \to \mathrm{Aut}\bigl(\St^\infty(\Phi, s^\bullet R)\bigr)
$$
such that every automorphism $\Xi(g)$ for $g \in \St(\Phi, R_s)$ has a strict representative.
\end{theorem}

Recall a possible construction of such strict representatives. Clearly, it suffices to construct a strict representative for \(\Xi(g)\), where $g = x_\alpha(\frac a {s^k})$ is a generator of $\St(\Phi, R_s)$. Take $\eta^*(i) = 2p(i + k)$ for sufficiently large constant \(p\) and define
$$
\eta_i \colon \St(\Phi, s^{2p(i + k)} R) \to \St(\Phi, s^i R)
$$
by the following rule. For $\beta \in \Phi$ such that \(\alpha + \beta \notin \Phi \cup \{0\}\) let 
$$\eta_i \bigl(x_\beta(s^{2p(i + k)} b)\bigr) = x_\beta(s^{2p(i + k)}b).$$
If $\alpha + \beta$ is a root, let
$$
\eta_i\bigl(x_\beta(s^{2p(i + k)} b)\bigr) = x_{\alpha + \beta}(N_{\alpha \beta} s^{2p(i + k) - k} ab)\, x_\beta(s^{2p(i + k)} b).
$$
Finally, if $-\alpha = \beta + \gamma$ is any decomposition into a sum of roots, then let
$$
\eta_i\bigl(x_{-\alpha}(s^{2p(i + k)} b)\bigr) = \bigl[x_{-\gamma}(N_{\alpha\beta} s^{p(i + k) - k} ab)\,
x_\beta(s^{p(i + k)} b),
x_{-\beta}(N_{\alpha\gamma} N_{\beta\gamma} s^{p(i + k) - k} a)\,
x_\gamma(N_{\beta \gamma} s^{p(i + k)})\bigr].
$$

It is proved in [LSV] that the homomorphisms $\eta_i$ are well-deined and they give an action of $\St(\Phi, R_s)$ on $\St^\infty(\Phi, s^\bullet R)$.

Such consruction shows that the choice of strict representatives is functorial is some natural sense. 
\begin{corollary}
\label{vorcor}
Let $f \colon B \to A$ be a homomorphism of integral domains, $h \in B$, $s = f(h) \neq 0$, and $u \in \St(\Phi, B_h)$. Then there is a strict representative $\eta$ of the automorphism $\Xi(u)$ of the pro-group $\St^\infty(\Phi, h^\bullet B)$ and a strict representative $\theta$ of the automorphism $\Xi\bigl(f(u)\bigr)$ of the pro-group $\St^\infty(\Phi, s^\bullet A)$ such that $\eta^*(i) = \theta^*(i)$ and
$$f\bigl(\eta_i(x)\bigr) = \theta_i\bigl(f(x)\bigr)$$
for all $x \in \St(\Phi, h^{\eta^*(0)}B)$ and for all \(i\).
\end{corollary}

\section{A patching theorem for \texorpdfstring{$\K_2(\Phi, R)$}{K2(Ф,R)}}

Пользуясь Теоремой~\ref{vor}, мы докажем Теорему~\ref{glueing} о склейке для $\St(\Phi,-)$. Впервые аналогичный результат для проективных модулей был доказан by Lindel-Lutkebohrnert и независимо by Mohan Kumar while proving Bass--Quillen conjecture for a power series ring over a field. Затем Линдел применил такую склейку при решения проблемы Басса--Квиллена для существенно гладких алгебр над совершенным полем.

As in the previous section $\Phi$ is a simply-laced root system of rank at least $3$.

\begin{theorem}\label{glueing}
Let $B$ be a subring of an integral domain $A$, $h\in B \setminus \{0\}$ be such that $A / hA = B / hB$. This is equivalent to $A = Ah^n + B$ and $Ah^n \cap B = Bh^n$ for all $n \in \mathbb N$. Consider the diagram
$$\begin{tikzcd}
\St(\Phi, B) \ar{r}{\iota} \ar{d}{\lambda_h} & \St(\Phi, A) \ar{d}{\lambda_h}\\
\St(\Phi, B_h) \ar{r}{\iota} & \St(\Phi, A_h).
\end{tikzcd}$$
Suppose that the identity
$$\iota(u) = \lambda_h(v) \in \St(\Phi, A_h)$$
holds for some $u \in \St(\Phi, B_h)$ and $v \in \St(\Phi, A)$. Then there is $w \in \St(\Phi, B)$ such that $u = \lambda_h(w)$ and $v = \iota(w)$.
\end{theorem}
\begin{proof}
Хорошо известно, что для работы с $\K_1$-функтором важно изучение различных разложений группы $\mathrm G(\Phi,\,R)$ в произведение подгрупп, см.~[??]. В том же самом смысле, для работы с $\K_2$-функтором оказывается важно изучение главных однородных $\St(\Phi,\,R)$-множеств, см.~[??].

Consider the set
$$
V = \St(\Phi, B_h) \times_{\St(\Phi, B)} \St(\Phi, A),
$$
by definition it is the factor-set of the Cartesian product $\St(\Phi, B_h) \times \St(\Phi, A)$ by the following equivalence relation: for every $u \in \St(\Phi, B_h)$, $v \in \St(\Phi, A)$, and $w \in \St(\Phi, B)$ let
$$
(u\, \iota(w), v) \sim (u, \lambda_h(w)\, v).
$$
We show that the canonical map $V \to \St(\Phi, A_h), (u, v) \mapsto \iota(u)\, \lambda_h(v)$ is a bijection.

Let us construct an action of the group $\St(\Phi, A_h)$ on $V$. Take $u \in \St(\Phi, B_h)$, $v \in \St(\Phi, A)$, $\alpha \in \Phi$, and $\frac c {h^s} \in A_h$. Choose a strict representative $\eta$ of the automorphism $\Xi(u)^{-1}$ of the pro-group $\St^\infty(\Phi, h^\bullet B)$ and a strict representative $\theta$ of the automorphism $\Xi\bigl(\iota(u)\bigr)^{-1}$ of the pro-group $\St^\infty(\Phi, h^\bullet A)$ using Corollart~\ref{vorcor} in such a way that $\eta^*(0) = \theta^*(0)$ and $$\iota\bigl(\eta_0(x)\bigr) = \theta_0\bigl(\iota(x)\bigr)$$ for all $x\in\St(\Phi, h^{\eta^*(0)}B)$.

Choose $n \geq \theta^*(0) + s$ and a decomposition $c = ah^n + b$ for $a \in A$, $b \in B$. Define
$$\textstyle
x_\alpha(\frac c {h^s}) (u, v) = \bigl(x_\alpha(\frac b {h^s})\, u, \theta_0(x_\alpha(ah^{n - s}))\, v\bigr).
$$
In lemma \ref{well-def} below we show that this construction is independent on all choices. Then the actions of all Steinberg generators $x_\alpha(\frac c {h^s})$ on $V$ are well-defined and it remains to check that they give an action of the Steinberg group $\St(\Phi, A_h)$.

At first we show that 
$$\textstyle
x_\alpha(\frac c {h^s})\, \bigl(x_\alpha(\frac{c'}{h^s})\, (u, v)\bigr) = x_\alpha(\frac{c+c'}{h^s})\, (u, v).
$$
Choose decompositions $c = ah^n + b$, $c' = a'h^n + b'$ as in the definition of the action, where \(n\) is sufficiently large. Since $x_\alpha(\frac{b'}{h^s})$ trivially acts on $x_\alpha(ah^{n-s})$ for sufficiently large $n$, there are strict representatives $\theta$ of $\Xi(\iota(u))^{-1}$ and $\theta'$ of $\Xi(x_\alpha(\frac{b'}{h^s})\, \iota(u))^{-1}$ such that
$$
\theta'_0\bigl(x_\alpha(ah^{n-s})\bigr) = \theta_0\bigl(x_\alpha(ah^{n-s})\bigr)
$$
and similarly for the corresponding $\eta$ and $\eta'$. Then
$$\textstyle
x_\alpha(\frac c {h^s})\, \bigl(x_\alpha(\frac{b'}{h^s})\, u, \theta_0(x_\alpha(a' h^{n-s}))\, v\bigr) = \bigl(x_\alpha(\frac{b + b'}{h^s})\, u, \theta'_0(x_\alpha(a h^{n - s}))\, \theta_0(x_\alpha(a' h^{n - s}))\, v\bigr),
$$
this proves the identity. The proof that $x_\alpha(\frac c {h^s})\, x_\beta(\frac{c'}{h^s})$ and $x_\beta(\frac{c'}{h^s})\, x_\alpha(\frac c {h^s})$ identically act for $\alpha + \beta \notin \Phi \cup \{0\}$ is the same.

In the case $\alpha + \beta \in \Phi$ we have to show that
$$\textstyle
f = x_\alpha(\frac c {h^s})\, x_\beta(\frac{c'}{h^s})
\text{ and }
g = x_{\alpha + \beta}(N_{\alpha \beta} \frac{cc'}{h^{2s}})\, x_\beta(\frac{c'}{h^s})\, x_\alpha(\frac c {h^s})
$$ 
act identically on $V$. Choose decompositions $c = ah^n + b$, $c' = a'h^n + b'$ and $cc' = (aa' h^n + ab' + a'b) h^n + bb'$ for sufficiently large \(n\).
Then
$$\textstyle
f\,(u, v) = \bigl(x_\alpha(\frac b {h^s})\, x_\beta(\frac{b'}{h^s})\, u,
\theta'_0(x_\alpha(ah^{n - s}))\, \theta_0(x_\beta(a'h^{n - s}))\, v\bigr),
$$
where $\theta$ is a strict representative of $\Xi(\iota(u))^{-1}$ and $\theta'$ is a strict representative of $\Xi(x_\beta(\frac{b'}{h^s})\, \iota(u))^{-1}$. It is possible to choose $\theta'$ in such a way that
$$
\theta'_0(x_\alpha(ah^{n - s})) = \theta_0\bigl(x_\alpha(ah^{n - s})\, x_{\alpha + \beta}(N_{\alpha\beta} ab'h^{n - 2s})\bigr).
$$
Next, 
\begin{multline*}\textstyle
g\, (u, v) = \bigl(x_{\alpha+\beta}(N_{\alpha \beta} \frac{bb'}{h^{2s}})\, x_\beta(\frac{b'}{h^s})\, x_\alpha(\frac b {h^s})\, u,\\
\theta'''_0\bigl(x_{\alpha + \beta}\bigl(N_{\alpha \beta} (aa' h^n + ab' + a'b) h^{n - 2s}\bigr)\bigr)\, \theta''_0(x_\beta(a'h^{n-s}))\, \theta_0(x_\alpha(ah^{n - s}))\, v\bigr),
\end{multline*}
where $\theta''$ is a strict representative of $\Xi\bigl(x_\alpha(\frac b {h^s})\, \iota(u)\bigr)^{-1}$ and $\theta'''$ is a strict representative of $\Xi\bigl(x_\beta(\frac{b'}{h^s})\, x_\alpha(\frac b{h^s})\, \iota(u)\bigr)^{-1}$. We may choose $\theta''$ and $\theta'''$ in such a way that 
$$
\theta''_0(x_\beta(a' h^{n - s})) = \theta_0\bigl(x_{\alpha + \beta}(-N_{\alpha \beta} a'bh^{n - 2s})\, x_\beta(a'h^{n - s})\bigr),
$$
and
$$
\theta'''_0\bigl(x_{\alpha + \beta}\bigl(N_{\alpha \beta} (aa'h^n + ab' + a'b) h^{n - 2s}\bigr)\bigr) = \theta_0\bigl(x_{\alpha + \beta}\bigl(N_{\alpha \beta} (aa'h^n + ab' + a'b) h^{n - 2s}\bigr)\bigr).
$$
It remains to check that
\begin{multline*}
x_\alpha(ah^{n - s})\, x_{\alpha + \beta}(N_{\alpha \beta} ab' h^{n - 2s})\, x_\beta(a' h^{n - s}) =\\
= x_{\alpha + \beta}\bigl(N_{\alpha \beta} (aa' h^n + ab' + a'b) h^{n - 2s}\bigr)\, x_{\alpha + \beta}(-N_{\alpha\beta} a'bh^{n - 2s})\, x_\beta(a'h^{n - s})\, x_\alpha(ah^{n - s}).
\end{multline*}
This is exactly a Steinberg relation.

It follows that the action of $\St(\Phi, A_h)$ on $V$ is well-defined. By construction, the canonical map \(V \to \St(\Phi, A_h)\) preserves the action, where \(\St(\Phi, A_h)\) acts on itself by left multiplication. Since \((1, 1) \in V\) maps to \(1 \in \St(\Phi, A_h)\), it remains to check transitivity of the action on \(V\). But the construction implies that \((u, v) = \iota(u)\, \bigl(\lambda_h(v)\, (1, 1)\bigr)\). Hence \(V \to \St(\Phi, A_h)\) is a bijection.

Now let \(u \in \St(\Phi, B_h)\) and \(v \in \St(\Phi, A)\) be such that \(\iota(u) = \lambda_h(v)\). The class of \((u, v^{-1})\) in \(V\) maps to \(1 \in \St(\Phi, A_h)\). Hence \((u, v^{-1}) \sim (1, 1)\).
\end{proof}

\begin{lemma}\label{well-def}
The action of \(x_\alpha(\frac c{h^s})\) on the element \((u, v) \in V\) is independent on the choices of a decomposition of \(c\), coherent strict representatives of \(\Sigma(u^{-1})\) and \(\Sigma(\iota(u^{-1}))\), and a representative \((u, v)\) of its equivalence class in \(V\).
\end{lemma}
\begin{proof}
Let us show that the action is independent on the choice of a decomposition of $c$. Indeed, let $k = \theta^*(0) + s$ and $c = a' h^k + b'$. Since $a' h^k - a h^n = b - b' \in Ah^k \cap B = Bh^k$, there is $d \in B$ such that $b = b' + dh^k$. Then
$$\textstyle
x_\alpha(\frac b {h^s})\, u = \bigl(x_\alpha(\frac{b'}{h^s})\, u\bigr) \bigl(u^{-1}\, x_\alpha(dh^{k - s})\, u\bigr).
$$
Since $\eta$ is a strict representative of $\Xi(u)^{-1}$, it follows that $$u^{-1}\, x_\alpha(dh^{k-s})\, u = \lambda_h \bigl(\eta_0(x_\alpha(dh^{k - s}))\bigr)$$
and
$$\textstyle
\bigl(x_\alpha(\frac b {h^s})\, u, \theta_0(x_\alpha(ah^{n - s}))\, v\bigr) \sim \bigl(x_\alpha(\frac{b'}{h^s})\, u, \iota\bigl(\eta_0(x_\alpha(dh^{k-s}))\bigr)\, \theta_0(x_\alpha(ah^{n - s}))\, v\bigr).
$$

A coherent choice of $\eta$ and $\theta$ guarantees that
$$
\iota\bigl(\eta_0(x_\alpha(dh^{k-s}))\bigr) = \theta_0(x_\alpha(dh^{k - s})),
$$
and, since $\theta_0$ is a homomorphism,
$$
\theta_0(x_\alpha(dh^{k - s}))\, \theta_0(x_\alpha(ah^{n - s})) = \theta_0(x_\alpha(dh^{k-s} + ah^{n - s})) = \theta_0(x_\alpha(a'h^{k - s})).
$$

Now we check that the action is independent on the choice of strict representatives. Let the strict representatives $\eta'$ and $\theta'$ be equivalent to $\eta$ and $\theta$ respectively. By definition of the equivalence, there is $n \geq \max(\theta^*(0) + s, {\theta'}^*(0) + s)$ such that 
$$
\theta_0(x_\alpha(ah^{n - s})) = \theta_0'(x_\alpha(ah^{n - s}))
$$
for all $a \in A$. Since the action is independent on the choice of a decomposition of $c$, it follows that it is also independent on the choices of $\eta$ and $\theta$.

Finally, we prove that the action is independent on the choice of $(u, v)$ in its equivalence class. Let $u = u' \lambda_h(w)$ for some $w \in \St(\Phi,\,B)$. Choose coherent strict representatives $\eta$, $\theta$, $\eta'$, $\theta'$ of $\Xi(u)^{-1}$, $\Xi(\iota(u))^{-1}$, $\Xi(u')^{-1}$, $\Xi(\iota(u'))^{-1}$, by Corollary~\ref{vorcor}. Clearly, for some strict representative $\zeta$ of $\Xi(\iota(w))^{-1}$ we have $\zeta_0(x) = \iota(w)^{-1}\, x\, \iota(w)$ for all $x \in \St(\Phi, h^{\zeta^*(0)} A)$.
Then
$$\textstyle
\bigl(x_\alpha(\frac b {h^s})\, u'\, \lambda_h(w), \theta_0(x_\alpha(ah^{n - s}))\, v\bigr) \sim \bigl(x_\alpha(\frac b {h^s})\, u', \iota(w)\, \theta_0(x_\alpha(ah^{n - s}))\, v\bigr),
$$
so
$$
\iota(w)\, \theta_0\bigl(x_\alpha(ah^{n-s})\bigr) = \theta'_0\bigl(x_\alpha(ah^{n-s})\bigr)\, \iota(w)
$$
for all \(a\) if \(n\) is large enough.
\end{proof}



\section{Formalism of Lindel--Popescu theorem}
В этом параграфе мы аксоматизируем рассуждение Ворста, который применил результаты Линдела о гипотезе Басса--Квиллена для существенно гладких алгебр, чтобы доказать аналог результата Линдела $\K_1$. Ставрова добавила к рассуждению Ворста результаты Попеску, и получила $\K_1$-аналог гипотезы Басса--Квиллена для регулярных колец, содержащих поле (в действительности, не только {\it линейный} $\K_1$-аналог, но и для функторов $\K_1^G$ построенных по изотропной редуктивной группе $G$).

Пусть $k$~--- коммутативное кольцо с $1$, $\mathcal A\mathsf{lg}_k$ обозначает категорию коммутативных $k$-алгебр, и $\mathcal G\mathsf{rp}$ обозначает категорию групп.
\begin{theorem}
\label{lpb}
Пусть ковариантный функтор 
$$
\K\colon\mathcal A\mathsf{lg}_k\rightarrow\mathcal G\mathsf{rp}
$$
удовлетворяет следующим свойствам.\\
%{\rm\bf (PSP)} Perfect subfield property. Если $F\in\mathcal A\mathsf{lg}_k$ является полем, то $F$ содержит совершенное подполе $L\subseteq F$, которое также является $k$-алгеброй.\\
{\rm\bf (CDC)} Commuting with directed colimits. Функтор $\K$ коммутирует с направленными копределами.\\
{\rm\bf (HIF)} Homotopy invariance for fields. Если $F\in\mathcal A\mathsf{lg}_k$ является полем, и $F[t]$ обозначает кольцо многочленов над $F$, то каноническое вложение $F\subseteq F[t]$ индуцирует изоморфизм
$$
\K(F)\cong\K(F[t]).
$$
{\rm\bf (HIL)} Homotopy invariance for localizations. Если $R\in\mathcal A\mathsf{lg}_k$ является областью, и каноническое вложение $R\subseteq R[t]$ индуцирует изоморфизм $\K(R)\cong\K(R[t])$, то для любого мультипликативного множества $S\subseteq R$ вложение $R_S\subseteq R_S[t]$ тоже индуцирует изоморфизм
$$
\K(R_S)\cong\K(R_S[t]).
$$
{\rm\bf (LGP)} Quillen--Suslin local-global principle. Если для $R\in\mathcal A\mathsf{lg}_k$, и $g\in\K(R[t])$ естественное отображение эвалюации $\mathrm{ev}_{t=0}\colon R[t]\rightarrow R$ и все максимальные локализации $\lambda_{\mathfrak m}\colon R[t]\rightarrow R_{\mathfrak m}[t]$, где $\mathfrak m\in\mathrm{Max}\,R$, переводят $g$ в $1$, то есть, $\K(\mathrm{ev}_{t=0})(g)=1\in\K(R)$, и $\K(\lambda_{\mathfrak m})(g)=1\in\K(R_{\mathfrak m}[t])$ для всех $\mathfrak m$, то $$g=1\in\K(R[t]).$$
{\rm\bf (LHT)} Local Horrocks' Theorem. Если $R\in\mathcal A\mathsf{lg}_k$ является локальной областью, то естественное отображение $\lambda_t\colon R[t]\rightarrow R[t,t^{-1}]$ индуцирует инъективное отображение 
$$
\K(\lambda_t)\colon\K(R[t])\rightarrowtail\K(R[t,t^{-1}]),
$$ 
и пересечение образов $\K(\lambda_t)$ и $\K(\lambda_{t^{-1}})$ совпадает с $\K(R)$, где $\lambda_{t^{-1}}\colon R[t^{-1}]\rightarrow R[t,t^{-1}]$ обозначает естественное отображение локализации, и $\K(R)$ вложен в $\K(R[t,t^{-1}])$, поскольку $R\rightarrow R[t,t^{-1}]$ является ретракцией,
$$
\mathrm{Im}\,\K(\lambda_t)\cap\mathrm{Im}\,\K(\lambda_{t^{-1}}) = \K(R).
$$
{\rm\bf (LGL)} Lindel's Glueing Lemma. Пусть $B\in\mathcal A\mathsf{lg}_k$~--- подобласть в области $A$, $h\in B\setminus0$ такой, что $A/h = B/h$, и рассмотрим диаграмму
$$\begin{tikzcd}
B \ar{r}{\iota} \ar{d}{\lambda_h} & A \ar{d}{\lambda_h}\\
B_h \ar{r}{\iota} & A_h.
\end{tikzcd}$$
Предположим, что для $u\in\K(B_h)$ и $v\in\K(A)$ верно, что $$\K(\iota)(u)=\K(\lambda_h)(v)\in\K(A_h).$$
Тогда найдётся $w\in\K(B)$ такой, что $u=\K(\lambda_h)(w)$, и $v=\K(\iota)(w)$.

Тогда для функтора $\K$ как следствие имеет место следующее свойство.\\
{\bf(ALP)} Analogue of Lindel--Popescu Theorem. Для любой алгебры $R\in\mathcal A\mathsf{lg}_k$, являющейся регулярным {\rm(}нётеровым{\rm)} кольцом, и содержащей совершенное поле $F\in\mathcal A\mathsf{lg}_k$, естественное отображение $R\subseteq R[t]$ индуцирует изоморфизм
$$
\K(R)\cong\K(R[t]).
$$
\end{theorem}

Доказательство теоремы разбито на ряд лемм, чтобы было проще следить, как используются разные свойства. Схематически взаимосвязь разных свойств, введённых в этом параграфе, можно изобразить при помощи следующей диаграммы.

\begin{center}
\begin{tikzpicture}
\node (alp) at ( 2,  0) {\textbf{(ALP)}};
\node (asp) at ( 0,  0) {\textbf{(ASP)}}
  edge[->] (alp);
\node (hil) at ( 2, -1) {\textbf{(HIL)}}
  edge[->] (alp);
\node (cdc) at ( 0, -1) {\textbf{(CDC)}}
  edge[->] (asp)
  edge[->] (alp);
\node (zgl) at (-2, -1) {\textbf{(ZGL)}}
  edge[->] (asp);
\node (lgl) at (-2, -2) {\textbf{(LGL)}}
  edge[->] (zgl)
  edge[->, bend right = 20] (alp);
\node (hif) at (-2,  0) {\textbf{(HIF)}}
  edge[->] (asp);
\node (ght) at ( 0,  1) {\textbf{(GHT)}}
  edge[->] (asp);
\node (lgp) at ( 0,  2) {\textbf{(LGP)}}
  edge[->] (ght)
  edge[->, bend left = 20] (alp);
\node (lht) at (-2,  1) {\textbf{(LHT)}}
  edge[->] (ght);
\end{tikzpicture}
\end{center}

Прежде всего, локально-глобальный принцип позволяет глобализовать теорему Хоррокса, см.~[??].
\begin{lemma}[Global Horrocks' Theorem]
\label{ght}
Пусть $\K\colon\mathcal A\mathsf{lg}_k\rightarrow\mathcal G\mathsf{rp}$ удовлетворяет свойствам {\bf(LGP)} и {\bf(LHT)} из формулировки Теоремы~\ref{lpb}. Тогда $\K$ удовлетворяет также следующему свойству.\\
{\rm\bf(GHT)} Если $R\in\mathcal A\mathsf{lg}_k$ является областью, то естественное отображение $\lambda_t\colon R[t]\rightarrow R[t,t^{-1}]$ индуцирует инъективное отображение 
$$
\K(\lambda_t)\colon\K(R[t])\rightarrowtail\K(R[t,t^{-1}]),
$$ 
и пересечение образов $\K(\lambda_t)$ и $\K(\lambda_{t^{-1}})$ совпадает с $\K(R)$, 
$$
\mathrm{Im}\,\K(\lambda_t)\cap\mathrm{Im}\,\K(\lambda_{t^{-1}}) = \K(R).
$$
\end{lemma}
\begin{proof}
Докажем первое утверждение. Действительно, пусть $g\in\K(R[t])$, и $\K(\lambda_t)(g)=1\in\K(R[t,t^{-1}])$. Поскольку $R\hookrightarrow R[t]$ является ретракцией, мы считаем $\K(R)$ вложенным в $\K(R[t])$, и рассмотрим $h=g\cdot\K(\mathrm{ev}_{t=0})(g)^{-1}$. Тогда $\K(\mathrm{ev}_{t=0})(h)=1$, кроме того, $K(\lambda_t)\big(\K(\lambda_{\mathfrak m})(h)\big)=1$ для любого $\mathfrak m\in\mathrm{Max}\,R$, поэтому из {\bf(LHT)} следует, что $\K(\lambda_{\mathfrak m})(h)=1$, а тогда из {\bf(LGP)} следует, что $h=1$, то есть, $g=\K(\mathrm{ev}_{t=0})(g)$. Однако по условию $\K(\lambda_t)(g)=1$, а $\K(R)$ вкладывается в $\K(R[t,t^{-1}])$, поэтому $g=1$.

Пусть теперь $\K(\lambda_t)(g)=\K(\lambda_{t^{-1}})(h)\in\K(R[t,t^{-1}])$. Обозначим $f=\K(\mathrm{ev}_{t=0})(g)$, тогда $\K(\lambda_{\mathfrak m})(gf^{-1})\in\K(R)$ по {\bf(LHT)} для любого $\mathfrak m\in\mathrm{Max}\,R$, а поскольку $\K(\mathrm{ev}_{t=0})(gf^{-1})=1$, то $\K(\lambda_{\mathfrak m})(gf^{-1})=1$. Тогда по {\bf(LGP)} мы заключаем, что $g=f\in\K(R)$.
\end{proof}

Также отметим, что частным случаем~{\bf(LGL)} является следующее свойство.
\begin{lemma}[Zariski Glueing Lemma]
\label{zgl}
Пусть $\K\colon\mathcal A\mathsf{lg}_k\rightarrow\mathcal G\mathsf{rp}$ удовлетворяет свойству {\bf(LGL)} из формулировки Теоремы~\ref{lpb}. Тогда он также удовлетворяет следующему свойству.\\
{\rm\bf(ZGL)}
Пусть $R\in\mathcal A\mathsf{lg}_k$~--- область, $a$, $b\in R$ порождают единичный идеал, $aR+bR=R$. Рассмотрим диаграмму
$$\begin{tikzcd}
R \ar{r}{\lambda_a} \ar{d}{\lambda_b} & R_a \ar{d}{\lambda_b}\\
R_b \ar{r}{\lambda_a} & R_{ab}.
\end{tikzcd}$$
Предположим, что для $u\in\K(R_a)$ и $v\in\K(R_b)$ верно, что $$\K(\lambda_b)(u)=\K(\lambda_a)(v)\in\K(R_{ab}).$$
Тогда найдётся $w\in\K(R)$ такой, что $u=\K(\lambda_a)(w)$, и $v=\K(\lambda_b)(w)$.
\end{lemma}
\begin{proof}
Положим $B=R$, $A=R_a$ и $h=b$. Заметим, что $Ah+B=A$, поскольку для любого $r/a^s\in A=R_a$ можно выбрать разложение единицы $xa^s+yb^s=1$, и тогда $r/a^s$ раскладывается в сумму $rx\in R=B$, и $ryb^s/a^s\in bR_a=hA$ (при $s=0$ можно взять $y=0$). Кроме того, заметим, что $Ah\cap B=Bh$. Действительно, если $rb/a^s=c\in B=R$, то есть, $rb=a^sc$, то, пользуясь разложением единицы $xa^s+yb^s=1$, получаем, что $c=cxa^s+cyb^s=xrb+cyb^s=(xr+cyb^{s-1})b\in bR=hB$ (при $s=0$ можно считать, что $y=0$). Остаётся применить {\bf(LGL)}.
\end{proof}

Следующая лемма в частных случаях была доказана в~[??].
\begin{lemma}
\label{lmp}
Пусть функтор $\K\colon\mathcal A\mathsf{lg}_k\rightarrow\mathcal G\mathsf{rp}$ удовлетворяет свойствам {\bf(ZGL)} из формулировки Леммы~\ref{zgl} и {\bf(GHT)} из формулировки Леммы~\ref{ght}. Тогда для любой $R\in\mathcal A\mathsf{lg}_k$ области, и любого унитального многочлена $f\in R[t]$ естественное отображение $\lambda_f\colon R\rightarrow R_f$ индуцирует инъективное отображение
$$
\K(R[t])\rightarrowtail\K(R[t]_f).
$$
\end{lemma}
\begin{proof}
Если $f=\sum_{i=0}^n a_it^i$, $a_n=1$, рассмотрим многочлен $$g=1+a_{n-1}t^{-1}+\ldots+a_0t^{-n}\in R[t^{-1}],$$ и рассмотрим коммутативную диаграмму
$$\begin{tikzcd}
R[t] \ar{r}{\lambda_t} \ar{d}{\lambda_f} & R[t, t^{-1}] \ar{d} & R[t^{-1}] \ar{l}[']{\lambda_{t^{-1}}} \ar{d}{\lambda_g}\\
R[t]_f \ar{r} & R[t, t^{-1}]_f = R[t, t^{-1}]_g & R[t^{-1}]_g. \ar{l}
\end{tikzcd}$$
Пусть $x\in\K(R[t])$, и $\K(\lambda_f)(x)=1$. Тогда образы элементов $\K(\lambda_t)(x)\in\K(R[t,t^{-1}])$ и $1\in\K(R[t^{-1}]_g)$ в $\K(R[t,t^{-1}]_g)$ совпадают, и по свойству {\bf(ZGL)}, применённому к правому квадрату, существует $y\in\K(R[t^{-1}])$ такой, что $\K(\lambda_{t^{-1}})(y)=\K(\lambda_t)(x)$, и $\K(\lambda_{g})(y)=1$. По свойству {\bf(GHT)} мы заключаем, что $x=y\in\K(R)$, но так как $\K(\lambda_{g})(y)=1$, то $y=1$.
\end{proof}

Пользуясь последней леммой, мы покажем, как вывести из свойства {\bf(HIF)} аналог проблемы Серра.
\begin{lemma}[Analogue of Serre's problem]
\label{asp}
Если функтор $\K\colon\mathcal A\mathsf{lg}_k\rightarrow\mathcal G\mathsf{rp}$ удовлетворяет свойствам {\bf(CDC)} и {\bf(HIF)} из формулировки Теоремы~\ref{lpb}, {\bf(ZGL)} из формулировки Леммы~\ref{zgl} и {\bf(GHT)} из формулировки Леммы~\ref{ght}, то он также удовлетворяет следующему свойству.\\
{\bf(ASP)} Для любого $F\in\mathcal A\mathsf{lg}_k$ являющейся полем, естественное вложение $F\subseteq F[t_1,\ldots,t_n]$ индуцирует изоморфизм
$$
\K(F)\cong\K(F[t_1,\ldots,t_n]),
$$
где $F[t_1,\ldots,t_n]$ обозначает кольцо многочленов.
\end{lemma}
\begin{proof}
Сначала покажем, что естественное вложение кольца $F[t_1,\ldots,t_n]$ в кольцо $F(t_1)[t_2,\ldots,t_n]$ индуцирует инъективное отображение
$$
\K(F[t_1,\ldots,t_n])\rightarrowtail\K(F(t_1)[t_2,\ldots,t_n]).
$$
Действительно, пусть $x\in\K(F[t_1,\ldots,t_n])$ переходит при этом отображении в единицу. Поскольку $F(t_1)[t_2,\ldots,t_n]$ является копределом $F[t_1,\ldots,t_n]_f$ по унитальным $f\in F[t_1]$, то по {\bf(CDC)} найдётся унитальный $f\in F[t_1]$ такой, что $\K(\lambda_f)(x)=1$. Тогда применяя Лемму~\ref{lmp} к кольцу $R=F[t_2,\ldots,t_n]$, получаем, что $x=1$. 

Теперь докажем утверждение леммы индукцией по числу полиномиальных переменных $n$. Ясно, что отображение $F(t_1)[t_2,\ldots,t_n]\rightarrow F(t_1)$, эвалюирующее $t_i$ в $0$ при $i>1$, индуцирует изоморфизм
$$
\K\big(F(t_1)[t_2,\ldots,t_n]\big)\cong\K\big(F(t_1)\big)
$$
по индукционному предположению. Тогда из коммутативности диаграммы
$$\begin{tikzcd}
\K\bigl(F[t_1, \ldots, t_n]\bigr) \ar[tail]{r} \ar{d}{t_i=0} & \K\bigl(F(t_1)[t_2, \ldots, t_n]\bigr) \ar{d}{\cong}\\
\K\bigl(F[t_1]\bigr) \ar{r} & \K\bigl(F(t_1)\bigr)
\end{tikzcd}$$
мы получаем, что $\K(F[t_1,\ldots,t_n])$ вкладывается в $\K\big(F(t_1)\big)$, и его образ содержится в образе $\K(F[t_1])$. Однако по {\bf(HIF)} мы знаем, что $\K(F[t_1])\cong\K(F)$, следовательно образ $\K(F[t_1,\ldots,t_n])$ совпадает с $\K(F)$.
\end{proof}

Следующие теоремы Линдела и Попеску являются ключевыми ингридиентами в доказательстве гипотезы Басса--Квиллена для регулярных колец, содержащих поле. 
\begin{theorem}[Lindel]
\label{lindel}
Пусть $F$~--- совершенное поле, и $A$~--- локальная область положительной размерности, существенно гладкая над $F$. Тогда $A$ содержит подобласть локальную подобласть $B$, являющуюся локализацией кольца многочленов над $F$ {\rm(}от нескольких переменных{\rm)}, и существует $h\in B\setminus A^\times$ такой, что $A/h=B/h$. 
\end{theorem}
См.~[Lindel, Lemma, Proposition~2 and~2'], [Vorst, Proposiion~3.2], а также [Bhatwadekar, Proposition~4.8].

\begin{theorem}[Popescu]
\label{popescu}
Пусть $F$~--- совершенное поле, и $R$~--- регулярная локальная область, содержащая $F$. Тогда $R$ является направленным копределом локальных областей, существенно гладких над $F$.
\end{theorem}
См.~[Popescu], [Swan].\\

Наконец, мы можем завершить доказательство Теоремы~\ref{lpb}. Рассуждение следует статьям [Vorst] и [Stavrova].

\begin{lemma}
Пусть функтор $\K\colon\mathcal A\mathsf{lg}_k\rightarrow\mathcal G\mathsf{rp}$ удовлетворяет свойствам {\bf(CDC)}, {\bf(HIL)}, {\bf(LGP)}, {\bf(LGL)} из формулировки Теоремы~\ref{lpb}, а также {\bf(ASP)} из формулировки Леммы~\ref{asp}. Тогда для него верно заключение Теоремы~\ref{lpb} {\bf(ALP)}. 
\end{lemma}
\begin{proof}
Сначала докажем заключение Теоремы~\ref{lpb} для области $A$, существенно гладкой над совершенным полем $F\in\mathcal A\mathsf{lg}_k$. Проведём индукцию по размерности $A$. Благодаря {\bf(LGP)} можно считать, что $A$ локально. Тогда если $d=\mathrm{dim}\,A$ равна нулю, то $A$ является полем, и мы можем применить {\bf(ASP)}. Для $d>0$ также будем считать, что $A$ локально. Пусть $g\in\K(A[t])$, и мы можем считать, что $\K(\mathrm{ev}_{t=0})(g)=1$. Выберем $B\subseteq A$ и $h\in B\setminus A^\times$ по Теореме~\ref{lindel}. Возможно, $A_h$ перестало быть локальным, однако $\mathrm{dim}\,A_h<d$ (ведь максимальный идеал $A$ становится единичным при локализации), и мы заключаем по индукционному предположению, что $\K(A_h[t])=\K(A_h)$. Тогда $\K(\lambda_h)(g)=1$, и по {\bf(LGL)} можно выбрать общий прообраз $f\in\K(B[t])$ для $g$ и $1\in\K(B_h[t])$. Из {\bf(HIL)} и {\bf(ASP)} следует, что $f\in\K(B)$, значит $g\in\K(A)$.

Теперь из {\bf(CDC)} и Теоремы~\ref{popescu} немедленно следует, что $\K(R[t])\cong\K(R)$ для любой локальной регулярной области, содержащей совершенное поле $F\in\mathcal A\mathsf{lg}_k$. Благодаря {\bf(LGP)}, из этого следует {\bf(ALP)}.
\end{proof}

\section{Main results}
\subsection{Analogue of Serre's problem for \texorpdfstring{$\K_2$}{K2}}

В этом параграфе $\Phi$ обозначает произвольную систему корней.

\begin{lemma}
\label{k2cdc} 
The functors $\Gsc(\Phi,\,-)$, $\St(\Phi,\,-)$ and $\K_2(\Phi,\,-)$ commute with filtered colimits.
\end{lemma}
\begin{proof}
The assertion for $\mathrm G_{\mathrm{sc}}(\Phi,\,-)$ follows from the fact that it is represented in the category $\catname{Ring}$ by a finitely presented Hopf $\ZZ$-algebra (see e.\,g. \cite[Lemma~10.127.3]{stacks-project}). The assertion for $\St(\Phi, -)$ is obvious from its definition. 

The assertion for the functor $\K_2(\Phi, -)$ now follows from the assertions for $\St(\Phi, -)$ and $\Gsc(\Phi, R)$ using the fact that filtered colimits commute with finite limits in $\catname{Grp}$ (in particular, they commute with kernels).
\end{proof}

\begin{comment}
Пусть $R=\mathrm{colim}_I R_i$~--- направленный копредел колец. Покажем, что 
$$ \St(\Phi,\,R)=\mathrm{colim}_I\St(\Phi,\,R_i). $$ Для этого можно проверить универсальное свойство копредела. Если $f_i\colon\St(\Phi,\,R_i)\rightarrow G$, и $x_\alpha(r)\in\St(\Phi,\,R)$, можно найти $r'\in R_i$ для некоторого $i$ такой, что $r'$ переходит в $r$, и определить $f(x_\alpha(r))=f_i(x_\alpha(r'))$. Легко видеть, что такое определение не зависит от выбора $i$, и сохраняет соотношения Стейнберга. Тогда $f$ будет искомым отображением из $\St(\Phi,\,R)$ в $G$, замыкающим диаграмму до коммутативной. Единственность такого $g$ также легко проверить на образующих. \end{comment}

\begin{comment}
Теперь утверждение про $\K_2$ следует из того, что направленный копредел является точным функтором в категории групп, но поскольку я не знаю, где это написано, то привожу доказательство. Рассмотрим естественное отображение $$ f\colon\mathrm{colim}_I\K_2(\Phi,\,R_i)\rightarrow\K_2(\Phi,\,R), $$ и докажем, что оно является биекцией. Если $x\in\K_2(\Phi,\,R)$, то найдётся $y\in\St(\Phi,\,R_i)$ для некоторого $i$ по предыдущему пункту, который переходит в $x$. Пусть $\phi$ обозначает естественное отображение $$ \phi\colon\St(\Phi,\,-)\rightarrow\mathrm G_{\mathrm{sc}}(\Phi,\,-). $$ Поскольку $\phi(y)$ в пределе переходит в $1$, можно считать, что $\phi(y)=1$, то есть $y\in\K_2(\Phi,\,R_i)$. Это доказывает сюръективность $f$. Пусть теперь $x\in\mathrm{colim}_I\K_2(\Phi,\,R_i)$ переходит в $1$ под действием $f$. Тогда образ $x$ в $x\in\mathrm{colim}_I\St(\Phi,\,R_i)$ также равен $1$, и используя конструкцию прямого предела в категории групп $$ \left(\bigsqcup_i\K_2(\Phi,\,R_i)/\sim\right)\rightarrow\left(\bigsqcup_i\St(\Phi,\,R_i)/\sim\right), $$ мы заключаем, что $x$ равен $1$.
\end{comment}

\begin{lemma}
\label{k2hil}
Если $R$ является областью, $\Phi$ имеет ранг хотя бы $3$ {\rm(}наверное, можно и $2$, но нужно куда-то сослаться или написать что-то более подробное в доказательстве{\rm)}, и каноническое вложение $R\subseteq R[t]$ индуцирует изоморфизм $\K_2(\Phi,\,R)\cong\K_2(\Phi,\,R[t])$, то для любого мультипликативного множества $S\subseteq R$ вложение $R_S\subseteq R_S[t]$ тоже индуцирует изоморфизм.
\end{lemma}
\begin{proof}
Пусть $g\in\K_2(\Phi,\,R_S[t])$, и $g(0)=1$. Тогда для некоторого $f\in S$ верно, что $g(ft)$ приходит из $\St(\Phi,\,R[t])$. Действительно, достаточно рассмотреть любое разложение $g$ в произведение $z_{\alpha}(r/s,\,tq)$, где $r,q\in R[t]$, $s\in S$, и разложить
$$
z_{\alpha}(r/s,\,f^2tq)=[x_{\alpha-\beta}(ftq),\,x_{\beta}(fN_{\alpha,\,\beta})]^{x_{-\alpha}(r/s)}
$$
в случае системы корней с простыми связями. Тогда ясно, что достаточно большой $f$ можно подобрать. Если в системе корней есть кратные связи, рассуждение аналогично.

Итак, пусть $h\in\St(\Phi,\,R[t]$, и $\lambda_S(h)=g(ft)$, $f\in S$. Ясно, что в действительности $h\in\K_2(\Phi,\,R[t])$, поэтому $h\in\K_2(\Phi,\,R)$. Таким образом, $g(ft)\in\K_2(\Phi,\,R_S)$, $f\in S$, и подставляя $t\mapsto f^{-1}t$ получаем, что $g\in\K_2(\Phi,\,R_S)$.
\end{proof}

Следующая Теорема доказана в~\cite[Korollar von Satz 1]{Re75}.

\begin{theorem}
Если $F$ является полем, то $\K_2(\Phi,\,F[t])\cong\K_2(\Phi,\,F)$.
\end{theorem}

\begin{theorem} \label{theorem:LP-for-K2}
Пусть $R$ является регулярным кольцом, содержащим поле, и $\Phi=\rA_l$, $l\geq4$, или $R$ является регулярным кольцом, содержащим поле характеристики $\neq2$, и $\Phi=\rD_l$, $l\geq7$. Тогда
$$
\K_2(\Phi,\,R[t])\cong\K_2(\Phi,\,R).
$$
\end{theorem}
\begin{proof}
Заметим, что если $k=\mathbb Z$ или $k=\mathbb Z[1/2]$, то $k$-алгебра, являющаяся полем, содержит простое подполе, которое совершенно и также является $k$-алгеброй. Тогда остаётся применить Теорему~\ref{lpb}.
\end{proof}

\begin{corollary}
Если $F$~--- поле, $\mathrm{char}\,F\neq2$, $l\geq7$, то
$$
\mathrm H_2\big(\mathrm{Spin}_{2l}(F[t_1,\ldots,t_n]),\,\mathbb Z\big)=\cfrac{F^\times\otimes F^\times}{a\otimes(1-a)}.
$$
\end{corollary}
\subsection{An analogue of Gersten's conjecture for $\K_2$.}

\subsection{\texorpdfstring{$\K_2(\Phi, R)$}{K2(R)} as motivic fundamental group}
For a ring $R$ we denote by $R[\Delta^\bullet]$ the standard simplicial ring, see e.\,g.~\cite{Jar83}. The application of a functor $G\colon\catname{Rings}\rightarrow\catname{Grps}$ to $R[\Delta^\bullet]$ yields a simplicial group $G(R[\Delta^\bullet])$, which in the context of $\mathbb{A}^1$-homotopy theory is usually called the {\it simplicial resolution} of $G$ and is denoted by $\mathrm{Sing}^{\mathbb{A}^1}_\bullet(G)(R)$.

The aim of this section is to show that Stein's group $\K_2(\Phi, R)$ can be interpreted as the fundamental group of the simplicial group $\Gsc(\Phi, R[\Delta^\bullet])$ under the assumption that $\Phi$ and $R$ satisfy the requirements of~\cref{theorem:LP-for-K2}. This is achieved in~\cref{theorem:pi1-GRDelta} below. This result parallels the computation of the fundamental group of the simplicial group $G(k[\Delta^\bullet])$ modeled on an arbitrary isotropic reductive group $G$, see~\cite[Proposition~3.2]{VW16}. As a corollary of~\cref{theorem:pi1-GRDelta} and a general representability result of A.~Asok, M.~Hoyois and M.~Wendt we obtain an interpretation of $\K_2(\Phi, R)$ in terms of $\mathbb{A}^1$-homotopy theory, see~\cref{cor:motivic-pi1}.

Recall from~\cite[\S~17]{May67} that the homotopy group $\pi_n(G)$ of a simplicial group $(G_\bullet, d_i, s_i)$ can be computed as $n$-th homology group of the normalized Moore complex
\[
1 \leftarrow N_0(G) \xleftarrow{\partial_1} N_1(G) \xleftarrow{\partial_2} N_2(G) \xleftarrow{\partial_3} \ldots\,,
\]
\iffalse\[\begin{tikzcd} 1 & N_0(G) \ar[l] & N_1(G) \ar{l}[swap]{\partial_1} & N_2(G) \ar{l}[swap]{\partial_2} & \ar{l}[swap]{\partial_3} \ldots, \end{tikzcd} \] \fi
in which $N_n(G) = \cap_{i=1}^n\Ker(d_i) \trianglelefteq G_n$ and the differential $\partial_k$ is obtained from $d_0\colon G_k \to G_{k-1}$ by restriction. In other words, $\pi_n(G) \cong \mathrm{H}_n(N_\bullet G) = \Ker(\partial_n) / \Im(\partial_{n+1})$.

\begin{prop}\label{prop:pi1-StDelta} Let $\Phi$ be an arbitrary irreducible root system. Then for any ring $R$ the simplicial groups $\E(\Phi,\,R[\Delta^\bullet])$ and $\St(\Phi,\,R[\Delta^\bullet])$ are connected. If, moreover, $\Phi$ has rank at least $2$ then the simplicial group $\St(\Phi, R[\Delta^\bullet])$ is simply-connected. \end{prop}
\begin{proof}
In our computations we identify the ring $R[\Delta^n] = R[t_0,\ldots t_n]/\langle \sum_{i=0}^n t_i -1 \rangle$ with $R[t_1, \ldots, t_n]$ via $t_0 = 1 - \sum_{i=1}^n t_i$. We also use concrete formulas from~\cite{Jar83} for the faces of $R[\Delta^\bullet]$.

The map $d_1\colon\St(\Phi,\,R[\Delta^1])\rightarrow\St(\Phi,\,R[\Delta^0])$ is given by $t_1\mapsto0$, therefore \[N_1\St(\Phi,\,R[\Delta^\bullet])=\Ker(d_1)=\St(\Phi,\,R[t_1],\,\langle t_1\rangle),\]
and $\partial_1\colon N_1\St(\Phi,\,R[\Delta^\bullet])\rightarrow N_0\St(\Phi,\,R[\Delta^\bullet])=\St(\Phi,\,R)$ is induced by $d_0$, which sends $t_1$ to $t_0=1$. 
It is clear that $x_\alpha(r)$ of $\St(\Phi, R)$ is the image of $x_\alpha(rt_1)$ under $\partial_1$.
This shows that $\St(\Phi, R[\Delta^\bullet])$ is connected. The argument for $\mathrm{E}(\Phi, R[\Delta^\bullet])$ is identical.

Now let us verify the second assertion. Notice that the kernel of $\partial_1$ coincides with the intersection $\overline{\St}(\Phi,\,R[t_1],\,\langle t_1\rangle)\cap\overline{\St}(\Phi,\,R[t_1],\,\langle t_1-1 \rangle )$, or, what is the same, with the kernel of the homomorphism
\[\St(\Phi,\,R[t_1])\rightarrow\St(\Phi,\,R)\times\St(\Phi,\,R)\]
sending $g(t_1)$ to $\big(g(0),\,g(1)\big)$. By~\cref{lem:fprod} this homomorphism can be identified with the homomorphism $\St(\Phi, R[t_1]) \to \St(\Phi, R\times R)$ induced by the ring homomorphism of evaluation of $t_1$ at $(0, 1)$.

Since the ideals $\langle t_1 \rangle$ and $\langle t_1-1 \rangle$ are coprime, we can identify $R\times R$ with $R[t_1]/t_1(t_1-1)$ by the Chinese remainder theorem. Thus, we obtain that $\Ker(\partial_1)$ coincides with $\overline{\St}(\Phi,\,R[t_1],\,\langle t_1(t_1-1) \rangle)$ and, in particular, is generated by $x_\alpha\big(t_1(t_1-1)f(t_1)\big)^{g(t_1)}$, where $\alpha \in \Phi$, $f\in R[t_1]$, $g \in \St(\Phi, R[t_1])$.

Notice that the face maps $d_1, d_2\colon\mathrm{St}(\Phi,\,R[\Delta^2])\rightarrow\mathrm{St}(\Phi,\,R[\Delta^1])$ are given by evaluations ($t_1\mapsto0$, $t_2\mapsto t_1$) and ($t_1\mapsto t_1$, $t_2\mapsto0$), respectively. Thus, we obtain that \[N_2\mathrm{St}(\Phi,\,R[\Delta^\bullet])=\Ker(d_1)\cap\Ker(d_2)=\overline{\St}(\Phi,\,R[t_1,\,t_2],\,\langle t_1\rangle )\cap\overline{\St}(\Phi,\,R[t_1,\,t_2],\,\langle t_2\rangle).\]
The differential $\partial_2$ is induced by the face map $d_0 \colon \St(\Phi, R[\Delta^2]) \to \St(\Phi, R[\Delta^1])$, which, in turn, is given by the evaluation ($t_1 \mapsto 1-t_1$, $t_2 \mapsto t_1$). 

It remains to see that the elements $x_{\alpha}\big(t_1t_2\,f(t_2)\big)^{g(t_2)}$ belong to $N_2\St(\Phi,\,R[\Delta^\bullet])$ and are mapped by $\partial_2$ onto the generating set of $\Ker(\partial_1)$ mentioned above. Thus, the normalized Moore complex for $\St(\Phi, R[\Delta^\bullet])$ is exact at $N_1$-term, which completes the proof of the proposition. \end{proof}

Let $f\colon G_\bullet\twoheadrightarrow Q_\bullet$ be a degreewise surjective morphism of simplicial groups (i.\,e. $f_n\colon G_n\to Q_n$ are surjective for all $n$). Recall from~\cite[Theorem~1.3]{Ina75} that in this situation the degreewise kernel $K_\bullet$ (i.\,e. the simplicial group given by $K_n = \Ker(f_n)$ with face and degeneracy maps induced from those of $G_\bullet$) yields a long exact sequence of groups
\begin{equation} \label{eq:simplicial-les} \begin{tikzcd} \ldots \ar{r} & \pi_{1}(G_\bullet) \ar{r} & \pi_1(Q_\bullet) \ar{r} & \pi_0(K_\bullet) \ar{r} & \pi_0(G_\bullet). \end{tikzcd} \end{equation}

\begin{theorem} \label{theorem:pi1-GRDelta}
 For $\Phi$ and $R$ as in~\cref{theorem:LP-for-K2} one has $\pi_1(\Gsc(\Phi, R[\Delta^\bullet])) = \K_2(\Phi, R)$.
\end{theorem}
\begin{proof}
First of all, notice that by the homotopy invariance for $\K_1$ (see e.\,g.~\cite[Theorem~1.3]{Sta14}) the simplicial group $\K_1(\Phi, R[\Delta^\bullet])$ is discrete. Applying the exact sequence~\eqref{eq:simplicial-les} to the canonical morphism $\Gsc(\Phi\,,R[\Delta^\bullet]) \to \K_1(\Phi\,,R[\Delta^\bullet])$, we obtain that $\pi_1(\Gsc(\Phi, R[\Delta^\bullet])) \cong \pi_1(\E(\Phi, R[\Delta^\bullet]))$.
 
Now consider the simplicial map $\mathrm{st}_\bullet \colon \St(\Phi, R[\Delta^\bullet]) \to \E(\Phi, R[\Delta^\bullet])$ given by canonical projections $\mathrm{st}$ in each degree. The application of~\eqref{eq:simplicial-les} yields an exact sequence of groups
\[
\pi_1\bigl(\St(\Phi, R[\Delta^\bullet])\bigr) \to \pi_1\bigl(\E(\Phi, R[\Delta^\bullet])\bigr) \to \pi_0\bigl(\K_2(\Phi, R[\Delta^\bullet])\bigr) \to \pi_0\bigl(\St(\Phi, R[\Delta^\bullet])\bigr).
\]
\iffalse \[
\begin{tikzcd} \pi_1(\St(\Phi, R[\Delta^\bullet])) \ar[r] & \pi_1(\E(\Phi, R[\Delta^\bullet])) \ar[r] & \pi_0(\K_2(\Phi, R[\Delta^\bullet])) \ar[r] & \pi_0(\St(\Phi, R[\Delta^\bullet])). \end{tikzcd}
\] \fi
 Proposition~\ref{prop:pi1-StDelta} implies that the first and the last groups in this exact sequence are trivial, so the two central groups are isomorphic. On the other hand, \cref{theorem:LP-for-K2} implies that $\K_2(\Phi, R[\Delta^\bullet])$ is discrete, so $\pi_0(\K_2(\Phi, R[\Delta^\bullet])) = \K_2(\Phi, R)$, as required.
\end{proof}

Now combining the above theorem with~\cite[Theorem~4.3.1]{AHW18} we obtain the following.
\begin{corollary} \label{cor:motivic-pi1} Let $k$ be an infinite field, $A$ be an arbitrary essentially smooth $k$-algebra. Then for $\Phi$ as above one has $\pi_1^{\mathbb{A}^1}(\Gsc(\Phi, -))(A) = [S^1 \wedge (\mathrm{Spec}(A))_+, \Gsc(\Phi, -)]_{ \mathbb{A}^1,*} = \K_2(\Phi, A).$
\end{corollary}



\printbibliography
\end{document}
